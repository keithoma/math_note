\documentclass[a4paper]{article}
\title{Topology}
\author{K}


% ---------------------------------------------------------------------
% P A C K A G E S
% ---------------------------------------------------------------------

% typography and formatting
\usepackage[english]{babel}
\usepackage[utf8]{inputenc}
\usepackage{geometry}
\usepackage{exsheets}
\usepackage{environ}
\usepackage{graphicx}
\usepackage{cutwin}

% mathematics
\usepackage{amsthm} % for theorems, and definitions
\usepackage{amssymb}
\usepackage{amsmath}
\usepackage{textcomp}
% \usepackage{MnSymbol} % for \cupdot

% extra
\usepackage{xcolor}
\usepackage{tikz}

% ---------------------------------------------------------------------
% S E T T I N G
% ---------------------------------------------------------------------

%maybe delete later, for colorbox
\usepackage{tcolorbox}
\newtcolorbox{defbox}{colback=blue!5!white,colframe=blue!75!black}
\newtcolorbox{defboxlight}{colback=cyan!5!white,colframe=cyan!75!black}
\newtcolorbox{thmbox}{colback=orange!5!white,colframe=orange!75!black}
\newtcolorbox{rembox}{colback=purple!5!white,colframe=purple!75!black}
\newtcolorbox{exmbox}{colback=gray!5!white,colframe=gray!75!black}

% typography and formatting
\geometry{margin=3cm}

\SetupExSheets{
  counter-format = ch.qu,
  counter-within = chapter,
  question/print = true,
  solution/print = true,
}

% mathematics
\newcounter{global}

\theoremstyle{definition}
\newtheorem{definition}{Definition}[]
\newtheorem{example}{Example}[definition]

\newtheorem{theorem}[definition]{Theorem}
\newtheorem{corollary}{Corollary}
\newtheorem{lemma}[definition]{Lemma}
\newtheorem{proposition}[definition]{Proposition}

\newtheorem*{remark}{Remark}

% extra
\definecolor{mathif}{HTML}{0000A0} % for conditions
\definecolor{maththen}{HTML}{CC5500} % for consequences
\definecolor{mathrem}{HTML}{8b008b} % for notes
\definecolor{mathobj}{HTML}{008800}

\usetikzlibrary{positioning}
\usetikzlibrary{shapes.geometric, arrows}

% ---------------------------------------------------------------------
% C O M M A N D S
% ---------------------------------------------------------------------

\newcommand{\norm}[1]{\left\lVert#1\right\rVert}
\newcommand{\rank}{\text{rank}}
\newcommand{\Vol}{\text{Vol}}

\newcommand{\set}[1]{\left\{\, #1 \,\right\}}
\newcommand{\makeset}[2]{\left\{\, #1 \mid #2 \,\right\}}

\newcommand*\diff{\mathop{}\!\mathrm{d}}
\newcommand*\Diff{\mathop{}\!\mathrm{D}}

\newcommand\restr[2]{{% we make the whole thing an ordinary symbol
  \left.\kern-\nulldelimiterspace % automatically resize the bar with \right
  #1 % the function
  \vphantom{\big|} % pretend it's a little taller at normal size
  \right|_{#2} % this is the delimiter
  }}

% ---------------------------------------------------------------------
% R E N D E R
% ---------------------------------------------------------------------

%\setlength\parindent{0pt}


\newif\ifshowproof
\showprooftrue

\NewEnviron{Proof}{%
    \ifshowproof%
        \begin{proof}%
            \BODY
        \end{proof}%
    \fi%
}%

\begin{document}
\section*{Problem 1}
\begin{proof}
    \(ii) \Rightarrow i)\) is self-evident. We will focus on \(i) \Rightarrow ii)\). Let \(\overline{\xi} \in \mathbb{R}^{d + 1}\) an arbitrage opportunity. Construct another arbitrage opportunity \(\overline{\xi_*} = (\xi_*^0, \xi_*^1, \ldots, \xi_*^d)\) by setting
    \begin{align*}
        \xi_*^0 =& \xi^0 - \overline{\xi} \cdot \overline{\pi} \\
        \xi_*^i =& \xi^i \text{ for all } i \geq 1 \text{.}
    \end{align*}
    For the newly constructed arbitrage opportunity, we have
    \begin{align*}
        \overline{\xi_*} \cdot \overline{\pi} =& \sum_{i = 0}^d \xi_*^i \cdot \pi^i \\
        =& \left(\xi_0 - \overline{\xi} \cdot \overline{\pi}\right) \pi^0 + \sum_{i = 1}^d \xi^i \cdot \pi^i \\
        =& \xi_0 \cdot \pi_0 + \sum_{i = 1}^d \xi^i \cdot \pi^i - \overline{\xi} \cdot \overline{\pi} \cdot \pi^0 \\
        =& \overline{\xi} \cdot \overline{\pi} - \overline{\xi} \cdot \overline{\pi} \qquad \text{ because \(\pi^0 = 1\)} \\
        =& 0 \text{.}
    \end{align*}
    In particular, \(\xi_*\) fullfills the given condition.

    Moreover, it is
    \begin{align*}
        \overline{\xi_*} \cdot \overline{S}(\omega) =& \sum_{i = 0}^d \xi_*^i S^i(\omega) \\
        =& (\xi_0 - \overline{\xi} \cdot \overline{\pi}) S^0(\omega) + \sum_{i = 1}^d \xi^i \cdot S^i(\omega) \\
        =& \overline{\xi} \cdot \overline{S}(\omega) - \overline{\xi} \overline{\pi} S^0 (\omega)
    \end{align*}
    Now, \(\overline{\xi} \cdot \overline{S}(\omega) \geq 0\) \(\mathbb{P}\)-almost surely by definition, \(\overline{\xi} \cdot \overline{\pi} \leq 0\) and \(S^0(\omega) > 0\). Thus, \(\overline{\xi_*} \cdot \overline{S}(\omega) \geq 0\) \(\mathbb{P}\)-almost surely.

    Futhermore, since \(\overline{\xi_*} \cdot \overline{S}(\omega) > \overline{\xi} \cdot \overline{S}(\omega)\), we also have \(\mathbb{P}(\overline{\xi_*} \cdot \overline{S}(\omega) > 0) \geq \mathbb{P}(\overline{\xi} \cdot \overline{S}(\omega) > 0) > 0\) as desired.

    Clearly, \(i)\) or \(ii)\) implies \(iii)\). The other direction is false. Consider a market with \(d = 1\), \(r = 0\), and \(\pi = (1)\). \(\Omega = \mathbb{N}\) and \(S(0) = 2\), and \(S(\omega) = 1\) for all \(\omega \neq 0\). \(\xi = (1, -1)\), so \(\overline{\xi} \cdot \pi = 0\) and yada yada yada, but the probability that you make money is \(0\).
\end{proof}
\newpage
\section*{Problem 2}
\subsection*{a)}
Simply consider \(\overline{\xi} = (-2, 0, 1)\). We have \(\overline{\xi} \cdot \overline{\pi} = 0\) and
\begin{align*}
    \overline{\xi} \cdot \overline{S}(\omega_1) =& -2 \cdot 1.1 + 0 + 3 = 0.8 \\
    \overline{\xi} \cdot \overline{S}(\omega_1) =& -2 \cdot 1.1 + 0 + 4 = 1.8 \\
\end{align*}

\subsection*{b)}
Consider \(\overline{\xi} = (-4, 1, 1)\). We have \(\overline{\xi} \cdot \overline{\pi} = -4 + 2 + 2 = 0\) and
\begin{align*}
    \overline{\xi} \cdot \overline{S}(\omega_1) =& -2 \cdot 1.1 + 3 + 1 = 1.8 \\    \overline{\xi} \cdot \overline{S}(\omega_2) =& -2 \cdot 1.1 + 1 + 3 = 1.8
\end{align*}
\end{document}