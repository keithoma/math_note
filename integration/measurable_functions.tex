\begin{definition}[Measurable Function]
    Let \((X, \mathcal{A}_X)\) and \((Y, \mathcal{A}_Y)\) be measurable spaces. A map \(f: X \rightarrow Y\) is called measurable if the pre-image of every measurable subset of \(Y\) under \(f\) is measurable subset of \(X\), i.e.
        \begin{align}
            B \in \mathcal{A}_Y \Rightarrow f^{-1}(B) \in \mathcal{A}_X \text{.}
        \end{align}
\end{definition}
%
%
%
%
%
\begin{definition}
    Let \((X \mathcal{A}_X)\) be a measurable space. A function \(f: \Omega \rightarrow \overline{\mathbb{R}}\) is called measurable if it is measurable with respect to the Borel \(\sigma\)-algebra on \(\overline{\mathbb{R}}\)
\end{definition}
%
%
%
%
%
%
\begin{definition}[Borel Measurable Maps]
    
\end{definition}
%
%
%
%
%
\begin{theorem}
    Let \((\Omega, \mathcal{A})\) be a measurable space, and \(\mathcal{B} = \sigma(\mathcal{E})\) for a generator \(\mathcal{E} \subset \mathcal{P}(\Omega)\). If for all \(E \in \mathcal{E}\) it is \(f^{-1}(E) \in \mathcal{A}\), then \(f\) is measurable.
\end{theorem}
\begin{example}
    Let \(f:(\mathbb{R}, \mathcal{B}) \rightarrow (\mathbb{R}, \mathcal{B})\) defined as
    \begin{align}
        f(x) := \begin{cases}
            1 x \in Q \\
            -1 x \notin Q
        \end{cases}
    \end{align}
    for a \(Q \notin \mathcal{B}(\mathbb{R})\). Then, \(f^{-1}({1})=Q \notin \mathcal{B}\) and therefore, \(f\) is not measurable even though \(|f| = 1\) is measurable.
\end{example}