\begin{definition}[Cross-section]
    Let \(\mathbb{R}^n = \mathbb{R}^k \times \mathbb{R}^l\) with \(n, k, l \in \mathbb{N}^*\), and \(A \subset \mathbb{R}^n\). Then for a \(y \in \mathbb{R}^l\)
    \begin{align}
        A_y := \left\{ x \in \mathbb{R}^k \middle| (x, y) \in A \right\}
    \end{align}
    is the \(l\)-dimensional cross-sections of \(A\).
\end{definition}
\begin{remark}
    Immediately from the definition above, we have
    \begin{align}
        A = \dot\bigcup_{y \in \mathbb{R}^l}(A_y, y) \text{.}
    \end{align}
    In other words, \( \left\{(A_y, y)\right\}_{y \in \mathbb{R}^l}\) is a patition of \(A\).
\end{remark}
\begin{theorem}[Cavalieri's principle]
    Let \(\mathbb{R}^n = \mathbb{R}^k \times \mathbb{R}^l\) with \(n, k, l \in \mathbb{N}^*\), let \(A \subset \mathbb{R}^k \times \mathbb{R}^l\) a Borel subset of \(\mathbb{R}^n\), and let \( \left\{(A_y, y)\right\}_{y \in \mathbb{R}^l}\) be a patition of \(A\) via cross-sections. Then we have the following
    \begin{enumerate}
        \item For all \(y \in \mathbb{R}^l\), \(A_y\) is Borel subset of \(\mathbb{R}^k\).
        \item Let \(F_A: \mathbb{R}^l \rightarrow [0, \infty], y \mapsto F_A(y) := Vol_K(A_y) = \lambda^k (A_y)\) be the \(k\)-dimensional volume of \(A_y\). Then \(F_A\) is Borel measurable on \(\mathbb{R}^l\).
        \item \(\text{Vol}_n(A) := \int_{\mathbb{R}^l} \text{Vol}_k (A_y)\)
    \end{enumerate}
\end{theorem}
\proof
\begin{enumerate}
    \item Fix \(y \in \mathbb{R}^l\)
\end{enumerate}
\begin{theorem}
    For \(K \subset \mathbb{R^n}\) compact, we have
    \begin{align}
        \text{Vol}_n (K) = \int_{\mathbb{R}} \text{Vol}_{n-1}(K_t)
    \end{align}
\end{theorem}
%
\begin{example}[Volume of a Cylinder]
    Let \(B \in \mathbb{R}^{n-1}\) be compact and \(Z := B \times [0, h] \subset \mathbb{R}^n\). \(Z\) is called the generalized \(n\)-dimensional cylinder with a basis \(B\) and height \(h\) where \(h \geq 0\). \\
    For the cross-sections we have
    \begin{align}
        Z_t = \begin{cases}
            B \quad \text{ if } t \in [0, h] \\
            \varnothing \quad\text{ else.}
        \end{cases}
    \end{align}
    and for the Volume of the Cylinder, we have
    \begin{align}
        \Vol_n(Z) &= \int_0^h \Vol_{n-1}(B) \diff t \\
        &= \Vol_{n-1}(B) \int_0^h \diff t \\
        &= \Vol_{n-1}(B) \cdot h
    \end{align}
    as \(\Vol_{n-1} (\varnothing) = 0\).
\end{example}