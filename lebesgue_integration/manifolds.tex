\begin{definition}
    \(M \subset \mathbb{R}^n\) is a \(k\)-dimensional submanifold, if
    \begin{itemize}
        \item For all \(a \in M\) there exists an open neighbourhood \(U\) around \(a\) in \(\mathbb{R}^n\) and there exists a \(n - k\) continuously partial differentiable \(f_j : U \rightarrow \mathbb{R}\) for \(j = 1, \dots, n-k\) such that
        \begin{align}
            M \cap U = \left\{ x \in U \middle| f_1(x) = \dots = f_{n-k}(x) = 0 \right\}
        \end{align}
        and for all \(x \in U\)
        \begin{align}
            \rank \frac{\partial (f_1, \dots, f_{n-k})}{\partial (x_1, \dots, x_n)}(x) = n-k
        \end{align}
    \end{itemize}
\end{definition}
%
\begin{example}
    Let's construct the simplest submanifold. Let \(n = 2\) and \(k = 1\).
    \begin{align}
        M = \left\{ x \in \mathbb{R}^2 \middle| f(x, y) = c \right\}
    \end{align}
\end{example}
%
\begin{theorem}
    If \(M \subset \mathbb{R}^n\) is a \(k\)-dimensional submanifold then the following are equivalent.
    \begin{enumerate}
        \item For all points \(a \in M\) there exists a open neighbourhood \(U \in \mathcal{U}_a(\mathbb{R})\), and there exists a function \(f_i: U \rightarrow \mathbb{R}\) with \(1 \leq i \leq n-k\)that is \(n-k\) continuously (partially) differentiable such that
        \begin{align}
            M \cap U = \left\{ x \in U \middle| f_1(x) = \dots = f_{n-k}(x) = 0 \right\}
        \end{align}
        and for all \(x \in U\) \(Df (x) = n-k\).
    \end{enumerate}
\end{theorem}
%
\begin{example}
    The figure eight is described by \(f: \mathbb{R} \rightarrow \mathbb{R}^2\), \(f(t):= (\cos t, \sin 2t)\). Define
    \begin{align}
        M := \left\{ x \in \mathbb{R} \middle | \cos x = 0, \sin 2x = 0 \right\}
    \end{align}
    then
    \begin{align}
        D\phi(x) = \begin{pmatrix}
            -\sin t \\
            2\cos 2t
        \end{pmatrix}
    \end{align}
\end{example}
%
\begin{definition}
    A submanifold is \(k\)-dimensional of the class \(\mathcal{C}^\alpha\) if the \(n-k\) functions that describe the submanifold is \(\alpha\) times continuously differentiable.
\end{definition}
%
\begin{theorem}
    Let \(M \subset \mathbb{R}^n\) a \(k\)-dimensional submanifold of the class \(\mathcal{C}^\alpha\). Let \(i = 1, 2\) \(T_i \subset \mathbb{R}^k\) open and \(\varphi_i: T_i \rightarrow V_i \subset M\) KARTEN, i.e. in parameter form of the class \(\mathcal{C}^\alpha\) with \(V:= V_1 \cap V_2 \neq \emptyset\).
\end{theorem}
%
\begin{question}
    Let \(f, g: \mathbb{R}^3 \rightarrow \mathbb{R}\) defined as
    \begin{align}
        & f(x, y, z) := x^2 + xy - y - z & g(x, y, z) := 2x^2 + 3xy - 2y - 3z &
    \end{align}
    Show that
    \begin{align}
        C := \left\{ (x, y, z) \in \mathbb{R}^3 \middle| f(x, y, z) = g(x, y, z) = 0 \right\}
    \end{align}
    is a submanifold of \(\mathbb{R}^3\) and that
    \begin{align}
        \phi: \mathbb{R} \rightarrow \mathbb{R}^3, \phi(t) := (t, t^2, t^3)
    \end{align}
    % is this word, parametization, correct?
    is a global parametization of \(C\).
\end{question}
\begin{solution}
    Define \(F: \mathbb{R}^3 \rightarrow \mathbb{R}^2\) as \(F(x, y, z) = (f(x, y, z), g(x, y, z))\), then \(C\) can be rewritten as
    \begin{align}
        C = \left\{ (x, y, z) \in \mathbb{R}^3 \middle| F(x, y, z) = 0 \right\} \text{.}
    \end{align}
    We have
    \begin{align}
        \partial_x f(x, y, z) &= 2x + y & \partial_x g(x, y, z) &= 4x + 3y \\
        \partial_y f(x, y, z) &= x - 1 & \partial_y g(x, y, z) &= 3x - 2 \\
        \partial_z f(x, y, z) &= -1 & \partial_z g(x, y, z) &= -3
    \end{align}
    therefore
    \begin{align}
        \Diff F (x, y, z) =
        \begin{pmatrix}
            2x + y & x - 1 & - 1 \\
            4x + 3y & 3x - 2 & -3 
        \end{pmatrix}
    \end{align}
    To check if \(\Diff F\) surjective, it is enough to show that
    \begin{align}
        \begin{pmatrix}
            x - 1 \\
            3x - 2
        \end{pmatrix} &
        \begin{pmatrix}
            -1 \\
            -3
        \end{pmatrix}
    \end{align}
    are linearely independent. For that, we compute the determinant of the matrix created by the two vectors.
    \begin{align}
        \det
        \begin{pmatrix}
            x  - 1 & -1 \\
            3x - 2 & -3
        \end{pmatrix}
        = -3x + 3 + 3x - 2 = 1
    \end{align}
    So, \(\Diff F\) has a rank of \(2\), therefore surjective. With this, \(C\) is a submanifold of \(\mathbb{R}^3\).
\end{solution}