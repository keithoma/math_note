\documentclass[a4paper]{book}
\title{Integration and Integration}
\author{K}


% ---------------------------------------------------------------------
% P A C K A G E S
% ---------------------------------------------------------------------

% typography and formatting
\usepackage[english]{babel}
\usepackage[utf8]{inputenc}
\usepackage{geometry}
\usepackage{exsheets}

% mathematics
\usepackage{amsthm} % for theorems, and definitions
\usepackage{amssymb}
\usepackage{amsmath}

% extra
\usepackage{xcolor}
\usepackage{tikz}

% ---------------------------------------------------------------------
% S E T T I N G
% ---------------------------------------------------------------------

% typography and formatting
\geometry{margin=3cm}

\SetupExSheets{
  counter-format = ch.qu,
  counter-within = chapter,
  question/print = true,
  solution/print = true,
}

% mathematics
\theoremstyle{definition}
\newtheorem{definition}{Definition}[chapter]
\newtheorem{example}{Example}[definition]

\newtheorem{theorem}{Theorem}[definition]
\newtheorem{corollary}{Corollary}
\newtheorem{lemma}{Lemma}[definition]

\newtheorem*{remark}{Remark}

% extra
\definecolor{mathif}{HTML}{0000A0} % for conditions
\definecolor{maththen}{HTML}{CC5500} % for consequences
\definecolor{mathrem}{HTML}{8b008b} % for notes

\usetikzlibrary{positioning}
\usetikzlibrary{shapes.geometric, arrows}

% ---------------------------------------------------------------------
% C O M M A N D S
% ---------------------------------------------------------------------

\newcommand{\norm}[1]{\left\lVert#1\right\rVert}

\begin{document}
\maketitle
\tableofcontents
%
%
%
%
%
\chapter*{Introduction}
\addcontentsline{toc}{chapter}{Introduction}
One problem of the Riemann integral is that some functions are not Riemann integratable.
\begin{example}[Dirichlet function]
    For \([a, b] \subset \mathbb{R}\), define the Dirchlet function as
    \begin{align}
        g: [a, b] \rightarrow \mathbb{R}, x \mapsto g(x) :=
        \begin{cases}
            1 \text{ for } x \in \mathbb{Q},\\
            0 \text{ for } x \in \mathbb{R} \setminus \mathbb{Q} \text{.}
        \end{cases}
    \end{align}
\end{example}
What are the properties a generalized concept of volumina should have?
\begin{enumerate}
    \item positive valued
    \item null empty set
    \item monotonous
    \item translationinvariance
    \item normalization
\end{enumerate}

\begin{definition}
Let \(\mu: \mathcal{P}(\mathbb{R}^n) \rightarrow \overline{\mathbb{R}}^{+}_0.\)
\begin{itemize}
    \item \(\mu\) is monotonous.
    \item \(\mu\) is translationinvariant.
    \item \(\mu\) is \(\sigma\)-additive.
\end{itemize}
\end{definition}
\begin{theorem}[Vitali's Theorem]    
\end{theorem}
%
%
%
%
%
\part{\(\sigma\)-algebra and measures}
%
\chapter{Family of Sets}
We have the following tree of inclusion.
\begin{figure}[h]
    \center
    \begin{tikzpicture}[node distance=2cm and 1cm]
        \tikzstyle{box} = [draw=none, minimum width=4cm]
        \tikzstyle{arrow} = [thick,-,>=stealth]
        \node [draw=none]                 (ring of sets) {ring of sets};
        \coordinate[below=of ring of sets] (c);
        \node [box, left=of c]      (algebra of sets) {algebra of sets};
        \node [box] at (c)     (s-ring)     {\(\sigma\)-ring};
        \node [box, right=of c] (monotone class)     {monotone class};
        \node [box, below=of c]     (s-algebra)    {\(\sigma\)-algebra};
        \node [box, below=of s-algebra] (b-algebra) {Borel \(\sigma\)-algebra};
        \draw [arrow] (ring of sets) -- (algebra of sets);
        \draw [arrow] (ring of sets) -- (s-ring);
        \draw [arrow] (algebra of sets) -- (s-algebra);
        \draw [arrow] (s-ring) -- (s-algebra);
        \draw [arrow] (monotone class) -- (s-algebra);
        \draw [arrow] (s-algebra) -- (b-algebra);
    \end{tikzpicture}
\end{figure}
NOTATION GUIDE:
\begin{enumerate}
    \item \(X\) as the superset
    \item \(\mathcal{P}(X)\) is the power set of \(X\).
    \item \(A, B \in \mathcal{P}(X)\) as subsets
    \item \(\mathcal{R}, \mathcal{A} \subset \mathcal{P}(X)\) system of subsets
\end{enumerate}
%
\section{Symmetric Difference}
\begin{definition}[Symmetric difference]
    Let {\color{mathif}\(A, B\)} be {\color{mathif}sets}. The binary set operation {\color{maththen}symmetric difference} is defined as
    \begin{align}
        A \triangle B := (A \setminus B) \cup (B \setminus A) \text{.}
    \end{align}
    In other words, \(x \in A \triangle B\) implies \(x\) is either in \(A\) or \(B\), but not in both.
\end{definition}
\begin{proposition}[Properties of Symmetric Difference]
    Let \(A, B, C, X\) and \(Y\) be {\color{mathif}sets}. Moreover, let \(A_i\) and \(X_i\) be {\color{mathif}sets} with an {\color{mathif}arbitary non-empty index set} \(i \in I\). Then, the following {\color{maththen}identities} hold.
    \begin{enumerate}
        \item \( A \triangle B = (A \cup B) \setminus (A \cap B) \).
        \item \( ( A \triangle B) \triangle C = A \triangle (B \triangle C) \). (Symmetric difference is {\color{mathrem}associative}.) 
        \item \( A \triangle B = B \triangle A \). (Symmetric difference is {\color{mathrem}commutative}.)
        \item \( A \triangle \emptyset = A\) and \( A \triangle A = \emptyset\)
        \item \( (A \triangle B) \cup C = (A \cup C) \triangle (B \cup C)\).
        \item \( A \cap B = \emptyset \Rightarrow A \triangle B = A \cup B \).
        \item \( B \subset A \Rightarrow A \triangle B = A \setminus B \).
        \item \(X \cap Y = \emptyset \Rightarrow A \cap B \subset (X \triangle A) \cup (Y \triangle B) \).
        \item \( (\bigcup_{i \in I} X_i) \triangle (\bigcup_{i \in I} A_i) \subset \bigcup_{i \in I} (X_i \triangle A_i) \)
    \end{enumerate}
\end{proposition}
\begin{Proof}
    Elementary.
\end{Proof}
%
%
%
%
%
\section{Ring of Sets}
\begin{definition}[Ring of sets]
    There are two equivalent definitions. Let {\color{mathif}\(X\)} be a {\color{mathif}set} and {\color{mathif}\(\mathcal{R} \subset \mathcal{P}(X)\)} a {\color{mathif}system of subsets}. Then {\color{maththen} \(\mathcal{R}\)}  is a {\color{maththen}ring of sets over \(X\)}, if
    \begin{enumerate}
        \item the following axioms are met.
        \begin{enumerate}
            \item \(\mathcal{R} \neq \emptyset\) (\(\mathcal{R}\) is {\color{mathrem}nonempty.})
            \item \(A, B \in \mathcal{R} \Rightarrow A \setminus B \in \mathcal{R}\) (\(\mathcal{R}\) is {\color{mathrem}closed under relative complement}.)
            \item \(A, B \in \mathcal{R} \Rightarrow A \cup B \in \mathcal{R}\) (\(\mathcal{R}\) is {\color{mathrem}closed under finite unions}.)
        \end{enumerate}
        \item \((\mathcal{R}, \triangle, \cap)\) is a ring in the algebraic sense, with \(\triangle\) as addition and \(\cap\) as multiplication.
    \end{enumerate}
\end{definition}
\begin{Proof}
    We show that the two definitions above are indeed equivalent.

    \((1 \Rightarrow 2)\) Let \(\mathcal{R}\) be nonempty, closed under the relative complement, and closed under finite unions. First, consider \((\mathcal{R}, \triangle)\). Let \(A, B \in \mathcal{R}\). It is
    \begin{enumerate}
        \item (Closure under addition) \(A \cup B \in \mathcal{R}\) because \(\mathcal{R}\) is closed under finite unions. We also have \(A \cap B = A \setminus (A \setminus B) \in \mathcal{R}\) as \(\mathcal{R}\) is closed under the relative complement. From these it follows that \( A \triangle B = (A \cup B) \setminus (A \cap B) \in \mathcal{R}\) by using the closure under the relative complement again.
        \item (Associativity)
        \item (Commutativity)
        \item (Neutral element) \(\emptyset\)
        \item (Inverse element) \(A\)
    \end{enumerate}
    Therefore, \((\mathcal{R}, \triangle)\) is an abelian group. Secondly, consider \((\mathcal{R}, \cap)\). \(\cap\) is associative and commutative. The identity element is the union of all sets (does this exist??).
\end{Proof}
%
\begin{remark}
    Since we have the identity \(A \cap B = A \setminus (A \setminus B)\), the condition that \(\mathcal{R}\) is closed under the relative complement, i.e.
    \begin{align}
        A, B \in \mathcal{R} \Rightarrow A \setminus B \in \mathcal{R}
    \end{align}
    can be replaced with closure under finite intersection, therefore
    \begin{align}
        A, B \in \mathcal{R} \Rightarrow A \cap B \in \mathcal{R} \text{.}
    \end{align}
\end{remark}
%%%%%%%%%% This is already implied in the definition
% \begin{proposition}[Properties of ring of sets]
%     Let \(\mathcal{R}\) be a ring of sets. It is
%     \begin{enumerate}
%         \item \(\emptyset \in \mathcal{R}\).
%         \item \(A, B \in \mathcal{R} \Rightarrow A \triangle B \in \mathcal{R}\)
%     \end{enumerate}
% \end{proposition}
%%%%%%%%%%
\begin{example}
    Let \(X\) be a set.
    \begin{enumerate}
        \item \(\mathcal{P}(X)\) and \(\{\emptyset, X\}\) are ring of sets.
        \item \(\{\emptyset\}\) is a ring of sets.
    \end{enumerate}
\end{example}
\section{Algebra of Sets}
\begin{definition}[Algebra of sets]
    There are two equivalent definitions. Let {\color{mathif}\(X\)} be a {\color{mathif}set} and {\color{mathif}\(\mathcal{R} \subset \mathcal{P}(X)\)} a {\color{mathif}system of subsets}. Then {\color{maththen} \(\mathcal{A}\)}  is a {\color{maththen}algebra of sets over \(X\)},
    \begin{enumerate}
        \item if \(\mathcal{A}\) is a {\color{mathif}ring of sets} that contains {\color{mathif}\(X\)}, or
        \item if the following axioms are met
        \begin{enumerate}
            \item \(\mathcal{A} \neq \emptyset\) (\(\mathcal{A}\) is {\color{mathrem}nonempty.})
            \item \(A \in \mathcal{A} \Rightarrow A^c \in \mathcal{A}\) (\(\mathcal{R}\) is {\color{mathrem}closed under the absolute complement}.)
            \item \(A, B \in \mathcal{A} \Rightarrow A \cup B \in \mathcal{A}\) (\(\mathcal{R}\) is {\color{mathrem}closed under finite unions}.)
        \end{enumerate}
    \end{enumerate}
\end{definition}
%
%
%
%
%
\section{\(\sigma\)-Ring}
\begin{definition}[\(\sigma\)-Ring]
    Let \(X\) be set and \(\mathcal{R} \subset \mathcal{P}(X)\) a system of subsets. \(\mathcal{R}\) is a \(\sigma\)-ring over \(X\), if
        \begin{enumerate}
            \item \(\mathcal{R} \neq \emptyset\). (\(\mathcal{A}\) is {\color{mathrem}nonempty.})
            \item \(A, B \in \mathcal{R} \Rightarrow A \setminus B \in \mathcal{R}\) ({\color{mathrem}closed under the relative complement}.)
            \item \(A_1, A_2, A_3, ... \in \mathcal{R} \Rightarrow \bigcup_{k=1}^\infty A_k \in \mathcal{R}\) ({\color{mathrem} Closed under countable unions.})
        \end{enumerate}
\end{definition}
%
%
%
%
%
\section{Monotone Class}
\begin{definition}[Notation for Monotonous Sequence of Sets]
    % Let \((X_k)_{k \in \mathbb{N}}\) and \((Y_k)_{k \in \mathbb{N}}\) be two sequence of sets. We write
    % \begin{align}
    %     X_k \uparrow X && Y_k \downarrow Y
    % \end{align}
    % if \(X_k\) and \(Y_k\) are monotonously increasing or decreasing, i.e. \(X_k \subset X_{k + 1}\) or \(Y_k \supset Y_{k + 1}\), and
\end{definition}
\begin{definition}[Monotone class]
    Let \(\mathcal{M} \subset \mathcal{P}(\Omega)\) a system of sets and \(k \in \mathbb{N}^*\). Then, \(\mathcal{M}\) is a monotone class, if
    \begin{enumerate}
        \item Let \(X_k \in \mathcal{M}\) with \(X_k \uparrow X\), then \(X \in \mathcal{M}\).
        \item Let \(Y_k \in \mathcal{M}\) with \(Y_k \downarrow X\), then \(Y \in \mathcal{M}\).
    \end{enumerate}
    Intersection of arbitary many monotonous class is again a monotonous class. Therefore, for all \(\mathcal{E} \subset \mathcal{P}(\Omega)\) with \(\mathcal{E} \neq \emptyset\) there exists the smallest monotonous class around \(\mathcal{E}\)
    \begin{align}
        \mathcal{M}_{\mathcal{E}} := \bigcap_{\mathcal{M} \text{ is monotonous class}, \mathcal{E} \subset \mathcal{M}} \mathcal{M}
    \end{align}
\end{definition}
\begin{remark}
    All \(\sigma\)-algebras are monotone class.
\end{remark}
\begin{theorem}
    Let \(\mathcal{A} \subset \mathcal{P}(\Omega)\) an algebra of sets. Then, the following are equivalent
    \begin{itemize}
        \item \(\mathcal{A}\) is a \(\sigma\)-algebra.
        \item For \(A_k \uparrow A\), \(A \in \mathcal{A}\).
    \end{itemize}
\end{theorem}
%
%
%
%
%
\section{\(\sigma\)-Algebra}
\textbf{CHEAT SHEET}
\begin{enumerate}
    \item closed under complementation (absolute and relative)
    \item closed under countable unions
    \item closed under countable intersections
    \item closed under symmetric differences
\end{enumerate}
\begin{definition}[\(\sigma\)-algebra]
    Let \(X\) be a set and \(\mathcal{A} \subset \mathcal{P}(X)\) a system of subsets. \(\mathcal{A}\) is a \(\sigma\)-algebra over \(X\), if
        \begin{enumerate}
            \item \(\mathcal{A} \neq \emptyset\).
            \item \(A \in \mathcal{A} \Rightarrow A^c \in \mathcal{A}\)
            \item \(A_1, A_2, A_3, ... \in \mathcal{A} \Rightarrow \bigcup_{k=1}^\infty A_k \in \mathcal{A}\)
        \end{enumerate}
\end{definition}
\begin{example}
    Trivial examples for the above structures.
\end{example}
\begin{definition}
    Let \(\mathcal{E} \subset \mathcal{P}(\Omega)\) be a system of sets. Define
    \begin{align}
        \mathcal{F}(\mathcal{E}) &:= \left\{ \mathcal{A} \subset \mathcal{P}(\Omega) \middle| \mathcal{E} \subset \mathcal{A}, \mathcal{A} \sigma\text{-Algebra} \right\} \\
        \left< \mathcal{E}  \right>^{\sigma} &:= \sigma(\mathcal{E}) := \bigcap_{\mathcal{A} \in \mathcal{F}(\mathcal{E})} \mathcal{A}
    \end{align}
    The first is the family of all \(\sigma\)-algebras that contain \(\mathcal{E}\).
    The second is the smallest \(\sigma\)-algebra that contains \(\mathcal{E}\).
\end{definition}
%
%
%
%
%
\section{Product Algebra??}
\begin{definition}
    Let \(\Omega_1\) and \(\Omega_1\) be sets; let \(\mathcal{R}_1 \subset \mathcal{P}(\Omega_1)\) and \(\mathcal{R}_2 \subset \mathcal{P}(\Omega_2)\) be ring of sets, and \(\Omega := \Omega_1 \times \Omega_2\). Define
    \begin{align}
        \mathcal{R} := \mathcal{R}_1 \boxtimes \mathcal{R}_2 := \left\{ \bigcup_{i=1}^m A_i \times B_i \middle| A_i \in \mathcal{R}_1, B_i \in \mathcal{R}_2, m \in \mathbb{N} \right\}
    \end{align}
    \(\mathcal{R}\) is a ring of sets over \(\Omega\).
\end{definition}
\begin{theorem}
    In above definition, if \(\mathcal{R}_1\) and \(\mathcal{R}_2\) are algebra of sets, then \(\mathcal{R}\) is a algebra of set.
\end{theorem}
\begin{theorem}
    \begin{align}
        \mathfrak{Q}(\mathbb{R}^n)
    \end{align}
    is a ring of sets.
\end{theorem}
\begin{remark}
    From \(\mathfrak{Q}(\mathbb{R}^n)\) we can construct one very important \(\sigma\)-algebra, the Borel-Algebra of \(\mathbb{R}^n\).
\end{remark}
\begin{definition}[Products of \(\sigma\)-algebras]
    Let \(\mathcal{A}_1\) and \(\mathcal{A}_2\) be \(\sigma\)-algebras on \(\Omega_1, \Omega_2\). Then, let
    \begin{align}
        \mathcal{A}_1 \otimes \mathcal{A}_2 = \sigma( \mathcal{A}_1 \boxtimes \mathcal{A}_2 )
    \end{align}
\end{definition}
\begin{example}
    \begin{align}
        \mathcal{B}(\mathbb{R}^{n+m}) = \mathcal{B}(\mathbb{R}^n) \otimes \mathcal{B}(\mathbb{R}^m)
    \end{align}
\end{example}
\begin{definition}
    Let \((X_k)_{k \in \mathbb{N}^*}\) be a sequence of sets with \(X_1 \subset X_2 \subset X_3 \subset \dots \) and \(X := \lim_{k \rightarrow \infty} := \bigcup_{k \in \mathbb{N}*} X_k\).
    Similar for monotonously decreasing.
\end{definition}

%
%
%
%
%
\section{Rectangles}
\begin{example}
    Let
    \begin{align}
        \mathfrak{Q}(\mathbb{R}) := \left\{ \bigcup_{i=1}^m [a_i, b_i)  \middle| m \in \mathbb{N}; a_i, b_i \in \mathbb{R} \right\}
    \end{align}
    be the set of all unions of finitely many right half open intervals on \(\mathbb{R}\). Then, \(\mathfrak{Q}(\mathbb{R})\) is a set of rings. Similary for the left half open sets, but not for open or closed intervals!
    \(\mathfrak{Q}(\mathbb{R})\) is neither \(\sigma\)-ring, \(\sigma\)-algebra nor an algebra of sets.
    One can generalize this to higher dimensions.
\end{example}
%
%
%
%
%
%
%
%
%
%
\section{Borel \(\sigma\)-algebra}
\begin{definition}
    Let \(\Omega\) be a set. A collection \(\mathcal{U} \subset \mathcal{P}(\Omega)\) of subsets of X is called a topology on \(X\) if it satisfies the following axioms.
    \begin{enumerate}
        \item \(\emptyset, X \in \mathcal{U}\).
        \item If \(n \in \mathbb{N}\) and \(U_1, \dots U_n \in \mathcal{U}\) then \(\bigcap_{i=1}^n U_i \in \mathcal{U}\).
        \item If \(I\) is any index set and \(U_i \in \mathcal{U}\) for \(i \in I\) then \(\bigcup_{i \in I} U_i \in \mathcal{U}\).
    \end{enumerate}
    A topological space is a pair \((\Omega, \mathcal{U})\) consisting of a set \(\Omega\) and a topology \(\mathcal{U} \in \mathcal{P}(\Omega)\).
\end{definition}
\begin{example}[Standard Topology on \(\overline{\mathbb{R}}\)]
    The set of open subsets \(\mathcal{T}\) of \(\overline{\mathbb{R}}\) is the standard topology on \(\overline{\mathbb{R}}\). Concretely, \(\mathcal{T}\) contains countable union of open intervals in \(\mathbb{R}\) and sets of the form \((a, \infty]\) or \([-\infty, b)\) for \(a, b \in \mathbb{R}\).
\end{example}
\begin{definition}[Borel algebra]
    Let \((\Omega, \mathcal{T})\) be a topological space, then \(\mathcal{B}(\Omega) := \sigma(\mathcal{T})\) is the Borel \(\sigma\)-algebra of \(\Omega\). The elments of \(\mathcal{B}\) are called Borel (measurable) sets.
    There are many ways to generate this algebra.
\end{definition}
\begin{theorem}
    Let \((\Omega, \mathcal{T})\) be a topological space. Then the following holds.
    \begin{enumerate}
        \item Every closed subset \(F \subset \Omega\) is a Borel set.
        \item Every countable union \(\bigcup_{i=1}^\infty F_i\) of closed subsets \(F_i \subset \Omega\) is a Borel set.
        \item Every countable intersection \(\bigcap_{i=1}^\infty F_i\) of open subsets \(F_i \subset \Omega\) is a Borel set.
    \end{enumerate}
\end{theorem}
\begin{theorem}
    It is
    \begin{align}
        \mathcal{B}(\mathbb{R}^n) = \sigma(\mathfrak{Q}(\mathbb{R}^n))
    \end{align}
    Moreover, define
    \begin{align}
        \mathfrak{Q}_{\mathbb{Q}}(\mathbb{R}^n) := \left\{ \bigcup_{i=1}^m [a_{1, i}, b_{1, i}) \times \dots [a_{n, i} \times b_{n, i}) \middle| m \in \mathbb{N}; a_{\nu, i}, b_{\nu, i} \in \mathbb{Q}; \nu = 1, \dots, n \right\}
    \end{align}
    the ring of sets of finite unions of quadern with rational edge points. Then, we even have
    \begin{align}
        \mathcal{R}(\mathbb{R}^n) = \sigma( \mathfrak{Q}_{\mathbb{Q}} (\mathbb{R}^n))
    \end{align}
\end{theorem}
\begin{lemma}
    Open subsets \(U \subset \mathbb{R}^n\) are disjoint union of countably many right half open dices with edge points in \(\mathbb{Q}^n\)
\end{lemma}
%
%
%
\section{Exercises}
\begin{question}
    Let \(X\) be a nonempty set and for all \(1 \leq i \leq m\) with \(m \in \mathbb{N}\) let \(A_i \subset X\) be a finite amount of subsets. Set
    \begin{align}
        S := A_1 \triangle A_2 \triangle \dots \triangle A_m \text{.}
    \end{align}

    Because of the associative property of the symmetric difference, \(S\) is uniquely defined regardless of the order of the operations.

    Show that \(x \in X\) belongs to \(S\) if and only if \(x\) belongs to an odd number of sets \(A_k\), i.e. when the number of indices \(k \in \{1, 2, \dots, m\}\) with \(x \in A_k\) is odd.
\end{question}
\begin{solution}
    % trivial 
\end{solution}
%
%
%
\begin{question}
    Let \(X\) be a nonempty set and \(R := \{f: X \rightarrow \mathbb{F}_2\}\) where \(\mathbb{F}_2 = \{0, 1\}\) is a field of two elements equipped with the common addition and the common multiplication. Moreover, define the operations
    \begin{align}
        (f \oplus g)(x) &:= f(x) + g(x) \\
        (f \otimes g)(x) &:= f(x) \cdot g(x) \text{.}
    \end{align}
    Show the following statements.
    \begin{enumerate}
        \item \((R, \oplus, \otimes)\) is a commutative ring with the identity element.
        \item The map \(\mathcal{P}(X) \rightarrow R, A \mapsto \chi_A\) that maps a subset \(A \subset X\) to its characteristic function is bijective.
        \item For all \(A, B \in \mathcal{P}(X)\) we have
        \begin{align}
            \chi_{A \triangle B} = \chi_A \oplus \chi_B && \chi_{A \cap B} = \chi_A \otimes \chi_B \text{.}
        \end{align}
        \item Conclude from the statements above that \(\mathcal{P}(X)\) is isomorphic to \(R\) as a ring with \(\triangle\) as its addition and with \(\cap\) as its multiplication.
        \item A subset \(\mathcal{R} \subset \mathcal{P}(X)\) is a ring of sets if and only if \(\mathcal{R}\) is a subring of \(\mathcal{P}(X)\) with respects to the ring structure defined above.
    \end{enumerate}
\end{question}
\begin{solution}
    
\end{solution}
%
%
%
\begin{question}
    %PRAESENZUEBUNG 2.1
\end{question}
%
%
%
\begin{question}
    %PREASENZUEBUNG 2.2
\end{question}
% UEBUNGSBLATT 2
\begin{question}
    Show explicitly that the following subsets generate the same \(\sigma\)-algebra on \(\mathbb{R}\).
    \begin{align}
        \mathcal{E}_1 &:= \{(a, b) \mid a, b \in \mathbb{R}, \, a \leq b\} &         \mathcal{E}_2 &:= \{[a, b) \mid a, b \in \mathbb{Q}, \, a \leq b\} \\
        \mathcal{E}_3 &:= \{[a, b) \mid a, b \in \mathbb{R}, \, a \leq b\} &
        \mathcal{E}_4 &:= \{(-\infty, b) \mid a, b \in \mathbb{Q}, \, a \leq b\}
    \end{align}
\end{question}
\begin{solution}
    We want to proof
    \begin{align}
        \sigma (\mathcal{E}_1) = \sigma (\mathcal{E}_2) = \sigma (\mathcal{E}_3) = \sigma (\mathcal{E}_4) \text{.}
    \end{align}
    We will do this by showing four inclusions. In each step, our goal is to show that an arbitary interval from the generator of the superset is included in the \(\sigma\)-algebra of the subset.
    \begin{enumerate}
    \item First we show \(\sigma (\mathcal{E}_1) \subset \sigma (\mathcal{E}_2)\). Fix \(a, b \in \mathbb{Q}\) with \(a \leq b\) and consider the interval \([a, b)\). If \(a = b\), then the interval is empty and \([a, b) = \varnothing \in \sigma (\mathcal{E}_1)\) immediately. Now let \(x, y \in \mathbb{R}\) with \(x < y < a\). The set \((x, a)^c \cap (y, b)\) is included in \(\sigma (\mathcal{E}_1)\) as \(\sigma\)-algebras are closed under absolute complements and intersections. We also have
    \begin{align}
        (x, a)^c \cap (y, b) = \left( (-\infty, x] \cup [a, \infty] \right) \cap (y, b) = [a, b) \text{.}
    \end{align}
    Therefore, it follows that \([a, b) \in \sigma (\mathcal{E}_1) \) and hence \(\sigma (\mathcal{E}_1) \subset \sigma (\mathcal{E}_2)\).
    \item Next, we show \( \sigma (\mathcal{E}_2) \subset \sigma (\mathcal{E}_3) \). As before, fix \(a, b \in \mathbb{R}\) with \(a \leq b\) and consider the interval \([a, b)\). If this interval is empty, it is included in \(\sigma (\mathcal{E}_2)\), so assume \(a < b\). Let \((a_k)_{k \in \mathbb{N}}\) and \((b_k)_{k \in \mathbb{N}}\) sequences in \(\mathbb{Q}\) with \(a < a_k\) and \(b_k < b\) and with \(a\) and \(b\) as their limits respectively. Since a \(\sigma\)-algebra is closed under countable unions, \(\bigcup_{k=1}^\infty [a_k, b_k)\) is included in \(\sigma (\mathcal{E}_2)\), but we also have
    \begin{align}
        \bigcup_{k=1}^\infty [a_k, b_k) = \lim_{k \rightarrow \infty} [a_k, b_k) = [a, b)
    \end{align}
    We conclude that \([a, b) \in \sigma (\mathcal{E}_2)\) and therefore, \(\sigma (\mathcal{E}_2) \subset \sigma (\mathcal{E}_3)\).
    \item Now we will show \( \sigma (\mathcal{E}_3) \subset \sigma (\mathcal{E}_4)\). Again, fix \(b \in \mathbb{Q}\) and consider \((-\infty, b)\). Let \((x_k)_{k \in \mathbb{N}}\) be a sequence in \(\mathbb{Q}\) with \(x_k < b\) for each \(k \in \mathbb{N}\) and diverging to negative infinity. As \(\sigma\)-algebras are closed under countable unions, we have \(\bigcup_{k=1}^\infty (x_k, b) \in \sigma( \mathcal{E}_3 )\). On the other hand, it is
    \begin{align}
        \bigcup_{k=1}^\infty (x_k, b) = \lim_{k \rightarrow \infty} (x_k, b) = (-\infty, b) \text{.}
    \end{align}
    This means that \((-\infty, b) \in \sigma (\mathcal{E}_3)\) and from this we have \(\sigma (\mathcal{E}_3) \subset \sigma (\mathcal{E}_4)\).
    \item Lastly, we want to show \(\mathcal{E}_4 \subset \mathcal{E}_1\). Fix \(a, b \in \mathbb{R}\) with \(a \leq b\) and consider \((a, b)\). Again, if \(a = b\), then the interval is empty and included in \(\sigma (\mathcal{E}_4) \). So assume \(a < b\). \\
    Let \( [x, y)\) with \(x, y \in \mathbb{Q}\) and \(x < y\) be a half left open interval. This is included in \(\sigma( \mathcal{E}_1)\) because \((-\infty, x)^c \cap (-\infty, y) = [x, y)\) and \(\sigma\)-algebras are closed under absolute complements and intersections. \\
    Now let \((x_k)_{k \in \mathbb{N}}\) be a sequence in \(\mathbb{Q}\) with \(x_k < y\) for all \(k \in \mathbb{N}\) and converging to \(x\), then \(\bigcap_{k = 1}^\infty [x, x_k) = \lim_{k \rightarrow \infty} [x, x_k]= \{x\}\) is also in \(\sigma(\mathcal{E}_1)\).
    \end{enumerate}
\end{solution}
%
\chapter{Measure}
\begin{definition}
    Let \(\mathcal{R} \subset \mathcal{P}(\Omega)\) a ring of sets, and let \(\mu: \mathcal{R} \rightarrow [0, \infty]\) be an application. \(\mu\) is called a content, if
    \begin{enumerate}
        \item \(\mu(\emptyset) = 0\).
        \item \(\mu(A \dot\cup B) = \mu(A) + \mu(B)\)
    \end{enumerate}
    An \(\sigma\)-additive content is called premeasure.\\
    A premeasure \(\mu: \mathcal{A} \rightarrow [0, \infty]\) on \(\sigma\)-algebra \(\mathcal{A}\) is called a measure.\\
    \(\mu\) is finite if for all \(A \in \mathcal{R}: \mu(A) < \infty\).\\
    \(\mu\) is \(\sigma\)-finite if there exists are sequence \((A_m)_{m \in \mathbb{N}^*}\) in \(\mathcal{R}\) with \(\mu(A_m) < \infty\) and \(\bigcup_{m \in \mathbb{N}^*}A_m = \Omega\).
\end{definition}
\begin{lemma}
    If \(\mu(A \cap B) < \infty\), then
    \begin{align}
        \mu(A \cap B) = \mu(A) + \mu(B) - \mu(A \cup B)
    \end{align}
\end{lemma}
\begin{theorem}[Properties of premeasure]
\end{theorem}
\begin{example}[Dirac-measure]
    Let \(\Omega \neq \emptyset\). Let \(\mathcal{A} \subset \mathcal{P}(\Omega)\) a \(\sigma\)-algebra. Define for all \(x \in \Omega\) a \(\delta_x: \mathcal{A} \rightarrow \mathbb{R}_0^+\) with
    \begin{align}
        \delta_x (A) := 
        \begin{cases}
            1 \text{, if \(x \in A\)} \\
            0 \text{, else.}
        \end{cases}
    \end{align}
    \(\delta_x\) is a finite measure, called the Dirac-measure.
\end{example}
\begin{definition}
    Let
    \begin{align}
    \mathfrak{Q}(\mathbb{R}^n) := \left\{ \dot\bigcup_{i=1}^m [a_{1, i}, b_{1, i}) \times \dots [a_{n, i} \times b_{n, i}) \middle| m \in \mathbb{N}; a_{\nu, i}, b_{\nu, i} \in \mathbb{R}; \nu = 1, \dots, n \right\}
    \end{align}
    define
    \begin{align}
        \lambda^n : \mathfrak{Q}(\mathbb{R}^n) \rightarrow \mathbb{R}_0^+, A \mapsto \lambda^n (A) := \sum_{i=1}^m \prod_{\nu = 1}^n (b_{\nu, i} - a_{\nu, i})
    \end{align}
    is a premeasure.
\end{definition}
\begin{definition}
    \begin{align}
        \mathcal{R}^{\uparrow} := \left\{A \in \mathcal{P}(\Omega) \middle| \exists (A_k)_{k \in \mathbb{N}^*} \subset \mathcal{R} \text{ with } A_k \uparrow A  \right\}
    \end{align}
    \(\mathcal{R}^{\uparrow}\) is the set of all \(A \in \mathcal{P}(\Omega)\) that can be expressed as countably many sets from \(\mathcal{R}\). \(\mathcal{R}^{\uparrow}\) is not a ring of sets.
\end{definition}
\begin{definition}
    Let \(\mu : \mathcal{R} \rightarrow [0, \infty]\) be a premeasure on \(\mathcal{R}\), and \(A_k \uparrow A\). Then,
    \begin{align}
        \tilde\mu : \mathcal{R}^\uparrow \rightarrow [0, \infty], A \mapsto := \tilde\mu(A) = \lim_{k \rightarrow \infty} \mu (A_k)
    \end{align}
    is an extension of \(\mu\) on \(\mathcal{R}^\uparrow\). This is not in general a premeasure.
\end{definition}
\begin{theorem}[Properties of the first extension]
    
\end{theorem}
\begin{definition}
    Let \(\mathcal{R} \subset \mathcal{P}(\Omega)\) a set of rings, \(\mu: \mathcal{R} \rightarrow [0, \infty]\) a \(\sigma\)-finite premeasure on \(\mathcal{R}\), and \(\tilde\mu: \mathcal{R}^\uparrow \rightarrow [0, \infty]\) the first extension on \(\mathcal{R}^\uparrow\). Moreover, let \(X \subset \Omega\) a subset of \(\Omega\). Then,
    \begin{align}
        \mu^* : \mathcal{P}(\Omega) \rightarrow [0, \infty], X \mapsto \mu^* (X) := \inf \left\{ \tilde\mu(A) \middle| A \in \mathcal{R}^\uparrow, X \subset A \right\}
    \end{align}
    is the outer measure.
\end{definition}
\begin{theorem}[Properties of the second extension]
    
\end{theorem}
Bla Bla bla
\begin{definition}[Lebesgue measure]
    
\end{definition}
%
%
%
%
%
\part{Lebesgue Integral}
\chapter{Measurable Functions}
\begin{definition}[Measurable Function]
    Let \((X, \mathcal{A}_X)\) and \((Y, \mathcal{A}_Y)\) be measurable spaces. A map \(f: X \rightarrow Y\) is called measurable if the pre-image of every measurable subset of \(Y\) under \(f\) is measurable subset of \(X\), i.e.
        \begin{align}
            B \in \mathcal{A}_Y \Rightarrow f^{-1}(B) \in \mathcal{A}_X \text{.}
        \end{align}
\end{definition}
%
%
%
%
%
\begin{definition}
    Let \((X \mathcal{A}_X)\) be a measurable space. A function \(f: X \rightarrow \overline{\mathbb{R}}\) is called measurable if it is measurable with respect to the Borel \(\sigma\)-algebra on \(\overline{\mathbb{R}}\)
\end{definition}
%
%
%
%
%
%
\begin{definition}[Borel Measurable Maps]
    
\end{definition}
%
%
%
%
%
\begin{theorem}
    Let \((\Omega, \mathcal{A})\) be a measurable space, and \(\mathcal{B} = \sigma(\mathcal{E})\) for a generator \(\mathcal{E} \subset \mathcal{P}(\Omega)\). If for all \(E \in \mathcal{E}\) it is \(f^{-1}(E) \in \mathcal{A}\), then \(f\) is measurable.
\end{theorem}
\begin{example}
    Let \(f:(\mathbb{R}, \mathcal{B}) \rightarrow (\mathbb{R}, \mathcal{B})\) defined as
    \begin{align}
        f(x) := \begin{cases}
            1 x \in Q \\
            -1 x \notin Q
        \end{cases}
    \end{align}
    for a \(Q \notin \mathcal{B}(\mathbb{R})\). Then, \(f^{-1}({1})=Q \notin \mathcal{B}\) and therefore, \(f\) is not measurable even though \(|f| = 1\) is measurable.
\end{example}
\chapter{Convergence Theorems}
\begin{theorem}[Beppo Levi]
    Let \((\Omega, \mathcal{A}, \mu)\) a measure space, and for \(k \in \mathbb{N}^*\), let \(f_k: \Omega \rightarrow \mathbb{R}\) be a sequence of integratable functions such that
    \begin{align}
        \forall x \in \Omega, \forall n \in \mathbb{N}: f_n(x) \leq f_{n+1}(x) \text{.}
    \end{align}
    Moreover, if there exists \(M \in \mathbb{R}\) with \(\forall k: \int f_k d \mu \leq M\), then
    \begin{align}
        f := \lim_{k \rightarrow \infty} f_k: \Omega \rightarrow \overline{\mathbb{R}}
    \end{align}
    integratable with
    \begin{align}
        \int f d\mu = \lim_{k \rightarrow \infty} \int f_k d\mu
    \end{align}
\end{theorem}
\begin{theorem}
    If the Riemann integral exists, it matches the Lebesgue integral.
\end{theorem}
\begin{theorem}
    Let \((\Omega, \mathcal{A}, \mu)\) a measure space, let \(g: X \rightarrow [0, \infty)\) be an integrable function, and let \(f_n: X \rightarrow \mathbb{R}\) be a sequence of integrable functions satisying
    \begin{align}
        |f_n(x)| \leq g(x) \text{ for all } x \in X \text{ and } n \in \mathbb{N}
    \end{align}
    and converging pointwise to \(f: X \rightarrow \mathbb{R}\). Then \(f\) is integrable and, for every \(E \in \mathcal{A}\)
    \begin{align}
        \int_E f d\mu = \lim_{n \rightarrow \infty} \int_E f_n d\mu
    \end{align}
\end{theorem}
%
%
%
%
%
\part{Applications}
\chapter{Cavalieri's Principle}
\begin{definition}[Cross-section]
    Let \(\mathbb{R}^n = \mathbb{R}^k \times \mathbb{R}^l\) with \(n, k, l \in \mathbb{N}^*\), and \(A \subset \mathbb{R}^n\). Then for a \(y \in \mathbb{R}^l\)
    \begin{align}
        A_y := \left\{ x \in \mathbb{R}^k \middle| (x, y) \in A \right\}
    \end{align}
    is the \(l\)-dimensional cross-sections of \(A\).
\end{definition}
\begin{remark}
    Immediately from the definition above, we have
    \begin{align}
        A = \dot\bigcup_{y \in \mathbb{R}^l}(A_y, y) \text{.}
    \end{align}
    In other words, \( \left\{(A_y, y)\right\}_{y \in \mathbb{R}^l}\) is a patition of \(A\).
\end{remark}
\begin{theorem}[Cavalieri's principle]
    Let \(\mathbb{R}^n = \mathbb{R}^k \times \mathbb{R}^l\) with \(n, k, l \in \mathbb{N}^*\), let \(A \subset \mathbb{R}^k \times \mathbb{R}^l\) a Borel subset of \(\mathbb{R}^n\), and let \( \left\{(A_y, y)\right\}_{y \in \mathbb{R}^l}\) be a patition of \(A\) via cross-sections. Then we have the following
    \begin{enumerate}
        \item For all \(y \in \mathbb{R}^l\), \(A_y\) is Borel subset of \(\mathbb{R}^k\).
        \item Let \(F_A: \mathbb{R}^l \rightarrow [0, \infty], y \mapsto F_A(y) := Vol_K(A_y) = \lambda^k (A_y)\) be the \(k\)-dimensional volume of \(A_y\). Then \(F_A\) is Borel measurable on \(\mathbb{R}^l\).
        \item \(\text{Vol}_n(A) := \int_{\mathbb{R}^l} \text{Vol}_k (A_y)\)
    \end{enumerate}
\end{theorem}
\proof
\begin{enumerate}
    \item Fix \(y \in \mathbb{R}^l\)
\end{enumerate}
\begin{theorem}
    For \(K \subset \mathbb{R^n}\) compact, we have
    \begin{align}
        \text{Vol}_n (K) = \int_{\mathbb{R}} \text{Vol}_{n-1}(K_t)
    \end{align}
\end{theorem}
%
\begin{example}[Volume of a Cylinder]
    Let \(B \in \mathbb{R}^{n-1}\) be compact and \(Z := B \times [0, h] \subset \mathbb{R}^n\). \(Z\) is called the generalized \(n\)-dimensional cylinder with a basis \(B\) and height \(h\) where \(h \geq 0\). \\
    For the cross-sections we have
    \begin{align}
        Z_t = \begin{cases}
            B \quad \text{ if } t \in [0, h] \\
            \varnothing \quad\text{ else.}
        \end{cases}
    \end{align}
    and for the Volume of the Cylinder, we have
    \begin{align}
        \Vol_n(Z) &= \int_0^h \Vol_{n-1}(B) \diff t \\
        &= \Vol_{n-1}(B) \int_0^h \diff t \\
        &= \Vol_{n-1}(B) \cdot h
    \end{align}
    as \(\Vol_{n-1} (\varnothing) = 0\).
\end{example}
%
\chapter{Finding Volume by Rotation}
\begin{definition}
    \(F: \mathbb{R}^n \rightarrow \overline{\mathbb{R}}\) is rotationally symmetric in \(\mathbb{R}^n\) if there exists a \(f: [0, \infty) \rightarrow \overline{\mathbb{R}}\) such that for all \(x \in \mathbb{R}^n\) it is \(F(x) = f(\norm{x})\).
\end{definition}
\begin{theorem}
    The volume of the unit sphare is
    \begin{align}
        \tau_n = \frac{\pi^{\frac{n}{2}}}{\Gamma(\frac{n}{2} + 1)}
    \end{align}
\end{theorem}
%
\begin{theorem}
    Let \(B \subset [0, \infty)\) a Borel subset and \(A := \left\{ x \in \mathbb{R}^n \middle| \norm{x} \in B \right\}\). Then the Lebesgue measure of \(A\) is
    \begin{align}
        \lambda^n (A) = n \tau_n \int_B r^{n-1} dr
    \end{align}
    where \(\tau_n\) is the volume of the unit sphere.
\end{theorem}
%
\begin{theorem}
    Let \(f:[0, \infty) \rightarrow \overline{\mathbb{R}}\) is Borel measurable. Then the following are equivalent.
    \begin{enumerate}
        \item \(F: \mathbb{R}^n \rightarrow \overline{\mathbb{R}}, x \mapsto F(x) := f(\norm{x})\) is Lebesgue integrable over \(\mathbb{R}^n\).
        \item \(r^{n-1} f: [0, \infty) \rightarrow \overline{\mathbb{R}}, r \mapsto r^{n-1}f(r)\) is Lebesgue integrable over \([0, \infty)\).
    \end{enumerate}
    Moreover, if one of the above is true, then we have the formula
    \begin{align}
        \int_{\mathbb{R}^n} f(\norm{x}) d^n x = n \tau_n \int_{[0, \infty)} r^{n-1}f(r)dr
    \end{align}
\end{theorem}
%
%
%
%
%
\begin{example}
    For a \(R \in \mathbb{R}^+\) and \(1 \leq i \leq n\) let 
    \begin{align}
        I_i := \int_{\norm{x} \leq R} x_i^2 d^nx \text{.}
    \end{align}
    We Immediately have \(I_i = I_j =: I\) for all \(i, j\).
    \begin{align}
        I &= \frac{1}{n} \sum_{i = 1}^n I_i\\
        &= \frac{1}{n} \sum_{i = 1}^n \int_{\norm{x} \leq R} x_i^2 d^n x\\
        &= \frac{1}{n} \int_{\norm{x} \leq R} \sum_{i = 1}^n x_i^2 d^n x\\
        &= \frac{1}{n} \int_{\norm{x} \leq R} \norm{x}^2 d^n x \\
    \end{align}
    Now with the formula above, we have
    \begin{align}
        I &= \frac{1}{n} \cdot n \cdot \tau_n \int_0^R r^{n-1} r^2 dr\\
        &= \tau_n \int_0^R r^{n+1} dr \\
        &=\tau_n \frac{R^{n+2}}{n+2}
    \end{align}
\end{example}
\begin{example}
    \begin{align}
        \int_0^\infty \exp(-x^2) = \frac{\sqrt{\pi}}{2}
    \end{align}
\end{example}
\proof
Define
\begin{align}
    I = \int_{-\infty}^\infty \exp(-x^2) dx
\end{align}
Consider
\begin{align}
    I^2 &= \left( \int_{-\infty}^\infty \exp(-x^2) dx \right) \left( \int_{-\infty}^\infty \exp(-y^2) dy \right) \\
    &= \int_{-\infty}^{\infty} \int_{-\infty}^{\infty} \exp(-x^2) \exp(-y^2) dx dy \\
    &= \int_{-\infty}^\infty \int_{-\infty}^\infty \exp(-(x^2 + y^2)) dx dy \\
    &= \int_{\mathbb{R}^2} e^{-\norm{x}^2} d^2\lambda \\
    &= \int_0^\infty r e^{-r^2} dr
\end{align}
%
%
%
%
%
\begin{example}
    Let \(B_1 := \left\{ x \in \mathbb{R}^2 \middle| \norm{x} < 1 \right\}\) be the open unit disk. Find the integral
    \begin{align}
        \int_{B_1} \frac{1}{\sqrt{1 - \norm{x}^2}} d\lambda^2(x)
    \end{align}
\end{example}
\proof Define \(f: [0, \infty) \rightarrow \overline{\mathbb{R}}\) as
\begin{align}
    f(x) = \frac{1}{\sqrt{1 - x^2}} \chi_{[0, 1)}(x) \text{.}
\end{align}
As \([0, 1)\) is a Borel set of \(\mathbb{R}\), \(\chi_{[0, 1)}\) is Borel measurable. On the other hand, \(\frac{1}{\sqrt{1 - x^2}}\) is continuous for all \(x \in [0, 1)\), so the composition of these two functions \(f\) is again Borel measurable.\\
Now consider, \(r f(r)\). We have
\begin{align}
    \int |r f(r)| dr &= \int_0^1 \frac{r}{\sqrt{1 - r^2}} dr \\
    &= -\sqrt{1 - r^2} \\
    &= 0 + 1 \\
    &= 1
\end{align}
\begin{example}
    Compute the following integral
    \begin{align}
        f(\xi, \eta) := \int_{B_1} \frac{\exp(i (x\xi + y\eta))}{\sqrt{1 - x^2 -y^2}} dx dy
    \end{align}
\end{example}
%
\chapter{Transformation Formula}
\begin{theorem}
    Suppose \(\phi: U \rightarrow V\) is a \(C^1\)-diffeomorphism between open subsets of \(\mathbb{R}^n\). If \(f: V \rightarrow \mathbb{R}\) is Lebesgue integrable OR continuous with a compact support, then
    \begin{align}
        \int_U (f \circ \phi) |\det(d\phi)| dm = \int_V f dm \text{.}
    \end{align}
    \end{theorem}
    \begin{example}
        (2D) From polar coordinates to cartesian coordinates.
        \begin{gather}
            \phi: \mathbb{R}_0^+ \times [0, 2\pi) \rightarrow \mathbb{R}^2, (r, \varphi) \mapsto \phi(r, \varphi) := (r \cos \varphi, r \sin \varphi) \\
            D \phi (r, \varphi) =         \begin{pmatrix}
                \cos \varphi & -r \sin \varphi \\
                \sin \varphi & r \cos \varphi
            \end{pmatrix}\\
            \det D\phi(r, \varphi) = r
        \end{gather}
        (3D) From spherical coordinates to cartesian coordinates.
        \begin{gather}
            \phi: \mathbb{R}_0^+ \times [0, \pi] \times [0, \pi) \rightarrow \mathbb{R}^3\\
            (r, \theta, \varphi) \mapsto \phi(r, \theta, \varphi) := (r \sin \theta \cos \varphi, r \sin \theta \sin \varphi, r \cos \theta)\\
            D\phi(r, \theta, \varphi) := \begin{pmatrix}
                \sin \theta \cos \varphi & r \cos \theta \cos \varphi & -r \sin \theta \sin \varphi \\
                \sin \theta \sin \varphi & r \cos \theta \sin \varphi & r \sin \theta \cos \varphi \\
                \cos \theta & -r \sin \theta & 0
            \end{pmatrix}\\
            \det D\phi(r, \theta, \varphi) = r^2 \sin \theta
        \end{gather}
        (3D) From cylindrical coordinates to cartesian coordinates.
        \begin{gather}
            \phi: \mathbb{R} \times \mathbb{R} \times [0, 2\pi) \rightarrow \mathbb{R}^3\\
            x = r \cos \theta \\
            y = r \sin \theta \\
            z = z \\
            D \phi (r, \theta, z) = \begin{pmatrix}
                \cos \theta & -r \sin \theta & 0 \\
                \sin \theta & r \cos \theta & 0 \\
                0 & 0 & 1
            \end{pmatrix}\\
            \det D \phi (r, \theta, z) = r
        \end{gather}
    \end{example}
%
%
%
%
%
\part{More Theory}
\chapter{Lebesgue Space}
\begin{definition}[\(L^p\)-Norm]
    Let \(X, \mathcal{A}, \mu\) a measure space, and \(f: X \rightarrow \overline{\mathbb{R}}\) measurable. Then for \(p \in [1, \infty)\) the \(L^p\)-norm is defined as
    \begin{align}
        \norm{f}_p := \left( \int_X |f|^p d\mu \right)^{\frac{1}{p}} \text{.}
    \end{align}
\end{definition}
%
\begin{theorem}[Holder Inequality]
    Let \(p, q \in (1, \infty)\) such that \(p^{-1} + q^{-1} =1\). Let \(f, g: X \rightarrow \overline{\mathbb{R}}\) measurable. Then we have
    \begin{align}
        \norm{fg}_1 \leq \norm{f}_p \cdot \norm{g}_q
    \end{align}
\end{theorem}
%
\begin{theorem}[Minkowski Inequality]
    Let \(f, g: X \rightarrow \overline{\mathbb{R}}\) measurable and \(f + g\) well defined on \(X\). Then
    \begin{align}
        \forall p \in [1, \infty) : \norm{f + g}_p \leq \norm{f}_p + \norm{g}_p
    \end{align}
\end{theorem}
\begin{definition}
    Let \(X, \mathcal{A}, \mu\) be a measure space and \(p \in [1, \infty)\). Define
    \begin{align}
        \mathcal{L}^p(X, \mathcal{A}, \mu) = \left\{f: X \rightarrow \mathbb{R} \middle| f \text{ is \(\mathcal{A}\)-measurable and } \norm{f}_p < \infty \right\}
    \end{align}
    aaa

\end{definition}
%
%
%
%
%
\part{Manifolds}
%
\begin{theorem}
    If \(M \subset \mathbb{R}^n\) is a \(k\)-dimensional submanifold then the following are equivalent.
    \begin{enumerate}
        \item For all points \(a \in M\) there exists a open neighbourhood \(U \in \mathcal{U}_a(\mathbb{R})\), and there exists a function \(f_i: U \rightarrow \mathbb{R}\) with \(1 \leq i \leq n-k\)that is \(n-k\) continuously (partially) differentiable such that
        \begin{align}
            M \cap U = \left\{ x \in U \middle| f_1(x) = \dots = f_{n-k}(x) = 0 \right\}
        \end{align}
        and for all \(x \in U\) \(Df (x) = n-k\).
    \end{enumerate}
\end{theorem}
%
\begin{example}
    The figure eight is described by \(f: \mathbb{R} \rightarrow \mathbb{R}^2\), \(f(t):= (\cos t, \sin 2t)\). Define
    \begin{align}
        M := \left\{ x \in \mathbb{R} \middle | \cos x = 0, \sin 2x = 0 \right\}
    \end{align}
    then
    \begin{align}
        D\phi(x) = \begin{pmatrix}
            -\sin t \\
            2\cos 2t
        \end{pmatrix}
    \end{align}
\end{example}
%
\begin{definition}
    A submanifold is \(k\)-dimensional of the class \(\mathcal{C}^\alpha\) if the \(n-k\) functions that describe the submanifold is \(\alpha\) times continuously differentiable.
\end{definition}
%
\begin{theorem}
    Let \(M \subset \mathbb{R}^n\) a \(k\)-dimensional submanifold of the class \(\mathcal{C}^\alpha\). Let \(i = 1, 2\) \(T_i \subset \mathbb{R}^k\) open and \(\varphi_i: T_i \rightarrow V_i \subset M\) KARTEN, i.e. in parameter form of the class \(\mathcal{C}^\alpha\) with \(V:= V_1 \cap V_2 \neq \emptyset\).
\end{theorem}
%
\begin{question}
    Let \(f, g: \mathbb{R}^3 \rightarrow \mathbb{R}\) defined as
    \begin{align}
        & f(x, y, z) := x^2 + xy - y - z & g(x, y, z) := 2x^2 + 3xy - 2y - 3z &
    \end{align}
    Show that
    \begin{align}
        C := \left\{ (x, y, z) \in \mathbb{R}^3 \middle| f(x, y, z) = g(x, y, z) = 0 \right\}
    \end{align}
    is a submanifold of \(\mathbb{R}^3\) and that
    \begin{align}
        \phi: \mathbb{R} \rightarrow \mathbb{R}^3, \phi(t) := (t, t^2, t^3)
    \end{align}
    % is this word, parametization, correct?
    is a global parametization of \(C\).
\end{question}
\begin{solution}
    Define \(F: \mathbb{R}^3 \rightarrow \mathbb{R}^2\) as \(F(x, y, z) = (f(x, y, z), g(x, y, z))\), then \(C\) can be rewritten as
    \begin{align}
        C = \left\{ (x, y, z) \in \mathbb{R}^3 \middle| F(x, y, z) = 0 \right\} \text{.}
    \end{align}
    We have
    \begin{align}
        \partial_x f(x, y, z) &= 2x + y & \partial_x g(x, y, z) &= 4x + 3y \\
        \partial_y f(x, y, z) &= x - 1 & \partial_y g(x, y, z) &= 3x - 2 \\
        \partial_z f(x, y, z) &= -1 & \partial_z g(x, y, z) &= -3
    \end{align}
    therefore
    \begin{align}
        \Diff F (x, y, z) =
        \begin{pmatrix}
            2x + y & x - 1 & - 1 \\
            4x + 3y & 3x - 2 & -3 
        \end{pmatrix}
    \end{align}
    To check if \(\Diff F\) surjective, it is enough to show that
    \begin{align}
        \begin{pmatrix}
            x - 1 \\
            3x - 2
        \end{pmatrix} &
        \begin{pmatrix}
            -1 \\
            -3
        \end{pmatrix}
    \end{align}
    are linearely independent. For that, we compute the determinant of the matrix created by the two vectors.
    \begin{align}
        \det
        \begin{pmatrix}
            x  - 1 & -1 \\
            3x - 2 & -3
        \end{pmatrix}
        = -3x + 3 + 3x - 2 = 1
    \end{align}
    So, \(\Diff F\) has a rank of \(2\), therefore surjective. With this, \(C\) is a submanifold of \(\mathbb{R}^3\).
\end{solution}
\end{document}