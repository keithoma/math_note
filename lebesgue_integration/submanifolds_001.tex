\begin{question}
    Let
    \begin{align}
        M = \{ (x, y, z) \in \mathbb{R}^3 \mid x^2 + y^2 - 2z^2 = 0, \, x + y + z - 1 = 0\} \text{.}
    \end{align}
    Show the following.
    \begin{enumerate}
        \item \(M\) is a onedimensional submanifold of \(\mathbb{R}^3\).
        \item Compute \(T_{(1, 1, -1)}M\).
    \end{enumerate}
\end{question}
\begin{solution}
    Define \(f: \mathbb{R}^3 \rightarrow \mathbb{R}^2\) as
    \begin{align}
        f(x, y, z) := (f_1(x, y, z), \, f_2(x, y, z)):= (x^2 + y^2 - 2z^2, \, x + y + z - 1) \text{.}
    \end{align}
    For the partial derivatives of \(f\), we have
    \begin{align}
        \partial_x f_1(x, y , z) = 2x && \partial_x f_2(x, y, z) = 1 \\
        \partial_y f_1(x, y, z) = 2y && \partial_y f_2(x, y, z) = 1 \\
        \partial_z f_1(x, y, z) = -4z && \partial_z f_2(x, y, z) = 1
    \end{align}
    hence we have
    \begin{align}
        \Diff f (x, y, z) = \begin{pmatrix}
            2x & 2y & -4z \\
            1 & 1 & 1
        \end{pmatrix}
    \end{align}
    This matrix has the rank \(1\) if \(x = y = -2z\). But if we look at the given equation, we have
    \begin{align}
        x + y + z - 1 &= x + x - \frac{1}{2}x - 1 \\
        &= \frac{3}{2} x - 1
    \end{align}
    As this equation should equal \(0\), we have \(x = \frac{2}{3}\). With that, we also have \(y = \frac{2}{3}\) and \(z = -\frac{1}{3}\). But if we plug it into the second equation, we get
    \begin{align}
        (\frac{2}{3})^2 + (\frac{2}{3})^2 - 2 (-\frac{1}{3})^2 &= \frac{8}{9} - \frac{2}{9} = \frac{6}{9} \neq 0
    \end{align}
    So for all \((x, y, z) \in M\), the differential \(\Diff f\) is surjective and by definition \(M\) is a onedimensional submanifold in \(\mathbb{R}^3\).
\end{solution}
%
\begin{question}
    Let \(R > 0\) and \(M\) be given by
    \begin{align}
        M = \{(\cosh(u) \cos(v), \, \cosh(u) \sin(v), \, u) \mid u \in (-R, R), \, v \in (-\pi, \pi)\} \text{.}
    \end{align}
    Show the following.
    \begin{enumerate}
        \item \(M\) is a two-dimensional submanifold of \(\mathbb{R}^2\).
        \item Compute \(\sigma_2 (M)\).
    \end{enumerate}
\end{question}
\begin{solution}
    We need to show that for all \(a \in M\) there exists an open neighbourhood \(U \in \mathcal{U}_a\) and \(T \subset \mathbb{R}^k\) open, the immersion \(\varphi: T \rightarrow \mathbb{R}^n\), such that \(\varphi: T \rightarrow \varphi(T) = M \cap U\) is homeomorphism.\\
    Define \(\varphi: (-R, R) \times (-\pi, \pi) \rightarrow \mathbb{R}^3\) as
    \begin{align}
        \varphi(u, v) = (\cosh(u) \cos(v), \, \cosh(u) \sin(v), \, u) \text{.}
    \end{align}
    We want to show that \(\varphi\) is a homeomorphism. The bijectivity is easy.
\end{solution}