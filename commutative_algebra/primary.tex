\documentclass{book}
\usepackage[utf8]{inputenc}
\usepackage[english]{babel}

% page layout
\usepackage{geometry}
    \geometry{
        a4paper,
        total={170mm,257mm},
        left=20mm,
        top=20mm,
    }

\usepackage{amsthm}

\theoremstyle{plain}
\newtheorem{thm}{Theorem}[chapter] % reset theorem numbering for each chapter
\newtheorem{lmm}{Lemma}
\newtheorem{prps}{Proposition}

\theoremstyle{definition}
\newtheorem{exmp}[thm]{Example} % same for example numbers
\newtheorem{exr}[thm]{Exercise}

\newtheoremstyle{custom_definition}% name of the style to be used
  {\topsep} % measure of space to leave above the theorem. E.g.: 3pt
  {\topsep} % measure of space to leave below the theorem. E.g.: 3pt
  {\normalfont} % name of font to use in the body of the theorem
  {} % measure of space to indent
  {\bfseries} % name of head font
  {.\newline} % punctuation between head and body
  {\topsep}% space after theorem head; " " = normal interword space
  {\thmname{#1}\thmnumber{ #2} --- \thmnote{#3}} % Manually specify head

\theoremstyle{custom_definition}
\newtheorem{defn}[thm]{Definition} 

\usepackage{amssymb}
\usepackage{amsmath}

\usepackage{tikz}
\usepackage{tikz-cd}
\usetikzlibrary{%
  matrix,%
  calc,%
  arrows%
}

%%%%%
\newcommand{\set}[1]{\left\{\, #1 \,\right\}}
\newcommand{\makeset}[2]{\left\{\, #1 \mathrel{\mid} #2 \,\right\}}


\newcommand{\Ker}[1]{ \mathrm{Ker} \left( #1 \right) }
\newcommand{\Coker}[1]{ \mathrm{Coker} \left( #1 \right) }
\newcommand{\Imag}[1]{ \mathrm{Im} \left( #1 \right)}

\newcommand{\bigslant}[2]{{\raisebox{.2em}{$#1$}\left/\raisebox{-.2em}{$#2$}\right.}}

\newcommand{\restr}[2]{{% we make the whole thing an ordinary symbol
  \left.\kern-\nulldelimiterspace % automatically resize the bar with \right
  #1 % the function
  \vphantom{\big|} % pretend it's a little taller at normal size
  \right|_{#2} % this is the delimiter
  }}

\begin{document}
\begin{defn}
    An ideal \(q\) in \(A\) is primary if \(q \neq A\) and if
    \begin{equation}
        xy \in q \Rightarrow \text{ either } x \in q \text{ or } y^n \in q \text{ for some } n > 0
    \end{equation}
\end{defn}

\begin{prps}
    \(q\) is primary \(\iff\) \(A/q \neq 0\) and every zero-divisor in \(A/q\) is nilpotent
\end{prps}

\begin{proof}
    Let \(q\) be a primary ideal.
    \begin{enumerate}
        \item Let \(x \in A/q\) be a zero-divisor, then there is a \(y \in A/q\) such that \((x + q)(y + q) = xy + q = \overline{0}\).
        \item So \(xy \in q\) and by definition, we either have \(x \in q\) or \(y^n \in q\) for some \(n > 0\).
        \item \(x \in q\) and \(y \in q\) cannot be, because we required \(x\) to be a zero-divisor in \(A/q\).
        \item The only other option is \(x^n = 0\) for some \(n > 0\).
        \item Hence, \(x\) is nilpotent in \(A/q\).
    \end{enumerate}
\end{proof}

\begin{prps}
    Every prime ideal is primary.
\end{prps}

\begin{prps}
    Contraction of primary ideals are primary.
\end{prps}

\begin{prps}
    Let \(q\) be a primary ideal in a ring \(A\). Then \(\sqrt{q}\) is the smallest prime ideal containing \(q\).
\end{prps}

\begin{proof}
    The nilradical of \(A\) is the intersection of all the prime ideals of \(A\).
\end{proof}

\begin{thm}[First Uniqueness Theorem]
    Let \(\mathfrak{a}\) be a decomposable ideal and let \(\mathfrak{a} = \cap_{i = 1}^{n} q_i\) be a minimal primary decomposition of \(\mathfrak{a}\). Let \(\mathfrak{p}_i = \sqrt{q_i}\). Then the \(\mathfrak{p}_i\) are precisely the prime ideals which occur in the set of ideals \(\sqrt{(\mathfrak{a} : x)}\) with \(x \in R\), and hence are independent of the particular decomposition of \(\mathfrak{a}\).
\end{thm}

\begin{exr}
    4.2
\end{exr}

\begin{proof}
    \begin{enumerate}
        \item If \(\mathfrak{a}\) is not decomposable, then the statement is vacuously true, so let \(\mathfrak{a}\) be decomposable, i.e.
        \begin{align}
            \mathfrak{a} = \bigcap_{i=1}^n q_i
        \end{align}
        We have
        \begin{align}
            \mathfrak{a} = \sqrt{\mathfrak{a}} = 
        \end{align}
        \item 
    \end{enumerate}
\end{proof}

\end{document}