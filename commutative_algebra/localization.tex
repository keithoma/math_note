\documentclass{book}
\usepackage[utf8]{inputenc}
\usepackage[english]{babel}

% page layout
\usepackage{geometry}
    \geometry{
        a4paper,
        total={170mm,257mm},
        left=20mm,
        top=20mm,
    }

\usepackage{amsthm}

\theoremstyle{plain}
\newtheorem{thm}{Theorem}[chapter] % reset theorem numbering for each chapter
\newtheorem{lmm}{Lemma}
\newtheorem{prps}{Proposition}

\theoremstyle{definition}
\newtheorem{exmp}[thm]{Example} % same for example numbers
\newtheorem{exr}[thm]{Exercise}

\newtheoremstyle{custom_definition}% name of the style to be used
  {\topsep} % measure of space to leave above the theorem. E.g.: 3pt
  {\topsep} % measure of space to leave below the theorem. E.g.: 3pt
  {\normalfont} % name of font to use in the body of the theorem
  {} % measure of space to indent
  {\bfseries} % name of head font
  {.\newline} % punctuation between head and body
  {\topsep}% space after theorem head; " " = normal interword space
  {\thmname{#1}\thmnumber{ #2} --- \thmnote{#3}} % Manually specify head

\theoremstyle{custom_definition}
\newtheorem{defn}[thm]{Definition} 

\usepackage{amssymb}
\usepackage{amsmath}

\usepackage{tikz}
\usepackage{tikz-cd}
\usetikzlibrary{%
  matrix,%
  calc,%
  arrows%
}

%%%%%
\newcommand{\set}[1]{\left\{\, #1 \,\right\}}
\newcommand{\makeset}[2]{\left\{\, #1 \mathrel{\mid} #2 \,\right\}}


\newcommand{\Ker}[1]{ \mathrm{Ker} \left( #1 \right) }
\newcommand{\Coker}[1]{ \mathrm{Coker} \left( #1 \right) }
\newcommand{\Imag}[1]{ \mathrm{Im} \left( #1 \right)}

\newcommand{\bigslant}[2]{{\raisebox{.2em}{$#1$}\left/\raisebox{-.2em}{$#2$}\right.}}

\newcommand{\restr}[2]{{% we make the whole thing an ordinary symbol
  \left.\kern-\nulldelimiterspace % automatically resize the bar with \right
  #1 % the function
  \vphantom{\big|} % pretend it's a little taller at normal size
  \right|_{#2} % this is the delimiter
  }}

\begin{document}
\begin{exr}
    Let \(S\) be a multiplicatively closed subset of a ring \(A\), and let \(M\) be a finitely generated \(A\)-module. Prove that \(S^{-1}M = 0\) if and only if there exists \(s \in S\) such that \(sM = 0\).
\end{exr}
\begin{proof}
    \begin{enumerate}
        \item Let \(S^{-1}M = 0\), then for all \(s \in S\) and for all \(m \in M\) we have that
        \begin{align}
            \frac{m}{s} = 0 \text{,}
        \end{align}
        or in other words, \((m, s) \equiv (0, s')\) for some \(s' \in S\).
        By definition, there exists a \(t \in S\) such that
        \begin{align}
            t(s\cdot 0 - s'm) = 0 &\iff t(0 - s'm) = 0 \\
            &\iff t s' m = 0 \text{.}
        \end{align}
        Choose \(t s'\) to be the factor and we get \(sM = 0\).
        \item If there is an \(s \in S\) such that \(sM = 0\), then we can write for all \(m \in M\) that \(s \cdot m = 0\). We have
        \begin{align}
            0 = s \cdot m = s (1 \cdot m - 1 \cdot 0)
        \end{align}
        which means again \((m, 1) \equiv (0, 1)\), hence \(S^{-1}M = 0\).
    \end{enumerate}
\end{proof}
\end{document}