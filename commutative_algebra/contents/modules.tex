\chapter{Modules}

\section{Exercise}

\begin{exr}
    Let \(R\) be a ring, \(\mathfrak{a}\) and \(\mathfrak{b}\) ideals, \(M\) and \(N\) modules. Set
    \begin{equation}
        \Lambda_{\mathfrak{a}}(M) := \makeset{m \in M}{\mathfrak{a} \subset \sqrt{\mathrm{Ann}(m)}}
    \end{equation}
    \begin{enumerate}
        \item If \(\mathfrak{a} \supset \mathfrak{b}\), then \(\Lambda_\mathfrak{a}(M) \subset \Lambda_\mathfrak{b}(M)\).
        \begin{proof}
            Clear.
        \end{proof}
        \item If \(M \subset N\), then \(\Lambda_\mathfrak{a}(M) =  \Lambda_\mathfrak{a}(N) \cap M\).
        \begin{proof}
            Let \(m \in \Lambda_\mathfrak{a}(M)\), then we have immediately \(m \in M\). \(m\) is also contained in \(\Lambda_\mathfrak{a}(N)\) because \(M \subset N\). Hence one side of the inclusion holds.

            For the other side, let \(m \in \Lambda_\mathfrak{a}(N)\) and \(m \in M\). The other inclusion follows immediately.
        \end{proof}
    \end{enumerate}
\end{exr}

\begin{exr}
    Let \(R\) be a ring, \(M\) a module, \(x \in \mathrm{Jac}(M)\), and \(m \in M\). If \((1 + x)m = 0\), then \(m = 0\).
\end{exr}