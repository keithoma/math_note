\begin{defn}[Topology and Topological Space]
    Let \(X\) be a nonempty set. A set \(\mathcal{T}\) of subsets of \(X\) is said to be a topology on \(X\) if
    \begin{enumerate}
        \item \(X\) and the empty set \(\emptyset\) belong to \(\mathcal{T}\)
        \item the union of arbitary many number of sets in \(\mathcal{T}\) belong to \(\mathcal{T}\)
        \item the intersection of any two sets in \(\mathcal{T}\) belongs to \(\mathcal{T}\)
    \end{enumerate}
    The pair \((X, \mathcal{T})\) is called a topological space.
\end{defn}

\begin{defn}[Discrete Topology]
    Let \(X\) be any nonemoty set and \(\mathcal{T}\) be the collection of all subsets of \(X\). Then \(\mathcal{T}\) is called the discrete topology on the set \(X\). The topological space \((X, \mathcal{T})\) is called a discrete space.
\end{defn}

\begin{defn}[Indiscrete Topology]
    Let \(X\) be any nonempty set and \(\mathcal{T} = \{\mathcal{T}, \emptyset\}\). Then \(\mathcal{T}\) is called the indiscrete topology and \((X, \mathcal{T})\) is said to be an indiscrete space.
\end{defn}

\begin{prps}
    If \((X, \mathcal{T})\) is a topological space such that for every \(x \in X\) the singleton set \(\{x\}\) is in \(\mathcal{T}\) then \(\mathcal{T}\) is the discrete topology.
\end{prps}

\begin{defn}
    Let \((X, \mathcal{T})\) be any topological space. Then the members of \(\mathcal{T}\) are said to be open sets.
\end{defn}

\begin{prps}
    If \((X, \mathcal{T})\) is any topological space, then
    \begin{enumerate}
        \item \(X\) and \(\emptyset\) are open sets.
        \item The union of arbitary many number of open sets is an open set.
        \item The intersection of finitely many number of open sets is an open set.
    \end{enumerate}
\end{prps}

\begin{defn}
    Let \((X, \mathcal{T})\) be a topological space. A subset \(S\) of \(X\) is said to be a closed set in \((X, \mathcal{T})\) if its complment in \(X\), namely \(X - S\) is open in \((X, \mathcal{T})\).
\end{defn}

\begin{prps}
    If \((X, \mathcal{T})\) is any topological space, then
    \begin{enumerate}
        \item \(\emptyset\) and \(X\) are closed set.
        \item The intersection of arbitary many number of closed sets is a closed set.
        \item The union of finitely many number of closed sets is a closed set.
    \end{enumerate}
\end{prps}

\begin{defn}
    A subset \(S\) of a topological space \((X, \mathcal{T})\) is said to be clopen if it is both open and closed in \((X, \mathcal{T})\).
\end{defn}

\begin{defn}
    Let \(X\) be any nonempty set. A topology \(\mathcal{T}\) on \(X\) is called the finite-closed topology or the cofinite topology if the closed subsets of \(X\) are \(X\) and all finite subsets of \(X\); that is, the open sets are \(\emptyset\) and all subsets of \(X\) which have finite complments.
\end{defn}

\begin{defn}[Euclidean Topology]
    A subset \(S\) of \(\mathbb{R}\) is said to be open in the euclidean topology on \(\mathbb{R}\) if for each \(x \in S\), there exist \(a, b \in \mathbb{R}\), with \(a < b\), such that \(x \in (a, b) \subseteq S\).
\end{defn}

\begin{prps}
    A subset \(S\) of \(\mathbb{R}\) is open if and only if it is a union of open intervals.
\end{prps}

\begin{defn}[Basis for a Topology]
    Let \((X, \mathcal{T})\) be a topological space. A collection \(\mathcal{B}\) of open subsets of \(X\) is said to be a basis for the topology \(\mathcal{T}\) if every open set is a union of members in \(\mathcal{B}\).
\end{defn}

\begin{exmp}
    Let \(\mathcal{B} = \makeset{(a, b)}{a, b \in \mathbb{R}, a < b}\). Then \(\mathcal{B}\) is a basis for the euclidean topology on \(\mathbb{R}\).
\end{exmp}

\begin{exmp}
    Let \((X, \mathcal{T})\) be a discrete space and \(\mathcal{B}\) the family of all singleton subsets of \(X\); that is, \(\mathcal{B} = \makeset{\{x\}}{x \in X}\).
\end{exmp}

\begin{exmp}
    Let \(X = \set{a, b, c, d, e, f}\) and
    \begin{equation}
        \mathcal{T}_1 = \set{X, \emptyset, \{a\}, \{c, d\}, \{a, c, d\}, \{b, c, d, e, f\}}
    \end{equation}
    Then \(\mathcal{B} = \set{{a}, {c, d}, {b, c, d, e, f}}\) is a basis for \(\mathcal{T}_1\) as \(\mathcal{B} \subseteq \mathcal{T}_1\) and every member of \(\mathcal{T}_1\) can be expressed as a union of members of \(\mathcal{B}\). Note that \(\mathcal{T}_1\) itself is also a basis for \(\mathcal{T}_1\).
\end{exmp}

\begin{prps}
    Let \(X\) be a nonempty set and let \(\mathcal{B}\) be a collection of subsets of \(X\). Then \(\mathcal{B}\) is a basis for a topology on \(X\) if and only if \(\mathcal{B}\) has the following properties:
    \begin{enumerate}
        \item \(X = \bigcup_{B \in \mathcal{B}} B\)
        \item for any \(B_1, B_2 \in \mathcal{B}\) the set \(B_1 \cap B_2\) is a union of members of \(\mathcal{B}\)
    \end{enumerate}
\end{prps}

\begin{prps}
    Let \((X, \mathcal{T})\) be a topological space. A family \(\mathcal{B}\) of open subsets of \(X\) is a basis for \(\mathcal{T}\) if and only if for any point \(x\) belonging to any open set \(U\) there is a \(B \in \mathcal{B}\) such that \(x \in B \subseteq U\)
\end{prps}

\begin{prps}
    Let \(\mathcal{B}\) be a basis for a topology \(\mathcal{T}\) on a set \(X\). Then a subset \(U\) of \(X\) is open if and only if for each \(x \in U\) there exists a \(B \in \mathcal{B}\) such that \(x \in B \subseteq U\).
\end{prps}

\begin{prps}
    Let \(\mathcal{B}_1\) and \(\mathcal{B}_2\) be bases for topologies \(\mathcal{T}_1\) and \(\mathcal{T}_2\) respectively, on a nonempty set \(X\). Then \(\mathcal{T}_1 = \mathcal{T}_2\) if and only if
    \begin{enumerate}
        \item for each \(B \in \mathcal{B}_1\) and each \(x \in \mathcal{B}\), there exists a \(B' \in \mathcal{B}_2\) such that \(x \in B' \subseteq B\)
        \item for each \(B \in \mathcal{B}_2\) and each \(x \in \mathcal{B}\), there exists a \(B' \in \mathcal{B}_1\) such that \(x \in B' \subseteq B\)
    \end{enumerate}
\end{prps}