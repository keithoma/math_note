\chapter{Spectrum}

\begin{defn}[Spectrum]
    Let \(R\) be a ring. We denote the set of all prime ideals of \(R\) by \(\mathrm{Spec}(R)\) and the set of all maximal ideals of \(R\) by \(\mathrm{Spm}(R)\).
\end{defn}

\begin{defn}[Variety]
    Let \(R\) be a ring and \(\mathfrak{a}\) an ideal in \(R\). Let \(\mathbf{V}(\mathfrak{a})\) denote the subset of \(\mathrm{Spec}(R)\) consisting of those primes that contain \(\mathfrak{a}\), i.e.
    \begin{equation}
        \mathbf{V}(\mathfrak{a}) := \makeset{\mathfrak{p} \in \mathrm{Spec}(R)}{\mathfrak{a} \subseteq \mathfrak{p}}\text{.}
    \end{equation}
    We call \(\mathbf{V}(\mathfrak{a})\) the variety of \(\mathfrak{a}\).
\end{defn}

\begin{prps}
    Let \(R\) be a ring, and \(\mathfrak{a}\) and \(\mathfrak{b}\) two ideals in \(R\).
    \begin{enumerate}
        \item If \(\mathfrak{a} \subset \mathfrak{b}\), then \(\mathbf{V}(b) \subset \mathbf{V}(a)\).
        \item If \(\mathbf{V}(b) \subset \mathbf{V}(a)\), then \(\mathfrak{a} \subset \sqrt{\mathfrak{b}}\).
        \item \(\mathbf{V}(\mathfrak{a}) = \mathbf{V}(\mathfrak{b})\) if and only if \(\sqrt{\mathfrak{a}} = \sqrt{\mathfrak{b}}\).
        \item \(\mathbf{V}(\mathfrak{a}) \cup \mathbf{V}(\mathfrak{b}) = \mathbf{V}(\mathfrak{a} \cap \mathfrak{b}) = \mathbf{V}(\mathfrak{a}\mathfrak{b})\).
        \item For any index set \(I\), it is \(\bigcap_{i \in I}\mathbf{V}(\mathfrak{a}_i) = \mathbf{V}(\sum_{i \in I}\mathfrak{a}_i)\).
        \item \(\mathbf{V}(R) = \emptyset\).
        \item \(\mathbf{V}(\langle 0 \rangle) = \mathrm{Spec}(R)\).
    \end{enumerate}
\end{prps}

\begin{defn}[Zariski Topology]
    Let \(\mathfrak{a} \subseteq A\) be an ideal. Declaring the sets
    \begin{equation}
        Z(\mathfrak{a}) := \makeset{\mathfrak{p} \in \mathrm{Spec}(A)}{\mathfrak{a} \subseteq \mathfrak{p}}
    \end{equation}
    to be closed induces a topology on \(\mathrm{Spec}(A)\), the Zariski Topology.

    Given an element \(f \in R\), we call the open set
    \begin{equation}
        D(f) := \mathrm{Spec}(R) - \mathbf{V}(\langle f \rangle)
    \end{equation}
    a principal open set. These sets form a basis for the topology of \(\mathrm{Spec}(R)\); indeed, given any prime \(\mathfrak{a} \not\subset \mathfrak{p}\)
\end{defn}

\section{Proofs}

\begin{prps}
    The Zariski topology is indeed a topology.
\end{prps}
\begin{proof}
    
\end{proof}

\section{Exercises}
\begin{exr}
    Let \(R\) be a ring and \(\mathfrak{p}, \mathfrak{q} \in \mathrm{Spec}(R)\). Show:
    \begin{enumerate}
        \item The closure \(\overline{\{\mathfrak{p}\}}\) of \(\mathfrak{p}\) is equal to \(\mathbf{V}(\mathfrak{p})\); that is, \(\mathfrak{q} \in \overline{\{\mathfrak{p}\}}\) if and only if \(\mathfrak{p} \subseteq \mathfrak{q}\).
    \end{enumerate}
\end{exr}