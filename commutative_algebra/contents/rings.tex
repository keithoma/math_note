\chapter{Rings and Ideals}

\section{Cheat Sheet}

\begin{defn}[Ring]
    A ring is a set \(R\) equipped with two binary operations \(+\) (addition) and \(\cdot\) (multiplication) satisfying the following three sets of axioms, called the ring axioms.
    \begin{enumerate}
      \item \((R, +)\) is an abelian group.
      \item \((R, \cdot)\) is a semigroup.
      \item Multiplication is distributive with respect to addition, meaning that
      \begin{itemize}
        \item \(a \cdot (b + c) = (a \cdot b) + (a \cdot c)\) for all \(a, b, c \in R\) (left distributivity).
        \item \((b + c) \cdot a = (b \cdot a) + (c \cdot a)\) for all \(a, b, c \in R\) (right distributivity).
      \end{itemize}
    \end{enumerate}
    A ring is called unitary if it contains the multiplicative identity and commutative if multiplication is commutative.
\end{defn}

\begin{defn}[Unit]
    
\end{defn}

\begin{defn}[Zerodivisors]
    
\end{defn}

\begin{defn}[Nilpotent]
    
\end{defn}

\begin{defn}[Idempotent]
    
\end{defn}

\begin{defn}[Ideal]
    
\end{defn}

\begin{defn}[Operations on Ideals]
    Let \(R\) be a ring \(\{\mathfrak{a}_i\}_{i \in I}\) be a collection of ideals in \(R\).
    \begin{enumerate}
        \item
        \begin{equation}
            \sum_{i \in I} \mathfrak{a}_i = \makeset{\sum_{i \in I}a_i}{a_i \in \mathfrak{a}_i \text{ for all \(i \in I\), and \(a_i = 0\) for almost all i}}
        \end{equation}
        \item The transporter of two ideals is defined By
        \begin{equation}
            (\mathfrak{a} : \mathfrak{b}) := \makeset{x \in R}{x \mathfrak{b} \subset \mathfrak{a}}
        \end{equation}
    \end{enumerate}
\end{defn}

\begin{defn}[Prime Ideal]
    
\end{defn}

\begin{defn}[Maximal Ideal]
    
\end{defn}

\begin{defn}[Quotient Ring]
    Given a ring \(A\) and two-sided ideal \(\mathfrak{a}\) in \(A\), we may define an congruence relation \(\sim\) on \(A\) as follows:
    \begin{equation}
        x \sim y :\Longleftrightarrow x - y \in \mathfrak{a} \text{.}
    \end{equation}
    The equivalence class of the element \(x\) in \(A\) is given by
    \begin{equation}
        [x] = x + \mathfrak{a} := \makeset{x + a}{a \in \mathfrak{a}}
    \end{equation}
    and the set of all such equivalence classes is denoted by \(A / \mathfrak{a}\); it becomes a ring, the factor ring or the quotient ring of \(A\) modulo \(\mathfrak{a}\), if one defines
    \begin{enumerate}
        \item \((a + \mathfrak{a}) + (b + \mathfrak{a}) = (a + b) + \mathfrak{a}\)
        \item \((a + \mathfrak{a}) (b + \mathfrak{a}) = (ab) + \mathfrak{a}\)
    \end{enumerate}
    The map \(\pi: R \longrightarrow A / \mathfrak{a}, \, x \mapsto \pi(x) := x + \mathfrak{a}\) is a surjective ring homomorphism and is sometimes called the natural quotient map or the canonical homomorphism.
\end{defn}

\begin{prps}[Universal Property]
    Let \(A\) and \(B\) be rings, \(\mathfrak{a}\) an ideal, and \(f: A \longrightarrow B\) a ring homomorphism with \(\mathfrak{a} \subseteq \Ker{f}\). Then there exists a unique ring homomorphism \(\tilde{f}: A / \mathfrak{a} \longrightarrow B\) such that \(f = \tilde{f} \circ \pi\).
\end{prps}

\begin{thm}
    There is a bijection between the sets
    \begin{equation}
        \makeset{\mathfrak{}}{}
    \end{equation}
\end{thm}

\begin{defn}[Integral Domain]
    
\end{defn}

\begin{thm}
    \begin{itemize}
        \item prime ideal, quotient is integral domain
        \item same as above, but if prime maximal, then quotient is a fields
        \item Maximal ideals are prime ideals.
    \end{itemize}
\end{thm}

\begin{defn}[Unique Factorization Domain]
    
\end{defn}

\begin{defn}[Principal Ideal Domain]
    
\end{defn}

\begin{prps}
    Commutative Rings \(\supset\) Unique Factorization Domain \(\supset\) Principal Ideal Domain \(\supset\) Fields
\end{prps}

\begin{thm}
    \begin{itemize}
        \item prime ideal, quotient is integral domain
        \item same as above, but if prime maximal, then quotient is a fields
        \item Maximal ideals are prime ideals.
        \item There is a 1:1 correspondence
        \begin{equation}
            \set{\text{Ideals in \(A / \mathfrak{a}\)}} \longleftrightarrow \makeset{\mathfrak{b} / \mathfrak{a}}{\mathfrak{a} \subseteq \mathfrak{b} \subseteq A}
        \end{equation}
    \end{itemize}
\end{thm}

\begin{proof}
    For the last point, it is easier to understand with extension and contraction.
\end{proof}

\section{Examples}

\begin{exmp}
    \begin{enumerate}
        \item \(\mathbb{Z}\)
        \item All fields.
        \item Let \(S\) be any set, then \((\mathcal{P}(S), \triangle, \cap)\) is a ring.
        \item continuous \(f: I \longrightarrow \mathbb{R}\) with a real interval I forms a ring.
        \item cartesian product of rings
    \end{enumerate}
\end{exmp}

\begin{exmp}
    Let \(S\) be any set, then \((2^S, \triangle, \cap)\) is a ring.
    \begin{enumerate}
        \item \(0 = \emptyset\) and \(-A = A\)
        \item The neutral element of the multiplication is \(S\).
        \item \((2^S)^\times = \{S\}\)
        \item \(\mathrm{ZD}(2^S) = 2^S - S\) since \(A \cap A^c = \emptyset\) (also minus the empty set) (this seems to be true for all boolean rings)
        \item \(\mathrm{Nil}(2^S) = \emptyset\) (seems to be true for all boolean rings)
        \item \(\langle A \rangle = 2^A\) contains all subset of \(A\)
    \end{enumerate}
\end{exmp}

\begin{exmp}[Integral Domains]
    \begin{enumerate}
        \item \(\mathbb{Z}[i]\) the Gaussian integers, \((\mathbb{Z}[i])^\times = {\{\pm 1, \pm i\}}\)
        \item \(\mathbb{Z}/n\mathbb{Z}\), \((\mathbb{Z}/n\mathbb{Z})^\times = \makeset{\overline{a}}{\mathrm{gcd}(a, n)=1}\), \(|(\mathbb{Z}/n\mathbb{Z})^\times| = \phi(n)\) (Euler totient function)
    \end{enumerate}
\end{exmp}

\section{Proofs}

\section{Exercises}
\begin{exr}
    Let \(\varphi: R \longrightarrow R'\) be a ring homomorphism, \(\mathfrak{a}_1\), \(\mathfrak{a}_2\), \(\mathfrak{a}_3\) ideals of \(R\), and \(\mathfrak{b}_1\),\(\mathfrak{b}_2\), \(\mathfrak{b}_3\) ideals of \(R'\). Prove the following statements. Prove the following statements:
    \begin{enumerate}
        \item \((\mathfrak{a}_1 + \mathfrak{a}_2)^e = \mathfrak{a}_1^e + \mathfrak{a}_2^e\)
        \begin{proof}
            We show \((\mathfrak{a}_1 + \mathfrak{a}_2)^e \subseteq \mathfrak{a}_1^e + \mathfrak{a}_2^e\). Let \(x \in (\mathfrak{a}_1 + \mathfrak{b}_2)^e\), then we have for some index set \(I\)
            \begin{equation}
              x = \sum_{i \in I} \lambda_i x_i \text{,}
            \end{equation}
            where \(\lambda_i \in R'\) and \(x_i \in \varphi(\mathfrak{a}_1 + \mathfrak{a}_2)\) for all \(i \in I\). For each \(i \in I\) we find and \(a_{i, 1}, a_{i, 2} \in \mathfrak{a}\) such that \(x_i = \varphi(a_{i, 1} +  a_{i, 2})\), hence
            \begin{align}
              x =& \sum_{i \in I} \lambda_i (a_{i, 1} +  a_{i, 2}) &  \\
              =& \sum_{i \in I} \lambda_i \left( \varphi(a_{i, 1}) + \varphi(a_{i, 2}) \right)&\text{(by linearity)} \\
              =& \sum_{i \in I} \lambda_i \varphi( a_{i, 1}) + \lambda_i \varphi( a_{i, 2}) & \text{(by distributivity)} \\
              =& \sum_{i \in I} \lambda_i \varphi( a_{i, 1}) + \sum_{i \in I} \lambda_i  \varphi( a_{i, 2}) & \text{(reordering the sum)} \text{.} \\
            \end{align}
            The last term is exactly the elements expressed by \(\mathfrak{a}_1^e + \mathfrak{a}_2^e\), therefore, \((\mathfrak{a}_1 + \mathfrak{a}_2)^e \subset \mathfrak{a}_1^e + \mathfrak{a}_2^e\).
      
            I think the above proof should work into both directions. If not, just notice that \(\mathfrak{a}_1^e \subset (\mathfrak{a}_1 + \mathfrak{a}_2)^e\).
          \end{proof}
          \item \((\mathfrak{b}_1 + \mathfrak{b}_2)^c \supseteq \mathfrak{b}_1^c + \mathfrak{b}_2^c\)
          \begin{proof}
            We have
            \begin{equation}
              (\mathfrak{b}_1 + \mathfrak{b}_2)^c = \makeset{x \in A}{\exists \, b_1 \in \mathfrak{b}_1 \exists \, b_2 \in \mathfrak{b}_2 : \varphi(x) = b_1 + b_2} \text{.}
            \end{equation}
            Now let \(x \in \mathfrak{b}_1^c + \mathfrak{b}_2^c\), then \(x = a_1 + a_2\) where \(\varphi(a_1) \in \mathfrak{b}_1\) and \(\varphi(a_2) \in \mathfrak{b}_2\). It is
            \begin{align}
              \varphi(x) =& \varphi(a_1 + a_2) & \\
              =& \varphi(a_1) + \varphi(a_2) & \text{(by additivity)}
            \end{align}
            Since \(\varphi(a_1) \in \mathfrak{b}_1\) and \(\varphi(a_2) \in \mathfrak{b}_2\) we have that \(x \in (\mathfrak{b}_1 + \mathfrak{b}_2)^c\).
          \end{proof}
    \end{enumerate}
\end{exr}