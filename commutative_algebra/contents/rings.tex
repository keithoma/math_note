\begin{defn}[Ring]
    A ring is a set \(R\) equipped with two binary operations \(+\) (addition) and \(\cdot\) (multiplication) satisfying the following three sets of axioms, called the ring axioms.
    \begin{enumerate}
      \item \((R, +)\) is an abelian group.
      \item \((R, \cdot)\) is a semigroup.
      \item Multiplication is distributive with respect to addition, meaning that
      \begin{itemize}
        \item \(a \cdot (b + c) = (a \cdot b) + (a \cdot c)\) for all \(a, b, c \in R\) (left distributivity).
        \item \((b + c) \cdot a = (b \cdot a) + (c \cdot a)\) for all \(a, b, c \in R\) (right distributivity).
      \end{itemize}
    \end{enumerate}
    A ring is called unitary if it contains the multiplicative identity and commutative if multiplication is commutative.
\end{defn}

\begin{defn}[Ideal]
    
\end{defn}

\begin{defn}[Ideal Arithmetic]
    
\end{defn}

\begin{defn}[Prime Ideal]
    
\end{defn}

\begin{defn}[Maximal Ideal]
    
\end{defn}

\begin{defn}[Quotient Ring]
    Given a ring \(A\) and two-sided ideal \(\mathfrak{a}\) in \(A\), we may define an congruence relation \(\sim\) on \(A\) as follows:
    \begin{equation}
        x \sim y :\Longleftrightarrow x - y \in \mathfrak{a} \text{.}
    \end{equation}
    The equivalence class of the element \(x\) in \(A\) is given by
    \begin{equation}
        [x] = x + \mathfrak{a} := \makeset{x + a}{a \in \mathfrak{a}}
    \end{equation}
    and the set of all such equivalence classes is denoted by \(A / \mathfrak{a}\); it becomes a ring, the factor ring or the quotient ring of \(A\) modulo \(\mathfrak{a}\), if one defines
    \begin{enumerate}
        \item \((a + \mathfrak{a}) + (b + \mathfrak{a}) = (a + b) + \mathfrak{a}\)
        \item \((a + \mathfrak{a}) (b + \mathfrak{a}) = (ab) + \mathfrak{a}\)
    \end{enumerate}
    The map \(\pi: R \longrightarrow A / \mathfrak{a}, \, x \mapsto \pi(x) := x + \mathfrak{a}\) is a surjective ring homomorphism and is sometimes called the natural quotient map or the canonical homomorphism.
\end{defn}

\begin{prps}[Universal Property]
    Let \(A\) and \(B\) be rings, \(\mathfrak{a}\) an ideal, and \(f: A \longrightarrow B\) a ring homomorphism with \(\mathfrak{a} \subseteq \Ker{f}\). Then there exists a unique ring homomorphism \(\tilde{f}: A / \mathfrak{a} \longrightarrow B\) such that \(f = \tilde{f} \circ \pi\).
\end{prps}

\begin{defn}[Integral Domain]
    
\end{defn}

\begin{thm}
    \begin{itemize}
        \item prime ideal, quotient is integral domain
        \item same as above, but if prime maximal, then quotient is a fields
        \item Maximal ideals are prime ideals.
        \item There is a 1:1 correspondence
        \begin{equation}
            \set{\text{Ideals in \(A / \mathfrak{a}\)}} \longleftrightarrow \makeset{\mathfrak{b} / \mathfrak{a}}{\mathfrak{a} \subseteq \mathfrak{b} \subseteq A}
        \end{equation}
    \end{itemize}
\end{thm}

