\documentclass{book}
\usepackage[utf8]{inputenc}
\usepackage[english]{babel}

% page layout
\usepackage{geometry}
    \geometry{
        a4paper,
        total={170mm,257mm},
        left=20mm,
        top=20mm,
    }

\usepackage{amsthm}

\theoremstyle{plain}
\newtheorem{thm}{Theorem}[chapter] % reset theorem numbering for each chapter
\newtheorem{lmm}{Lemma}
\newtheorem{prps}{Proposition}

\theoremstyle{definition}
\newtheorem{exmp}[thm]{Example} % same for example numbers
\newtheorem{exr}[thm]{Exercise}

\newtheoremstyle{custom_definition}% name of the style to be used
  {\topsep} % measure of space to leave above the theorem. E.g.: 3pt
  {\topsep} % measure of space to leave below the theorem. E.g.: 3pt
  {\normalfont} % name of font to use in the body of the theorem
  {} % measure of space to indent
  {\bfseries} % name of head font
  {.\newline} % punctuation between head and body
  {\topsep}% space after theorem head; " " = normal interword space
  {\thmname{#1}\thmnumber{ #2} --- \thmnote{#3}} % Manually specify head

\theoremstyle{custom_definition}
\newtheorem{definition}[thm]{Definition} 

\usepackage{amssymb}
\usepackage{amsmath}

\usepackage{tikz}
\usepackage{tikz-cd}
\usetikzlibrary{%
  matrix,%
  calc,%
  arrows%
}

%%%%%
\newcommand{\set}[1]{\left\{\, #1 \,\right\}}
\newcommand{\makeset}[2]{\left\{\, #1 \mathrel{\mid} #2 \,\right\}}


\newcommand{\Ker}[1]{ \mathrm{Ker} \left( #1 \right) }
\newcommand{\Coker}[1]{ \mathrm{Coker} \left( #1 \right) }
\newcommand{\Imag}[1]{ \mathrm{Im} \left( #1 \right)}

\newcommand{\bigslant}[2]{{\raisebox{.2em}{$#1$}\left/\raisebox{-.2em}{$#2$}\right.}}

\newcommand{\restr}[2]{{% we make the whole thing an ordinary symbol
  \left.\kern-\nulldelimiterspace % automatically resize the bar with \right
  #1 % the function
  \vphantom{\big|} % pretend it's a little taller at normal size
  \right|_{#2} % this is the delimiter
  }}

\begin{document}

\chapter{Rings}

Concepts
\begin{enumerate}
    \item zero-Divisors
    \item nilpotent
    \item ideals
\end{enumerate}

Properties of Ring
\begin{enumerate}
    \item local ring
    \item noetherian ring
    \item artinian ring
    \item principal ideal domain
    \item unique factorization domain
    \item integral domain
\end{enumerate}

\part{Ideals}
\chapter{Ideal Operation}
\begin{definition}
    Let \(R\) be a ring and \(\{\mathfrak{a}_i\}_{i \in I}\) a collection of ideals for an arbitary index set \(I\).
    \begin{enumerate}
        \item The sum of ideals is the smallest ideal in \(R\) containing each \(\mathfrak{a}_i\), i.e.
        \begin{equation}
            \sum_{i \in I} \mathfrak{a}_i := \makeset{\sum_{i \in I} a_i}{a_i \in \mathfrak{a}_i \text{ for all } i \in I \text{, and } a_i = 0 \text{ for almost all } i \in I} \text{.}
        \end{equation}
        If \(\mathfrak{a}\) and \(\mathfrak{b}\) are ideals, then
        \begin{equation}
            \mathfrak{a} + \mathfrak{b} = \makeset{a + b}{a \in \mathfrak{a} \text{ and } b \in \mathfrak{b}}\text{.}
        \end{equation}
        \item The product of ideals is the smallest ideal in \(R\)
    \end{enumerate}

\end{definition}
\end{document}