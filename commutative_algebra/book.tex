\documentclass{book}
\usepackage[utf8]{inputenc}
\usepackage[english]{babel}

\usepackage{amsmath}

\usepackage{amssymb}

% page layout
\usepackage{geometry}
 \geometry{
 a4paper,
 total={170mm,257mm},
 left=20mm,
 top=20mm,
 }

% theorems
\usepackage{amsthm}

\newtheoremstyle{custom_definition}% name of the style to be used
  {\topsep} % measure of space to leave above the theorem. E.g.: 3pt
  {\topsep} % measure of space to leave below the theorem. E.g.: 3pt
  {\normalfont} % name of font to use in the body of the theorem
  {} % measure of space to indent
  {\bfseries} % name of head font
  {.\newline} % punctuation between head and body
  {\topsep}% space after theorem head; " " = normal interword space
  {\thmname{#1}\thmnumber{ #2} --- \thmnote{#3}} % Manually specify head

  \newtheoremstyle{custom_theorem}% name of the style to be used
  {\topsep} % measure of space to leave above the theorem. E.g.: 3pt
  {\topsep} % measure of space to leave below the theorem. E.g.: 3pt
  {\itshape} % name of font to use in the body of the theorem
  {} % measure of space to indent
  {\bfseries} % name of head font
  {.\newline} % punctuation between head and body
  {\topsep}% space after theorem head; " " = normal interword space
  {\thmname{#1}\thmnumber{ #2} --- \thmnote{#3}} % Manually specify head

\theoremstyle{custom_definition}
\newtheorem{definition}{Definition}

\theoremstyle{custom_theorem}
\newtheorem{lemma}{Lemma}
\newtheorem{theorem}{Theorem}

\newcommand{\bigslant}[2]{{\raisebox{.2em}{$#1$}\left/\raisebox{-.2em}{$#2$}\right.}}

\usepackage[lastexercise]{exercise}

\begin{document}
    \begin{definition}[Group of Units]
        Let \(A\) be a ring. An element \(a \in A\) is called an unit if there is an element \(b \in A\) such that \(a \cdot b = 1\).

        We denote the set of all units as following.
        \begin{align}
            A^\times := \{\, a \in A \, \mid\, \exists b \in A : a \cdot b = 1 \,\}
        \end{align}
    \end{definition}

    \(A^\times\) forms a group.

    \begin{enumerate}
        \item Let \(a, b \in A^\times\). Then, there are \(a^\prime\) and \(b^\prime\) in \(A^\times\) such that \(a \cdot a^\prime = 1\) and \(b \cdot b^\prime = 1\) respectively. We have \(a \cdot b \cdot a^\prime \cdot b^\prime = 1\) hence \(a \cdot b \in A^\times\). In other words, \(A^\times\) is closed under multiplication.
        \item Associativity is inherited from the ring \(A\).
        \item The identity element is \(1\). It is included in \(A^\times\) as \(1 \cdot 1 = 1\). And the identity property \(a \cdot 1 = a\) for all \(a \in A^\times\) is inherited from \(A\).
        \item Let \(a \in A^\times\). Then, there is a \(b \in A^\times\) such that \(a \cdot b = 1\). This \(b\) is precisely the inverse element of \(a\).
    \end{enumerate}

    If \(A\) is commutativ, then \(A^\prime\) is commutative.

    My guess is that \(A^\prime\) being a commutative group does not imply that \(A\) is commutative.

    Also, if \(A\) isn't commutative, there probably is a left unit group and a right unit group. Or are they the same?

    Examples:

    \begin{enumerate}
        \item \(\mathbb{Z}^\times = \{ -1, 1 \}\)
        \item For any field \(\mathbb{K}\), it is \(\mathbb{K}^\times = \mathbb{K}\).
        \item Let \(A = \text{Mat}_{2 \times 2}(\mathbb{R})\). Then, the group of units \(A^\times\) is the set of all invertible matricies also called the general linear group \(\text{GL}_2(\mathbb{R})\). This should be true of the general case \(A = \text{Mat}_{n \times n}{(\mathbb{K})}\).
        \item Let \(\mathbb{Q}[X]\) be a polynomial ring.
    \end{enumerate}

    \begin{definition}[Set of Zero Divisors]
        \begin{align}
            \text{ZD}(A) := \{\, a \in A \,\mid\, \exists b \in A \setminus \{0\} : a \cdot b = 0 \,\}.
        \end{align}
    \end{definition}

    Examples:
    \begin{enumerate}
        \item \(\text{ZD}(\mathbb{Z}) = \{0\}\).
        \item For any field \(\mathbb{K}\), it is \(\text{ZD}(\mathbb{K}) = \{0\}\).
        \item 
    \end{enumerate}

    Proof of above:
    Let \(\mathbb{K}\) be a field and assume there is a nonzero \(x \in \mathbb{K}\) such that \(x \cdot b = 0\) for a \(b \in \mathbb{K}\). The issue here is that \(\mathbb{K}\) contains the inverse of \(b\) and so we have \(x = 0 \cdot b^{-1} = 0\).

    \begin{definition}[Integral Domain]
        A ring \(A\) with \(\text{ZD}(A) = \{0\}\) is called an integral domain.
    \end{definition}

    \begin{definition}[Set of Nilpotent Elements]
        \begin{align}
            \text{Nil}(A) := \{\, a \in A \,\mid\, \exists n \in \mathbb{N} : a^n = 0 \,\}
        \end{align}
    \end{definition}

    \begin{definition}[Reduced Ring]
        A ring \(A\) with \(\text{Nil}(A) = \{0\}\) is called a reduced ring.
    \end{definition}


    Here some lemmas.

    \(A \setminus \text{ZD}(A)\) is a semigroup containing \(A^\times\).

    Proof:

    \begin{enumerate}
        \item Let \(x, y \in A \setminus \text{ZD}(A)\). Then \(x \cdot a \neq 0\) and \(y \cdot b \neq 0\) for all \(a, b \in A\). Assume there exists a \(c \in A\) such that \(x \cdot y \cdot c = 0\). This implies \(x \cdot c = 0\) or \(y \cdot c = 0\), but this is impossible. Conclude \(x \cdot y \in A \setminus \text{ZD}(A)\).
        \item Let \(x \in A^\times\). By definition we have for some \(a \in A\) that \(x \cdot a = 1\). Assume \(x \in \text{ZD}(A)\). Then we have \(x \cdot b = 0\) for some \(b \in A \setminus \{0\}\). With the previous equation we get
        \begin{align}
            x \cdot a = 1 & \iff x \cdot a \cdot b = 1 \cdot b \\
            & \iff x \cdot b \cdot a = b \\
            & \iff 0 = b
        \end{align}
        But this is a contradiction. Hence \(x \not\in \text{ZD}(A)\).
        \item We have to prove associativity and the identity element, but both are clear.
    \end{enumerate}

    More lemma: cancelation lemma, clear.

    Here is one interesting:

    \(\text{Nil}(A)\) is an ideal in \(A\).

    Proof. Let \(x \in \text{Nil}(A)\) and \(a \in A\). Then \(x \cdot a \in \text{Nil}(A)\) (duh, obviously).

    We have to show that \(\text{Nil}(A)\) is an addtive subgroup of \(A\).

    \begin{enumerate}
        \item Let \(x, y \in \text{Nil}(A)\). Then \(a^n = 0\) and \(b^m = 0\) for some \(n \in \mathbb{N}\). With the binominal theorem we get \((a + b)^{n + m}= 0\)
    \end{enumerate}

    I need the latex thingy for quotient ring.

    Another lemma. The set \(A_{\text{red}} := A / \text{Nil}(A)\) is a reduced ring.

    Proof. Assume there is an \(\overline{x} \in \text{Nil}(A_{\text{red}})\) but \(\overline{x} \neq 0\). So \(\overline{x}^n = 0\) for a suitable \(n \in \mathbb{N}\). We have \(0 = \overline{x}^n = (x + \text{Nil}(A))^n = \)



    \begin{definition}[Sum of Ideals]
        Let \(A\) be a ring and \(\{\mathfrak{a}_i\}_{i \in I}\) be a collection of ideals. We define the smallest ideal in \(A\) which contains each \(\mathfrak{a}_i\) by \(\sum_{i \in I} \mathfrak{a}_i\), i.e.
        \begin{align}
            \sum_{i \in I} \mathfrak{a}_i := \left\{ \sum_{i \in I} a_i \, \,\mid\, a_i \in \mathfrak{a}_i \text{ for all \(i \in I\), and \(a_i = 0\) for almost all i}\,\right\}
        \end{align}
    \end{definition}

    This makes sense to me.

    \begin{definition}[Intersection of Ideals]
        We define the largest ideal in \(A\) containing each \(\mathfrak{a}_i\) by
        \begin{align}
            \bigcap_{i \in I} \mathfrak{a}_i
        \end{align}       
    \end{definition}

    \begin{definition}[Product of Ideals]
        
    \end{definition}

    \begin{definition}[Radical of Ideals]
        The radical of an ideal \(\mathfrak{a}\) is given by
        \begin{align}
            \sqrt{\mathfrak{a}} := \left\{\, b \in A \, \mid \, \exists n \in \mathbb{N} : b^n = a \,\right\}
        \end{align}
    \end{definition}

    Again some lemmas.

    \(\sqrt{\mathfrak{a}}\) is an ideal.
    \begin{enumerate}
        \item We prove that \(\sqrt{\mathfrak{a}}\) is an additive subgroup of \(A\).
        \begin{enumerate}
            \item Let \(x, y \in \sqrt{\mathfrak{a}}\). Then for some \(n, m \in \mathbb{N}\) we have that \(x^n = y^m = a\). Consider
            \begin{align}
                (x + y)^{n + m}
            \end{align}
            This is a sum and product out of the elements in \(\sqrt{\mathfrak{a}}\).
            \item Associativity and identity is inherited.
            \item Inverse element is clear.
        \end{enumerate}
        This is also clear.
    \end{enumerate}

    Alternate way:

    If \(\mathfrak{a} = A\), then \(\sqrt{\mathfrak{a}} = A\) and this is an ideal. Consider the case \(\mathfrak{a} \neq A\). Let \(\pi: A \longrightarrow A / \mathfrak{a}\) be the natural projection. Since \(A / \mathfrak{a}\) is an ideal, we can apply the lemma above and we know that \(\text{Nil}(A / \mathfrak{a})\) is an ideal.

    The point here is that
    \begin{align}
        \pi^{-1} (\text{Nil}(A / \mathfrak{a})) = \sqrt{\mathfrak{a}}
    \end{align}


    The Chinese Remainder theorem

    Let \(A\) be a ring, \(n \geq 2\) and \(\mathfrak{a}_1, \ldots, \mathfrak{a}_n\) be ideals in \(A\).
    \begin{enumerate}
        \item If the \(\mathfrak{a}_i\) are pairwise coprime, then \(\prod_{i=1}^n \mathfrak{a}_i = \bigcap_{i=1}^n \mathfrak{a}_i\)
    \end{enumerate}
    something

    \begin{theorem}[Universal Property of the Quotient]
        For a ring homomorphism \(\varphi: R \longrightarrow S\), and an ideal \(I \subseteq \ker{\varphi}\), there exists a unique homomorphism \(\overline{\varphi}: \bigslant{R}{I} \longrightarrow S\) such that \(\overline{\varphi} \circ \pi = \varphi\), where \(\pi: R \longrightarrow \bigslant{R}{I}\) is the canonical projection.
    \end{theorem}
    \begin{proof}
        If there is such a \(\overline{\varphi}\), then for all \(x \in R\) it must satisfy
        \begin{equation}
            \varphi(x) = \overline{\varphi}(\pi(x)) = \overline{\varphi}(\overline{x}) \text{.}
        \end{equation}
        We just have to make sure that the above \(\overline{\varphi}\) is well defined. Let \(x, y \in R\) such that \(\overline{x} = \overline{y}\), then by definition of the quotient ring we have that \(x - y \in \mathfrak{a} \subseteq \ker \varphi\). So \(\varphi(x - y) = 0\) and since \(\varphi\) is a ring homomorphism we have \(\varphi(x) = \varphi(y)\).
    \end{proof}
    \begin{lemma}[Prime Ideals and Integral Domains]
        \(\mathfrak{p}\) is a prime ideal if and only if \(\bigslant{R}{\mathfrak{p}}\) is an integral domain.
    \end{lemma}
    \begin{proof}
    Apparantly this is a tautology? See Bosch 41.
    \end{proof}
    \begin{lemma}
        \(\mathfrak{m}\) is a maximal ideal if and only if \(\bigslant{R}{m}\) is a field.
    \end{lemma}
    \begin{lemma}
        All maximal ideals are prime ideals.
    \end{lemma}
    \begin{lemma}
        Let \(R\) be a ring, \(\mathfrak{a}\) an ideal in \(R\). Moreover, define the sets
        \begin{align}
            X &:= \{\, \text{Ideals in \(\bigslant{R}{\mathfrak{a}}\)} \,\} \\
            Y &:= \{\, \bigslant{\mathfrak{b}}{\mathfrak{a}} \,\mid\, \mathfrak{b} \subset R, \, \mathfrak{b} \supseteq \mathfrak{a} \,\}
        \end{align}
        https://math.stackexchange.com/questions/69578/bijection-between-ideals-of-r-i-and-ideals-containing-i?rq=1
    \end{lemma}
\end{document}


