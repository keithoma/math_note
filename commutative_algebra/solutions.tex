\documentclass{book}
\usepackage[utf8]{inputenc}
\usepackage[english]{babel}

% page layout
\usepackage{geometry}
    \geometry{
        a4paper,
        total={170mm,257mm},
        left=20mm,
        top=20mm,
    }

\usepackage{amsthm}

\theoremstyle{plain}
\newtheorem{theorem}{Theorem}[chapter] % reset theorem numbering for each chapter
\newtheorem{lemma}{Lemma}
\newtheorem{proposition}{Proposition}

\theoremstyle{definition}
\newtheorem{example}[theorem]{Example} % same for example numbers
\newtheorem{exercise}[theorem]{Exercise}

\newtheoremstyle{custom_definition}% name of the style to be used
  {\topsep} % measure of space to leave above the theorem. E.g.: 3pt
  {\topsep} % measure of space to leave below the theorem. E.g.: 3pt
  {\normalfont} % name of font to use in the body of the theorem
  {} % measure of space to indent
  {\bfseries} % name of head font
  {.\newline} % punctuation between head and body
  {\topsep}% space after theorem head; " " = normal interword space
  {\thmname{#1}\thmnumber{ #2} --- \thmnote{#3}} % Manually specify head

\theoremstyle{custom_definition}
\newtheorem{definition}[theorem]{Definition} 

\usepackage{amssymb}
\usepackage{amsmath}

\usepackage{tikz}
\usepackage{tikz-cd}
\usetikzlibrary{%
  matrix,%
  calc,%
  arrows%
}

%%%%%
\newcommand{\set}[1]{\left\{\, #1 \,\right\}}
\newcommand{\makeset}[2]{\left\{\, #1 \mathrel{\mid} #2 \,\right\}}


\newcommand{\Ker}[1]{ \mathrm{Ker} \left( #1 \right) }
\newcommand{\Coker}[1]{ \mathrm{Coker} \left( #1 \right) }
\newcommand{\Imag}[1]{ \mathrm{Im} \left( #1 \right)}

\newcommand{\bigslant}[2]{{\raisebox{.2em}{$#1$}\left/\raisebox{-.2em}{$#2$}\right.}}

\newcommand{\restr}[2]{{% we make the whole thing an ordinary symbol
  \left.\kern-\nulldelimiterspace % automatically resize the bar with \right
  #1 % the function
  \vphantom{\big|} % pretend it's a little taller at normal size
  \right|_{#2} % this is the delimiter
  }}

\begin{document}

\begin{exercise}
    
\end{exercise}

\begin{definition}[Saturation]
    Let \(A\) be a ring and \(S\) be a multiplicative closed subset. The complement in \(A\) of the union of prime ideals that do not meet \(S\) is denoted by \(\overline{S}\) and is called the saturation of \(S\). It is the smallest and unique multiplicativly closed subset that contains \(S\).
\end{definition}

\begin{exercise}[3.8]
    \begin{itemize}
        \item i) \(\Rightarrow\) ii): Let \(\phi: S^{-1}A \longrightarrow T^{-1}A\) be the ring homomorphism that maps \(a/s \in S^{-1}A\) to \(a/s\) as an element of \(T^{-1}A\). We show that if \(\phi\) is bijective, then for each \(t \in T\), \(t/1\) is a unit in \(S^{-1}A\).
        \item ii) \(\Rightarrow\) iii): For each \(t \in T\), \(t/1\) is a unit in \(S^{-1}A\). We show that for each \(t \in T\) there exists \(x \in A\) such that \(xt \in S\).
        \begin{enumerate}
            \item Let \(t \in T\), then \(t/1\) is a unit in \(S^{-1}A\), so there exists a \(a/s \in S^{-1}A\) such that \(t/1 \cdot a/s = 1\).
            \item We have
            \begin{align}
                & 1 = \frac{t}{1} \cdot \frac{a}{s} = \frac{t a}{s} \\
                \iff & 1 \cdot s = \frac{t a}{s} s \\
                \iff & s = t a \text{.}
            \end{align}
            \item If we set \(x := a\), then there exists a \(x \in A\) such that \(xt \in S\).
        \end{enumerate}
        \item iii) \(\Rightarrow\) iv): For each \(t \in T\) there exists \(x \in A\) such that \(xt \in S\). We show that \(T\) is contained in the saturation of \(S\).
        \begin{enumerate}
            \item Let \(t \in T\), \(x \in A\) such that \(xt \in S\), and \(\mathfrak{p}\) be a prime ideal that contains \(t\).
            \item Then \(xt \in \mathfrak{p}\), so \(\mathfrak{p} \cap S \neq \emptyset\), or in other words, prime ideals that contain \(t\) do not meet \(S\).
            \item Hence, \(t\) is also not contained in the union of prime ideals that do not meet \(S\).
            \item But it is contained in the complement of the union of prime ideals that do not meet \(S\).
            \item So \(t\) is in the saturation of \(S\).
        \end{enumerate}
        \item iv) \(\Rightarrow\) v): Let \(T\) be contained in the saturation of \(S\). We show that every prime ideal that meets \(T\) also meets \(S\).
        \begin{enumerate}
            \item Let \(\mathfrak{p}\) be a prime ideal such that \(\mathfrak{p} \cap T \neq \emptyset\).
            \item Then there is a \(x \in \mathfrak{p}\) with \(x \in T\).
            \item \(x\) is also in \(\overline{S}\) because \(T\) is contained in the saturation of \(S\).
            \item This means that \(x\) is not in the union of prime ideals that do not meet \(S\).
            \item So if \(x \in \mathfrak{p}\), then \(\mathfrak{p}\) must meet \(S\).
        \end{enumerate}
        \item v) \(\Rightarrow\) i): Every prime ideal that meets \(T\) also meets \(S\). Let \(\phi: S^{-1}A \longrightarrow T^{-1}A\) be the ring homomorphism which maps \(a/s \in S^{-1}A\) to \(a/s\) as an element of \(T^{-1}A\). We show \(\phi\) is bijective.
        \begin{enumerate}
            \item 
        \end{enumerate}
    \end{itemize}
\end{exercise}

\end{document}