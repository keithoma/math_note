\documentclass{book}
\usepackage[utf8]{inputenc}
\usepackage[english]{babel}

% page layout
\usepackage{geometry}
    \geometry{
        a4paper,
        total={170mm,257mm},
        left=20mm,
        top=20mm,
    }

\usepackage{amsthm}

\theoremstyle{plain}
\newtheorem{thm}{Theorem}[chapter] % reset theorem numbering for each chapter
\newtheorem{lmm}{Lemma}
\newtheorem{prps}{prps}

\theoremstyle{definition}
\newtheorem{exmp}[thm]{Example} % same for example numbers
\newtheorem{exr}[thm]{Exercise}

\newtheoremstyle{custom_definition}% name of the style to be used
  {\topsep} % measure of space to leave above the theorem. E.g.: 3pt
  {\topsep} % measure of space to leave below the theorem. E.g.: 3pt
  {\normalfont} % name of font to use in the body of the theorem
  {} % measure of space to indent
  {\bfseries} % name of head font
  {.\newline} % punctuation between head and body
  {\topsep}% space after theorem head; " " = normal interword space
  {\thmname{#1}\thmnumber{ #2} --- \thmnote{#3}} % Manually specify head

\theoremstyle{custom_definition}
\newtheorem{defn}[thm]{Definition} 

\usepackage{amssymb}
\usepackage{amsmath}

%%%%%
\newcommand{\set}[1]{\left\{\, #1 \,\right\}}
\newcommand{\makeset}[2]{\left\{\, #1 \mathrel{\Big|} #2 \,\right\}}

\newcommand{\bigslant}[2]{{\raisebox{.2em}{$#1$}\left/\raisebox{-.2em}{$#2$}\right.}}


\begin{document}
%
%
%
\chapter{Commutative Rings}

List of Definitions
\begin{enumerate}
  \item Rings
  \item Ring Homomorphism
  \item Ideal
  \item prime ideal
  \item coprime
  \item irreducible
  \item zero divisor
  \item nilpotent
  \item spe
  \item jacobson radical
  \item ideal operation
\end{enumerate}

\begin{defn}[Ring]
  A ring is a set \(R\) equipped with two binary operations \(+\) (addition) and \(\cdot\) (multiplication) satisfying the following three sets of axioms, called the ring axioms.
  \begin{enumerate}
    \item \((R, +)\) is an abelian group.
    \item \((R, \cdot)\) is a semigroup.
    \item Multiplication is distributive with respect to addition, meaning that
    \begin{itemize}
      \item \(a \cdot (b + c) = (a \cdot b) + (a \cdot c)\) for all \(a, b, c \in R\) (left distributivity).
      \item \((b + c) \cdot a = (b \cdot a) + (c \cdot a)\) for all \(a, b, c \in R\) (right distributivity).
    \end{itemize}
  \end{enumerate}
  A ring is called unitary if it contains the multiplicative identity and commutative if multiplication is commutative.
\end{defn}
From here, all rings are unitary commutative rings.

\begin{defn}[Ring Homomorphism]
\end{defn}

\begin{defn}[Ideal]
\end{defn}

\begin{defn}[Coprime]
  Let \(R\) be a ring. We say two elements \(x, y \in R\) are coprime if one of the following equivalent condition hold.
  \begin{enumerate}
    \item For each \(z \in R\) there exist \(a, b \in R\) such that \(ax + by = z\) (Bézout's identity).
    \item \(y + \langle x \rangle\) is a unit in \(\bigslant{R}{\langle x \rangle}\).
    \item The principal ideals generated by the elements are comaximal, i.e. \(\langle x \rangle + \langle y \rangle = \langle 1 \rangle = R\).
  \end{enumerate}
  Similarly, two ideals \(\mathfrak{a}\) and \(\mathfrak{b}\) in \(R\) are coprime if they are comaximal.
\end{defn}

\begin{defn}[Prime Ideal]
\end{defn}

\begin{defn}[Zero Divisor]
\end{defn}

\part{Exercises}

\begin{exr}
  Let \(\varphi: A \longrightarrow B\) be a ring homomorphism, \(\mathfrak{a}_1, \mathfrak{a}_2, \mathfrak{a}_3\) ideals in \(A\), and \(\mathfrak{b}_1, \mathfrak{b}_2, \mathfrak{b}_3\) ideals of \(B\). Prove the following statements.
  \begin{enumerate}
    \item \((\mathfrak{a}_1 + \mathfrak{a}_2)^e = (\mathfrak{a}_1)^e + (\mathfrak{a}_2)^e\).
    \begin{proof}
      We show \((\mathfrak{a}_1 + \mathfrak{a}_2)^e \subseteq (\mathfrak{a}_1)^e + (\mathfrak{a}_2)^e\). Let \(x \in (\mathfrak{a}_1 + \mathfrak{b}_2)^e\), then we have for some index set \(I\)
      \begin{equation}
        x = \sum_{i \in I} \lambda_i x_i \text{,}
      \end{equation}
      where \(\lambda_i \in B\) and \(x_i \in \varphi(\mathfrak{a}_1 + \mathfrak{a}_2)\) for all \(i \in I\). For each \(i \in I\) it is \(x_i = \varphi(\mu_{i, 1} a_{i, 1} + \mu_{i, 2} a_{i, 2})\), hence
      \begin{align}
        x =& \sum_{i \in I} \lambda_i \varphi(\mu_{i, 1} a_{i, 1} + \mu_{i, 2} a_{i, 2}) &  \\
        =& \sum_{i \in I} \lambda_i \left( \varphi(\mu_{i, 1} a_{i, 1}) + \varphi(\mu_{i, 2} a_{i, 2}) \right)&\text{(by linearity)} \\
        =& \sum_{i \in I} \lambda_i \left( \mu_{i, 1} \varphi( a_{i, 1}) + \mu_{i, 2} \varphi( a_{i, 2}) \right)&\text{(by linearity)} \\
        =& \sum_{i \in I} \lambda_i \mu_{i, 1}\varphi( a_{i, 1}) + \lambda_i \mu_{i, 2}\varphi( a_{i, 2}) & \text{(by distributivity)} \\
        =& \sum_{i \in I} \lambda_i \mu_{i, 1} \varphi( a_{i, 1}) + \sum_{i \in I} \lambda_i \mu_{i, 2} \varphi( a_{i, 2}) & \text{(reordering the sum)} \text{.} \\
      \end{align}
      The last term is exactly the elements expressed by \(\mathfrak{a}_1^e + \mathfrak{a}_2^e\), therefore, \((\mathfrak{a}_1 + \mathfrak{a}_2)^e \subseteq (\mathfrak{a}_1)^e + (\mathfrak{a}_2)^e\).

      I think the above proof should work into both directions.
    \end{proof}
    \item \((\mathfrak{b}_1 + \mathfrak{b}_2)^c \supseteq \mathfrak{b}_1^c + \mathfrak{b}_2^c\)
    \begin{proof}
      We have
      \begin{equation}
        (\mathfrak{b}_1 + \mathfrak{b}_2)^c = \makeset{x \in A}{\exists \, b_1 \in \mathfrak{b}_1 \exists \, b_2 \in \mathfrak{b}_2 : \varphi(x) = b_1 + b_2} \text{.}
      \end{equation}
      Now let \(x \in \mathfrak{b}_1^c + \mathfrak{b}_2^c\), then \(x = a_1 + a_2\) where \(\varphi(a_1) \in \mathfrak{b}_1\) and \(\varphi(a_2) \in \mathfrak{b}_2\). It is
      \begin{align}
        \varphi(x) =& \varphi(a_1 + a_2) & \\
        =& \varphi(a_1) + \varphi(a_2) & \text{(by additivity)}
      \end{align}
      Since \(\varphi(a_1) \in \mathfrak{b}_1\) and \(\varphi(a_2) \in \mathfrak{b}_2\) we have that \(x \in (\mathfrak{b}_1 + \mathfrak{b}_2)^c\).
    \end{proof}
  \end{enumerate}
\end{exr}

\begin{exr}
  Let \(\varphi: A \longrightarrow B\) be a ring homomorphism, \(\mathfrak{a}\) an ideal of \(A\), and \(\mathfrak{b}\) an ideal of \(B\). Prove the following statements:
  \begin{enumerate}
    \item Then \(\mathfrak{a} \subseteq \mathfrak{a}^{ec}\).
    \begin{proof}
      It is
      \begin{align}
        \mathfrak{a}^{ec} = & \makeset{x \in A}{\varphi(x) \in \mathfrak{a}^e} \\
        = & \makeset{x \in A}{\varphi(x) \in \langle \varphi(\mathfrak{a}) \rangle} \\
        = & \makeset{x \in A}{\forall i \in I \, \exists a_i \in \mathfrak{a}_1 : \varphi(x) = \sum_{i \in I} \lambda_i \varphi(a_i)} \text{.}
      \end{align}
      Let \(a \in \mathfrak{a}\) and choose \(I = \set{1}\), \(\lambda_1\), and \(a_i = a\), then \(a \in \mathfrak{a}^{ec}\).
    \end{proof}
    \item \(\mathfrak{b}^{ce} \subseteq \mathfrak{b}\).
    \item \(\mathfrak{a}^{ece} = \mathfrak{a}^e\).
    \item \(\mathfrak{b}^{cec} = \mathfrak{b}^c\).
    \item If \(\mathfrak{b}\) is an extension, then \(\mathfrak{b}^c\) is the largest ideal of \(A\) with extension \(\mathfrak{b}\).
    \item If two extensions have the same contraction, then they are equal.
    \begin{proof}
      a
    \end{proof}
  \end{enumerate}
\end{exr}

\begin{exr}
  Let \(A\) be a ring, \(A[\mathcal{X}, \mathcal{Y}]\) the polynomial ring in two sets of variables \(\mathcal{X}\) and \(\mathcal{Y}\). Show that \(\langle \mathcal{X} \rangle\) is prime if and only if \(A\) is a domain.
\end{exr}
\begin{proof}
  It should be noted here, that \(A[\mathcal{X}]\) does not contain \(X_1 X_2\) for example. It does contain \(X_1 + X_2\) however. The rest is easy.
\end{proof}
\begin{exr}
  Show that, in a PID, nonzero elements \(x\) and \(y\) are relatively prime (share no prime factor) if and only if they're coprime.
\end{exr}

\begin{exr}
  Let \(\mathfrak{a}\) and \(\mathfrak{b}\) be ideals, and \(\mathfrak{p}\) a prime ideal. Prove that these conditions are equivalent:
  \begin{enumerate}
    \item \(\mathfrak{a} \subseteq \mathfrak{p}\) or \(\mathfrak{b} \subseteq \mathfrak{p}\)
    \item \(\mathfrak{a} \cap \mathfrak{b} \subseteq \mathfrak{p}\)
    \item \(\mathfrak{a}\mathfrak{b} \subseteq \mathfrak{p}\)
  \end{enumerate}
\end{exr}
\begin{proof}
  (1) to (2) is easy. Same for (2) to (3). For (3) to (1) show it with contradiction.
\end{proof}

\begin{exr}
  Let \(A\) be a ring, \(\mathfrak{p}\) a prime ideal, and \(\mathfrak{m}_1, \ldots, \mathfrak{m}_n\) maximal ideals with \(\mathfrak{m}_1, \ldots, \mathfrak{m}_n = 0\). Show \(\mathfrak{p} = \mathfrak{m}_i\) for some \(i\).
\end{exr}
\begin{proof}
  By induction. Proof first for \(m_1 m_2\), the rest is clear.
\end{proof}

\begin{exr}
  Let \(A\) be a ring, \(\mathfrak{p}\) a prime, and \(\mathfrak{a}_1, \ldots, \mathfrak{a}_n\) ideals.
  \begin{enumerate}
    \item If \(\bigcap_{i=1}^n \mathfrak{a}_i \subseteq \mathfrak{p}\), then \(\mathfrak{a}_j \subseteq \mathfrak{p}\) for some \(j\).
    \begin{proof}
      If \(\mathfrak{a}_1 \cap \mathfrak{a}_2 \subseteq \mathfrak{p}\), then by the exercise above we have the desired result. The rest is induction.
    \end{proof}
    \item If \(\bigcap_{i=1}^n \mathfrak{a}_i = \mathfrak{p}\), then \(\mathfrak{a}_j \subseteq \mathfrak{p}\) for some \(j\).
    \begin{proof}
      Clear.
    \end{proof}
  \end{enumerate}
\end{exr}

\begin{exr}
  Let \(A\) be a ring, \(\mathcal{S}\) the set of all ideals that consist entirely of zerodivisors. Show that \(\mathcal{S}\) has maximal elements and they're prime. Conclude that \(\text{ZD}(A)\) is a union of primes.
\end{exr}

\begin{exr}
  Exercise 2.27, proof is silly
\end{exr}

\begin{exr}
  Let \(A_1 \times A_2\) be a product of two rings. Show that \(A_1 \times A_2\) is a domain if and only if either \(A_1\) or \(A_2\) is a domain and the other is \(0\).
\end{exr}
\begin{proof}
  The back implication is clear.

  For the other implication, assume neither is integral domain, this leads to an obvious contradiction.

  Now assume neither is 0. Choose \((a, 0) and (0, b)\), contradiction.
\end{proof}

\begin{exr}
  Let \(A_1 \times A_2\) be a product of rings, \(\mathfrak{p} \subset A_1 \times A_2\) an ideal. Show that \(\mathfrak{p}\) is prime if and only if either \(\mathfrak{p} = \mathfrak{p}_1 \times A_2\) with \(\mathfrak{p}_1 \subseteq A_1\) prime or \(\mathfrak{p} = A_1 \times \mathfrak{p}_2\) with \(\mathfrak{p}_2 \subseteq A_2\) prime.
\end{exr}

\begin{proof}
  If \(\mathfrak{p}\) is prime, then for each \((x, y) \in \mathfrak{p}\) we have that \((x, 1) \in \mathfrak{p}\) or \((1, y) \in \mathfrak{p}\). From this the first implication follows.

  For the other side is clear.
\end{proof}

\begin{exr}
  Let \(A\) be a domain, and \(x, y \in A\) with \(\langle x \rangle = \langle y \rangle\). Show \(x = uy\) for some unit \(u\).
\end{exr}
\begin{proof}
  From \(\langle x \rangle = \langle y \rangle\) we get that \(r x = s y\) for some \(r, s \in A\). Because \(A\) is a domain, we have \(\frac{r}{s} x = y\). This is a unit because \(\frac{r}{s} \cdot \frac{s}{r} = 1\).
\end{proof}

\begin{exr}
  Let \(k\) be a field, \(R\) a nonzero ring, \(\varphi: k \longrightarrow R\) a ring map. Prove \(\varphi\) is injective.
\end{exr}

\begin{proof}
  The trick here is to know that the kernel is an ideal. Since the kernel contains \(0\), it must also contain the ideal generated by it. Now, in all fields is the zeroideal maximal, hence the kernel is already maximal and contains only \(0\). From that we conclude \(\varphi\) is injective.
\end{proof}

\begin{exr}
  Let \(A\) be a ring, \(\mathfrak{p}\) a prime, \(\mathcal{X}\) a set of variables. Let \(\mathfrak{p}[\mathcal{X}]\) denote the set of polynomials with coefficients in \(\mathfrak{p}\). Prove these statements:
  \begin{enumerate}
    \item \(\mathfrak{p}R[\mathcal{X}]\) and \(\mathfrak{p}[\mathcal{X}]\) and \(\mathfrak{p}R[\mathcal{X}] + \langle \mathcal{X} \rangle\) are primes of \(R[\mathcal{X}]\), which contract to \(\mathfrak{p}\).
    \begin{proof}
      We have \(\bigslant{R[\mathcal{X}]}{\mathfrak{p}R[\mathcal{X}]} \simeq \left(\bigslant{R}{\mathfrak{p}}\right)[\mathcal{X}]\). The latter one is a domain because it is the polynomial ring of a domain. Therefore, \(\mathfrak{p}R[\mathcal{X}]\) is prime.

      We also have \(\mathfrak{p}R[\mathcal{X}] = \mathfrak{p}[\mathcal{X}]\).

      For the contraction let \(\varphi: R \longrightarrow R[\mathcal{X}]\). Then \(\varphi^{-1} \left( \mathfrak{p}[\mathcal{X}] \right) = \mathfrak{p}\).

      For the last part, consider
      \begin{equation}
        \varphi: R[\mathcal{X}] \longrightarrow \bigslant{R}{\mathfrak{p}}
      \end{equation}
      with the natural definition \(\varphi\left(a_0 + a_1 X + \ldots + a_n X^n\right) = a_0 + \mathfrak{p}\). Then, the kernel is all the polynomials with \(a_0 \in \mathfrak{p}\) so \(\mathfrak{p}R[\mathcal{X}] + \langle \mathcal{X} \rangle\). Since \(\varphi\) is obviously surjective, we have the isomorphism
      \begin{equation}
        \bigslant{R[\mathcal{X}]}{\mathfrak{p}R[\mathcal{X}]+ \langle \mathcal{X} \rangle} \simeq \bigslant{R}{\mathfrak{p}}
      \end{equation}
      The latter is a domain, so is the former, hence \(\mathfrak{p}R[\mathcal{X}]+ \langle \mathcal{X} \rangle\) is prime.

      Again for the contraction we have \(\varphi^{-1}(\mathfrak{p}R[\mathcal{X}] + \langle \mathcal{X} \rangle) = \mathfrak{p}\) (because we are basically only caring about \(a_0\)).
    \end{proof}
    \item Assume \(\mathfrak{p}\) is maximal. Then \(\mathfrak{p}R[\mathcal{X}] + \langle \mathcal{X} \rangle\) is maximal.
    \begin{proof}
      From above, we have an isomorphism
      \begin{equation}
        \bigslant{R[\mathcal{X}]}{\mathfrak{p}R[\mathcal{X}]+ \langle \mathcal{X} \rangle} \simeq \bigslant{R}{\mathfrak{p}}
      \end{equation}
      therefore, if \(\mathfrak{p}\) is maximal so is \(\mathfrak{p}R[\mathcal{X}] + \langle \mathcal{X} \rangle\).
    \end{proof}
  \end{enumerate}
\end{exr}

\begin{exr}
  Let \(R\) be a ring, \(X\) a variable, \(H \in P := R[X]\), and \(a \in R\). Given \(n \geq 1\), show \((X - a)^n\) and \(H\) are coprime if and only if \(H(a)\) is a unit.
\end{exr}
\begin{proof}
  Let \((X - a)^n\) and \(H\) be coprime. This means that
  \begin{align}
    \langle (X - a)^n \rangle + \langle H \rangle = R[X]
  \end{align}
  With the ring homomorphism \(\varphi\) that substitutes \(X\) with \(a\) we have
  \begin{align}
    R = \left( \langle (X - a)^n \rangle + \langle H \rangle \right)^e = \langle (X - a)^n \rangle^e + \langle H \rangle^e = \langle 0 \rangle + \langle H(a) \rangle 
  \end{align}
  hence \(H(a)\) must be a unit.

  Let \(H(a)\) be a unit in \(R\). Consider the map \(\varphi_a: R[X] \longrightarrow R\) with
  \begin{equation}
    p(X) \mapsto \varphi_a(p(X)) := p(a) \text{.}
  \end{equation}
  \(\varphi_a\) is a ring homomorphism because
  \begin{align}
    \varphi_a(p(X) + q(X)) = p(a) + q(a) = \varphi_a(p(X)) + \varphi_a(q(X))
  \end{align}
  and I'm to lazy the show it for the multiplication and \(\varphi_a(1) = 1\). Morover, \(\varphi_a\) is surjective. So we have \(\varphi^{-1}_a(R^\times) \subseteq (R[X])^\times\). From there, if \(H(a)\) is a unit, so must \(H(X)\) be. In that case, \(H(X)\) and \((X - a)^n\) are obviously coprime.
\end{proof}

\begin{exr}
  Let \(R\) be a ring, \(X\) a variable, \(F \in P := R[X]\), and \(a \in R\). Set \(F' := \partial F / \partial X\). Show the following statements are equivalent:
  \begin{enumerate}
    \item \(a\) is a supersimple root of \(F\). (\(a\) is a supersimple root if \(F(a) = 0\) and \(F'(a) \neq 0\) is a unit.)
    \item \(a\) is a root of \(F\), and \(X - a\) and \(F'\) are coprime.
    \item \(F = (X - a)G\) for some \(G\) in \(P\) coprime to \(X - a\).
  \end{enumerate}
  Show that if 3. holds, then \(G\) is unique.
\end{exr}

\begin{proof}
  "1. to 2.": Immideately, we have that \(a\) is a root of \(F\). Since \(F'(a)\) is a unit, by the previous exercise, we have that \((X - a)^n\) and \(F'\) are coprimes. In particular, if we choose \(n = 1\), we get the desired result.

  "2. to 3.": We have \(F'= G(X) + (X - a)G'\) and since this is coprime to \(X - a\) we have for \(\lambda, \mu \in R[X]\)
  \begin{align}
    \lambda (X - a) + \mu F'(X) =& 1 \\
    \lambda (X - a) + \mu (X - a)G'(X) + \mu G(X) =& 1 \\
    (\lambda + \mu G'(X) ) (X - a) + \mu G(X) =& 1
  \end{align}
  If we set \(\lambda + \mu G'(X)\) as the factor, we see that \(X - a\) and \(G\) are again coprime.

  "3. to 1.": Clearly, \(a\) is a root of \(F\). We also have \(\lambda G(X) + \mu (X - a) = 1\), so if we substitute \(X\) for a, we get the desired result.
\end{proof}

\begin{exr}
  Let \(R\) be a ring, \(\mathfrak{p}\) a prime, \(\mathcal{X}\) a set of variables, \(F, G \in R[\mathcal{X}]\). Let \(c(F), c(G), c(FG)\) be the ideals of \(R\) generated by the coefficients of \(F, G, FG\).
  \begin{enumerate}
    \item Show that if \(\mathfrak{p}\) doesn't contain either \(c(F)\) or \(c(G)\), then \(\mathfrak{p}\) doesn't contain \(c(FG)\).
    \begin{proof}
      Assume \(c(FG) \subseteq \mathfrak{p}\), then because of \(\mathfrak{p}\) is prime, we must have that the coefficients of \(F\) are also in \(\mathfrak{p}\).
    \end{proof}
    \item Show that if \(c(F) = R\) and \(c(G) = R\), then \(c(FG) = R\).
    \begin{proof}
      
    \end{proof}
  \end{enumerate}
\end{exr}

\chapter{Radicals}

\begin{exr}%3.10
  Let \(\varphi: R \longrightarrow R'\) be a ring homomorphism, \(\mathfrak{p}\) an ideal of \(R\). Show:
  \begin{enumerate}
    \item there is an ideal \(\mathfrak{q}\) of \(R'\) with \(\varphi^{-1}(\mathfrak{q}) = \mathfrak{p}\) if and only if \(\varphi^{-1}(\mathfrak{p}R') = \mathfrak{p}\).
    \begin{proof}
      Choose \(\mathfrak{q} := \langle \varphi(\mathfrak{p}) \rangle = \mathfrak{p}R'\), then \(\mathfrak{p} = \varphi^{-1}(\mathfrak{q}) = \varphi^{-1}(\mathfrak{p}R')\).

      Solution from the book: Given \(\mathfrak{q}\) note \(\varphi(\mathfrak{p}) \subseteq \mathfrak{q}\), as always \(\varphi(\varphi^{-1} (\mathfrak{q})) = \mathfrak{q}\). So \(\mathfrak{p}R' \subseteq \mathfrak{q}\). Hence \(\varphi^{-1}(\mathfrak{p}R \subseteq \varphi^{-1}(\mathfrak{q}) = \mathfrak{p}\)

      On the other hand, if \(\varphi^{-1}(\mathfrak{p}R') = \mathfrak{p}\), then define \(\mathfrak{q} := \mathfrak{p}R'\) and we have \(\varphi^{-1}(\mathfrak{q}) = \mathfrak{p}\).
    \end{proof}
  \end{enumerate}
\end{exr}

\begin{exr}
  Use Zorn's lemma to prove that any prime ideal \(\mathfrak{p}\) contains a prime ideal \(\mathfrak{q}\) that is minimal containing any given subset \(\mathfrak{s} \subseteq \mathfrak{p}\).
\end{exr}

\begin{proof}
  
\end{proof}

\begin{exr}
  Let \(R\) be a ring, \(\mathfrak{a} \subseteq \text{Jac}(R)\) an ideal, \(w \in R\), and \(w' \in \bigslant{R}{\mathfrak{a}}\) its residue. Prove that \(w \in R^\times\) if and only if \(w' \in (\bigslant{R}{\mathfrak{a}})^\times\). What if \(\mathfrak{a} \neq \notin \text{Jac}(R)\)?
\end{exr}

\begin{proof}
  Let \(w \in R^\times\)
\end{proof}

\chapter{Exact Sequences}

\begin{exr}
  Let \(M'\) and \(M''\) be modules, \(N \subseteq M'\) a submodule. Prove
  \begin{equation}
    \bigslant{M}{N} \simeq \bigslant{M'}{N \oplus M''}
  \end{equation}
\end{exr}

\begin{proof}
  Consider the two sequences
  \begin{align}
    0 \longrightarrow N \longrightarrow M' \longrightarrow \bigslant{M'}{N} \longrightarrow 0 \\
    0 \longrightarrow 0 \longrightarrow M'' \longrightarrow M'' \longrightarrow M'' \longrightarrow 0 \text{.}
  \end{align}
  with
  \begin{align}
    f_0(x) = 0 & f_1(x) = x & f_2(x) = x + N & f_3(x) = 0 \\
    g_0(x) = 0 & g_1(x) = 0 & g_2(x) = x & g_3 = 0 \text{,}
  \end{align}
  then the two sequences are exact because
  \begin{align}
    \text{im}(f_0) = 0 = \text{ker}(f_1) & \text{im}(f_1) = N = \text{ker}(f_2) & \text{im}(f_2) = \bigslant{M'}{N} = \text{ker}(f_3) \\
    \text{im}(g_0) = 0 = \text{ker}(g_1) & \text{im}(g_1) = 0 \text{ker}(g_2) & \text{im}(g_2) = M'' = \text{ker}(g_3).
  \end{align}
  Since both sequences are exact, so is their direct sum. Note that \(N \oplus 0 \simeq N\) and the maps between the chain are the composition of \(f\) and \(g\).
  \begin{equation}
    0 \longrightarrow N \longrightarrow M' \oplus M'' \longrightarrow \bigslant{M'}{N} \oplus M'' \longrightarrow 0
  \end{equation}
  Both \(f_2\) and \(g_2\) are surjective, so with the first isomorphism theorem, we have
  \begin{equation}
    \bigslant{M'}{N} \oplus M'' \simeq \bigslant{M}{N}
  \end{equation}
  as desired.
\end{proof}

\begin{exr}
  Let \(M'\) and \(M''\) be modules, and set \(M = M' \oplus M''\). Let \(N\) be a submodule of \(M\) containing \(M'\), and set \(N'' := N \cup M''\). Prove \(N = M' \oplus N''\).
\end{exr}

\begin{proof}
  Consider the sequence
  \begin{equation}
    N \cup M'' \longrightarrow N \longrightarrow M'
  \end{equation}
  This splits because the first part has a retraction and the second part is surjective, therefore we have the isomorphism as desired.
\end{proof}
\end{document}