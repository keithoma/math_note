\documentclass{book}
\usepackage[utf8]{inputenc}
\usepackage[english]{babel}

% page layout
\usepackage{geometry}
    \geometry{
        a4paper,
        total={170mm,257mm},
        left=20mm,
        top=20mm,
    }

\usepackage{amsthm}

\theoremstyle{plain}
\newtheorem{thm}{Theorem}[chapter] % reset theorem numbering for each chapter
\newtheorem{lmm}{Lemma}
\newtheorem{prps}{prps}

\theoremstyle{definition}
\newtheorem{exmp}[thm]{Example} % same for example numbers
\newtheorem{exr}[thm]{Exercise}

\newtheoremstyle{custom_definition}% name of the style to be used
  {\topsep} % measure of space to leave above the theorem. E.g.: 3pt
  {\topsep} % measure of space to leave below the theorem. E.g.: 3pt
  {\normalfont} % name of font to use in the body of the theorem
  {} % measure of space to indent
  {\bfseries} % name of head font
  {.\newline} % punctuation between head and body
  {\topsep}% space after theorem head; " " = normal interword space
  {\thmname{#1}\thmnumber{ #2} --- \thmnote{#3}} % Manually specify head

\theoremstyle{custom_definition}
\newtheorem{defn}[thm]{Definition} 

\usepackage{amssymb}
\usepackage{amsmath}

%%%%%
\newcommand{\set}[1]{\left\{\, #1 \,\right\}}
\newcommand{\makeset}[2]{\left\{\, #1 \mathrel{\Big|} #2 \,\right\}}

\newcommand{\bigslant}[2]{{\raisebox{.2em}{$#1$}\left/\raisebox{-.2em}{$#2$}\right.}}


\begin{document}
%
%
%
\chapter{Commutative Rings}

Definitions
\begin{enumerate}
  \item prime, coprime, relatively prime, irreducible
\end{enumerate}


\begin{exr}
  Let \(\varphi: A \longrightarrow B\) be a ring homomorphism, \(\mathfrak{a}_1, \mathfrak{a}_2, \mathfrak{a}_3\) ideals in \(A\), and \(\mathfrak{b}_1, \mathfrak{b}_2, \mathfrak{b}_3\) ideals of \(B\). Prove the following statements.
  \begin{enumerate}
    \item \((\mathfrak{a}_1 + \mathfrak{a}_2)^e = (\mathfrak{a}_1)^e + (\mathfrak{a}_2)^e\).
    \begin{proof}
      We show \((\mathfrak{a}_1 + \mathfrak{a}_2)^e \subseteq (\mathfrak{a}_1)^e + (\mathfrak{a}_2)^e\). Let \(x \in (\mathfrak{a}_1 + \mathfrak{b}_2)^e\), then we have for some index set \(I\)
      \begin{equation}
        x = \sum_{i \in I} \lambda_i x_i \text{,}
      \end{equation}
      where \(\lambda_i \in B\) and \(x_i \in \varphi(\mathfrak{a}_1 + \mathfrak{a}_2)\) for all \(i \in I\). For each \(i \in I\) it is \(x_i = \varphi(\mu_{i, 1} a_{i, 1} + \mu_{i, 2} a_{i, 2})\), hence
      \begin{align}
        x =& \sum_{i \in I} \lambda_i \varphi(\mu_{i, 1} a_{i, 1} + \mu_{i, 2} a_{i, 2}) &  \\
        =& \sum_{i \in I} \lambda_i \left( \varphi(\mu_{i, 1} a_{i, 1}) + \varphi(\mu_{i, 2} a_{i, 2}) \right)&\text{(by linearity)} \\
        =& \sum_{i \in I} \lambda_i \left( \mu_{i, 1} \varphi( a_{i, 1}) + \mu_{i, 2} \varphi( a_{i, 2}) \right)&\text{(by linearity)} \\
        =& \sum_{i \in I} \lambda_i \mu_{i, 1}\varphi( a_{i, 1}) + \lambda_i \mu_{i, 2}\varphi( a_{i, 2}) & \text{(by distributivity)} \\
        =& \sum_{i \in I} \lambda_i \mu_{i, 1} \varphi( a_{i, 1}) + \sum_{i \in I} \lambda_i \mu_{i, 2} \varphi( a_{i, 2}) & \text{(reordering the sum)} \text{.} \\
      \end{align}
      The last term is exactly the elements expressed by \(\mathfrak{a}_1^e + \mathfrak{a}_2^e\), therefore, \((\mathfrak{a}_1 + \mathfrak{a}_2)^e \subseteq (\mathfrak{a}_1)^e + (\mathfrak{a}_2)^e\).

      I think the above proof should work into both directions.
    \end{proof}
    \item \((\mathfrak{b}_1 + \mathfrak{b}_2)^c \supseteq \mathfrak{b}_1^c + \mathfrak{b}_2^c\)
    \begin{proof}
      We have
      \begin{equation}
        (\mathfrak{b}_1 + \mathfrak{b}_2)^c = \makeset{x \in A}{\exists \, b_1 \in \mathfrak{b}_1 \exists \, b_2 \in \mathfrak{b}_2 : \varphi(x) = b_1 + b_2} \text{.}
      \end{equation}
      Now let \(x \in \mathfrak{b}_1^c + \mathfrak{b}_2^c\), then \(x = a_1 + a_2\) where \(\varphi(a_1) \in \mathfrak{b}_1\) and \(\varphi(a_2) \in \mathfrak{b}_2\). It is
      \begin{align}
        \varphi(x) =& \varphi(a_1 + a_2) & \\
        =& \varphi(a_1) + \varphi(a_2) & \text{(by additivity)}
      \end{align}
      Since \(\varphi(a_1) \in \mathfrak{b}_1\) and \(\varphi(a_2) \in \mathfrak{b}_2\) we have that \(x \in (\mathfrak{b}_1 + \mathfrak{b}_2)^c\).
    \end{proof}
  \end{enumerate}
\end{exr}

\begin{exr}
  Let \(\varphi: A \longrightarrow B\) be a ring homomorphism, \(\mathfrak{a}\) an ideal of \(A\), and \(\mathfrak{b}\) an ideal of \(B\). Prove the following statements:
  \begin{enumerate}
    \item Then \(\mathfrak{a} \subseteq \mathfrak{a}^{ec}\).
    \begin{proof}
      It is
      \begin{align}
        \mathfrak{a}^{ec} = & \makeset{x \in A}{\varphi(x) \in \mathfrak{a}^e} \\
        = & \makeset{x \in A}{\varphi(x) \in \langle \varphi(\mathfrak{a}) \rangle} \\
        = & \makeset{x \in A}{\forall i \in I \, \exists a_i \in \mathfrak{a}_1 : \varphi(x) = \sum_{i \in I} \lambda_i \varphi(a_i)} \text{.}
      \end{align}
      Let \(a \in \mathfrak{a}\) and choose \(I = \set{1}\), \(\lambda_1\), and \(a_i = a\), then \(a \in \mathfrak{a}^{ec}\).
    \end{proof}
    \item \(\mathfrak{b}^{ce} \subseteq \mathfrak{b}\).
    \item \(\mathfrak{a}^{ece} = \mathfrak{a}^e\).
    \item \(\mathfrak{b}^{cec} = \mathfrak{b}^c\).
    \item If \(\mathfrak{b}\) is an extension, then \(\mathfrak{b}^c\) is the largest ideal of \(A\) with extension \(\mathfrak{b}\).
    \item If two extensions have the same contraction, then they are equal.
    \begin{proof}
      a
    \end{proof}
  \end{enumerate}
\end{exr}

\begin{exr}
  Let \(A\) be a ring, \(A[\mathcal{X}, \mathcal{Y}]\) the polynomial ring in two sets of variables \(\mathcal{X}\) and \(\mathcal{Y}\). Show that \(\langle \mathcal{X} \rangle\) is prime if and only if \(A\) is a domain.
\end{exr}
\begin{proof}
  It should be noted here, that \(A[\mathcal{X}]\) does not contain \(X_1 X_2\) for example. It does contain \(X_1 + X_2\) however. The rest is easy.
\end{proof}
\begin{exr}
  Show that, in a PID, nonzero elements \(x\) and \(y\) are relatively prime (share no prime factor) if and only if they're coprime.
\end{exr}

\begin{exr}
  Let \(\mathfrak{a}\) and \(\mathfrak{b}\) be ideals, and \(\mathfrak{p}\) a prime ideal. Prove that these conditions are equivalent:
  \begin{enumerate}
    \item \(\mathfrak{a} \subseteq \mathfrak{p}\) or \(\mathfrak{b} \subseteq \mathfrak{p}\)
    \item \(\mathfrak{a} \cap \mathfrak{b} \subseteq \mathfrak{p}\)
    \item \(\mathfrak{a}\mathfrak{b} \subseteq \mathfrak{p}\)
  \end{enumerate}
\end{exr}
\begin{proof}
  (1) to (2) is easy. Same for (2) to (3). For (3) to (1) show it with contradiction.
\end{proof}

\begin{exr}
  Let \(A\) be a ring, \(\mathfrak{p}\) a prime ideal, and \(\mathfrak{m}_1, \ldots, \mathfrak{m}_n\) maximal ideals with \(\mathfrak{m}_1, \ldots, \mathfrak{m}_n = 0\). Show \(\mathfrak{p} = \mathfrak{m}_i\) for some \(i\).
\end{exr}
\begin{proof}
  By induction. Proof first for \(m_1 m_2\), the rest is clear.
\end{proof}

\begin{exr}
  Let \(A\) be a ring, \(\mathfrak{p}\) a prime, and \(\mathfrak{a}_1, \ldots, \mathfrak{a}_n\) ideals.
  \begin{enumerate}
    \item If \(\bigcap_{i=1}^n \mathfrak{a}_i \subseteq \mathfrak{p}\), then \(\mathfrak{a}_j \subseteq \mathfrak{p}\) for some \(j\).
    \begin{proof}
      If \(\mathfrak{a}_1 \cap \mathfrak{a}_2 \subseteq \mathfrak{p}\), then by the exercise above we have the desired result. The rest is induction.
    \end{proof}
    \item If \(\bigcap_{i=1}^n \mathfrak{a}_i = \mathfrak{p}\), then \(\mathfrak{a}_j \subseteq \mathfrak{p}\) for some \(j\).
    \begin{proof}
      Clear.
    \end{proof}
  \end{enumerate}
\end{exr}

\begin{exr}
  Let \(A\) be a ring, \(\mathcal{S}\) the set of all ideals that consist entirely of zerodivisors. Show that \(\mathcal{S}\) has maximal elements and they're prime. Conclude that \(\text{ZD}(A)\) is a union of primes.
\end{exr}

\begin{exr}
  Exercise 2.27, proof is silly
\end{exr}

\begin{exr}
  Let \(A_1 \times A_2\) be a product of two rings. Show that \(A_1 \times A_2\) is a domain if and only if either \(A_1\) or \(A_2\) is a domain and the other is \(0\).
\end{exr}
\begin{proof}
  The back implication is clear.

  For the other implication, assume neither is integral domain, this leads to an obvious contradiction.

  Now assume neither is 0. Choose \((a, 0) and (0, b)\), contradiction.
\end{proof}
\end{document}