\documentclass{book}
\usepackage[utf8]{inputenc}
\usepackage[english]{babel}

% page layout
\usepackage{geometry}
    \geometry{
        a4paper,
        total={170mm,257mm},
        left=20mm,
        top=20mm,
    }

\usepackage{amsthm}

\theoremstyle{plain}
\newtheorem{thm}{Theorem}[chapter] % reset theorem numbering for each chapter
\newtheorem{lmm}{Lemma}
\newtheorem{prps}{Proposition}

\theoremstyle{definition}
\newtheorem{exmp}[thm]{Example} % same for example numbers
\newtheorem{exr}[thm]{Exercise}

\newtheoremstyle{custom_definition}% name of the style to be used
  {\topsep} % measure of space to leave above the theorem. E.g.: 3pt
  {\topsep} % measure of space to leave below the theorem. E.g.: 3pt
  {\normalfont} % name of font to use in the body of the theorem
  {} % measure of space to indent
  {\bfseries} % name of head font
  {.\newline} % punctuation between head and body
  {\topsep}% space after theorem head; " " = normal interword space
  {\thmname{#1}\thmnumber{ #2} --- \thmnote{#3}} % Manually specify head

\theoremstyle{custom_definition}
\newtheorem{defn}[thm]{Definition} 

\usepackage{amssymb}
\usepackage{amsmath}

\usepackage{tikz}
\usepackage{tikz-cd}
\usetikzlibrary{%
  matrix,%
  calc,%
  arrows%
}

%%%%%
\newcommand{\set}[1]{\left\{\, #1 \,\right\}}
\newcommand{\makeset}[2]{\left\{\, #1 \mathrel{\mid} #2 \,\right\}}


\newcommand{\Ker}[1]{ \mathrm{Ker} \left( #1 \right) }
\newcommand{\Coker}[1]{ \mathrm{Coker} \left( #1 \right) }
\newcommand{\Imag}[1]{ \mathrm{Im} \left( #1 \right)}

\newcommand{\bigslant}[2]{{\raisebox{.2em}{$#1$}\left/\raisebox{-.2em}{$#2$}\right.}}

\newcommand{\restr}[2]{{% we make the whole thing an ordinary symbol
  \left.\kern-\nulldelimiterspace % automatically resize the bar with \right
  #1 % the function
  \vphantom{\big|} % pretend it's a little taller at normal size
  \right|_{#2} % this is the delimiter
  }}

\begin{document}
\part{Zero}
\documentclass[a4paper]{book}
\title{Commutative Ring Theory}
\author{Kei Thoma}


% ---------------------------------------------------------------------
% P A C K A G E S
% ---------------------------------------------------------------------

% typography and formatting
\usepackage[english]{babel}
\usepackage[utf8]{inputenc}
\usepackage{geometry}
\usepackage{exsheets}
\usepackage{environ}
\usepackage{graphicx}
\usepackage{cutwin}
\usepackage{pifont}

% mathematics
\usepackage{xfrac}  
\usepackage{amsthm} % for theorems, and definitions
\usepackage{amssymb}
\usepackage{amsmath}
\usepackage{textcomp}
\usepackage{mathtools}
% \usepackage{MnSymbol} % for \cupdot

% extra
\usepackage{xcolor}
\usepackage{tikz}

% ---------------------------------------------------------------------
% S E T T I N G
% ---------------------------------------------------------------------

%maybe delete later, for colorbox
\usepackage{tcolorbox}
\newtcolorbox{defbox}{colback=blue!5!white,colframe=blue!75!black}
\newtcolorbox{defboxlight}{colback=cyan!5!white,colframe=cyan!75!black}
\newtcolorbox{thmbox}{colback=orange!5!white,colframe=orange!75!black}
\newtcolorbox{rembox}{colback=purple!5!white,colframe=purple!75!black}
\newtcolorbox{exmbox}{colback=gray!5!white,colframe=gray!75!black}
\newtcolorbox{intbox}{colback=violet!5!white,colframe=violet!75!black}

% typography and formatting
\geometry{margin=3cm}

\SetupExSheets{
  counter-format = ch.qu,
  counter-within = chapter,
  question/print = true,
  solution/print = true,
}

% mathematics
\newcounter{global}

\theoremstyle{definition}
\newtheorem{definition}{Definition}[]
\newtheorem{example}{Example}[definition]

\newtheorem{theorem}[definition]{Theorem}
\newtheorem{corollary}{Corollary}
\newtheorem{lemma}[definition]{Lemma}
\newtheorem{proposition}[definition]{Proposition}

\newtheorem*{remark}{Remark}
\newtheorem*{intuition}{Intuition}

% extra
\definecolor{mathif}{HTML}{0000A0} % for conditions
\definecolor{maththen}{HTML}{CC5500} % for consequences
\definecolor{mathrem}{HTML}{8b008b} % for notes
\definecolor{mathobj}{HTML}{008800}

\usetikzlibrary{positioning}
\usetikzlibrary{shapes.geometric, arrows}

% ---------------------------------------------------------------------
% C O M M A N D S
% ---------------------------------------------------------------------

\newcommand{\norm}[1]{\left\lVert#1\right\rVert}
\newcommand{\rank}{\text{rank}}
\newcommand{\Vol}{\text{Vol}}

\newcommand{\set}[1]{\left\{\, #1 \,\right\}}
\newcommand{\makeset}[2]{\left\{\, #1 \mid #2 \,\right\}}

\newcommand*\diff{\mathop{}\!\mathrm{d}}
\newcommand*\Diff{\mathop{}\!\mathrm{D}}

\newcommand\restr[2]{{% we make the whole thing an ordinary symbol
  \left.\kern-\nulldelimiterspace % automatically resize the bar with \right
  #1 % the function
  \vphantom{\big|} % pretend it's a little taller at normal size
  \right|_{#2} % this is the delimiter
  }}

% ---------------------------------------------------------------------
% R E N D E R
% ---------------------------------------------------------------------

\newif\ifshowproof
\showprooftrue

\NewEnviron{Proof}{%
    \ifshowproof%
        \begin{proof}%
            \BODY
        \end{proof}%
    \fi%
}%

\begin{document}
\maketitle
\tableofcontents
\chapter{Introduction and Motivation}
\chapter{Metric Spaces}
\chapter{Topological Spaces}
\chapter{Products, Sequential Continuity, and Nets}

\begin{thmbox}
    \begin{lemma}[Lemma 4.15]
        In any space \(X\), a subset \(A \subset X\) is open if and only if every point \(x \in A\) has a neighbourhood \(\mathcal{V} \subset X\) that is contained in \(A\).
    \end{lemma}
\end{thmbox}
\begin{proof}
    ``\(\Rightarrow\)'': If \(A\) is open, then \(A\) itself can be taken as the desired neighbourhood of every \(x \in A\).
    ``\(\Leftarrow\)'': Let every point \(x \in A\) have a neighbourhood \(\mathcal{V} \subset X\) that is contained in \(A\). Denote the open sets of these neighbourhoods by \(\mathcal{U}_x\). Then, \(A\) is the union of all these open sets \(\mathcal{U}_x\) and thus open.
\end{proof}

\begin{thmbox}
    \begin{lemma}[Lemma 4.16]
        In any first-countable topological space \(X\), a subspace \(A \subset X\) is not open if and only if there exists a point \(x \in A\) and a sequence \(x_n \in X \setminus A\) such that \(x_n \rightarrow x\).
    \end{lemma}
\end{thmbox}
\begin{proof}
    ``\(\Leftarrow\)'': (Proof by contraposition.) If \(A \subset X\) is open, then for every \(x \in A\) and sequence \(x_n \in X\) converging to \(x\), we cannot have \(x_n \in X \setminus A\) for all \(n\) since \(A\) is a neighbourhood of \(x\). This is true so far for all topological spaces, with or without first-countability axiom, but the latter will be needed to prove the converse.

    ``\(\Rightarrow\)'': So suppose now that \(A \subset X\) is not open, which by Lemma 4.15, means there exists a point \(x \in A\) such that no neighbourhood \(\mathcal{V} \subset X\) of \(x\) is contained in \(A\). Fix a countable neighbourhood base \(\mathcal{U}_1, \mathcal{U}_2, \ldots\) for \(x\). XXX

    Observe that since none of the neighbourhoods \(\mathcal{U}_n\) can be contained in \(A\), there exists a sequence of points
    \begin{align*}
        x_n \in \mathcal{U}_n \text{ such that } x_n \not\in A \text{.}
    \end{align*}
    This sequence converges to \(x\) since every neighbourhood \(\mathcal{V} \subset X\) of \(x\) contains one of \(\mathcal{U}_N\), implying that for all \(n \geq N\),
    \begin{align*}
        x_n \in \mathcal{U}_n \subset \mathcal{U}_n \subset \mathcal{V} \text{.}
    \end{align*}
\end{proof}


\begin{defbox}
    \begin{definition}
        A {\color{maththen}directed set} \((I, \prec)\) consists of a set \(I\) with a partial order \(\prec\) such that for every pair \(\alpha, \beta \in I\), there exists an element \(\gamma \in I\) with \(\gamma \prec \alpha\) and \(\gamma \prec \beta\).
    \end{definition}
\end{defbox}

\begin{defbox}
    \begin{definition}
        Given a space \(X\), a net \(\{x_\alpha\}_{\alpha \in I}\) in \(X\) is a function \(I \longrightarrow X: \alpha \mapsto x_\alpha\) where \((I, \prec)\) is a directed set.
    \end{definition}
\end{defbox}

\chapter{Compactness}

\begin{defbox}
    \begin{definition}
        A {\color{mathobj}subset} \(A \subset X\) is {\color{maththen}compact} if every open cover of \(A\) has a finite subcover, i.e. given an arbitary open cover \(\{\mathcal{U}_\alpha\}_{\alpha \in I}\) of \(A\), one can always find a finite subset \(\{\alpha_1, \ldots, \alpha_N\} \subset I\) such that \(A \subset \mathcal{U}_{\alpha_1} \cup \cdots \cup \mathcal{U}_{\alpha_N}\). We say that \(X\) itself is a {\color{maththen}compact space} if \(X\) is compact subset of itself.
    \end{definition}
\end{defbox}
\end{document}
\part{Commutative Rings}
\chapter{Rings and Ideals}

\section{Cheat Sheet}

\begin{defn}[Ring]
    A ring is a set \(R\) equipped with two binary operations \(+\) (addition) and \(\cdot\) (multiplication) satisfying the following three sets of axioms, called the ring axioms.
    \begin{enumerate}
      \item \((R, +)\) is an abelian group.
      \item \((R, \cdot)\) is a semigroup.
      \item Multiplication is distributive with respect to addition, meaning that
      \begin{itemize}
        \item \(a \cdot (b + c) = (a \cdot b) + (a \cdot c)\) for all \(a, b, c \in R\) (left distributivity).
        \item \((b + c) \cdot a = (b \cdot a) + (c \cdot a)\) for all \(a, b, c \in R\) (right distributivity).
      \end{itemize}
    \end{enumerate}
    A ring is called unitary if it contains the multiplicative identity and commutative if multiplication is commutative.
\end{defn}

\begin{defn}[Unit]
    
\end{defn}

\begin{defn}[Zerodivisors]
    
\end{defn}

\begin{defn}[Nilpotent]
    
\end{defn}

\begin{defn}[Idempotent]
    
\end{defn}

\begin{defn}[Ideal]
    
\end{defn}

\begin{defn}[Operations on Ideals]
    Let \(R\) be a ring \(\{\mathfrak{a}_i\}_{i \in I}\) be a collection of ideals in \(R\).
    \begin{enumerate}
        \item
        \begin{equation}
            \sum_{i \in I} \mathfrak{a}_i = \makeset{\sum_{i \in I}a_i}{a_i \in \mathfrak{a}_i \text{ for all \(i \in I\), and \(a_i = 0\) for almost all i}}
        \end{equation}
        \item The transporter of two ideals is defined By
        \begin{equation}
            (\mathfrak{a} : \mathfrak{b}) := \makeset{x \in R}{x \mathfrak{b} \subset \mathfrak{a}}
        \end{equation}
    \end{enumerate}
\end{defn}

\begin{defn}[Prime Ideal]
    
\end{defn}

\begin{defn}[Maximal Ideal]
    
\end{defn}

\begin{defn}[Quotient Ring]
    Given a ring \(A\) and two-sided ideal \(\mathfrak{a}\) in \(A\), we may define an congruence relation \(\sim\) on \(A\) as follows:
    \begin{equation}
        x \sim y :\Longleftrightarrow x - y \in \mathfrak{a} \text{.}
    \end{equation}
    The equivalence class of the element \(x\) in \(A\) is given by
    \begin{equation}
        [x] = x + \mathfrak{a} := \makeset{x + a}{a \in \mathfrak{a}}
    \end{equation}
    and the set of all such equivalence classes is denoted by \(A / \mathfrak{a}\); it becomes a ring, the factor ring or the quotient ring of \(A\) modulo \(\mathfrak{a}\), if one defines
    \begin{enumerate}
        \item \((a + \mathfrak{a}) + (b + \mathfrak{a}) = (a + b) + \mathfrak{a}\)
        \item \((a + \mathfrak{a}) (b + \mathfrak{a}) = (ab) + \mathfrak{a}\)
    \end{enumerate}
    The map \(\pi: R \longrightarrow A / \mathfrak{a}, \, x \mapsto \pi(x) := x + \mathfrak{a}\) is a surjective ring homomorphism and is sometimes called the natural quotient map or the canonical homomorphism.
\end{defn}

\begin{prps}[Universal Property]
    Let \(A\) and \(B\) be rings, \(\mathfrak{a}\) an ideal, and \(f: A \longrightarrow B\) a ring homomorphism with \(\mathfrak{a} \subseteq \Ker{f}\). Then there exists a unique ring homomorphism \(\tilde{f}: A / \mathfrak{a} \longrightarrow B\) such that \(f = \tilde{f} \circ \pi\).
\end{prps}

\begin{defn}[Integral Domain]
    
\end{defn}

\begin{thm}
    \begin{itemize}
        \item prime ideal, quotient is integral domain
        \item same as above, but if prime maximal, then quotient is a fields
        \item Maximal ideals are prime ideals.
        \item There is a 1:1 correspondence
        \begin{equation}
            \set{\text{Ideals in \(A / \mathfrak{a}\)}} \longleftrightarrow \makeset{\mathfrak{b} / \mathfrak{a}}{\mathfrak{a} \subseteq \mathfrak{b} \subseteq A}
        \end{equation}
    \end{itemize}
\end{thm}

\begin{defn}[Unique Factorization Domain]
    
\end{defn}

\begin{defn}[Principal Ideal Domain]
    
\end{defn}

\begin{prps}
    Commutative Rings \(\supset\) Unique Factorization Domain \(\supset\) Principal Ideal Domain \(\supset\) Fields
\end{prps}

\begin{thm}
    \begin{itemize}
        \item prime ideal, quotient is integral domain
        \item same as above, but if prime maximal, then quotient is a fields
        \item Maximal ideals are prime ideals.
        \item There is a 1:1 correspondence
        \begin{equation}
            \set{\text{Ideals in \(A / \mathfrak{a}\)}} \longleftrightarrow \makeset{\mathfrak{b} / \mathfrak{a}}{\mathfrak{a} \subseteq \mathfrak{b} \subseteq A}
        \end{equation}
    \end{itemize}
\end{thm}

\section{Examples}

\begin{exmp}
    \begin{enumerate}
        \item \(\mathbb{Z}\)
        \item All fields.
        \item Let \(S\) be any set, then \((\mathcal{P}(S), \triangle, \cap)\) is a ring.
        \item continuous \(f: I \longrightarrow \mathbb{R}\) with a real interval I forms a ring.
        \item cartesian product of rings
    \end{enumerate}
\end{exmp}

\begin{exmp}
    Let \(S\) be any set, then \((2^S, \triangle, \cap)\) is a ring.
    \begin{enumerate}
        \item \(0 = \emptyset\) and \(-A = A\)
        \item The neutral element of the multiplication is \(S\).
        \item \((2^S)^\times = \{S\}\)
        \item \(\mathrm{ZD}(2^S) = 2^S - S\) since \(A \cap A^c = \emptyset\) (also minus the empty set) (this seems to be true for all boolean rings)
        \item \(\mathrm{Nil}(2^S) = \emptyset\) (seems to be true for all boolean rings)
        \item \(\langle A \rangle = 2^A\) contains all subset of \(A\)
    \end{enumerate}
\end{exmp}

\section{Proofs}

\section{Exercises}
\chapter{Radicals}

\section{Cheat Sheet}

\section{Proofs}

\section{Exercises}

\begin{exr}
    Let \(R\) be a ring, \(\mathfrak{a} \subset \mathrm{Jac}(R)\) an ideal, \(u \in R\), and \(u + \mathfrak{a}\) its residue in \(R\). Prove that \(u \in R^\times\) if and only if \(u + \mathfrak{a} \in (R / \mathfrak{a})^\times\). What if \(\mathfrak{a} \not\subset \mathrm{Jac}(R)\)?
\end{exr}
\chapter{Zariski Topology}

\begin{defn}[Spectrum]
    Let \(R\) be a ring. We denote the set of all prime ideals of \(R\) by \(\mathrm{Spec}(R)\) and the set of all maximal ideals of \(R\) by \(\mathrm{Spm}(R)\).
\end{defn}

\begin{defn}[Variety]
    Let \(R\) be a ring and \(\mathfrak{a}\) an ideal in \(R\). Let \(\mathbf{V}(\mathfrak{a})\) denote the subset of \(\mathrm{Spec}(R)\) consisting of those primes that contain \(\mathfrak{a}\), i.e.
    \begin{equation}
        \mathbf{V}(\mathfrak{a}) := \makeset{\mathfrak{p} \in \mathrm{Spec}(R)}{\mathfrak{a} \subseteq \mathfrak{p}}\text{.}
    \end{equation}
    We call \(\mathbf{V}(\mathfrak{a})\) the variety of \(\mathfrak{a}\).
\end{defn}

\begin{prps}
    Let \(R\) be a ring, and \(\mathfrak{a}\) and \(\mathfrak{b}\) two ideals in \(R\).
    \begin{enumerate}
        \item If \(\mathfrak{a} \subset \mathfrak{b}\), then \(\mathbf{V}(b) \subset \mathbf{V}(a)\).
        \item If \(\mathbf{V}(b) \subset \mathbf{V}(a)\), then \(\mathfrak{a} \subset \sqrt{\mathfrak{b}}\).
        \item \(\mathbf{V}(\mathfrak{a}) = \mathbf{V}(\mathfrak{b})\) if and only if \(\sqrt{\mathfrak{a}} = \sqrt{\mathfrak{b}}\).
        \item \(\mathbf{V}(\mathfrak{a}) \cup \mathbf{V}(\mathfrak{b}) = \mathbf{V}(\mathfrak{a} \cap \mathfrak{b}) = \mathbf{V}(\mathfrak{a}\mathfrak{b})\).
        \item For any index set \(I\), it is \(\bigcap_{i \in I}\mathbf{V}(\mathfrak{a}_i) = \mathbf{V}(\sum_{i \in I}\mathfrak{a}_i)\).
        \item \(\mathbf{V}(\langle 0 \rangle) = \mathrm{Spec}(R)\).
    \end{enumerate}
\end{prps}

\begin{defn}[Zariski Topology]
    Declaring \(\mathbf{V}(\mathfrak{a})\) to be closed sets induces a topology on \(\mathrm{Spec}(R)\), the Zariski topology.

    Given an element \(f \in R\), we call the open set
    \begin{equation}
        \mathbf{D}(f) := \mathrm{Spec}(R) - \mathbf{V}(\langle f \rangle)
    \end{equation}
    a principal open set. These sets form a basis for the topology of \(\mathrm{Spec}(R)\); indeed, given any prime \(\mathfrak{a} \not\subset \mathfrak{p}\), there is an \(f \in \mathfrak{a} - \mathfrak{p}\), and so \(\mathfrak{p} \in \mathbf{D}(f) \subset \mathrm{Spec}(R) - \mathbf{V}(\mathfrak{a})\). Further, \(f, g \not\in \mathfrak{p}\) if and only if \(fg \not\in \mathfrak{p}\) for any \(f, g \in R\) and prime \(\mathfrak{p}\), in other words
    \begin{equation}
        \mathbf{D}(f) \cap \mathbf{D}(g) = \mathbf{D}(fg)
    \end{equation}
\end{defn}

\section{Proofs}

\begin{prps}
    Let \(R\) be a ring, and \(\mathfrak{a}\) and \(\mathfrak{b}\) two ideals in \(R\).
    \begin{enumerate}
        \item If \(\mathfrak{a} \subset \mathfrak{b}\), then \(\mathbf{V}(b) \subset \mathbf{V}(a)\).
        \item If \(\mathbf{V}(b) \subset \mathbf{V}(a)\), then \(\mathfrak{a} \subset \sqrt{\mathfrak{b}}\).
        \item \(\mathbf{V}(\mathfrak{a}) = \mathbf{V}(\mathfrak{b})\) if and only if \(\sqrt{\mathfrak{a}} = \sqrt{\mathfrak{b}}\).
        \item \(\mathbf{V}(\mathfrak{a}) \cup \mathbf{V}(\mathfrak{b}) = \mathbf{V}(\mathfrak{a} \cap \mathfrak{b}) = \mathbf{V}(\mathfrak{a}\mathfrak{b})\).
        \begin{proof}
            \begin{align}
                \mathbf{V}(\mathfrak{a}) \cup \mathbf{V}(\mathfrak{b}) &= \makeset{\mathfrak{p} \in \mathrm{Spec}(R)}{\mathfrak{a} \subset \mathfrak{p}} \cup \makeset{\mathfrak{p} \in \mathrm{Spec}(R)}{\mathfrak{b} \subset \mathfrak{p}} \\
                &= \makeset{\mathfrak{p} \in \mathrm{Spec}(R)}{\mathfrak{a} \subset \mathfrak{p} \text{ or } \mathfrak{b} \subset \mathfrak{p}} \\
                &= \makeset{\mathfrak{p} \in \mathrm{Spec}(R)}{\mathfrak{a} \cap \mathfrak{b} \subset \mathfrak{p}} \\
                &= \mathbf{V}(\mathfrak{a} \cap \mathfrak{b})
            \end{align}
        \end{proof}
        \item For any index set \(I\), it is \(\bigcap_{i \in I}\mathbf{V}(\mathfrak{a}_i) = \mathbf{V}(\sum_{i \in I}\mathfrak{a}_i)\).
        \item \(\mathbf{V}(R) = \emptyset\).
        \begin{proof}
            \(\mathbf{V}(R) = \makeset{\mathfrak{p} \in \mathrm{Spec}(R)}{R \subset \mathfrak{p}} = \emptyset\) because by definition a prime ideal must not be the whole ring.
        \end{proof}
        \item \(\mathbf{V}(\langle 0 \rangle) = \mathrm{Spec}(R)\).
        \begin{proof}
            \(\mathbf{V}(\langle 0 \rangle) = \makeset{\mathfrak{p} \in \mathrm{Spec}(R)}{\langle 0 \rangle \subset \mathfrak{p}} = \mathrm{Spec}(R)\) because all ideals contain the zeroideal.
        \end{proof}
    \end{enumerate}
\end{prps}

\begin{prps}
    The Zariski topology is indeed a topology.
\end{prps}
\begin{proof}
    
\end{proof}

\section{Exercises}
\begin{exr}
    Let \(R\) be a ring and \(\mathfrak{p}, \mathfrak{q} \in \mathrm{Spec}(R)\). Show:
    \begin{enumerate}
        \item The closure \(\overline{\{\mathfrak{p}\}}\) of \(\mathfrak{p}\) is equal to \(\mathbf{V}(\mathfrak{p})\); that is, \(\mathfrak{q} \in \overline{\{\mathfrak{p}\}}\) if and only if \(\mathfrak{p} \subseteq \mathfrak{q}\).
        \begin{proof}
            Let \(\mathfrak{q} \in \overline{\{\mathfrak{p}\}}\). If \(f \in R - \mathfrak{p}\), then \(\mathfrak{q} \in \mathbf{D}(f)\).
        \end{proof}
    \end{enumerate}
\end{exr}

\begin{exr}
    Describe \(\mathrm{Spec}(\mathbb{R})\), \(\mathrm{Spec}(\mathbb{Z})\), \(\mathrm{Spec}(\mathbb{C}[X])\), and \(\mathrm{Spec}(\mathbb{R}[X])\).
\end{exr}
\begin{proof}
    \begin{enumerate}
        \item \(\mathrm{Spec}(\mathbb{R}) = \set{\langle 0 \rangle}\) because the only ideals in a field are the zeroideal and the field itself.
        \item \(\mathrm{Spec}(\mathbb{Z}) = \makeset{p \mathbb{Z}}{\text{\(p\) is a prime number}}\).
        \item \(\mathrm{Spec}(\mathbb{C}[X]) = \makeset{\langle X - z \rangle}{z \in \mathbb{C}}\) because \(\mathbb{C}[X]\) is a PID and because of the fundamental theorem of algebra.
        \item \(\mathrm{Spec}(\mathbb{R}[X])\) has the ideals above and all polynomials of degree two with complex roots.
    \end{enumerate}
    For any PID \(R\), the points \(x_p\) of \(\mathrm{Spec}(R)\) represents the ideals \(\langle p \rangle\) with \(p\) prime or \(0\). The closed sets are the \(\mathbf{V}(\langle a \rangle)\) with \(a \in R\); moreover, \(\mathbf{V}(\langle a \rangle) = \emptyset\) if \(a\) is a unit, \(\mathbf{V}(\langle 0 \rangle) = R\), and \(\mathbf{V}(\langle a \rangle) = x_{p_1} \cup \ldots \cup x_{p_s}\) if \(a = p_1^{n_1} \cdots p_s^{n_s}\) with \(p_i\) a prime and \(n_i \leq 1\).
\end{proof}
\begin{exr}
    Let \(R\) be a ring, and let \(X_1, X_2 \subset \mathrm{Spec}(R)\) closed subsets. Show that the following four statements are equivalent:
    \begin{enumerate}
        \item Then \(X_1 \sqcup X_2 = \mathrm{Spec}(R)\); that is, \(X_1 \cup X_2 = \mathrm{Spec}(R)\) and \(X_1 \cap X_2 = \emptyset\).
        \item There are complementary idempotents \(e_1, e_2 \in R\) with \(\mathrm{V}(\langle e_i \rangle) = X_i\).
    \end{enumerate}
\end{exr}
\begin{proof}
    "1. to 2." Since \(X_1\) and \(X_2\) are closed subsets, there are ideals \(\mathfrak{a}_1\) and \(\mathfrak{a}_2\) such that
    \begin{align}
        \mathrm{Spec}(R) = \mathbf{V}(\mathfrak{a}_1) \cup \mathbf{V}(\mathfrak{a}_2) = \mathbf{V}(\mathfrak{a}_1 \mathfrak{a}_2) \\
        \emptyset = \mathbf{V}(\mathfrak{a}_1) \cap \mathbf{V}(\mathfrak{a}_2) = \mathbf{V}(\mathfrak{a_1} + \mathfrak{a}_2)
    \end{align}
    If two variety are equal, the radical of the generating ideals are equal, hence \(\sqrt{\langle 0 \rangle} = \sqrt{\mathfrak{a}_1 \mathfrak{a}_2}\) and \(\sqrt{R} = \sqrt{\mathfrak{a}_1 + \mathfrak{a}_2}\).
\end{proof}
\part{Modules}
\end{document}