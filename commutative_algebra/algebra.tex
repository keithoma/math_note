\documentclass{book}
\usepackage[utf8]{inputenc}
\usepackage[english]{babel}

% page layout
\usepackage{geometry}
    \geometry{
        a4paper,
        total={170mm,257mm},
        left=20mm,
        top=20mm,
    }

\usepackage{amsthm}

\theoremstyle{plain}
\newtheorem{thm}{Theorem}[chapter] % reset theorem numbering for each chapter
\newtheorem{lmm}{Lemma}
\newtheorem{prps}{Proposition}

\theoremstyle{definition}
\newtheorem{exmp}[thm]{Example} % same for example numbers
\newtheorem{exr}[thm]{Exercise}

\newtheoremstyle{custom_definition}% name of the style to be used
  {\topsep} % measure of space to leave above the theorem. E.g.: 3pt
  {\topsep} % measure of space to leave below the theorem. E.g.: 3pt
  {\normalfont} % name of font to use in the body of the theorem
  {} % measure of space to indent
  {\bfseries} % name of head font
  {.\newline} % punctuation between head and body
  {\topsep}% space after theorem head; " " = normal interword space
  {\thmname{#1}\thmnumber{ #2} --- \thmnote{#3}} % Manually specify head

\theoremstyle{custom_definition}
\newtheorem{defn}[thm]{Definition} 

\usepackage{amssymb}
\usepackage{amsmath}

\usepackage{tikz}
\usepackage{tikz-cd}
\usetikzlibrary{%
  matrix,%
  calc,%
  arrows%
}

%%%%%
\newcommand{\set}[1]{\left\{\, #1 \,\right\}}
\newcommand{\makeset}[2]{\left\{\, #1 \mathrel{\mid} #2 \,\right\}}


\newcommand{\Ker}[1]{ \mathrm{Ker} \left( #1 \right) }
\newcommand{\Coker}[1]{ \mathrm{Coker} \left( #1 \right) }
\newcommand{\Imag}[1]{ \mathrm{Im} \left( #1 \right)}

\newcommand{\bigslant}[2]{{\raisebox{.2em}{$#1$}\left/\raisebox{-.2em}{$#2$}\right.}}

\newcommand{\restr}[2]{{% we make the whole thing an ordinary symbol
  \left.\kern-\nulldelimiterspace % automatically resize the bar with \right
  #1 % the function
  \vphantom{\big|} % pretend it's a little taller at normal size
  \right|_{#2} % this is the delimiter
  }}

\begin{document}
\part{Topology}
\begin{defn}[Topology and Topological Space]
    Let \(X\) be a nonempty set. A set \(\mathcal{T}\) of subsets of \(X\) is said to be a topology on \(X\) if
    \begin{enumerate}
        \item \(X\) and the empty set \(\emptyset\) belong to \(\mathcal{T}\)
        \item the union of arbitary many number of sets in \(\mathcal{T}\) belong to \(\mathcal{T}\)
        \item the intersection of any two sets in \(\mathcal{T}\) belongs to \(\mathcal{T}\)
    \end{enumerate}
    The pair \((X, \mathcal{T})\) is called a topological space.
\end{defn}

\begin{defn}[Discrete Topology]
    Let \(X\) be any nonemoty set and \(\mathcal{T}\) be the collection of all subsets of \(X\). Then \(\mathcal{T}\) is called the discrete topology on the set \(X\). The topological space \((X, \mathcal{T})\) is called a discrete space.
\end{defn}

\begin{defn}[Indiscrete Topology]
    Let \(X\) be any nonempty set and \(\mathcal{T} = \{\mathcal{T}, \emptyset\}\). Then \(\mathcal{T}\) is called the indiscrete topology and \((X, \mathcal{T})\) is said to be an indiscrete space.
\end{defn}

\begin{prps}
    If \((X, \mathcal{T})\) is a topological space such that for every \(x \in X\) the singleton set \(\{x\}\) is in \(\mathcal{T}\) then \(\mathcal{T}\) is the discrete topology.
\end{prps}

\begin{defn}
    Let \((X, \mathcal{T})\) be any topological space. Then the members of \(\mathcal{T}\) are said to be open sets.
\end{defn}

\begin{prps}
    If \((X, \mathcal{T})\) is any topological space, then
    \begin{enumerate}
        \item \(X\) and \(\emptyset\) are open sets.
        \item The union of arbitary many number of open sets is an open set.
        \item The intersection of finitely many number of open sets is an open set.
    \end{enumerate}
\end{prps}

\begin{defn}
    Let \((X, \mathcal{T})\) be a topological space. A subset \(S\) of \(X\) is said to be a closed set in \((X, \mathcal{T})\) if its complment in \(X\), namely \(X - S\) is open in \((X, \mathcal{T})\).
\end{defn}

\begin{prps}
    If \((X, \mathcal{T})\) is any topological space, then
    \begin{enumerate}
        \item \(\emptyset\) and \(X\) are closed set.
        \item The intersection of arbitary many number of closed sets is a closed set.
        \item The union of finitely many number of closed sets is a closed set.
    \end{enumerate}
\end{prps}

\begin{defn}
    A subset \(S\) of a topological space \((X, \mathcal{T})\) is said to be clopen if it is both open and closed in \((X, \mathcal{T})\).
\end{defn}

\begin{defn}
    Let \(X\) be any nonempty set. A topology \(\mathcal{T}\) on \(X\) is called the finite-closed topology or the cofinite topology if the closed subsets of \(X\) are \(X\) and all finite subsets of \(X\); that is, the open sets are \(\emptyset\) and all subsets of \(X\) which have finite complments.
\end{defn}

\begin{defn}[Euclidean Topology]
    A subset \(S\) of \(\mathbb{R}\) is said to be open in the euclidean topology on \(\mathbb{R}\) if for each \(x \in S\), there exist \(a, b \in \mathbb{R}\), with \(a < b\), such that \(x \in (a, b) \subseteq S\).
\end{defn}

\begin{prps}
    A subset \(S\) of \(\mathbb{R}\) is open if and only if it is a union of open intervals.
\end{prps}

\begin{defn}[Basis for a Topology]
    Let \((X, \mathcal{T})\) be a topological space. A collection \(\mathcal{B}\) of open subsets of \(X\) is said to be a basis for the topology \(\mathcal{T}\) if every open set is a union of members in \(\mathcal{B}\).
\end{defn}

\begin{exmp}
    Let \(\mathcal{B} = \makeset{(a, b)}{a, b \in \mathbb{R}, a < b}\). Then \(\mathcal{B}\) is a basis for the euclidean topology on \(\mathbb{R}\).
\end{exmp}

\begin{exmp}
    Let \((X, \mathcal{T})\) be a discrete space and \(\mathcal{B}\) the family of all singleton subsets of \(X\); that is, \(\mathcal{B} = \makeset{\{x\}}{x \in X}\).
\end{exmp}

\begin{exmp}
    Let \(X = \set{a, b, c, d, e, f}\) and
    \begin{equation}
        \mathcal{T}_1 = \set{X, \emptyset, \{a\}, \{c, d\}, \{a, c, d\}, \{b, c, d, e, f\}}
    \end{equation}
    Then \(\mathcal{B} = \set{{a}, {c, d}, {b, c, d, e, f}}\) is a basis for \(\mathcal{T}_1\) as \(\mathcal{B} \subseteq \mathcal{T}_1\) and every member of \(\mathcal{T}_1\) can be expressed as a union of members of \(\mathcal{B}\). Note that \(\mathcal{T}_1\) itself is also a basis for \(\mathcal{T}_1\).
\end{exmp}

\begin{prps}
    Let \(X\) be a nonempty set and let \(\mathcal{B}\) be a collection of subsets of \(X\). Then \(\mathcal{B}\) is a basis for a topology on \(X\) if and only if \(\mathcal{B}\) has the following properties:
    \begin{enumerate}
        \item \(X = \bigcup_{B \in \mathcal{B}} B\)
        \item for any \(B_1, B_2 \in \mathcal{B}\) the set \(B_1 \cap B_2\) is a union of members of \(\mathcal{B}\)
    \end{enumerate}
\end{prps}

\begin{prps}
    Let \((X, \mathcal{T})\) be a topological space. A family \(\mathcal{B}\) of open subsets of \(X\) is a basis for \(\mathcal{T}\) if and only if for any point \(x\) belonging to any open set \(U\) there is a \(B \in \mathcal{B}\) such that \(x \in B \subseteq U\)
\end{prps}

\begin{prps}
    Let \(\mathcal{B}\) be a basis for a topology \(\mathcal{T}\) on a set \(X\). Then a subset \(U\) of \(X\) is open if and only if for each \(x \in U\) there exists a \(B \in \mathcal{B}\) such that \(x \in B \subseteq U\).
\end{prps}

\begin{prps}
    Let \(\mathcal{B}_1\) and \(\mathcal{B}_2\) be bases for topologies \(\mathcal{T}_1\) and \(\mathcal{T}_2\) respectively, on a nonempty set \(X\). Then \(\mathcal{T}_1 = \mathcal{T}_2\) if and only if
    \begin{enumerate}
        \item for each \(B \in \mathcal{B}_1\) and each \(x \in \mathcal{B}\), there exists a \(B' \in \mathcal{B}_2\) such that \(x \in B' \subseteq B\)
        \item for each \(B \in \mathcal{B}_2\) and each \(x \in \mathcal{B}\), there exists a \(B' \in \mathcal{B}_1\) such that \(x \in B' \subseteq B\)
    \end{enumerate}
\end{prps}
\part{Other Stuff}
\begin{defn}[Euler Totient Function]
    \textit{The Euler totient function counts the positive integers up to a given integer \(n\) that are relatively prime to \(n\)}.
\end{defn}
\part{Commutative Rings}
\begin{defn}[Ring]
    A ring is a set \(R\) equipped with two binary operations \(+\) (addition) and \(\cdot\) (multiplication) satisfying the following three sets of axioms, called the ring axioms.
    \begin{enumerate}
      \item \((R, +)\) is an abelian group.
      \item \((R, \cdot)\) is a semigroup.
      \item Multiplication is distributive with respect to addition, meaning that
      \begin{itemize}
        \item \(a \cdot (b + c) = (a \cdot b) + (a \cdot c)\) for all \(a, b, c \in R\) (left distributivity).
        \item \((b + c) \cdot a = (b \cdot a) + (c \cdot a)\) for all \(a, b, c \in R\) (right distributivity).
      \end{itemize}
    \end{enumerate}
    A ring is called unitary if it contains the multiplicative identity and commutative if multiplication is commutative.
\end{defn}

\begin{defn}[Ideal]
    
\end{defn}

\begin{defn}[Ideal Arithmetic]
    
\end{defn}

\begin{defn}[Prime Ideal]
    
\end{defn}

\begin{defn}[Maximal Ideal]
    
\end{defn}

\begin{defn}[Spectrum]
    
\end{defn}

\begin{defn}[Variety]
    Let \(A\) be a ring and \(\mathfrak{a}\) an ideal. Let \(\mathbf{V}(\mathfrak{a})\) denote the subset of \(\mathrm{Spec}(A)\) consisting of those primes that contain \(\mathfrak{a}\), i.e.
    \begin{equation}
        \mathbf{V}(\mathfrak{a}) := \makeset{S \in \mathcal{P}(\mathrm{Spec}(A))}{\text{for all \(\mathfrak{p} \in S\) it is \(\mathfrak{a} \subseteq \mathfrak{p}\)}}
    \end{equation}
    We call \(\mathbf{V}(\mathfrak{a})\) the variety of \(\mathfrak{a}\).
\end{defn}

\begin{defn}[Zariski Topology]
    Let \(\mathfrak{a} \subseteq A\) be an ideal. Declaring the sets
    \begin{equation}
        Z(\mathfrak{a}) := \makeset{\mathfrak{p} \in \mathrm{Spec}(A)}{\mathfrak{a} \subseteq \mathfrak{p}}
    \end{equation}
    to be closed induces a topology on \(\mathrm{Spec}(A)\), the Zariski Topology.
\end{defn}

\begin{defn}[Quotient Ring]
    Given a ring \(A\) and two-sided ideal \(\mathfrak{a}\) in \(A\), we may define an congruence relation \(\sim\) on \(A\) as follows:
    \begin{equation}
        x \sim y :\Longleftrightarrow x - y \in \mathfrak{a} \text{.}
    \end{equation}
    The equivalence class of the element \(x\) in \(A\) is given by
    \begin{equation}
        [x] = x + \mathfrak{a} := \makeset{x + a}{a \in \mathfrak{a}}
    \end{equation}
    and the set of all such equivalence classes is denoted by \(A / \mathfrak{a}\); it becomes a ring, the factor ring or the quotient ring of \(A\) modulo \(\mathfrak{a}\), if one defines
    \begin{enumerate}
        \item \((a + \mathfrak{a}) + (b + \mathfrak{a}) = (a + b) + \mathfrak{a}\)
        \item \((a + \mathfrak{a}) (b + \mathfrak{a}) = (ab) + \mathfrak{a}\)
    \end{enumerate}
    The map \(\pi: R \longrightarrow A / \mathfrak{a}, \, x \mapsto \pi(x) := x + \mathfrak{a}\) is a surjective ring homomorphism and is sometimes called the natural quotient map or the canonical homomorphism.
\end{defn}

\begin{prps}[Universal Property]
    Let \(A\) and \(B\) be rings, \(\mathfrak{a}\) an ideal, and \(f: A \longrightarrow B\) a ring homomorphism with \(\mathfrak{a} \subseteq \Ker{f}\). Then there exists a unique ring homomorphism \(\tilde{f}: A / \mathfrak{a} \longrightarrow B\) such that \(f = \tilde{f} \circ \pi\).
\end{prps}

\begin{defn}[Integral Domain]
    
\end{defn}

\begin{thm}
    \begin{itemize}
        \item prime ideal, quotient is integral domain
        \item same as above, but if prime maximal, then quotient is a fields
        \item Maximal ideals are prime ideals.
        \item There is a 1:1 correspondence
        \begin{equation}
            \set{\text{Ideals in \(A / \mathfrak{a}\)}} \longleftrightarrow \makeset{\mathfrak{b} / \mathfrak{a}}{\mathfrak{a} \subseteq \mathfrak{b} \subseteq A}
        \end{equation}
    \end{itemize}
\end{thm}

\section{Exercises}
\begin{exr}
    Let \(R\) be a ring and \(\mathfrak{p}, \mathfrak{q} \in \mathrm{Spec}(R)\). Show:
    \begin{enumerate}
        \item The closure \(\overline{\{\mathfrak{p}\}}\) of \(\mathfrak{p}\) is equal to \(\mathbf{V}(\mathfrak{p})\); that is, \(\mathfrak{q} \in \overline{\{\mathfrak{p}\}}\) if and only if \(\mathfrak{p} \subseteq \mathfrak{q}\).
    \end{enumerate}
\end{exr}
\chapter{Radicals}

\section{Cheat Sheet}

\section{Proofs}

\section{Exercises}

\begin{exr}
    Let \(R\) be a ring, \(\mathfrak{a} \subset \mathrm{Jac}(R)\) an ideal, \(u \in R\), and \(u + \mathfrak{a}\) its residue in \(R\). Prove that \(u \in R^\times\) if and only if \(u + \mathfrak{a} \in (R / \mathfrak{a})^\times\). What if \(\mathfrak{a} \not\subset \mathrm{Jac}(R)\)?
\end{exr}
\begin{proof}
    If \(u \in R^\times\), then we have immediately \(u \in (R/\mathfrak{a})^\times\) without using the condition \(\mathfrak{a} \subset \mathrm{Jac}(R)\).
\end{proof}
\chapter{Spectrum}

\begin{defn}[Spectrum]
    Let \(R\) be a ring. We denote the set of all prime ideals of \(R\) by \(\mathrm{Spec}(R)\) and the set of all maximal ideals of \(R\) by \(\mathrm{Spm}(R)\).
\end{defn}

\begin{defn}[Variety]
    Let \(R\) be a ring and \(\mathfrak{a}\) an ideal in \(R\). Let \(\mathbf{V}(\mathfrak{a})\) denote the subset of \(\mathrm{Spec}(R)\) consisting of those primes that contain \(\mathfrak{a}\), i.e.
    \begin{equation}
        \mathbf{V}(\mathfrak{a}) := \makeset{\mathfrak{p} \in \mathrm{Spec}(R)}{\mathfrak{a} \subseteq \mathfrak{p}}\text{.}
    \end{equation}
    We call \(\mathbf{V}(\mathfrak{a})\) the variety of \(\mathfrak{a}\).
\end{defn}

\begin{prps}
    Let \(R\) be a ring, and \(\mathfrak{a}\) and \(\mathfrak{b}\) two ideals in \(R\).
    \begin{enumerate}
        \item If \(\mathfrak{a} \subset \mathfrak{b}\), then \(\mathbf{V}(b) \subset \mathbf{V}(a)\).
        \item If \(\mathbf{V}(b) \subset \mathbf{V}(a)\), then \(\mathfrak{a} \subset \sqrt{\mathfrak{b}}\).
        \item \(\mathbf{V}(\mathfrak{a}) = \mathbf{V}(\mathfrak{b})\) if and only if \(\sqrt{\mathfrak{a}} = \sqrt{\mathfrak{b}}\).
        \item \(\mathbf{V}(\mathfrak{a}) \cup \mathbf{V}(\mathfrak{b}) = \mathbf{V}(\mathfrak{a} \cap \mathfrak{b}) = \mathbf{V}(\mathfrak{a}\mathfrak{b})\).
        \item For any index set \(I\), it is \(\bigcap_{i \in I}\mathbf{V}(\mathfrak{a}_i) = \mathbf{V}(\sum_{i \in I}\mathfrak{a}_i)\).
        \item \(\mathbf{V}(R) = \emptyset\).
        \item \(\mathbf{V}(\langle 0 \rangle) = \mathrm{Spec}(R)\).
    \end{enumerate}
\end{prps}

\begin{defn}[Zariski Topology]
    Let \(\mathfrak{a} \subseteq A\) be an ideal. Declaring the sets
    \begin{equation}
        Z(\mathfrak{a}) := \makeset{\mathfrak{p} \in \mathrm{Spec}(A)}{\mathfrak{a} \subseteq \mathfrak{p}}
    \end{equation}
    to be closed induces a topology on \(\mathrm{Spec}(A)\), the Zariski Topology.

    Given an element \(f \in R\), we call the open set
    \begin{equation}
        D(f) := \mathrm{Spec}(R) - \mathbf{V}(\langle f \rangle)
    \end{equation}
    a principal open set. These sets form a basis for the topology of \(\mathrm{Spec}(R)\); indeed, given any prime \(\mathfrak{a} \not\subset \mathfrak{p}\)
\end{defn}

\section{Proofs}

\begin{prps}
    The Zariski topology is indeed a topology.
\end{prps}
\begin{proof}
    
\end{proof}

\section{Exercises}
\begin{exr}
    Let \(R\) be a ring and \(\mathfrak{p}, \mathfrak{q} \in \mathrm{Spec}(R)\). Show:
    \begin{enumerate}
        \item The closure \(\overline{\{\mathfrak{p}\}}\) of \(\mathfrak{p}\) is equal to \(\mathbf{V}(\mathfrak{p})\); that is, \(\mathfrak{q} \in \overline{\{\mathfrak{p}\}}\) if and only if \(\mathfrak{p} \subseteq \mathfrak{q}\).
    \end{enumerate}
\end{exr}
\part{Modules}
\chapter{Modules}
\subsection*{Definition and Theorems}
\subsubsection*{Introduction}
\begin{defbox}
    \begin{definition}[Module]
    \end{definition}
\end{defbox}

\begin{exmbox}
    \begin{example}
        \begin{enumerate}
            \item If \(A\) is a field, then an \(A\)-module is a vector space.
            \item A \(\mathbb{Z}\)-module is just an abelian group.
        \end{enumerate}
    \end{example}
\end{exmbox}

\begin{defbox}
    \begin{definition}[Submodules]
        Let \(M\) be an \(A\)-module. A subset \(N\) of \(M\) is called a submodule if \((N, +)\) is a subgroup of \(M\) and for all \(n \in N\) and for all \(a \in A\) it is \(a \cdot n \in N\).
    \end{definition}
\end{defbox}


\begin{thmbox}
    \begin{proposition}
        Let \(A\) be a ring. If \(A\) is viewed as a module over itself, then its submodules are exactly its ideals, i.e.
        \begin{align*}
            \makeset{N}{N \text{ is a submodule of } A} = \makeset{\mathfrak{a}}{\mathfrak{a} \text{ is an ideal of } A}\text{.}
        \end{align*}
    \end{proposition}
\end{thmbox}


\begin{defbox}
    \begin{definition}[Homomorphism of Modules]
        
    \end{definition}
\end{defbox}

\begin{thmbox}
    \begin{proposition}
        Let \(M\) and \(N\) be an \(A\)-module, and \(\varphi: M \rightarrow N\) be an \(A\)-module homomorphism.
        \begin{enumerate}
            \item \(\mathrm{im}(\varphi)\) is a submodule of \(M\).
            \item \(\mathrm{ker}(\varphi)\) is a submodule of \(N\).
            \item For any submodule \(N^\prime\) of \(N\), its preimage \(\varphi^{-1}(N^\prime)\) is a submodule of \(M\).
        \end{enumerate}
    \end{proposition}
\end{thmbox}


\subsubsection*{Free and Finitely Generated}

\begin{defbox}
    \begin{definition}
        An \(A\)-module is finitely generated if there exists a finite set \(\{m_1, \ldots, m_n\}\) with \(n \in \mathbb{N}^+\) in \(M\) such that for any \(x\) in \(M\), there exits \(\lambda_1, \ldots, \lambda_n\) in \(A\) with
        \begin{align*}
            x = \lambda_1 m_1 + \cdots + \lambda_n m_n
        \end{align*}
    \end{definition}
\end{defbox}

\begin{thmbox}
    \begin{lemma}
        An \(A\)-module is finitely generated if and only if there exists a surjective \(A\)-module homomorphism
        \begin{align*}
            A^n \longrightarrow M
        \end{align*}
        for some \(n \in \mathbb{N}^+\).
    \end{lemma}
\end{thmbox}

\begin{defbox}
    \begin{definition}
        Let \(M\) be an \(A\)-module. A set \(B \subset M\) is a basis of \(M\) if
        \begin{enumerate}
            \item \(B\) is a generating set for \(M\)
            \item \(B\) is linearly independent
        \end{enumerate}
        A free module is a module with a basis.
    \end{definition}
\end{defbox}

\begin{rembox}
    \begin{remark}
        An \(A\)-module being free does \textbf{not} imply the module being finitely generated. Similary, an \(A\)-module being finitely generated does \textbf{not} imply the module being free.
    \end{remark}
\end{rembox}

\begin{exmbox}
    \begin{example}
        Two examples to illustrate the remark above.
        \begin{enumerate}
            \item As an \(\mathbb{Z}\)-module, \(\mathbb{Z} / 2 \mathbb{Z}\) is finitely generated but is not free.
            \item As an \(\mathbb{Z}\)-module, \(\bigoplus_{\mathbb{N}} \mathbb{Z}\) is free, but is not finitely generated.
        \end{enumerate}
    \end{example}
\end{exmbox}

\begin{proof}
    \begin{enumerate}
        \item \(\{1\}\) is a generating set of \(\mathbb{Z}/2\mathbb{Z}\) since \(1 \cdot 1 = 1\) and \(2 \cdot 1 = 0\). However, \(\{1\}\) and ...
    \end{enumerate}
\end{proof}


% ████████  ██████  ██████  ███████ ██  ██████  ███    ██ 
%    ██    ██    ██ ██   ██ ██      ██ ██    ██ ████   ██ 
%    ██    ██    ██ ██████  ███████ ██ ██    ██ ██ ██  ██ 
%    ██    ██    ██ ██   ██      ██ ██ ██    ██ ██  ██ ██ 
%    ██     ██████  ██   ██ ███████ ██  ██████  ██   ████ 
%
%  █████  ███    ██ ██████  
% ██   ██ ████   ██ ██   ██ 
% ███████ ██ ██  ██ ██   ██ 
% ██   ██ ██  ██ ██ ██   ██ 
% ██   ██ ██   ████ ██████  
%
%  █████  ███    ██ ███    ██ ██ ██   ██ ██ ██       █████  ████████  ██████  ██████  
% ██   ██ ████   ██ ████   ██ ██ ██   ██ ██ ██      ██   ██    ██    ██    ██ ██   ██ 
% ███████ ██ ██  ██ ██ ██  ██ ██ ███████ ██ ██      ███████    ██    ██    ██ ██████  
% ██   ██ ██  ██ ██ ██  ██ ██ ██ ██   ██ ██ ██      ██   ██    ██    ██    ██ ██   ██ 
% ██   ██ ██   ████ ██   ████ ██ ██   ██ ██ ███████ ██   ██    ██     ██████  ██   ██ 




\subsubsection*{Torsion and Annihilator}
\begin{defbox}
    \begin{definition}
        \begin{align*}
            \mathrm{Tor}(M) = \makeset{m \in M}{\text{there is an } a \in A \setminus \{ 0 \} \text{ such that} a \cdot m = 0}
        \end{align*}
    \end{definition}
\end{defbox}

\begin{example}
    \begin{enumerate}
        \item Let \(\mathbb{Z}\) be a module over itself. It is \(\mathrm{Tor}(\mathbb{Z}) = \{0\}\).
        \item Let \(n \in \mathbb{N}^+\) and consider the \(\mathbb{Z}\)-module \(\mathbb{Z}^n\). It is
    \end{enumerate}
\end{example}

\begin{thmbox}
    \begin{lemma}
        If \(M\) is a free \(A\)-module, then it is torsion-free, i.e. \(\mathrm{Tor}(M) = \{0\}\).
    \end{lemma}
\end{thmbox}
\begin{proof}
    
\end{proof}

\begin{defbox}
    \begin{definition}[Annihilator]
        
    \end{definition}
\end{defbox}

\begin{defbox}
    \begin{definition}[Radical]
        
    \end{definition}
\end{defbox}

\begin{defbox}
    \begin{definition}[Simple Modules]
        Let \(A\) be a ring. A nonzero \(A\)-module \(M\) is called simple if the only submodules are \(\{0\}\) and \(M\) itself.
    \end{definition}
\end{defbox}

\begin{example}
    If \(M\) is a simple \(A\)-module, then any \(f \in \mathrm{Hom}_A (M, M) \setminus \{0\}\) is an isomorphism.
\end{example}

\begin{proof}
    Fix an \(f \in \mathrm{Hom}_A (M, M) \setminus \{0\}\). Since \(\mathrm{ker}(f)\) is a submodule of \(M\), it must be either \(\{0\}\) or whole \(M\). But \(\mathrm{ker}(f) = M\) would mean that \(f = 0\) which was explicitly excluded, thus \(\mathrm{ker}(f) = \{0\}\). By the isomorphism theorem, we also have \(\mathrm{im}(f) \cong \sfrac{M}{\mathrm{ker}(f)} \cong M\). Therefore, \(f\) is bijective.
\end{proof}

\begin{defbox}
    \begin{definition}[Indecomposable]
        Let \(A\) be a ring. A nonzero \(A\)-module \(M\) is called indecomposable if it cannot be written as a direct sum of two non-zero submodules.
    \end{definition}
\end{defbox}

\begin{thmbox}
    \begin{proposition}
        Every simple module is indecomposable.
    \end{proposition}
\end{thmbox}

\begin{exmbox}
    \begin{example}
        Not all indecomposable modules are simple. For example, \(\mathbb{Z}\) is indecomposable, but is not simple.
    \end{example}
\end{exmbox}

\newpage
\section{Exercises and Notes}

\begin{example}
    Let \(f: M \rightarrow N\) be a surjective homomorphism of two finitely generated \(A\)-modules.

    \begin{enumerate}
        \item If \(N \cong A^n\) is a free \(A\)-module, show that \(M \cong \mathrm{ker}(f) \oplus N\).
        
        \begin{proof}
            Since \(N\) is finitely generated, let \((e_1, \ldots, e_n)\) be a set of generators. 
        \end{proof}
    \end{enumerate}
\end{example}

\begin{example}
    Let \(A\) be a ring, \(\mathfrak{a}\) and \(\mathfrak{b}\) ideals, \(M\) and \(N\) \(A\)-modules. Set
    \begin{align*}
        \Gamma_\mathfrak{a}(M) := \makeset{m \in M}{\mathfrak{a} \subset \sqrt{\mathrm{Ann}(m)}} \text{.}
    \end{align*}
    Prove the following statements.
    \begin{enumerate}
        \item If \(\mathfrak{a} \supset \mathfrak{b}\), then \(\Gamma_\mathfrak{a}(M) \subset \Gamma_\mathfrak{b}(M)\).
        
        \begin{proof}
            The proof is a matter of verification. Let \(m \in \Gamma_\mathfrak{a}(M)\). It is
            \begin{align*}
                m \in \Gamma_\mathfrak{a}(M) & \Rightarrow \mathfrak{a} \subset \sqrt{\mathrm{Ann}(m)} \\
                & \Rightarrow \text{For all } a \in \mathfrak{a} \text{ there is a } n \in \mathbb{N}^+ \text{ such that } a^n \in \mathrm{Ann}(m) \text{.}\\
                & \Rightarrow \text{For all } a \in \mathfrak{a} \text{ there is a } n \in \mathbb{N}^+ \text{ such that } a^n \cdot m = 0 \text{.}
                %                
                \intertext{Since \(\mathfrak{a} \supset \mathfrak{b}\), the last statement is true for all \(a \in \mathfrak{b}\). We have}
                %
                & \Rightarrow \text{For all } a \in \mathfrak{b} \text{ there is a } n \in \mathbb{N}^+ \text{ such that } a^n \cdot m = 0 \text{.} \\
                & \Rightarrow \text{For all } a \in \mathfrak{b} \text{ there is a } n \in \mathbb{N}^+ \text{ such that } a^n \in \mathrm{Ann}(m) \text{.}\\
                & \Rightarrow \mathfrak{b} \subset \sqrt{\mathrm{Ann}(m)} \\
                & \Rightarrow m \in \Gamma_\mathfrak{b}(M)
            \end{align*}
            Thus, \(\Gamma_\mathfrak{a}(M) \subset \Gamma_\mathfrak{b}(M)\).
        \end{proof}

        \item If \(M \subset N\), then \(\Gamma_\mathfrak{a}(M) = \Gamma_\mathfrak{a}(N) \cap M\).
        
        \begin{proof}
            Again, the proof is a matter of verification.

            ``\(\subset\)'': \(M \subset N\) implies \(\Gamma_\mathfrak{a}(M) \subset \Gamma_\mathfrak{a}(N)\). Moreover, it is \(\Gamma_\mathfrak{a}(M) \subset M\). Thus, \(\Gamma_\mathfrak{a}(M) \subset \Gamma_\mathfrak{a}(N) \cap M\).

            ``\(\supset\)'': Let \(m \in \Gamma_\mathfrak{a}(N) \cap M\). It is
            
            \begin{align*}
                m \in \Gamma_\mathfrak{a}(N) \cap M & \Rightarrow \mathfrak{a} \subset \sqrt{\mathrm{Ann}(m)} \text{ and } m \in M \text{.} \\
                & \Rightarrow m \in \Gamma_\mathfrak{a}(M) \text{.}
            \end{align*}

            Hence, \(\Gamma_\mathfrak{a}(N) \cap M \subset \Gamma_\mathfrak{a}(M)\).
        \end{proof}
            \item In general, it is \(\Gamma_\mathfrak{a}(\Gamma_\mathfrak{b}(M)) = \Gamma_{\mathfrak{a} + \mathfrak{b}}(M) = \Gamma_\mathfrak{a}(M) \cap \Gamma_\mathfrak{b}(M)\).
            \item In general, it is \(\Gamma_\mathfrak{a}(M) = \Gamma_{\sqrt{\mathfrak{a}}}(M)\).
            \item If \(\mathfrak{a}\) is finitely generated, then
            \begin{align*}
                \Gamma_\mathfrak{a}(M) = \bigcup_{n \geq 1} \makeset{m \in M}{\mathfrak{a}^n m = 0} \text{.}
            \end{align*}
    \end{enumerate}
\end{example}

\begin{example}
    Let \(A\) be a ring, \(M\) a module, \(x \in \mathrm{Rad}(M)\), and \(m \in M\). If \((1 + x)m = 0\), then \(m = 0\).
\end{example}

\begin{proof}
    By definition of radical of a module, it is
    \begin{align*}
        \mathrm{Rad} \left(\sfrac{A}{\mathrm{Ann}(M)}\right) = \sfrac{\mathrm{Rad}(M)}{\mathrm{Ann}(M)} \text{.}
    \end{align*}
    Thus, if \(x \in \mathrm{Rad}(M)\), then its residue \(x^\prime := x + \mathrm{Ann}(M)\) lies in \(\mathrm{Rad}\left(\sfrac{A}{\mathrm{Ann}(M)}\right)\) which means \(x^\prime\) is nilpotent. SOME THEOREM yields \((1 + x^\prime)\) is an unit in \(\sfrac{A}{\mathrm{Ann}(M)}\).
\end{proof}

\chapter{Exact Sequences}

\begin{defn}[Exact Sequence]
    
\end{defn}
\chapter{Tensor Product}

\section{Definition}

\begin{defn}[Bilinear Maps]
    Let \(R\) be a ring, and \(M, N, P\) modules. We call a map
    \begin{equation}
        \alpha: M \times N \longrightarrow P
    \end{equation}
    bilinear if it is linear in each variable. Denote all these maps by \(\mathrm{Bil}_R(M, N; P)\). It is an \(R\)-module with sum and scalar multiplication performed valuewise.
\end{defn}

\begin{defn}[Tensor Product]
    Let \(R\) be a ring, and \(M\) and \(N\) be \(R\)-modules.
\end{defn}

\section{Proofs}

\begin{prps}
    \(\mathrm{Bil}_R(M, N; P)\) is a \(R\)-module.
\end{prps}

\section{Exercises}

\begin{exr}
    Let \(R\) be a ring, and \(X\) and \(Y\) variables, then \(R[X] \otimes R[Y] \simeq R[X, Y]\).
\end{exr}

\begin{proof}
    
\end{proof}
\chapter{Flatness}

\begin{lmm}[9.1]
    Let \(R\) be a ring, \(\alpha: M \longrightarrow N\) a module homomorphism. Then there is a commutative diagram with two short exact sequences involving N
\end{lmm}
\end{document}