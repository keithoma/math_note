\documentclass[a4paper]{book}
\title{Topology}
\author{K}


% ---------------------------------------------------------------------
% P A C K A G E S
% ---------------------------------------------------------------------

% typography and formatting
\usepackage[english]{babel}
\usepackage[utf8]{inputenc}
\usepackage{geometry}
\usepackage{exsheets}
\usepackage{environ}
\usepackage{graphicx}
\usepackage{cutwin}

% mathematics
\usepackage{xfrac}  
\usepackage{amsthm} % for theorems, and definitions
\usepackage{amssymb}
\usepackage{amsmath}
\usepackage{textcomp}
% \usepackage{MnSymbol} % for \cupdot

% extra
\usepackage{xcolor}
\usepackage{tikz}

% ---------------------------------------------------------------------
% S E T T I N G
% ---------------------------------------------------------------------

%maybe delete later, for colorbox
\usepackage{tcolorbox}
\newtcolorbox{defbox}{colback=blue!5!white,colframe=blue!75!black}
\newtcolorbox{defboxlight}{colback=cyan!5!white,colframe=cyan!75!black}
\newtcolorbox{thmbox}{colback=orange!5!white,colframe=orange!75!black}
\newtcolorbox{rembox}{colback=purple!5!white,colframe=purple!75!black}
\newtcolorbox{exmbox}{colback=gray!5!white,colframe=gray!75!black}

% typography and formatting
\geometry{margin=3cm}

\SetupExSheets{
  counter-format = ch.qu,
  counter-within = chapter,
  question/print = true,
  solution/print = true,
}

% mathematics
\newcounter{global}

\theoremstyle{definition}
\newtheorem{definition}{Definition}[]
\newtheorem{example}{Example}[definition]

\newtheorem{theorem}[definition]{Theorem}
\newtheorem{corollary}{Corollary}
\newtheorem{lemma}[definition]{Lemma}
\newtheorem{proposition}[definition]{Proposition}

\newtheorem*{remark}{Remark}

% extra
\definecolor{mathif}{HTML}{0000A0} % for conditions
\definecolor{maththen}{HTML}{CC5500} % for consequences
\definecolor{mathrem}{HTML}{8b008b} % for notes
\definecolor{mathobj}{HTML}{008800}

\usetikzlibrary{positioning}
\usetikzlibrary{shapes.geometric, arrows}

% ---------------------------------------------------------------------
% C O M M A N D S
% ---------------------------------------------------------------------

\newcommand{\norm}[1]{\left\lVert#1\right\rVert}
\newcommand{\rank}{\text{rank}}
\newcommand{\Vol}{\text{Vol}}

\newcommand{\set}[1]{\left\{\, #1 \,\right\}}
\newcommand{\makeset}[2]{\left\{\, #1 \mid #2 \,\right\}}

\newcommand*\diff{\mathop{}\!\mathrm{d}}
\newcommand*\Diff{\mathop{}\!\mathrm{D}}

\newcommand\restr[2]{{% we make the whole thing an ordinary symbol
  \left.\kern-\nulldelimiterspace % automatically resize the bar with \right
  #1 % the function
  \vphantom{\big|} % pretend it's a little taller at normal size
  \right|_{#2} % this is the delimiter
  }}

% ---------------------------------------------------------------------
% R E N D E R
% ---------------------------------------------------------------------

\newif\ifshowproof
\showprooftrue

\NewEnviron{Proof}{%
    \ifshowproof%
        \begin{proof}%
            \BODY
        \end{proof}%
    \fi%
}%

\begin{document}
\maketitle
\tableofcontents
%%%%%%%%%%%%%%%%%%%%%%%%%%%%%%%%%%%%%%%%%%%%%%%%%%%%%%%%%%%%%%%%%%%%%%%%%%%%%%%
\part{Rings}
\chapter{Rings}
\section{Definition and Theorems}
\begin{defbox}
    \begin{definition}[Ring]
        A ring is a set \(A\) equipped with two binary operations \(+\) (addition) and \(\cdot\) (multiplication) satisfying the following three sets of axioms, called the ring axioms.
    \begin{enumerate}
      \item \((A, +)\) is an abelian group.
      \item \((A, \cdot)\) is a semigroup.
      \item Multiplication is distributive with respect to addition, meaning that
      \begin{itemize}
        \item \(a \cdot (b + c) = (a \cdot b) + (a \cdot c)\) for all \(a, b, c \in A\) (left distributivity).
        \item \((b + c) \cdot a = (b \cdot a) + (c \cdot a)\) for all \(a, b, c \in A\) (right distributivity).
      \end{itemize}
    \end{enumerate}
    A ring is called unitary if it contains the multiplicative identity and commutative if multiplication is commutative.
    \end{definition}
\end{defbox}

\chapter{Ideals}

\begin{defbox}
    \begin{definition}[Ideal]
        Let \(A\) be a ring. A subset \(\mathfrak{a} \subset A\) is called an ideal if it satisfies the following two conditions.
        \begin{enumerate}
            \item \((\mathfrak{a}, +)\) is a subgroup of \((A, +)\).
            \item For every \(r \in A\) and every \(x \in \mathfrak{a}\), it is \(rx \in \mathfrak{a}\).
        \end{enumerate}
        \
        Given a subset \(S \subset A\), by the ideal \((S)\) that \(S\) generates, we mean the smallest ideal containing \(S\). If an ideal is generated by a subset \(S \subset A\), then the elements of this subset are called generators.

        \
        An ideal that is generated by a single element is called principal.

        \
        If an ideal \(\mathfrak{a}\) is not the whole ring \(A\), then the ideal is called proper.
    \end{definition}
\end{defbox}

\begin{defbox}
    \begin{definition}[Ideal Operation]
        Let \(\mathfrak{a}\) and \(\mathfrak{b}\) be ideals of a ring \(A\).
        \begin{enumerate}
            \item The sum of two ideals \(\mathfrak{a}\) and \(\mathfrak{b}\) is defined by
            \begin{align*}
                \mathfrak{a} + \mathfrak{b} = \makeset{a + b}{a \in \mathfrak{a} \text{ and } b \in \mathfrak{b}} = (\mathfrak{a}, \mathfrak{b})
            \end{align*}
            which is again an ideal. It is the smallest ideal in \(A\) that contains \(\mathfrak{a}\) and \(\mathfrak{b}\).
            \item The product of an ideal
            \item The intersection of
            \item The radical of an ideal \(\mathfrak{a}\) is defined by
            \begin{align*}
                \sqrt{\mathfrak{a}} = \makeset{x \in A}{x^n \in \mathfrak{a} \text{ for some } n \in \mathbb{N}^+}
            \end{align*}
            which is again an ideal.
            \item The transporter
        \end{enumerate}
    \end{definition}
\end{defbox}

\begin{proof}
    We verify the statements made in the definition.
    \begin{enumerate}
        \item \begin{enumerate}
            \item ``\(\mathfrak{a} + \mathfrak{b}\) is an ideal.'':
        \end{enumerate}
    \end{enumerate}
\end{proof}

\begin{exmbox}
    \begin{example}
        The union of two ideals is \textbf{not} an ideal in general. Consider \((2)\) and \((3)\) in \(\mathbb{Z}\). If \((2) \cup (3)\) was an ideal, then \(3 - 2 = 1\) would be contained in \((2) \cup (3)\). But \(1 \not\in (2)\) and \(1 \not\in (3)\), thus \(1 \not\in (2) \cup (3)\).
    \end{example}
\end{exmbox}

\begin{thmbox}
    \begin{proposition}
        Let \(\mathfrak{a}\) be an ideal of \(A\).
        \begin{enumerate}
            \item \(\mathfrak{a} = A\) if and only if \(1 \in \mathfrak{a}\) if and only if \(\mathfrak{a}\) contains an unit.
            \item \(\mathfrak{a}^2 \subset \mathfrak{a}\).
            \item \(\mathfrak{a} \cdot \mathfrak{b} \subset \mathfrak{a} \cap \mathfrak{b} \subset \mathfrak{a} + \mathfrak{b}\).
            \item \(\mathfrak{a} \subset \mathfrak{a} + \mathfrak{b}\) and \(\mathfrak{b} \subset \mathfrak{a} + \mathfrak{b}\).
        \end{enumerate}
    \end{proposition}
\end{thmbox}

\begin{thmbox}
    \begin{proposition}
        Let \(\mathfrak{a}\) and \(\mathfrak{b}\) be two ideals of a ring \(A\).
        \begin{enumerate}
            \item \(\mathfrak{a} \subset \sqrt{\mathfrak{a}}\).
            \item \(\sqrt{\sqrt{\mathfrak{a}}} = \sqrt{\mathfrak{a}}\).
            \item If \(\mathfrak{a} \subset \mathfrak{b}\), then \(\sqrt{\mathfrak{a}} \subset \sqrt{\mathfrak{b}}\).
            \item \(\sqrt{\mathfrak{a}} = A\) if and only if \(\mathfrak{a} = A\).
            \item \(\sqrt{\mathfrak{a} \cdot \mathfrak{b}} = \sqrt{\mathfrak{a} \cap \mathfrak{b}} = \sqrt{\mathfrak{a}} \cap \sqrt{\mathfrak{b}}\).
            \item \(\sqrt{\mathfrak{a} + \mathfrak{b}} = \sqrt{\sqrt{\mathfrak{a}} + \sqrt{\mathfrak{b}}}\).
            \item If \(\mathfrak{a} = \mathfrak{p}^n\) for some prime ideal \(\mathfrak{p}\) and \(n \in \mathbb{N}^+\), then \(\sqrt{\mathfrak{a}} = \mathfrak{p}\).
        \end{enumerate}
    \end{proposition}
\end{thmbox}

\begin{proof}
    We verify each statement.
    \begin{enumerate}
        \item Let \(x \in \mathfrak{a}\), then trivially, \(x^1 \in \mathfrak{a}\), so \(x \in \sqrt{\mathfrak{a}}\).
        


        \item Since \(\sqrt{\sqrt{\mathfrak{a}}} \supset \sqrt{\mathfrak{a}}\) from above, it suffices to verify the other inclusion. Let \(x \in \sqrt{\sqrt{\mathfrak{a}}}\), then \(x^n \in \sqrt{\mathfrak{a}}\) and in turn, \(\left(x^n\right)^m \in \mathfrak{a}\). Thus, \(x^{nm} \in \mathfrak{a}\), therefore, \(x \in \sqrt{\mathfrak{a}}\).
        
        
        
        \item Suppose \(\mathfrak{a} \subset \mathfrak{b}\) and let \(x \in \sqrt{\mathfrak{a}}\). Then, \(x^n \in \mathfrak{a}\) for some \(n \in \mathbb{N}^+\), thus \(x^n \in \mathfrak{b}\). It follows that \(x \in \sqrt{\mathfrak{b}}\).
        


        \item ``\(\Rightarrow\)'': Let \(\sqrt{\mathfrak{a}} = A\), then for all \(x \in A\), we have that \(x^n \in \mathfrak{a}\) for some \(n \in \mathbb{N}^+\). In particular, \(1^n \in \mathfrak{a}\), but \(1^n = 1\) for all \(n \in \mathbb{N}^+\). Thus, \(\mathfrak{a} = A\).
        
        ``\(\Leftarrow\)'': On the other hand, let \(\mathfrak{a} = A\). In general, it is \(\mathfrak{a} \subset \sqrt{\mathfrak{a}}\), therefore \(A \subset \sqrt{\mathfrak{a}}\) which immediately yields the desired equality \(A = \sqrt{\mathfrak{a}}\).



        \item ``\(\sqrt{\mathfrak{a} \cdot \mathfrak{b}} \subset \sqrt{\mathfrak{a} \cap \mathfrak{b}}\)'': If \(x \in \sqrt{\mathfrak{a} \cdot \mathfrak{b}}\), then \(x^n \in \mathfrak{a} \cdot \mathfrak{b}\) for some \(n \in \mathbb{N}^+\). Since \(\mathfrak{a} \cdot \mathfrak{b} \subset \mathfrak{a} \cap \mathfrak{b}\), we have \(x^n \in \mathfrak{a} \cap \mathfrak{b}\), and it follows that \(x \in \sqrt{\mathfrak{a} \cap \mathfrak{b}}\).
        
        ``\(\sqrt{\mathfrak{a} \cdot \mathfrak{b}} \supset \sqrt{\mathfrak{a} \cap \mathfrak{b}}\)'': Let \(x \in \sqrt{\mathfrak{a} \cap \mathfrak{b}}\), then \(x^n \in \mathfrak{a} \cap \mathfrak{b}\) for some \(n \in \mathbb{N}^+\). Hence it is \(x^n \in \mathfrak{a}\) and \(x^n \in \mathfrak{b}\), therefore \(x^n \cdot x^n = x^{2n} \in \mathfrak{a} \cdot \mathfrak{b}\). Conclude \(x \in \sqrt{\mathfrak{a} \cdot \mathfrak{b}}\).

        ``\(\sqrt{\mathfrak{a} \cap \mathfrak{b}} \subset \sqrt{\mathfrak{a}} \cap \sqrt{\mathfrak{b}}\)'': If \(x \in \sqrt{\mathfrak{a} \cap \mathfrak{b}}\), then \(x^n \in \mathfrak{a} \cap \mathfrak{b}\), thus \(x^n \in \mathfrak{a}\) and \(x^n \in \mathfrak{b}\). We may write \(x \in \sqrt{\mathfrak{a}}\) and \(x \in \sqrt{\mathfrak{b}}\), therefore \(x \in \sqrt{\mathfrak{a}} \cap \sqrt{\mathfrak{b}}\).

        ``\(\sqrt{\mathfrak{a} \cap \mathfrak{b}} \supset \sqrt{\mathfrak{a}} \cap \sqrt{\mathfrak{b}}\)'': Finally, let \(x \in \sqrt{\mathfrak{a}} \cap \sqrt{\mathfrak{b}}\). Then, \(x \sqrt{\mathfrak{a}}\) and \(x \sqrt{\mathfrak{b}}\), so \(x^n \in \mathfrak{a}\) and \(x^m \in \mathfrak{b}\) for some \(n, m \in \mathbb{N}^+\). Say \(n \geq m\), then \(x^n \in \mathfrak{b}\). This yields \(x^n \in \mathfrak{a} \cap \mathfrak{b}\), thus \(x \in \sqrt{\mathfrak{a} \cap \mathfrak{b}}\).



        \item ``\(\sqrt{\mathfrak{a} + \mathfrak{b}} \subset \sqrt{\sqrt{\mathfrak{a}} + \sqrt{\mathfrak{b}}}\)'': Let \(x \in \sqrt{\mathfrak{a} + \mathfrak{b}}\), then \(x^n \in \mathfrak{a} + \mathfrak{b}\) for some \(n \in \mathbb{N}^+\). By definition of sum of ideals, we have that \(x^n = a + b\) for some \(a \in \mathfrak{a}\) and \(b \in \mathfrak{b}\). Since \(\mathfrak{a} \subset \sqrt{\mathfrak{a}}\) and \(\mathfrak{b} \subset \sqrt{\mathfrak{b}}\), we have \(x^n \in \sqrt{\mathfrak{a}} + \sqrt{\mathfrak{b}}\), thus \(x \in \sqrt{\sqrt{\mathfrak{a}} + \sqrt{\mathfrak{b}}}\).
        
        ``\(\sqrt{\mathfrak{a} + \mathfrak{b}} \supset \sqrt{\sqrt{\mathfrak{a}} + \sqrt{\mathfrak{b}}}\)'': Now let \(x \in \sqrt{\sqrt{\mathfrak{a}} + \sqrt{\mathfrak{b}}}\), then \(x^n \in \sqrt{\mathfrak{a}} + \sqrt{\mathfrak{b}}\) for some \(n \in \mathbb{N}^+\). Hence there exists \(a \in \sqrt{\mathfrak{a}}\) and \(b \in \sqrt{\mathfrak{b}}\) such that \(x^n = a + b\). We have that \(a^p \in \mathfrak{a}\) and \(b^q \in \mathfrak{b}\) for some \(p, q \in \mathbb{N}^+\). Consider
        \begin{align*}
            \left(x^n\right)^{(p + q -1)} &= (a + b)^{(p + q -1)} \\ 
            &= \sum_{k=0}^{p + q - 1} \binom{p + q - 1}{k} a^k \cdot b^{p + q - 1 - k} \text{.}
        \end{align*}
        For each \(k \in \set{0, 1, \ldots, p + q -1}\), we have \(a^k \in \mathfrak{a}\) or \(b^{p + q - 1} \in \mathfrak{b}\). Thus, the whole sum lies in \(\mathfrak{a} + \mathfrak{b}\) or in other words \(x^{n(p + q - 1)} \in \mathfrak{a} + \mathfrak{b}\). Conclude \(x \in \sqrt{\mathfrak{a} + \mathfrak{b}}\).



        \item ``\(\sqrt{\mathfrak{a}} \subset \mathfrak{p}\)'': Let \(x \in \sqrt{\mathfrak{a}}\), then \(x^m \in \mathfrak{a}\) for some \(m \in \mathbb{N}^+\). Because \(\mathfrak{a} = \mathfrak{p}^n\), we have \(x^m \in \mathfrak{p}^n\). We also have \(\mathfrak{p}^n \subset \mathfrak{p}\), thus \(x^m \in \mathfrak{p}\) and since \(\mathfrak{p}\) is prime, \(x \in \mathfrak{p}\).
        
        ``\(\sqrt{\mathfrak{a}} \supset \mathfrak{p}\)'': On the other hand, if \(x \in \mathfrak{p}\), then \(x^n \in \mathfrak{p}^n = \mathfrak{a}\), therefore \(x \in \sqrt{\mathfrak{a}}\).
    \end{enumerate}
\end{proof}

\begin{thmbox}
    \begin{proposition}
        \begin{enumerate}
            \item \(\mathfrak{a} \subset (\mathfrak{a} : \mathfrak{b})\).
        \end{enumerate}
    \end{proposition}
\end{thmbox}

\begin{exmbox}
    \begin{example}
        Does \(\sqrt{\mathfrak{a}^2} = \mathfrak{a}\) hold?
    \end{example}
\end{exmbox}

\chapter{Anatomy of Rings}

\begin{defbox}
    \begin{definition}[Nilpotent Element and Nilradical]
        An element \(x\) of a ring \(A\) is called nilpotent if there exists some positive integer \(n \in \mathbb{N}^+\), called the index or the degree, such that \(x^n = 0\).

        The set of all nilpotent elements is called the nilradical of the ring and is denoted by \(\mathrm{Nil}(A)\).
    \end{definition}
\end{defbox}

\begin{defbox}
    \begin{definition}[Reduced Ring]
        A ring \(A\) is called reduced ring if it has no non-zero nilpotent elements.
    \end{definition}
\end{defbox}

\begin{thmbox}
    \begin{proposition}
        Let \(A\) and \(B\) be two rings and \(A^\prime \subset A\) be a subring of \(A\).
        \begin{enumerate}
            \item If \(A\) is reduced, then \(A^\prime\) is also reduced.
            \item If \(A\) and \(B\) are reduced, then \(A \times B\) is also reduced.
            
            (XXX DOES THIS ALSO HOLD FOR ARBITARY MANY PRODUCTS?)
        \end{enumerate}
    \end{proposition}
\end{thmbox}

\section{Exercises and Notes}

\begin{example}
    Let \(K\) be a field and \(A = \sfrac{K[X, Y]}{(X - XY^2, Y^3)}\).
    \begin{enumerate}
        \item Compute the nilradical \(\mathrm{Nil}(A)\).
        \begin{proof}[Solution]
            Denote \((X - XY^2, Y^3) =: \mathfrak{a}\).
            \begin{align*}
                X + \mathfrak{a} &= XY^2 + \mathfrak{a} && \text{because } X - XY^2 \Rightarrow X \sim XY^2 \text{.} \\
                &= XY^2 Y^2 + \mathfrak{a} && \text{because } XY^2 - XY^2Y^2 = Y^2 (X - XY^2) = 0 \Rightarrow XY^2 \sim XY^2Y^2 \\
                &= XY \cdot Y^3 + \mathfrak{a} \\
                &= XY \cdot 0 + \mathfrak{a} \\
                &= 0 + \mathfrak{a} \text{.}
            \end{align*}
            Thus, \(X \in (X - XY^2, Y^3)\). We have therefore the isomorphism \(\sfrac{K[X, Y]}{(X-XY^2, Y^3)} \simeq \sfrac{K[Y]}{(Y^3)}\). [I WANT A ELEGANT REASON FOR THIS. PROBABLY ISOMORPHISM THEOREM.]

            Clearly, \(Y \in \mathrm{Nil}(A)\) or in other words \((Y) \subset \mathrm{Nil}(A)\). But we also have that \(\sfrac{K[Y]}{(Y)} = K\) which is a field, therefore \((Y)\) is a maximal ideal. Because \(1 \not\in \mathrm{Nil}(A)\) conclude \(\mathrm{Nil}(A) = (Y)\).
        \end{proof}
    \end{enumerate}
\end{example}

\chapter{Polynomial Rings}

\chapter{Quotient}

\chapter{Localization}

\section{Definition and Theorems}

\begin{defbox}
    \begin{definition}[Multiplicative Subset]
        A subset \(S\) of a ring \(A\) is called a multiplicative subset if the following conditions hold.
        \begin{enumerate}
            \item \(1 \in S\).
            \item For all \(x, y \in S\) it is \(xy \in S\).
        \end{enumerate}
    \end{definition}
\end{defbox}

\begin{exmbox}
    \begin{example}
        Let \(A\) be a ring. Important examples of a multiplicative subset include the following.
        \begin{enumerate}
            \item The set of units \(A^\times\) is a multiplicative subset.
            \item The set of non-zero-divisors \(A \setminus \mathrm{ZD}(A)\) is a multiplicative subset.
        \end{enumerate}
    \end{example}
\end{exmbox}

\begin{example}
    Let \(A\) be a ring. Other examples of multiplicative subsets are the following.
    \begin{enumerate}
        \item For any element \(x \in A\), the set generated by its power \(\set{1, x, x^2, x^3, \ldots}\) is a multiplicative subset.
        \item For any ideal \(\mathfrak{a} \subset A\), the set \(1 + \mathfrak{a}\) is a multiplicative subset.
    \end{enumerate}
\end{example}

\begin{thmbox}
    \begin{lemma}
        An ideal \(\mathfrak{p}\) of a ring \(A\) is prime if and only if its complement \(A \setminus \mathfrak{p}\) is a multiplicative subset.
    \end{lemma}
\end{thmbox}

\begin{defbox}
    \begin{definition}[Localization]
        \(S^{-1}A\) is again a ring.
    \end{definition}
\end{defbox}

\begin{thmbox}
    \begin{lemma}
        Let \(A\) be a ring and \(S\) a multiplicative subset, then the following are equivalent.
        \begin{enumerate}
            \item \(S^{-1}A = 0\).
            \item \(S\) contains a nilpotent element.
            \item \(0 \in S\).
        \end{enumerate}
    \end{lemma}
\end{thmbox}
\begin{proof}
    ``\(1. \Rightarrow 2.\)'': Let \(S^{-1}A = 0\), then for all \(x \in A\) and \(s \in S\) it is \((x, s) \sim (0, 1)\), thus \(x \cdot u = 0\) for some \(u \in S\). In particular, this holds for \(x = 1\), therefore \(1 \cdot u = 0\). Since a unit can never be a zero divisor, we must have \(u = 0\) which is nilpotent and lies in \(S\).

    ``\(1. \Leftarrow 2.\)'': On the other hand, let \(x \in S\) be nilpotent, i.e. \(x^n = 0\) for some \(n \in \mathbb{N}^+\). Because \(S\) is multiplicatively closed \(x^n = 0\) lies in \(S\). Fix an element \((y, s) \in S^{-1}A\), then \(y \cdot 1 \cdot 0 = 0 \cdot s \cdot 0\). Hence \((y, s) \sim (0, 1)\) and we have \(S^{-1}A = 0\).

    ``\(2. \Rightarrow 3.\)'': Again, let \(x \in S\) be nilpotent, thus \(x^n = 0\) for some \(n \in \mathbb{N}^+\). \(S\) is multiplicatively closed and we have \(x^n = 0 \in S\).

    ``\(2. \Leftarrow 3.\)'': If \(0 \in S\), then \(S\) simply contains a nilpotent element because \(0\) is nilpotent.
\end{proof}


\begin{rembox}
    \begin{remark}
        In the lemma above, the condition \(0 \not\in S\) is required because if \(S\) contains \(0\), then \(S^{-1}A = 0\) and by definition, an integral domain is a nonzero ring.
    \end{remark}
\end{rembox}


\begin{thmbox}
    \begin{proposition}
        Let \(A\) be a ring. \(A\) is reduced if and only if all its localizations \(A_\mathfrak{p}\) at \(\mathfrak{p} \in \mathrm{Spec} \, A\) is reduced.
    \end{proposition}
\end{thmbox}

\begin{proof}
    ``\(\Rightarrow\)'': We prove the statement by contrapositive. Let \(A_\mathfrak{p}\) be not reduced for all \(\mathfrak{p} \in \mathrm{Spec} \, A\). Thus, in all \(A_\mathfrak{p}\), there is an element, say \(x / s\) that is nilpotent and not zero, i.e. \((x / s)^n = 0\) for some \(n \in \mathbb{N}^+\). By the definition of localization, we get \(x^n \cdot u = 0\) for some \(u \in A \setminus \mathfrak{p}\). Now, \(u \in A \setminus \mathfrak{p}\) cannot be zero, because if it was, \(A_\mathfrak{p} = 0\) which is reduced. Thus, \(x\) is nilpotent and \(A\) is not reduced.
\end{proof}








%%%%%%%%%%%%%%%%%%%%%%%%%%%%%
\begin{thmbox}
    \begin{lemma}
        Let \(A\) be a ring and \(S \subset A\) be a multiplicative subset that does not contain \(0\).
        
        \begin{enumerate}
            \item \(A\) is an integral domain if and only if \(S^{-1}A\) is an integral domain.
            \item \(A\) is a unique factorization domain if and only if \(S^{-1}A\) is a unique factorization domain.
        \end{enumerate}
    \end{lemma}
\end{thmbox}



\begin{proof}
    ``\(\Rightarrow\)'': Let \(A\) be an integral domain. Since \(S\) does not contain \(0\), the localization \(S^{-1}A\) is a nonzero ring (see EXAMPLE). Let \((x, s) \in S^{-1}A \setminus \{0\}\) be a nonzero element and suppose there is a \((y, t) \in S^{-1}A\) with \((x, s) \cdot (y, t) = 0\). It is \((xy, st) = (0, 1)\) and thus \(xy \cdot u = 0\) for some \(u \in S\). Because \(x\) was nonzero and \(S\) does not contain \(0\) we must have \(y = 0\). Hence \(S^{-1}A\) is an integral domain.

    ``\(\Leftarrow\)'': On the other hand, let \(S^{-1}A\) be an integral domain. JUST USE THE CANONIC MAPPING \(\varphi_S: A \longrightarrow S^{-1}A\).
\end{proof}

\section{Exercises and Notes}

\begin{example}
    Let \(A_1\) and \(A_2\) be rings. Consider \(A = A_1 \times A_2\) and set \(S := \set{(1, 1), (1, 0)}\). Prove \(A_1 \simeq S^{-1}A\).
\end{example}

\begin{proof}[Solution]
    I don't understand the solution?
\end{proof}


\begin{example}
    Find all intermediate rings \(\mathbb{Z} \subset A \subset \mathbb{Q}\), and describe each \(A\) as a localization of \(\mathbb{Z}\). As a starter, prove \(\mathbb{Z}\left[\frac{2}{3}\right] = S_3^{-1} \mathbb{Z}\) where \(S_3 := \makeset{3^i}{i \in \mathbb{N}^+}\).
\end{example}

\chapter{Hierarchy of Rings}
\section{Definition and Theorems}
\subsection{Integral Domains}



\part{Modules}

\begin{defbox}
    \begin{definition}[Module]
    \end{definition}
\end{defbox}

\begin{exmbox}
    \begin{example}
        \begin{enumerate}
            \item If \(A\) is a field, then an \(A\)-module is a vector space.
            \item A \(\mathbb{Z}\)-module is just an abelian group.
        \end{enumerate}
    \end{example}
\end{exmbox}

\begin{defbox}
    \begin{definition}[Annihilator]
        
    \end{definition}
\end{defbox}

\begin{defbox}
    \begin{definition}[Radical]
        
    \end{definition}
\end{defbox}

\begin{defbox}
    \begin{definition}[Simple Modules]
        Let \(A\) be a ring. A nonzero \(A\)-module \(M\) is called simple if the only submodules are \(\{0\}\) and \(M\) itself.
    \end{definition}
\end{defbox}

\begin{example}
    If \(M\) is a simple \(A\)-module, then any \(f \in \mathrm{Hom}_A (M, M) \setminus \{0\}\) is an isomorphism.
\end{example}

\begin{proof}
    Fix an \(f \in \mathrm{Hom}_A (M, M) \setminus \{0\}\). Since \(\mathrm{ker}(f)\) is a submodule of \(M\), it must be either \(\{0\}\) or whole \(M\). But \(\mathrm{ker}(f) = M\) would mean that \(f = 0\) which was explicitly excluded, thus \(\mathrm{ker}(f) = \{0\}\). By the isomorphism theorem, we also have \(\mathrm{im}(f) \cong \sfrac{M}{\mathrm{ker}(f)} \cong M\). Therefore, \(f\) is bijective.
\end{proof}

\begin{defbox}
    \begin{definition}[Indecomposable]
        Let \(A\) be a ring. A nonzero \(A\)-module \(M\) is called indecomposable if it cannot be written as a direct sum of two non-zero submodules.
    \end{definition}
\end{defbox}

\begin{thmbox}
    \begin{proposition}
        Every simple module is indecomposable.
    \end{proposition}
\end{thmbox}

\begin{exmbox}
    \begin{example}
        Not all indecomposable modules are simple. For example, \(\mathbb{Z}\) is indecomposable, but is not simple.
    \end{example}
\end{exmbox}

\newpage
\section{Exercises and Notes}

\begin{example}
    Let \(f: M \rightarrow N\) be a surjective homomorphism of two finitely generated \(A\)-modules.

    \begin{enumerate}
        \item If \(N \cong A^n\) is a free \(A\)-module, show that \(M \cong \mathrm{ker}(f) \oplus N\).
        
        \begin{proof}
            Since \(N\) is finitely generated, let \((e_1, \ldots, e_n)\) be a set of generators. 
        \end{proof}
    \end{enumerate}
\end{example}

\begin{example}
    Let \(A\) be a ring, \(\mathfrak{a}\) and \(\mathfrak{b}\) ideals, \(M\) and \(N\) \(A\)-modules. Set
    \begin{align*}
        \Gamma_\mathfrak{a}(M) := \makeset{m \in M}{\mathfrak{a} \subset \sqrt{\mathrm{Ann}(m)}} \text{.}
    \end{align*}
    Prove the following statements.
    \begin{enumerate}
        \item If \(\mathfrak{a} \supset \mathfrak{b}\), then \(\Gamma_\mathfrak{a}(M) \subset \Gamma_\mathfrak{b}(M)\).
        
        \begin{proof}
            The proof is a matter of verification. Let \(m \in \Gamma_\mathfrak{a}(M)\). It is
            \begin{align*}
                m \in \Gamma_\mathfrak{a}(M) & \Rightarrow \mathfrak{a} \subset \sqrt{\mathrm{Ann}(m)} \\
                & \Rightarrow \text{For all } a \in \mathfrak{a} \text{ there is a } n \in \mathbb{N}^+ \text{ such that } a^n \in \mathrm{Ann}(m) \text{.}\\
                & \Rightarrow \text{For all } a \in \mathfrak{a} \text{ there is a } n \in \mathbb{N}^+ \text{ such that } a^n \cdot m = 0 \text{.}
                %                
                \intertext{Since \(\mathfrak{a} \supset \mathfrak{b}\), the last statement is true for all \(a \in \mathfrak{b}\). We have}
                %
                & \Rightarrow \text{For all } a \in \mathfrak{b} \text{ there is a } n \in \mathbb{N}^+ \text{ such that } a^n \cdot m = 0 \text{.} \\
                & \Rightarrow \text{For all } a \in \mathfrak{b} \text{ there is a } n \in \mathbb{N}^+ \text{ such that } a^n \in \mathrm{Ann}(m) \text{.}\\
                & \Rightarrow \mathfrak{b} \subset \sqrt{\mathrm{Ann}(m)} \\
                & \Rightarrow m \in \Gamma_\mathfrak{b}(M)
            \end{align*}
            Thus, \(\Gamma_\mathfrak{a}(M) \subset \Gamma_\mathfrak{b}(M)\).
        \end{proof}

        \item If \(M \subset N\), then \(\Gamma_\mathfrak{a}(M) = \Gamma_\mathfrak{a}(N) \cap M\).
        
        \begin{proof}
            Again, the proof is a matter of verification.

            ``\(\subset\)'': \(M \subset N\) implies \(\Gamma_\mathfrak{a}(M) \subset \Gamma_\mathfrak{a}(N)\). Moreover, it is \(\Gamma_\mathfrak{a}(M) \subset M\). Thus, \(\Gamma_\mathfrak{a}(M) \subset \Gamma_\mathfrak{a}(N) \cap M\).

            ``\(\supset\)'': Let \(m \in \Gamma_\mathfrak{a}(N) \cap M\). It is
            
            \begin{align*}
                m \in \Gamma_\mathfrak{a}(N) \cap M & \Rightarrow \mathfrak{a} \subset \sqrt{\mathrm{Ann}(m)} \text{ and } m \in M \text{.} \\
                & \Rightarrow m \in \Gamma_\mathfrak{a}(M) \text{.}
            \end{align*}

            Hence, \(\Gamma_\mathfrak{a}(N) \cap M \subset \Gamma_\mathfrak{a}(M)\).
        \end{proof}
            \item In general, it is \(\Gamma_\mathfrak{a}(\Gamma_\mathfrak{b}(M)) = \Gamma_{\mathfrak{a} + \mathfrak{b}}(M) = \Gamma_\mathfrak{a}(M) \cap \Gamma_\mathfrak{b}(M)\).
            \item In general, it is \(\Gamma_\mathfrak{a}(M) = \Gamma_{\sqrt{\mathfrak{a}}}(M)\).
            \item If \(\mathfrak{a}\) is finitely generated, then
            \begin{align*}
                \Gamma_\mathfrak{a}(M) = \bigcup_{n \geq 1} \makeset{m \in M}{\mathfrak{a}^n m = 0} \text{.}
            \end{align*}
    \end{enumerate}
\end{example}

\begin{example}
    Let \(A\) be a ring, \(M\) a module, \(x \in \mathrm{Rad}(M)\), and \(m \in M\). If \((1 + x)m = 0\), then \(m = 0\).
\end{example}

\begin{proof}
    By definition of radical of a module, it is
    \begin{align*}
        \mathrm{Rad} \left(\sfrac{A}{\mathrm{Ann}(M)}\right) = \sfrac{\mathrm{Rad}(M)}{\mathrm{Ann}(M)} \text{.}
    \end{align*}
    Thus, if \(x \in \mathrm{Rad}(M)\), then its residue \(x^\prime := x + \mathrm{Ann}(M)\) lies in \(\mathrm{Rad}\left(\sfrac{A}{\mathrm{Ann}(M)}\right)\) which means \(x^\prime\) is nilpotent. SOME THEOREM yields \((1 + x^\prime)\) is an unit in \(\sfrac{A}{\mathrm{Ann}(M)}\).
\end{proof}

\chapter{Tensor Product}
\section{Definition and Theorems}

\begin{defbox}
    \begin{definition}
        Let \(M\) and \(N\) be \(A\)-modules. Their tensor product is a pair \((M \otimes_A N, \theta)\) where
        \begin{enumerate}
            \item \(M \otimes_A N\) is an \(A\)-module.
            \item \(\theta: M \times N \rightarrow M \otimes_A N\) is an \(A\)-bilinear mapping.
        \end{enumerate}
        satisfying the universal property, for every pair \((P, \omega)\) of an \(A\)-module and an \(A\)-bilinear mapping \(\omega: M \times N \rightarrow P\), there exists a unique \(A\)-module homomorphism \(f: M \otimes_A N \rightarrow P\) with \(\omega = f \circ \theta\).
    \end{definition}
\end{defbox}

\begin{defbox}
    \begin{definition}
        Let \(M\) and \(N\) be \(A\)-modules. Their tensor product is the pair \((M \otimes_A N, \theta)\), where
        \begin{enumerate}
            \item \(M \otimes_A N\) is the quotient of the free \(A\)-module \(A^{M \times N}\) on the direct product \(M \times N\), by the submodule generated by the set of elements of the form:
            \begin{align*}
                (\lambda m_1 + m_2, n) - \lambda (m_1, n) - (m_2, n) \\
                (m, \lambda n_1 + n_2) - \lambda (m, n_1) - (m, n_2)
            \end{align*}
            for \(m, m_1, m_2 \in M\); \(n, n_1, n_2 \in N\); and \(\lambda \in A\), where we denote \((m, n)\) for its image under the canonical mapping \(M \times N \rightarrow A^{(M \times N)}\).
            \item \(\theta: M \times N \rightarrow M \otimes_A N\) is the composition of the canonical mapping \(M \times N \rightarrow A^{(M \times N)}\) with the quotient module homomorphism \(A^{(M \times N)} \rightarrow M \otimes_A N\).
        \end{enumerate}
    \end{definition}
\end{defbox}

\begin{exmbox}
    \begin{example}
        It is \(\mathbb{Z}/2\mathbb{Z} \otimes \mathbb{Z}/3\mathbb{Z} = 0\).
    \end{example}
\end{exmbox}
\begin{proof}
    Let's show this in multiple concrete ways.
    \newline
    \textbf{Method 1:}
    I want to do this conretely. First, we have
    \begin{align*}
        \mathbb{Z}/2\mathbb{Z} \times \mathbb{Z}/3\mathbb{Z} = \set{(0, 0); (0, 1);, (0, 2); (1, 0); (1, 1); (1, 2)} \text{.}
    \end{align*}
    Thus, the elements of \(\mathbb{Z}^{(\mathbb{Z}/2\mathbb{Z} \times \mathbb{Z}/3\mathbb{Z})}\) are in the form
    \begin{align*}
        (x_{(0, 0)}, x_{(0, 1)}, x_{(0, 2)}, x_{(1, 0)}, x_{(1, 1)}, x_{(1, 2)})
    \end{align*}
    where \(x_{(i, j)} \in \mathbb{Z}\) with \(i \in \{0, 1\}\) and \(j \in \{0, 1, 2\}\).

    Now, we want to find the submodule generated by the rules in the definition.

    \begin{enumerate}
        \item Set \(m_1 = m_2 = n = \lambda = 0\), then
        \begin{align*}
            (0 \cdot 0 + 0, 0) + 0 \cdot (0, 0) - (0, 0) = (0, 0) = 1 \cdot (0, 0) \rightarrow (1, 0, 0, 0, 0, 0) \text{.}
        \end{align*}
        \item Set \(m = n_2 = 0\), \(n_1 = 1\), and \(\lambda = 2\), then
        \begin{align*}
            (0, 2 \cdot 1 + 0) - 2 \cdot (0, 1) - (0, 0) &= (0, 2) - (2 \cdot 0, 1) \\
            &= (0, 2) - (0, 1) \\
            &= (0, 1) \\
            &= 1 \cdot (0, 1) \\
            &\rightarrow (0, 1, 0, 0, 0, 0)
        \end{align*}
        \item I think the rest is clear for now.
    \end{enumerate}
    We may conclude that the submodule generated by the rules defined is the whole module, thus \(\mathbb{Z}/2\mathbb{Z} \otimes \mathbb{Z}/3\mathbb{Z} = 0\).
    %
    \newline
    \textbf{Method 2:}
    https://www.math.brown.edu/reschwar/M153/tensor.pdf
\end{proof}


\begin{thmbox}
    \begin{proposition}
        Let \(A\) be a ring, and \(M, N\) and \(P\) be \(A\)-modules.
        \begin{enumerate}
            \item (identity) \(A \otimes_A M = M\).
            \item (commutative law) \(M \otimes_A N = N \otimes_A M\).
        \end{enumerate}
    \end{proposition}
\end{thmbox}
\begin{proof}
    As in the proposition, let \(A\) be a ring, and \(M, N\) and \(P\) be \(A\)-modules.
    \begin{enumerate}
        \item Define \(\beta: A \times M \rightarrow M\) by \(\beta(x, m) := xm\). Clearly, \(\beta\) is bilinear.
    \end{enumerate}
\end{proof}


\newpage
\section{Exercises and Notes}

\begin{example}
    Let \(A \rightarrow B \rightarrow C\) be ring homomorphisms and \(M\) and \(N\) be \(A\)-modules. Show the following.
    \begin{enumerate}
        \item \((M \otimes_A B) \otimes_B C \cong M \otimes_A C\)
        
        \begin{proof}
            It is
            \begin{align*}
                (M \otimes_A B) \otimes_B C & \cong M \otimes_A (B \otimes_B C) \\
                & \cong M \otimes_A C
            \end{align*}
        \end{proof}

        \item \((M \otimes_A N) \otimes_A B \cong (M \otimes_A B) \otimes_B (N \otimes_A B)\)
        
        \begin{proof}
            trivial
        \end{proof}
    \end{enumerate}
\end{example}

\begin{example}
    Let \(A\) be a ring.
    \begin{enumerate}
        \item If \(M, N\) are \(A\)-modules, then \(\mathrm{Hom}_A(M, N)\) may be viewed as an \(A\)-module via
        \begin{align*}
            a \cdot \varphi := (m \mapsto a \cdot \varphi(m))
        \end{align*}
        for \(a \in A\) and \(\varphi \in \mathrm{Hom}_A(M, N)\).

        \begin{proof}
            this is trivial
        \end{proof}
        \item If \(M, N, L\) are \(A\)-modules, then there exists a natural isomorphism of \(A\)-modules
        \begin{align*}
            \mathrm{Hom}_A(L \otimes_A M, N) \cong \mathrm{Hom}_A(L, \mathrm{Hom}_A(M, N))
        \end{align*}
    \end{enumerate}
\end{example}

\begin{example}
    Let \(A\) be a ring, \(\mathfrak{a}\) an ideal of \(A\), and \(M\) an \(A\)-module.
    \begin{enumerate}
        \item Show that \(M / \mathfrak{a} M \cong M \otimes_A A / \mathfrak{a}\).
        \begin{proof}
            Define \(\varphi: M \otimes_A A / \mathfrak{a} \rightarrow M / \mathfrak{a}M \) by
            \begin{align*}
                m \otimes_A \overline{x} \mapsto x \cdot m + \mathfrak{a}M \text{.}
            \end{align*}
        \end{proof}
    \end{enumerate}
\end{example}

\end{document}