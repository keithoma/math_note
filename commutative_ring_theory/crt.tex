\documentclass[a4paper]{book}
\title{Topology}
\author{K}


% ---------------------------------------------------------------------
% P A C K A G E S
% ---------------------------------------------------------------------

% typography and formatting
\usepackage[english]{babel}
\usepackage[utf8]{inputenc}
\usepackage{geometry}
\usepackage{exsheets}
\usepackage{environ}
\usepackage{graphicx}
\usepackage{cutwin}

% mathematics
\usepackage{xfrac}  
\usepackage{amsthm} % for theorems, and definitions
\usepackage{amssymb}
\usepackage{amsmath}
\usepackage{textcomp}
% \usepackage{MnSymbol} % for \cupdot

% extra
\usepackage{xcolor}
\usepackage{tikz}

% ---------------------------------------------------------------------
% S E T T I N G
% ---------------------------------------------------------------------

%maybe delete later, for colorbox
\usepackage{tcolorbox}
\newtcolorbox{defbox}{colback=blue!5!white,colframe=blue!75!black}
\newtcolorbox{defboxlight}{colback=cyan!5!white,colframe=cyan!75!black}
\newtcolorbox{thmbox}{colback=orange!5!white,colframe=orange!75!black}
\newtcolorbox{rembox}{colback=purple!5!white,colframe=purple!75!black}
\newtcolorbox{exmbox}{colback=gray!5!white,colframe=gray!75!black}

% typography and formatting
\geometry{margin=3cm}

\SetupExSheets{
  counter-format = ch.qu,
  counter-within = chapter,
  question/print = true,
  solution/print = true,
}

% mathematics
\newcounter{global}

\theoremstyle{definition}
\newtheorem{definition}{Definition}[]
\newtheorem{example}{Example}[definition]

\newtheorem{theorem}[definition]{Theorem}
\newtheorem{corollary}{Corollary}
\newtheorem{lemma}[definition]{Lemma}
\newtheorem{proposition}[definition]{Proposition}

\newtheorem*{remark}{Remark}

% extra
\definecolor{mathif}{HTML}{0000A0} % for conditions
\definecolor{maththen}{HTML}{CC5500} % for consequences
\definecolor{mathrem}{HTML}{8b008b} % for notes
\definecolor{mathobj}{HTML}{008800}

\usetikzlibrary{positioning}
\usetikzlibrary{shapes.geometric, arrows}

% ---------------------------------------------------------------------
% C O M M A N D S
% ---------------------------------------------------------------------

\newcommand{\norm}[1]{\left\lVert#1\right\rVert}
\newcommand{\rank}{\text{rank}}
\newcommand{\Vol}{\text{Vol}}

\newcommand{\set}[1]{\left\{\, #1 \,\right\}}
\newcommand{\makeset}[2]{\left\{\, #1 \mid #2 \,\right\}}

\newcommand*\diff{\mathop{}\!\mathrm{d}}
\newcommand*\Diff{\mathop{}\!\mathrm{D}}

\newcommand\restr[2]{{% we make the whole thing an ordinary symbol
  \left.\kern-\nulldelimiterspace % automatically resize the bar with \right
  #1 % the function
  \vphantom{\big|} % pretend it's a little taller at normal size
  \right|_{#2} % this is the delimiter
  }}

% ---------------------------------------------------------------------
% R E N D E R
% ---------------------------------------------------------------------

\newif\ifshowproof
\showprooftrue

\NewEnviron{Proof}{%
    \ifshowproof%
        \begin{proof}%
            \BODY
        \end{proof}%
    \fi%
}%

\begin{document}
\maketitle
\tableofcontents
%%%%%%%%%%%%%%%%%%%%%%%%%%%%%%%%%%%%%%%%%%%%%%%%%%%%%%%%%%%%%%%%%%%%%%%%%%%%%%%
\chapter{Rings}
\section{Definition and Theorems}
\begin{defbox}
    \begin{definition}[Ring]
        A ring is a set \(A\) equipped with two binary operations \(+\) (addition) and \(\cdot\) (multiplication) satisfying the following three sets of axioms, called the ring axioms.
    \begin{enumerate}
      \item \((A, +)\) is an abelian group.
      \item \((A, \cdot)\) is a semigroup.
      \item Multiplication is distributive with respect to addition, meaning that
      \begin{itemize}
        \item \(a \cdot (b + c) = (a \cdot b) + (a \cdot c)\) for all \(a, b, c \in A\) (left distributivity).
        \item \((b + c) \cdot a = (b \cdot a) + (c \cdot a)\) for all \(a, b, c \in A\) (right distributivity).
      \end{itemize}
    \end{enumerate}
    A ring is called unitary if it contains the multiplicative identity and commutative if multiplication is commutative.
    \end{definition}
\end{defbox}

\chapter{Ideals}

\begin{defbox}
    \begin{definition}[Ideal]
        
    \end{definition}
\end{defbox}

\begin{defbox}
    \begin{definition}[Ideal Operation]
        Let \(\mathfrak{a}\) and \(\mathfrak{b}\) be ideals of a ring \(A\).
        \begin{enumerate}
            \item The sum of two ideals \(\mathfrak{a}\) and \(\mathfrak{b}\) is defined by
            \begin{align*}
                \mathfrak{a} + \mathfrak{b} = \makeset{a + b}{a \in \mathfrak{a} \text{ and } b \in \mathfrak{b}}
            \end{align*}
            which is again an ideal. It is the smallest ideal in \(A\) that contains \(\mathfrak{a}\) and \(\mathfrak{b}\).
            \item The product of an ideal
            \item The intersection of
            \item The radical of an ideal \(\mathfrak{a}\) is defined by
            \begin{align*}
                \sqrt{\mathfrak{a}} = \makeset{x \in A}{x^n \in \mathfrak{a} \text{ for some } n \in \mathbb{N}^+}
            \end{align*}
            which is again an ideal.
            \item The transporter
        \end{enumerate}
    \end{definition}
\end{defbox}

\begin{thmbox}
    \begin{proposition}
        Let \(\mathfrak{a}\) and \(\mathfrak{b}\) be two ideals of a ring \(A\).
        \begin{enumerate}
            \item \(\sqrt{\mathfrak{a}} = A\) if and only if \(\mathfrak{a} = A\).
            \item \(\sqrt{\mathfrak{a} \cdot \mathfrak{b}} = \sqrt{\mathfrak{a} \cap \mathfrak{b}} = \sqrt{\mathfrak{a}} \cap \sqrt{\mathfrak{b}}\).
        \end{enumerate}
    \end{proposition}
\end{thmbox}

\begin{proof}
    a.
    \begin{enumerate}
        \item \textbf{``\(\Rightarrow\)'':} Let \(\sqrt{\mathfrak{a}} = A\), then for all \(x \in A\), we have that \(x^n \in \mathfrak{a}\) for some \(n \in \mathbb{N}^+\). In particular, \(1^n \in \mathfrak{a}\), but \(1^n = 1\) for all \(n \in \mathbb{N}^+\). Thus, \(\mathfrak{a} = A\).
        
        \textbf{``\(\Leftarrow\)'':} On the other hand, let \(\mathfrak{a} = A\). In general, it is \(\mathfrak{a} \subset \sqrt{\mathfrak{a}}\), therefore \(A \subset \sqrt{\mathfrak{a}}\) which immediately yields the desired equality \(A = \sqrt{\mathfrak{a}}\).
        \item ``\(\sqrt{\mathfrak{a} \cdot \mathfrak{b}} \subset \sqrt{\mathfrak{a} \cap \mathfrak{b}}\)'': If \(x \in \sqrt{\mathfrak{a} \cdot \mathfrak{b}}\), then \(x^n \in \mathfrak{a} \cdot \mathfrak{b}\) for some \(n \in \mathbb{N}^+\). Since \(\mathfrak{a} \cdot \mathfrak{b} \subset \mathfrak{a} \cap \mathfrak{b}\), we have \(x^n \in \mathfrak{a} \cap \mathfrak{b}\), and it follows that \(x \in \sqrt{\mathfrak{a} \cap \mathfrak{b}}\).
        
        ``\(\sqrt{\mathfrak{a} \cdot \mathfrak{b}} \supset \sqrt{\mathfrak{a} \cap \mathfrak{b}}\)'': Let \(x \in \sqrt{\mathfrak{a} \cap \mathfrak{b}}\), then \(x^n \in \mathfrak{a} \cap \mathfrak{b}\) for some \(n \in \mathbb{N}^+\). Hence it is \(x^n \in \mathfrak{a}\) and \(x^n \in \mathfrak{b}\), therefore \(x^n \cdot x^n = x^{2n} \in \mathfrak{a} \cdot \mathfrak{b}\). Conclude \(x \in \sqrt{\mathfrak{a} \cdot \mathfrak{b}}\).
    \end{enumerate}
\end{proof}

\chapter{Anatomy of Rings}

\begin{defbox}
    \begin{definition}[Nilpotent Element and Nilradical]
        An element \(x\) of a ring \(A\) is called nilpotent if there exists some positive integer \(n \in \mathbb{N}^+\), called the index or the degree, such that \(x^n = 0\).

        The set of all nilpotent elements is called the nilradical of the ring and is denoted by \(\mathrm{Nil}(A)\).
    \end{definition}
\end{defbox}

\section{Exercises and Notes}

\begin{example}
    Let \(K\) be a field and \(A = \sfrac{K[X, Y]}{(X - XY^2, Y^3)}\).
    \begin{enumerate}
        \item Compute the nilradical \(\mathrm{Nil}(A)\).
        \begin{proof}[Solution]
            Denote \((X - XY^2, Y^3) =: \mathfrak{a}\).
            \begin{align*}
                X + \mathfrak{a} &= XY^2 + \mathfrak{a} && \text{because } X - XY^2 \Rightarrow X \sim XY^2 \text{.} \\
                &= XY^2 Y^2 + \mathfrak{a} && \text{because } XY^2 - XY^2Y^2 = Y^2 (X - XY^2) = 0 \Rightarrow XY^2 \sim XY^2Y^2 \\
                &= XY \cdot Y^3 + \mathfrak{a} \\
                &= XY \cdot 0 + \mathfrak{a} \\
                &= 0 + \mathfrak{a} \text{.}
            \end{align*}
            Thus, \(X \in (X - XY^2, Y^3)\). We have therefore the isomorphism \(\sfrac{K[X, Y]}{(X-XY^2, Y^3)} \simeq \sfrac{K[Y]}{(Y^3)}\). [I WANT A ELEGANT REASON FOR THIS. PROBABLY ISOMORPHISM THEOREM.]

            Clearly, \(Y \in \mathrm{Nil}(A)\) or in other words \((Y) \subset \mathrm{Nil}(A)\). But we also have that \(\sfrac{K[Y]}{(Y)} = K\) which is a field, therefore \((Y)\) is a maximal ideal. Because \(1 \not\in \mathrm{Nil}(A)\) conclude \(\mathrm{Nil}(A) = (Y)\).
        \end{proof}
    \end{enumerate}
\end{example}

\chapter{Hierarchy of Rings}



\end{document}