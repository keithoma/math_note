\documentclass[a4paper]{book}
\title{Commutative Ring Theory}
\author{Kei Thoma}


% ---------------------------------------------------------------------
% P A C K A G E S
% ---------------------------------------------------------------------

% typography and formatting
\usepackage[english]{babel}
\usepackage[utf8]{inputenc}
\usepackage{geometry}
\usepackage{exsheets}
\usepackage{environ}
\usepackage{graphicx}
\usepackage{cutwin}
\usepackage{pifont}

% mathematics
\usepackage{xfrac}  
\usepackage{amsthm} % for theorems, and definitions
\usepackage{amssymb}
\usepackage{amsmath}
\usepackage{textcomp}
\usepackage{mathtools}
% \usepackage{MnSymbol} % for \cupdot

% extra
\usepackage{xcolor}
\usepackage{tikz}

% ---------------------------------------------------------------------
% S E T T I N G
% ---------------------------------------------------------------------

%maybe delete later, for colorbox
\usepackage{tcolorbox}
\newtcolorbox{defbox}{colback=blue!5!white,colframe=blue!75!black}
\newtcolorbox{defboxlight}{colback=cyan!5!white,colframe=cyan!75!black}
\newtcolorbox{thmbox}{colback=orange!5!white,colframe=orange!75!black}
\newtcolorbox{rembox}{colback=purple!5!white,colframe=purple!75!black}
\newtcolorbox{exmbox}{colback=gray!5!white,colframe=gray!75!black}
\newtcolorbox{intbox}{colback=violet!5!white,colframe=violet!75!black}

% typography and formatting
\geometry{margin=3cm}

\SetupExSheets{
  counter-format = ch.qu,
  counter-within = chapter,
  question/print = true,
  solution/print = true,
}

% mathematics
\newcounter{global}

\theoremstyle{definition}
\newtheorem{definition}{Definition}[]
\newtheorem{example}{Example}[definition]

\newtheorem{theorem}[definition]{Theorem}
\newtheorem{corollary}{Corollary}
\newtheorem{lemma}[definition]{Lemma}
\newtheorem{proposition}[definition]{Proposition}

\newtheorem*{remark}{Remark}
\newtheorem*{intuition}{Intuition}

% extra
\definecolor{mathif}{HTML}{0000A0} % for conditions
\definecolor{maththen}{HTML}{CC5500} % for consequences
\definecolor{mathrem}{HTML}{8b008b} % for notes
\definecolor{mathobj}{HTML}{008800}

\usetikzlibrary{positioning}
\usetikzlibrary{shapes.geometric, arrows}

% ---------------------------------------------------------------------
% C O M M A N D S
% ---------------------------------------------------------------------

\newcommand{\norm}[1]{\left\lVert#1\right\rVert}
\newcommand{\rank}{\text{rank}}
\newcommand{\Vol}{\text{Vol}}

\newcommand{\set}[1]{\left\{\, #1 \,\right\}}
\newcommand{\makeset}[2]{\left\{\, #1 \mid #2 \,\right\}}

\newcommand*\diff{\mathop{}\!\mathrm{d}}
\newcommand*\Diff{\mathop{}\!\mathrm{D}}

\newcommand\restr[2]{{% we make the whole thing an ordinary symbol
  \left.\kern-\nulldelimiterspace % automatically resize the bar with \right
  #1 % the function
  \vphantom{\big|} % pretend it's a little taller at normal size
  \right|_{#2} % this is the delimiter
  }}

% ---------------------------------------------------------------------
% R E N D E R
% ---------------------------------------------------------------------

\newif\ifshowproof
\showprooftrue

\NewEnviron{Proof}{%
    \ifshowproof%
        \begin{proof}%
            \BODY
        \end{proof}%
    \fi%
}%

\begin{document}
\maketitle
\tableofcontents
%%%%%%%%%%%%%%%%%%%%%%%%%%%%%%%%%%%%%%%%%%%%%%%%%%%%%%%%%%%%%%%%%%%%%%%%%%%%%%%

\newpage
Notes taken from
\begin{itemize}
    \item my courses
    \item Altman, Kleinman: A Term of Commutative Algebra
    \item Wikipedia
    \item Math Stackexchange
\end{itemize}

\chapter{To Do}
\begin{enumerate}
    \item add addendum
    \begin{enumerate}
        \item for abelian group
        \item for semigroup
    \end{enumerate}
    \item after definition of ring, add that identity and inverse are unique as a remark
    \item after ring homomorphism, add general properties of image, kernel etc.
\end{enumerate}

\part{Rings}
\chapter{Rings and Homomorphisms}
\subsection*{Definition and Theorems}
\subsubsection*{Rings}
\begin{defbox}
    \begin{definition}[Ring]
        A ring is a set \(A\) equipped with two binary operations \(+\) (addition) and \(\cdot\) (multiplication) satisfying the following three sets of axioms, called the ring axioms.
    \begin{enumerate}
      \item \((A, +)\) is an abelian group.
      \item \((A, \cdot)\) is a semigroup.
      \item Multiplication is distributive with respect to addition, meaning that
      \begin{itemize}
        \item \(a \cdot (b + c) = (a \cdot b) + (a \cdot c)\) for all \(a, b, c \in A\) (left distributivity).
        \item \((b + c) \cdot a = (b \cdot a) + (c \cdot a)\) for all \(a, b, c \in A\) (right distributivity).
      \end{itemize}
    \end{enumerate}
    A ring is called unitary if it contains the multiplicative identity and commutative if multiplication is commutative.
    \end{definition}
\end{defbox}
%
%
%
\begin{intbox}
    \begin{intuition}
        A ring may be understood as the generalization of the integers.

        Another way to see rings is a less well behaved field where the theory of dividing is due to rings missing the multiplicative identity richer.
    \end{intuition}
\end{intbox}
%
%
%
\begin{rembox}
    \begin{remark}
        In this text, we will primarily be concerned with commutative unitary rings, and thus, for brevity sake, we simply write ``ring'' and mean a commutative unitary ring.
    \end{remark}
\end{rembox}
%
%
%
\begin{exmbox}
    \begin{example}
        Some important examples of rings include the following.
        \begin{enumerate}
            \item The prototypical example is the ring of integers \(\mathbb{Z}\) with the two operations being of addition and multiplication.
            \item Any field is a ring. In particular, the rational numbers \(\mathbb{Q}\), the real numbers \(\mathbb{R}\), and the complex numbers \(\mathbb{C}\) are rings.
            \item The zero ring or trivial ring is the unique ring consisting of one element \(0\) with the operations \(+\) and \(\cdot\) defined such that \(0 + 0 = 0\) and \(0 \cdot 0 = 0\). It is the unique ring in which the additive and the multiplicative identity coincide.
            \item the set of polynomials
            \item an example of a finite ring
            \item If \(S\) is a set, then the power set \(\mathcal{P}(S)\) of \(S\) becomes a ring if we define addition to be the symmetric difference of sets and multiplication to be intersection.
        \end{enumerate}
    \end{example}
\end{exmbox}
%
%
%
\begin{example}
    Moreover, we have some examples of rings that are non-commutative or non-unitary.
    \begin{enumerate}
        \item Matrix ring is non-commutative
    \end{enumerate}
\end{example}
%
%
%
\begin{example}
    Counterexamples of rings include the following.
    \begin{enumerate}
        \item The set of natural numbers \(\mathbb{N}\) with the usual operations is not a ring, since \((\mathbb{N}, +)\) is not even a group.
        \item Trivially, the emptyset regardless of the operations is not a ring.
    \end{enumerate}
\end{example}
%
%
%
\begin{defbox}
    \begin{definition}[Subring]
        A subset \(S\) of \(A\) is called a subring if any of the following equivalent conditions holds.
    \end{definition}
\end{defbox}
%
%
%
\begin{thmbox}
    \begin{proposition}
        Let \(A\) be a ring and \(R\) and \(S\) subrings of \(A\).
        \begin{enumerate}
            \item (ANY?) intersection stable
            \item cartesian product is again a ring
        \end{enumerate}
    \end{proposition}
\end{thmbox}
%
%
%
\begin{exmbox}
    \begin{example}
        \begin{enumerate}
            \item Complement, of course not.
            \item union, of course not.
            \item difference, of course not
            \item symmetric difference, of course not
        \end{enumerate}
    \end{example}
\end{exmbox}

\subsubsection*{Ring Homomorphisms}
\begin{defbox}
    \begin{definition}[Ring Homomorphism]
        A homomorphism from ring \((A, +, \cdot)\) to a ring \((B, \boxplus, \boxdot)\) is a map \(\varphi\) from \(A\) to \(B\) that preserves the ring operations.
    \end{definition}
\end{defbox}
%
%
%
\begin{example}
    examples of ring homomorphism.
\end{example}
%
%
%
\begin{thmbox}
    \begin{proposition}
        Let \(f: A \rightarrow B\) be a ring homomorphism.
        \begin{enumerate}
            \item A ring homomorphism preserves the additive identity, i.e. \(f(0_A) = 0_B\).
        \end{enumerate}
    \end{proposition}
\end{thmbox}
\newpage
\subsection*{Notes}


\chapter{Ideals}

\subsection*{Definition and Theorems}
\subsubsection*{Ideals}
\begin{defbox}
    \begin{definition}[Ideal]
        Let \(A\) be a ring. A subset \(\mathfrak{a} \subset A\) is called an ideal if it satisfies the following two conditions.
        \begin{enumerate}
            \item \((\mathfrak{a}, +)\) is a subgroup of \((A, +)\).
            \item For every \(r \in A\) and every \(x \in \mathfrak{a}\), it is \(rx \in \mathfrak{a}\).
        \end{enumerate}
        \
        Given a subset \(S \subset A\), by the ideal \((S)\) that \(S\) generates, we mean the smallest ideal containing \(S\). If an ideal is generated by a subset \(S \subset A\), then the elements of this subset are called generators.

        \
        An ideal that is generated by a single element is called principal.

        \
        If an ideal \(\mathfrak{a}\) is not the whole ring \(A\), then the ideal is called proper.
    \end{definition}
\end{defbox}

\subsubsection*{Ideal Operations}

\begin{defbox}
    \begin{definition}[Ideal Operations]
        Let \(\mathfrak{a}\) and \(\mathfrak{b}\) be ideals of a ring \(A\).
        \begin{enumerate}
            \item The sum of two ideals \(\mathfrak{a}\) and \(\mathfrak{b}\) is defined by
            \begin{align*}
                \mathfrak{a} + \mathfrak{b} = \makeset{a + b}{a \in \mathfrak{a} \text{ and } b \in \mathfrak{b}} = (\mathfrak{a}, \mathfrak{b})
            \end{align*}
            which is again an ideal. It is the smallest ideal in \(A\) that contains \(\mathfrak{a}\) and \(\mathfrak{b}\).
            \item The product of an ideal
            \item The intersection of
            \item The radical of an ideal \(\mathfrak{a}\) is defined by
            \begin{align*}
                \sqrt{\mathfrak{a}} = \makeset{x \in A}{x^n \in \mathfrak{a} \text{ for some } n \in \mathbb{N}^+}
            \end{align*}
            which is again an ideal.
            \item The transporter
        \end{enumerate}
    \end{definition}
\end{defbox}

\begin{proof}
    We verify the statements made in the definition.
    \begin{enumerate}
        \item \begin{enumerate}
            \item ``\(\mathfrak{a} + \mathfrak{b}\) is an ideal.'':
        \end{enumerate}
    \end{enumerate}
\end{proof}

\begin{exmbox}
    \begin{example}
        The union of two ideals is \textbf{not} an ideal in general. Consider \((2)\) and \((3)\) in \(\mathbb{Z}\). If \((2) \cup (3)\) was an ideal, then \(3 - 2 = 1\) would be contained in \((2) \cup (3)\). But \(1 \not\in (2)\) and \(1 \not\in (3)\), thus \(1 \not\in (2) \cup (3)\).
    \end{example}
\end{exmbox}

\begin{thmbox}
    \begin{proposition}
        Let \(\mathfrak{a}\) be an ideal of \(A\).
        \begin{enumerate}
            \item \(\mathfrak{a} = A\) if and only if \(1 \in \mathfrak{a}\) if and only if \(\mathfrak{a}\) contains an unit.
            \item \(\mathfrak{a}^2 \subset \mathfrak{a}\).
            \item \(\mathfrak{a} \cdot \mathfrak{b} \subset \mathfrak{a} \cap \mathfrak{b} \subset \mathfrak{a} + \mathfrak{b}\).
            \item \(\mathfrak{a} \subset \mathfrak{a} + \mathfrak{b}\) and \(\mathfrak{b} \subset \mathfrak{a} + \mathfrak{b}\).
        \end{enumerate}
    \end{proposition}
\end{thmbox}

\begin{thmbox}
    \begin{proposition}
        Let \(\mathfrak{a}\) and \(\mathfrak{b}\) be two ideals of a ring \(A\).
        \begin{enumerate}
            \item \(\mathfrak{a} \subset \sqrt{\mathfrak{a}}\).
            \item \(\sqrt{\sqrt{\mathfrak{a}}} = \sqrt{\mathfrak{a}}\).
            \item If \(\mathfrak{a} \subset \mathfrak{b}\), then \(\sqrt{\mathfrak{a}} \subset \sqrt{\mathfrak{b}}\).
            \item \(\sqrt{\mathfrak{a}} = A\) if and only if \(\mathfrak{a} = A\).
            \item \(\sqrt{\mathfrak{a} \cdot \mathfrak{b}} = \sqrt{\mathfrak{a} \cap \mathfrak{b}} = \sqrt{\mathfrak{a}} \cap \sqrt{\mathfrak{b}}\).
            \item \(\sqrt{\mathfrak{a} + \mathfrak{b}} = \sqrt{\sqrt{\mathfrak{a}} + \sqrt{\mathfrak{b}}}\).
            \item If \(\mathfrak{a} = \mathfrak{p}^n\) for some prime ideal \(\mathfrak{p}\) and \(n \in \mathbb{N}^+\), then \(\sqrt{\mathfrak{a}} = \mathfrak{p}\).
        \end{enumerate}
    \end{proposition}
\end{thmbox}

\begin{proof}
    We verify each statement.
    \begin{enumerate}
        \item Let \(x \in \mathfrak{a}\), then trivially, \(x^1 \in \mathfrak{a}\), so \(x \in \sqrt{\mathfrak{a}}\).
        


        \item Since \(\sqrt{\sqrt{\mathfrak{a}}} \supset \sqrt{\mathfrak{a}}\) from above, it suffices to verify the other inclusion. Let \(x \in \sqrt{\sqrt{\mathfrak{a}}}\), then \(x^n \in \sqrt{\mathfrak{a}}\) and in turn, \(\left(x^n\right)^m \in \mathfrak{a}\). Thus, \(x^{nm} \in \mathfrak{a}\), therefore, \(x \in \sqrt{\mathfrak{a}}\).
        
        
        
        \item Suppose \(\mathfrak{a} \subset \mathfrak{b}\) and let \(x \in \sqrt{\mathfrak{a}}\). Then, \(x^n \in \mathfrak{a}\) for some \(n \in \mathbb{N}^+\), thus \(x^n \in \mathfrak{b}\). It follows that \(x \in \sqrt{\mathfrak{b}}\).
        


        \item ``\(\Rightarrow\)'': Let \(\sqrt{\mathfrak{a}} = A\), then for all \(x \in A\), we have that \(x^n \in \mathfrak{a}\) for some \(n \in \mathbb{N}^+\). In particular, \(1^n \in \mathfrak{a}\), but \(1^n = 1\) for all \(n \in \mathbb{N}^+\). Thus, \(\mathfrak{a} = A\).
        
        ``\(\Leftarrow\)'': On the other hand, let \(\mathfrak{a} = A\). In general, it is \(\mathfrak{a} \subset \sqrt{\mathfrak{a}}\), therefore \(A \subset \sqrt{\mathfrak{a}}\) which immediately yields the desired equality \(A = \sqrt{\mathfrak{a}}\).



        \item ``\(\sqrt{\mathfrak{a} \cdot \mathfrak{b}} \subset \sqrt{\mathfrak{a} \cap \mathfrak{b}}\)'': If \(x \in \sqrt{\mathfrak{a} \cdot \mathfrak{b}}\), then \(x^n \in \mathfrak{a} \cdot \mathfrak{b}\) for some \(n \in \mathbb{N}^+\). Since \(\mathfrak{a} \cdot \mathfrak{b} \subset \mathfrak{a} \cap \mathfrak{b}\), we have \(x^n \in \mathfrak{a} \cap \mathfrak{b}\), and it follows that \(x \in \sqrt{\mathfrak{a} \cap \mathfrak{b}}\).
        
        ``\(\sqrt{\mathfrak{a} \cdot \mathfrak{b}} \supset \sqrt{\mathfrak{a} \cap \mathfrak{b}}\)'': Let \(x \in \sqrt{\mathfrak{a} \cap \mathfrak{b}}\), then \(x^n \in \mathfrak{a} \cap \mathfrak{b}\) for some \(n \in \mathbb{N}^+\). Hence it is \(x^n \in \mathfrak{a}\) and \(x^n \in \mathfrak{b}\), therefore \(x^n \cdot x^n = x^{2n} \in \mathfrak{a} \cdot \mathfrak{b}\). Conclude \(x \in \sqrt{\mathfrak{a} \cdot \mathfrak{b}}\).

        ``\(\sqrt{\mathfrak{a} \cap \mathfrak{b}} \subset \sqrt{\mathfrak{a}} \cap \sqrt{\mathfrak{b}}\)'': If \(x \in \sqrt{\mathfrak{a} \cap \mathfrak{b}}\), then \(x^n \in \mathfrak{a} \cap \mathfrak{b}\), thus \(x^n \in \mathfrak{a}\) and \(x^n \in \mathfrak{b}\). We may write \(x \in \sqrt{\mathfrak{a}}\) and \(x \in \sqrt{\mathfrak{b}}\), therefore \(x \in \sqrt{\mathfrak{a}} \cap \sqrt{\mathfrak{b}}\).

        ``\(\sqrt{\mathfrak{a} \cap \mathfrak{b}} \supset \sqrt{\mathfrak{a}} \cap \sqrt{\mathfrak{b}}\)'': Finally, let \(x \in \sqrt{\mathfrak{a}} \cap \sqrt{\mathfrak{b}}\). Then, \(x \sqrt{\mathfrak{a}}\) and \(x \sqrt{\mathfrak{b}}\), so \(x^n \in \mathfrak{a}\) and \(x^m \in \mathfrak{b}\) for some \(n, m \in \mathbb{N}^+\). Say \(n \geq m\), then \(x^n \in \mathfrak{b}\). This yields \(x^n \in \mathfrak{a} \cap \mathfrak{b}\), thus \(x \in \sqrt{\mathfrak{a} \cap \mathfrak{b}}\).



        \item ``\(\sqrt{\mathfrak{a} + \mathfrak{b}} \subset \sqrt{\sqrt{\mathfrak{a}} + \sqrt{\mathfrak{b}}}\)'': Let \(x \in \sqrt{\mathfrak{a} + \mathfrak{b}}\), then \(x^n \in \mathfrak{a} + \mathfrak{b}\) for some \(n \in \mathbb{N}^+\). By definition of sum of ideals, we have that \(x^n = a + b\) for some \(a \in \mathfrak{a}\) and \(b \in \mathfrak{b}\). Since \(\mathfrak{a} \subset \sqrt{\mathfrak{a}}\) and \(\mathfrak{b} \subset \sqrt{\mathfrak{b}}\), we have \(x^n \in \sqrt{\mathfrak{a}} + \sqrt{\mathfrak{b}}\), thus \(x \in \sqrt{\sqrt{\mathfrak{a}} + \sqrt{\mathfrak{b}}}\).
        
        ``\(\sqrt{\mathfrak{a} + \mathfrak{b}} \supset \sqrt{\sqrt{\mathfrak{a}} + \sqrt{\mathfrak{b}}}\)'': Now let \(x \in \sqrt{\sqrt{\mathfrak{a}} + \sqrt{\mathfrak{b}}}\), then \(x^n \in \sqrt{\mathfrak{a}} + \sqrt{\mathfrak{b}}\) for some \(n \in \mathbb{N}^+\). Hence there exists \(a \in \sqrt{\mathfrak{a}}\) and \(b \in \sqrt{\mathfrak{b}}\) such that \(x^n = a + b\). We have that \(a^p \in \mathfrak{a}\) and \(b^q \in \mathfrak{b}\) for some \(p, q \in \mathbb{N}^+\). Consider
        \begin{align*}
            \left(x^n\right)^{(p + q -1)} &= (a + b)^{(p + q -1)} \\ 
            &= \sum_{k=0}^{p + q - 1} \binom{p + q - 1}{k} a^k \cdot b^{p + q - 1 - k} \text{.}
        \end{align*}
        For each \(k \in \set{0, 1, \ldots, p + q -1}\), we have \(a^k \in \mathfrak{a}\) or \(b^{p + q - 1} \in \mathfrak{b}\). Thus, the whole sum lies in \(\mathfrak{a} + \mathfrak{b}\) or in other words \(x^{n(p + q - 1)} \in \mathfrak{a} + \mathfrak{b}\). Conclude \(x \in \sqrt{\mathfrak{a} + \mathfrak{b}}\).



        \item ``\(\sqrt{\mathfrak{a}} \subset \mathfrak{p}\)'': Let \(x \in \sqrt{\mathfrak{a}}\), then \(x^m \in \mathfrak{a}\) for some \(m \in \mathbb{N}^+\). Because \(\mathfrak{a} = \mathfrak{p}^n\), we have \(x^m \in \mathfrak{p}^n\). We also have \(\mathfrak{p}^n \subset \mathfrak{p}\), thus \(x^m \in \mathfrak{p}\) and since \(\mathfrak{p}\) is prime, \(x \in \mathfrak{p}\).
        
        ``\(\sqrt{\mathfrak{a}} \supset \mathfrak{p}\)'': On the other hand, if \(x \in \mathfrak{p}\), then \(x^n \in \mathfrak{p}^n = \mathfrak{a}\), therefore \(x \in \sqrt{\mathfrak{a}}\).
    \end{enumerate}
\end{proof}

\begin{thmbox}
    \begin{proposition}
        \begin{enumerate}
            \item \(\mathfrak{a} \subset (\mathfrak{a} : \mathfrak{b})\).
        \end{enumerate}
    \end{proposition}
\end{thmbox}

\begin{exmbox}
    \begin{example}
        Does \(\sqrt{\mathfrak{a}^2} = \mathfrak{a}\) hold?
    \end{example}
\end{exmbox}


\begin{thmbox}
    \begin{proposition}
        Let \(A_1, \ldots, A_n\) be rings for \(n \in \mathbb{N}^+\) and denote \(A := A_1 \times \cdots \times A_n\). The ideals in \(A\) are exactly in the form \(\mathfrak{a}_1 \times \cdots \times \mathfrak{a}_n\) where \(\mathfrak{a}_i\) is an ideal in \(A_i\) for \(1 \leq i \leq n\), i.e.
        \begin{align*}
            \set{\text{ideals in } A} = \prod_{i = 1}^n \set{\text{ideals in } A_i}
        \end{align*}
        Add stuff for spectrums XXX.
    \end{proposition}
\end{thmbox}

\subsubsection*{Prime Ideals}

\begin{defbox}
    \begin{definition}[Prime Ideals]
        
    \end{definition}
\end{defbox}

\begin{exmbox}
    \begin{example}
        \begin{enumerate}
            \item The intersection of two prime ideals are not prime in general. Consider \((2)\) and \((3)\) in the ring \(\mathbb{Z}\), then \((2) \cap (3) = (6)\) is not a prime ideal.
        \end{enumerate}
    \end{example}
\end{exmbox}

\begin{thmbox}
    \begin{lemma}
        An ideal \(\mathfrak{a}\) of a ring \(A\) is prime if and only if \(A / \mathfrak{a}\) is an integral domain.
    \end{lemma}
\end{thmbox}
\begin{proof}
    ``\(\Rightarrow\)'': Let \(\mathfrak{a}\) be a prime ideal and consider two elements \(x + \mathfrak{a}\) and \(y + \mathfrak{a}\). If \((x + \mathfrak{a})(y + \mathfrak{a}) = 0\), then \(xy + \mathfrak{a} = 0\), thus \(xy \in \mathfrak{a}\). Since \(\mathfrak{a}\) was prime, this implies \(x \in \mathfrak{a}\) or \(y \in \mathfrak{a}\). In either case, this means \((x + \mathfrak{a})\) or \(y + \mathfrak{a}\) was already \(0\), and therefore \(A / \mathfrak{a}\) has no nonzero zero divisors which means it is an integral domain.

    ``\(\Leftarrow\)'': 
\end{proof}

\subsubsection*{Maximal Ideals}

\begin{defbox}
    \begin{definition}[Maximal Ideals]
        
    \end{definition}
\end{defbox}

\begin{thmbox}
    \begin{lemma}
        Every non-zero ring has a maximal ideal.
    \end{lemma}
\end{thmbox}
\begin{proof}
    
\end{proof}
\begin{rembox}
    \begin{remark}
        Stated the lemma above differently, for any ring \(A\), it is \(\mathrm{Spm}(A) = \varnothing\) if and only if \(A\) is trivial.
    \end{remark}    
\end{rembox}
\begin{thmbox}
    \begin{corollary}
        Any proper ideal is contained in a maximal ideal.
    \end{corollary}
\end{thmbox}

\begin{thmbox}
    \begin{lemma}
        An ideal \(\mathfrak{a}\) of a ring \(A\) is maximal if and only if \(A / \mathfrak{a}\) is a field.
    \end{lemma}
\end{thmbox}
\begin{proof}
    ``\(\Rightarrow\)'': Let \(\mathfrak{a}\) be a maximal ideal 
\end{proof}

\subsubsection*{Radical Ideals}

\begin{defbox}
    \begin{definition}
        An ideal \(\mathfrak{a}\) is called a radical ideal if it coincides with its radical, i.e. if \(\mathfrak{a} = \sqrt{\mathfrak{a}}\).
    \end{definition}
\end{defbox}

\subsubsection*{Principal Ideals}


\subsubsection*{Move}

\begin{thmbox}
    \begin{proposition}
        In a finite ring, every prime ring is maximal, i.e. if \(A\) is a finite ring, then
        \begin{align*}
            \mathrm{Spec}(A) = \mathrm{Spm}(A) \text{.}
        \end{align*}
    \end{proposition}
\end{thmbox}
\begin{proof}
    
\end{proof}

\chapter{Anatomy of Rings}

\subsubsection*{Zero Divisor}
\begin{defbox}
    \begin{definition}[Zero Divisor]
        An element \(a\) of a ring \(A\) is called a zero divisor if one of the following equivalent conditions hold.
        \begin{enumerate}
            \item There exists a nonzero \(x \in A\) such that \(ax = 0\).
            \item The map \(A \rightarrow A\) that sends \(x\) to \(ax\) is not injective.
        \end{enumerate} 
    \end{definition}
\end{defbox}

\subsubsection*{Group of Units}
\begin{defbox}
    \begin{definition}[Group of Units]
        
    \end{definition}
\end{defbox}

\subsubsection*{Nilpotent Elements}
\begin{defbox}
    \begin{definition}[Nilpotent Element and Nilradical]
        An element \(x\) of a ring \(A\) is called nilpotent if there exists some positive integer \(n \in \mathbb{N}^+\), called the index or the degree, such that \(x^n = 0\).

        The set of all nilpotent elements is called the nilradical of the ring and is denoted by \(\mathrm{Nil}(A)\).
    \end{definition}
\end{defbox}

\begin{defbox}
    \begin{definition}[Reduced Ring]
        A ring \(A\) is called reduced ring if it has no non-zero nilpotent elements.
    \end{definition}
\end{defbox}

\begin{thmbox}
    \begin{proposition}
        Let \(A\) and \(B\) be two rings and \(A^\prime \subset A\) be a subring of \(A\).
        \begin{enumerate}
            \item If \(A\) is reduced, then \(A^\prime\) is also reduced.
            \item If \(A\) and \(B\) are reduced, then \(A \times B\) is also reduced.
            
            (XXX DOES THIS ALSO HOLD FOR ARBITARY MANY PRODUCTS?)
        \end{enumerate}
    \end{proposition}
\end{thmbox}

\subsubsection*{Irreducible and Prime Elements}

\begin{defbox}
    \begin{definition}[Irreducible Element]
        An element \(a\) of an integral domain \(A\) is a nonzero element that is
        \begin{enumerate}
            \item not invertible, i.e. \(a\) is not a unit, and
            \item is not a product of two non-invertible elements.
        \end{enumerate}
        REWRITE THIS DEFINITION
    \end{definition}
\end{defbox}

\begin{defbox}
    \begin{definition}[Prime Element]
        A non-zero non-unit element \(a\) of a ring \(A\) is called prime if whenever \(a \, |  \, bc\) for some \(b\) and \(c\) in \(A\), then it implies \(a \, | \, b\) or \(a \, | \, c\).
    \end{definition}
\end{defbox}

\begin{thmbox}
    \begin{proposition}
        In an integral domain, every prime element is irreducible.
    \end{proposition}
\end{thmbox}

\begin{exmbox}
    \begin{example}
        The converse of the above proposition is not true in general.
    \end{example}
\end{exmbox}

\section{Exercises and Notes}

\begin{example}
    Let \(K\) be a field and \(A = \sfrac{K[X, Y]}{(X - XY^2, Y^3)}\).
    \begin{enumerate}
        \item Compute the nilradical \(\mathrm{Nil}(A)\).
        \begin{proof}[Solution]
            Denote \((X - XY^2, Y^3) =: \mathfrak{a}\).
            \begin{align*}
                X + \mathfrak{a} &= XY^2 + \mathfrak{a} && \text{because } X - XY^2 \Rightarrow X \sim XY^2 \text{.} \\
                &= XY^2 Y^2 + \mathfrak{a} && \text{because } XY^2 - XY^2Y^2 = Y^2 (X - XY^2) = 0 \Rightarrow XY^2 \sim XY^2Y^2 \\
                &= XY \cdot Y^3 + \mathfrak{a} \\
                &= XY \cdot 0 + \mathfrak{a} \\
                &= 0 + \mathfrak{a} \text{.}
            \end{align*}
            Thus, \(X \in (X - XY^2, Y^3)\). We have therefore the isomorphism \(\sfrac{K[X, Y]}{(X-XY^2, Y^3)} \simeq \sfrac{K[Y]}{(Y^3)}\). [I WANT A ELEGANT REASON FOR THIS. PROBABLY ISOMORPHISM THEOREM.]

            Clearly, \(Y \in \mathrm{Nil}(A)\) or in other words \((Y) \subset \mathrm{Nil}(A)\). But we also have that \(\sfrac{K[Y]}{(Y)} = K\) which is a field, therefore \((Y)\) is a maximal ideal. Because \(1 \not\in \mathrm{Nil}(A)\) conclude \(\mathrm{Nil}(A) = (Y)\).
        \end{proof}
    \end{enumerate}
\end{example}

\chapter{Polynomial Rings}

\chapter{Quotient}

\begin{thmbox}
    \begin{lemma}
        We have
        \begin{align*}
            \set{\text{Ideals of } A / \mathfrak{a}} \cong \set{\text{Ideals of } A \text{ that contain } \mathfrak{a}}
        \end{align*}
    \end{lemma}
\end{thmbox}

\chapter{Localization}

\subsection*{Definition and Theorems}
\subsubsection{Multiplicative Subsets}


\begin{defbox}
    \begin{definition}[Multiplicative Subset]
        A subset \(S\) of a ring \(A\) is called a multiplicative subset if the following conditions hold.
        \begin{enumerate}
            \item \(1 \in S\).
            \item For all \(x, y \in S\) it is \(xy \in S\).
        \end{enumerate}
    \end{definition}
\end{defbox}

\begin{exmbox}
    \begin{example}
        Let \(A\) be a ring. Trivially, the following subsets of \(A\).are multiplicative subsets.
        \begin{enumerate}
            \item \(A\) itself is a multiplicative subset.
            \item \(\{1\}\) is a multiplicative subset.
            \item \(\{0, 1\}\) is a multiplicative subset.
        \end{enumerate}
    \end{example}
\end{exmbox}

\begin{exmbox}
    \begin{example}
        Let \(A\) be a ring. Important examples of a multiplicative subset include the following.
        \begin{enumerate}
            \item The set of units \(A^\times\) is a multiplicative subset.
            \item The set of non-zero-divisors \(A \setminus \mathrm{ZD}(A)\) is a multiplicative subset.
        \end{enumerate}
    \end{example}
\end{exmbox}

\begin{proof}
    Let \(A\) be a ring.
    \begin{enumerate}
        \item We show \(A^\times\) is a multiplicative subset. Clearly, \(1\) is a unit and thus lies in \(A^\times\). Let \(x\) and \(y\) be units in \(A\), then there are some \(x^{-1}\) and \(y^{-1}\) in \(A\) with \(x \cdot x^{-1} = 1\) and \(y \cdot y^{-1}\). Then, \(xy \cdot x^{-1} \cdot y^{-1} = x x^{-1} \cdot y y^{-1} = 1\), so \(xy\) is a unit and \(A^\times\) is multiplicatively closed.
    \end{enumerate}
\end{proof}

\begin{example}
    Let \(A\) be a ring. Other examples of multiplicative subsets are the following.
    \begin{enumerate}
        \item Let \(S\) be a multiplicative subset. Then, \(S \cup \{0\}\) is also multiplicative subset.
        \item For any element \(x \in A\), the set generated by its power \(\set{1, x, x^2, x^3, \ldots}\) is a multiplicative subset.
        \item For any ideal \(\mathfrak{a} \subset A\), the set \(1 + \mathfrak{a}\) is a multiplicative subset.
    \end{enumerate}
\end{example}

\begin{thmbox}
    \begin{lemma}
        An ideal \(\mathfrak{p}\) of a ring \(A\) is prime if and only if its complement \(A \setminus \mathfrak{p}\) is a multiplicative subset.
    \end{lemma}
\end{thmbox}
\begin{proof}
    Let \(A\) be a ring and \(\mathfrak{p}\) be an ideal in \(A\).

    ``\(\Rightarrow\)'': Suppose \(\mathfrak{p}\) is prime. By definition, \(1 \not\in \mathfrak{p}\), hence \(1\) lies in the complement \(A \setminus \mathfrak{p}\). Now let \(x,y \in A \setminus \mathfrak{p}\) and assume \(xy \not\in A \setminus \mathfrak{p}\). In this case, \(xy \in \mathfrak{p}\) and since \(\mathfrak{p}\) is prime, we must have \(x \in \mathfrak{p}\) or \(x \in \mathfrak{p}\) both of which are contradictions.

    ``\(\Leftarrow\)'': On the other hand, let \(A \setminus \mathfrak{p}\) be a multiplicative subset. Fix a \(xy \in \mathfrak{p}\) and assume \(x, y \not\in \mathfrak{p}\). We have that \(x, y \in A \setminus \mathfrak{p}\) and since \(A \setminus \mathfrak{p}\) is a multiplicative subset, it is \(xy \in A \setminus \mathfrak{p}\). This implies \(xy \not\in \mathfrak{p}\) which is a contradiction.
\end{proof}

\begin{rembox}
    \begin{remark}
        The lemma does not imply that any complement of a multiplicative subset is a prime ideal. Only if the complement of a multiplicative subset is already an ideal it is prime. Thus, constructing multiplicative subsets through complements of primitive ideals are not exhaustive.
    \end{remark}
\end{rembox}
\begin{exmbox}
    \begin{example}
        Consider \(\mathbb{Z}\) and the multiplicative subset \(\{1\}\). The complement \(\mathbb{Z} \setminus \{1\}\) is not an ideal.
    \end{example}
\end{exmbox}

\begin{thmbox}
    \begin{proposition}
        intersection is again multiplicative

        cartesian product?
    \end{proposition}
\end{thmbox}
\begin{exmbox}
    \begin{example}
        subsets?
        unions
        symmetric difference
    \end{example}
\end{exmbox}

\subsubsection{Localization}
\begin{defbox}
    \begin{definition}[Localization]
        \(S^{-1}A\) is again a ring.
    \end{definition}
\end{defbox}

\begin{thmbox}
    \begin{lemma}[Universal Property of Localization]
        Let \(A\) and \(B\) be two rings, \(S\) a multiplicative subset of \(A\), and \(f: A \rightarrow B\) a ring homomorphism that maps every element of \(S\) to a unit in \(B\). In this case, there exists a unique ring homomorphism \(g: S^{-1}A \rightarrow B\) such that \(f = g \circ \varphi\).
    \end{lemma}
\end{thmbox}

\begin{thmbox}
    \begin{lemma}
        Let \(A\) be a ring and \(S\) a multiplicative subset, then the following are equivalent.
        \begin{enumerate}
            \item \(S^{-1}A = 0\).
            \item \(S\) contains a nilpotent element.
            \item \(0 \in S\).
        \end{enumerate}
    \end{lemma}
\end{thmbox}
\begin{proof}
    ``\(1. \Rightarrow 2.\)'': Let \(S^{-1}A = 0\), then for all \(x \in A\) and \(s \in S\) it is \((x, s) \sim (0, 1)\), thus \(x \cdot u = 0\) for some \(u \in S\). In particular, this holds for \(x = 1\), therefore \(1 \cdot u = 0\). Since a unit can never be a zero divisor, we must have \(u = 0\) which is nilpotent and lies in \(S\).

    ``\(1. \Leftarrow 2.\)'': On the other hand, let \(x \in S\) be nilpotent, i.e. \(x^n = 0\) for some \(n \in \mathbb{N}^+\). Because \(S\) is multiplicatively closed \(x^n = 0\) lies in \(S\). Fix an element \((y, s) \in S^{-1}A\), then \(y \cdot 1 \cdot 0 = 0 \cdot s \cdot 0\). Hence \((y, s) \sim (0, 1)\) and we have \(S^{-1}A = 0\).

    ``\(2. \Rightarrow 3.\)'': Again, let \(x \in S\) be nilpotent, thus \(x^n = 0\) for some \(n \in \mathbb{N}^+\). \(S\) is multiplicatively closed and we have \(x^n = 0 \in S\).

    ``\(2. \Leftarrow 3.\)'': If \(0 \in S\), then \(S\) simply contains a nilpotent element because \(0\) is nilpotent.
\end{proof}



\begin{example}
    Some concrete examples of localization include the following.
    \begin{enumerate}
        \item 
    \end{enumerate}
\end{example}


\begin{thmbox}
    \begin{proposition}
        Let \(A\) be a ring. \(A\) is reduced if and only if all its localizations \(A_\mathfrak{p}\) at \(\mathfrak{p} \in \mathrm{Spec} \, A\) is reduced.
    \end{proposition}
\end{thmbox}

\begin{proof}
    ``\(\Rightarrow\)'': We prove the statement by contrapositive. Let \(A_\mathfrak{p}\) be not reduced for all \(\mathfrak{p} \in \mathrm{Spec} \, A\). Thus, in all \(A_\mathfrak{p}\), there is an element, say \(x / s\) that is nilpotent and not zero, i.e. \((x / s)^n = 0\) for some \(n \in \mathbb{N}^+\). By the definition of localization, we get \(x^n \cdot u = 0\) for some \(u \in A \setminus \mathfrak{p}\). Now, \(u \in A \setminus \mathfrak{p}\) cannot be zero, because if it was, \(A_\mathfrak{p} = 0\) which is reduced. Thus, \(x\) is nilpotent and \(A\) is not reduced.
\end{proof}








%%%%%%%%%%%%%%%%%%%%%%%%%%%%%
\subsubsection*{Interactions}
\begin{thmbox}
    \begin{proposition}
        Let \(A\) be a ring and \(S \subset A\) be a multiplicative subset that does not contain \(0\).
        
        \begin{enumerate}
            \item \(A\) is an integral domain if and only if \(S^{-1}A\) is an integral domain.
            \item \(A\) is a unique factorization domain if and only if \(S^{-1}A\) is a unique factorization domain.
        \end{enumerate}
    \end{proposition}
\end{thmbox}



\begin{proof}
    ``\(\Rightarrow\)'': Let \(A\) be an integral domain. Since \(S\) does not contain \(0\), the localization \(S^{-1}A\) is a nonzero ring (see EXAMPLE). Let \((x, s) \in S^{-1}A \setminus \{0\}\) be a nonzero element and suppose there is a \((y, t) \in S^{-1}A\) with \((x, s) \cdot (y, t) = 0\). It is \((xy, st) = (0, 1)\) and thus \(xy \cdot u = 0\) for some \(u \in S\). Because \(x\) was nonzero and \(S\) does not contain \(0\) we must have \(y = 0\). Hence \(S^{-1}A\) is an integral domain.

    ``\(\Leftarrow\)'': On the other hand, let \(S^{-1}A\) be an integral domain. JUST USE THE CANONIC MAPPING \(\varphi_S: A \longrightarrow S^{-1}A\).
\end{proof}

\begin{rembox}
    \begin{remark}
        In the lemma above, the condition \(0 \not\in S\) is required because if \(S\) contains \(0\), then \(S^{-1}A = 0\) and by definition, an integral domain is a nonzero ring.
    \end{remark}
\end{rembox}

\begin{thmbox}
    \begin{proposition}
        Let \(A\) be a ring, \(S\) a multiplicative subset, and \(\mathfrak{a}_1, \ldots, \mathfrak{a}_n\) for \(n \in \mathbb{N}^+\) ideals in \(A\). It is
        \begin{align*}
            \left(\bigcap_{i=1}^n \mathfrak{a}_i\right) A_S &= \left(\bigcap_{i=1}^n \mathfrak{a}_i A_S\right)
            \intertext{or written differently}
            \left(\mathfrak{a}_1 \cap \cdots \cap \mathfrak{a}_n\right) A_S &= \mathfrak{a}_1 A_S \cap \cdots \cap \mathfrak{a}_n A_S \text{.}
        \end{align*}
    \end{proposition}
\end{thmbox}
\begin{proof}
    By induction, we reduce the case to \(n = 2\), that is, we want to show \((\mathfrak{a}_1 \cap \mathfrak{a}_2) A_S = \mathfrak{a}_1 A_S \cap \mathfrak{a}_2 A_S\). The inclusion \((\mathfrak{a}_1 \cap \mathfrak{a}_2) \xhookrightarrow{} \mathfrak{a}_1\) induces a natural inclusion \((\mathfrak{a}_1 \cap \mathfrak{a}_2) A_S \xhookrightarrow{} \mathfrak{a}_1 A_S\) which can be extended to a injective map \(f: (\mathfrak{a}_1 \cap \mathfrak{a}_2)A_S \rightarrow \mathfrak{a}_1 A_S \cap \mathfrak{a}_2 A_S\). It suffies to show \(f\) is surjective. Let \(y \in \mathfrak{a}_1 A_S \cap \mathfrak{a}_2 A_S\). We have
    \begin{align*}
        y = \frac{a_1}{s} = \frac{a_2}{t}
    \end{align*}
    with \(a_1 \in \mathfrak{a}_2\), \(a_2 \in \mathfrak{a}_2\), and \(s, t \in S\). Thus it is \(a_1 t u = a_2 s u\) for some \(u \in S\). Since \(a_1\) lies in \(\mathfrak{a}_1\), we have \(a_1 t u \in \mathfrak{a}_1\), and similary \(a_2 s u \in \mathfrak{a}_2\), hence \(a_1 t u \in \mathfrak{a}_1 \cap \mathfrak{a}_2\). But \(t\) and \(u\) are invertible in \(A_S\), therefore
    \begin{align*}
        \frac{a_1}{s} = \frac{a_1 t u}{s t u} \in (\mathfrak{a}_1 \cap \mathfrak{a}_2) A_S
    \end{align*}
    thus \(f\) is surjective.
\end{proof}

\begin{exmbox}
    \begin{example}
        Consider \(\mathbb{Q}[X]\)
    \end{example}
\end{exmbox}

\newpage
\subsection*{Exercises and Notes}

\begin{example}
    Let \(A_1\) and \(A_2\) be rings. Consider \(A = A_1 \times A_2\) and set \(S := \set{(1, 1), (1, 0)}\). Prove \(A_1 \simeq S^{-1}A\).
\end{example}

\begin{proof}[Solution]
    I don't understand the solution?
\end{proof}


\begin{example}
    Find all intermediate rings \(\mathbb{Z} \subset A \subset \mathbb{Q}\), and describe each \(A\) as a localization of \(\mathbb{Z}\). As a starter, prove \(\mathbb{Z}\left[\frac{2}{3}\right] = S_3^{-1} \mathbb{Z}\) where \(S_3 := \makeset{3^i}{i \in \mathbb{N}^+}\).
\end{example}

\chapter{Hierarchy of Rings}

\newpage
\section{Integral Domains}
\subsection*{Definitions and Theorems}
\begin{defbox}
    \begin{definition}[Integral Domains]
        An integral domain \(A\) is a nonzero ring satisfying the following equivalent conditions.
        \begin{enumerate}
            \item The product of two nonzero elements is nonzero, i.e. for all \(a\) and \(b\) in \(A\) it is \(ab \neq 0\).
            \item The zero ideal \((0)\) is a prime ideal.
            \item Every nonzero element is cancellable under multiplication, i.e. \(ab = ac\) implies \(b = c\).
        \end{enumerate}
    \end{definition}
\end{defbox}

\begin{thmbox}
    \begin{lemma}
        Let \(A\) be a ring and \(\mathfrak{p}\) an ideal. Then, \(\mathfrak{p}\) is a prime ideal if and only if \(A / \mathfrak{p}\) is an integral domain.
    \end{lemma}
\end{thmbox}


\subsection*{Notes}

\newpage
\section{Unique Factorization Domains}
\subsection*{Definitions and Theorems}
\subsection*{Notes}

\newpage
\section{Principal Ideal Domains}
\subsection*{Definitions and Theorems}

\begin{defbox}
    \begin{definition}[Principal Ideal Domains]
        A principal ideal domain is an integral domain in which every ideal is principal.
    \end{definition}
\end{defbox}

\begin{thmbox}
    \begin{lemma}
        Let \(A\) be a principal ideal domain and \(\mathfrak{a}\) be an ideal in \(A\). The quotient \(A/\mathfrak{a}\) is a principal ideal ring.
    \end{lemma}
\end{thmbox}

\begin{rembox}
    \begin{remark}
        In the above lemma, the quotient \(A / \mathfrak{a}\) need not be an principal ideal domain because \(A / \mathfrak{a}\) is not even be an integral domain if \(\mathfrak{a}\) is not a prime ideal.
    \end{remark}
\end{rembox}

\begin{exmbox}
    \begin{example}
        \(\mathbb{Z}/6\mathbb{Z}\) is a principal ideal ring, but not a principal ideal domain.
    \end{example}
\end{exmbox}

\subsection*{Notes}

\newpage
\section{Euclidean Domains}
\subsection*{Definitions and Theorems}
\subsection*{Notes}




\chapter{Classification of Rings}
\section{Definition and Theorems}
\subsection{Noetherian Ring}

\begin{thmbox}
    \begin{lemma}
        All principal ideal domains are Noetherian.
    \end{lemma}
\end{thmbox}

\begin{rembox}
    \begin{remark}
        By the lemma above, it follows that any
        \begin{enumerate}
            \item Euclidean domains
            \item fields
        \end{enumerate}
        are Noetherian.
    \end{remark}
\end{rembox}

\begin{example}
    
\end{example}

\begin{example}
    
\end{example}

\begin{thmbox}
    \begin{theorem}[Hilbert's Basis Theorem]
        If \(A\) is a Noetherian ring, then the polynomial ring with finitely many variables \(A[X_1, \ldots, X_n]\) is Noetherian. In particular, if \(A\) is Noetherian, so is \(A[X]\).
    \end{theorem}
\end{thmbox}

\begin{thmbox}
    \begin{corollary}
        If \(A\) is Noetherian, the power series ring \(A[[X]]\) is Noetherian.
    \end{corollary}
\end{thmbox}

\begin{rembox}
    \begin{remark}
        The polynomial ring with infinitely many variables \(A[X_1, X_2, \ldots]\) is never Noetherian.
    \end{remark}
\end{rembox}

\newpage
\section{Artinian Rings}

\subsection*{Definition and Theorems}

\begin{defbox}
    \begin{definition}[Artinian Rings]
        
    \end{definition}
\end{defbox}

\begin{exmbox}
    \begin{example}
        \begin{enumerate}
            \item Any field is Artinian.
            \item Any finite ring is Artinian.
        \end{enumerate}
    \end{example}
\end{exmbox}

\begin{thmbox}
    \begin{proposition}
        \begin{enumerate}
            \item A quotient of an Artinian ring is Artinian.
            \item A localization of an Artinian ring is Artinian.
        \end{enumerate}
    \end{proposition}
\end{thmbox}

\begin{thmbox}
    \begin{lemma}
        An integral domain is Artinian if and only if it is a field.
    \end{lemma}
\end{thmbox}
\begin{proof}
    Let \(A\) be an integral domain.

    ``\(\Rightarrow\)'': Since \(A\) is an Artinian, the descending chain
    \begin{align*}
        (x) \supset (x^2) \supset \cdots \supset (x^n) \supset (x^{n+1}) \supset \cdots
    \end{align*}
    becomes stationary, that is \((x^n) = (x^{n+1})\) for some \(n \in \mathbb{N}^+\). It follows that there is a \(b \in A\) such that \(x^n = b x^{n+1}\). We have
    \begin{align*}
        x^n = b x^{n+1} &\iff 0 = b x^{n+1} - x^{n}\\
        &\iff 0 = bx^n (x - 1)
    \end{align*}
    Since \(A\) is an integral domain, \(bx^n\) cannot be zero,thus \(x - 1 = 0\) or in other words \(x\) is a unit. Hence \(A\) is a field.

    ``\(\Leftarrow\)'': All fields are already Artinian.
\end{proof}


\begin{thmbox}
    \begin{proposition}
        Let \(A\) be an Artinian ring. Then, we have the following
        \begin{enumerate}
            \item The spectrum \(\mathrm{Spec}(A)\) of \(A\) and the maximal spectrum \(\mathrm{Spm}(A)\) of \(A\) are both finite.
            \item It is \(\mathrm{Spec}(A) = \mathrm{Spm}(A)\).
            \item For some \(n \in \mathbb{N}^+\), it is \((\mathrm{Jac}(A))^n = 0\).
            \item There are maximal ideals \(\mathfrak{m}_1, \ldots, \mathfrak{m}_n\) in \(\mathrm{Spm}(A)\) such that \(\prod_{i = 1}^n \mathfrak{m}_i = 0\).
            \item \(A\) is Noetherian.
            \item \(A\) has finite rank.
        \end{enumerate}
    \end{proposition}
\end{thmbox}

\begin{proof}
    \begin{enumerate}
        \item Let \((\mathfrak{m}_k)_{i \in \mathbb{N}^+}\) be a sequence of maximal ideals and set
        \begin{align*}
            I_k = \prod_{i = 1}^k \mathfrak{m}_i \text{.}
        \end{align*}
        Since \(A\) is Artinian, the chain \(I_0 \supset I_1 \supset \cdots \supset I_k \supset I_{k+1} \supset \cdots\) becomes stationary. Hence \(I_k = I_{k + 1}\) for some \(k \in \mathbb{N}^+\).
        \item Since \(\mathrm{Spec}(A) \supset \mathrm{Spm}(A)\) is immediately clear, we show the other direction of the inclusion. Let \(\mathfrak{p}\) be a prime ideal and consider \(A/\mathfrak{p}\). It is an integral domain because \(\mathfrak{p}\) is a prime ideal and it is also Artinian because a quotient of an Artinian ring is Artinian. Therefore, \(A/\mathfrak{p}\) is a field, hence \(\mathfrak{p}\) is a maximal ideal.
    \end{enumerate}
\end{proof}

\begin{thmbox}
    \begin{lemma}
        A ring is Artinian if and only if it is Noetherian and \(\mathrm{Spec}(A) = \mathrm{Spm}(A)\).
    \end{lemma}
\end{thmbox}

\begin{thmbox}
    \begin{theorem}
        
    \end{theorem}
\end{thmbox}

\subsection*{Exercise and Notes}

\begin{example}
    Given a prime \(p \in \mathbb{Z}\), find all Artinian rings \(A\) with \(p^2\) elements (up to isomorphisms).
\end{example}

\begin{proof}
    Let \(A\) be an Artinian ring with \(p^2\) elements where \(p \in \mathbb{Z}\) is prime. By the structure theorem of Artinian rings, we have that \(A\) is a product of local Artinian rings. Since \(p^2\) has two prime factors, this product can involve at most two factors. Thus, we have two cases.

    \textbf{Case 1:} In this case, \(A = A_1 \times A_2\) for two local Artinian rings \(A_1\) and \(A_2\) with both having exactly \(p\) elements. A ring with \(p\) elements is isomorphic to \(\mathbb{F}_p\). We may conclude \(A = \mathbb{F}_p \times \mathbb{F}_p\).

    \textbf{Case 2:} If \(A\) has only one factor, \(A\) must be a local ring, i.e. it has a unique maximal ideal \(\mathfrak{m}\) with \(\mathfrak{m}^n = 0\) for some \(\mathbb{N}^+\). Choose such \(n\) to be minimal and consider the chain \(R \supset \mathfrak{m} \supset \mathfrak{m}^2 \supset 0\). Taking the quotient at each step we obtain 
\end{proof}

\chapter{Zariski Topology}

\begin{thmbox}
    \begin{lemma}
        Let \(A\) be a ring, then the following are equivalent.
        \begin{enumerate}
            \item \(\mathrm{Spec}(A)\) is irreducible.
            \item \(\mathrm{Nil}(A)\) is prime.
            \item \(A / \mathrm{Nil}(A)\) is an integral domain.
        \end{enumerate}
        In this case, \(\mathrm{Spec}(A)\) is connected.
    \end{lemma}
\end{thmbox}

\chapter{Summary}

Here, I want to summarize the interactions of

structures
\begin{enumerate}
    \item integral domains
    \item 
\end{enumerate}

\begin{center}
    \begin{tabular}{ | c | c | c | c | }
        \hline
        & Integral Domain & Unique Factorization Domain & Principal Ideal Domain \\
        \hline
        Polynomial Ring & \(\iff\) & \(\Downarrow \quad \not\Rightarrow\) & \(\not\Downarrow\) \\
        \hline
        Localization & \(\Downarrow\) & &
    \end{tabular}
\end{center}

\part{Modules}
\chapter{Modules}
\subsection*{Definition and Theorems}
\subsubsection*{Introduction}
\begin{defbox}
    \begin{definition}[Module]
    \end{definition}
\end{defbox}

\begin{exmbox}
    \begin{example}
        \begin{enumerate}
            \item If \(A\) is a field, then an \(A\)-module is a vector space.
            \item A \(\mathbb{Z}\)-module is just an abelian group.
        \end{enumerate}
    \end{example}
\end{exmbox}

\begin{defbox}
    \begin{definition}[Submodules]
        Let \(M\) be an \(A\)-module. A subset \(N\) of \(M\) is called a submodule if \((N, +)\) is a subgroup of \(M\) and for all \(n \in N\) and for all \(a \in A\) it is \(a \cdot n \in N\).
    \end{definition}
\end{defbox}


\begin{thmbox}
    \begin{proposition}
        Let \(A\) be a ring. If \(A\) is viewed as a module over itself, then its submodules are exactly its ideals, i.e.
        \begin{align*}
            \makeset{N}{N \text{ is a submodule of } A} = \makeset{\mathfrak{a}}{\mathfrak{a} \text{ is an ideal of } A}\text{.}
        \end{align*}
    \end{proposition}
\end{thmbox}


\begin{defbox}
    \begin{definition}[Homomorphism of Modules]
        
    \end{definition}
\end{defbox}

\begin{thmbox}
    \begin{proposition}
        Let \(M\) and \(N\) be an \(A\)-module, and \(\varphi: M \rightarrow N\) be an \(A\)-module homomorphism.
        \begin{enumerate}
            \item \(\mathrm{im}(\varphi)\) is a submodule of \(M\).
            \item \(\mathrm{ker}(\varphi)\) is a submodule of \(N\).
            \item For any submodule \(N^\prime\) of \(N\), its preimage \(\varphi^{-1}(N^\prime)\) is a submodule of \(M\).
        \end{enumerate}
    \end{proposition}
\end{thmbox}


\subsubsection*{Free and Finitely Generated}

\begin{defbox}
    \begin{definition}
        An \(A\)-module is finitely generated if there exists a finite set \(\{m_1, \ldots, m_n\}\) with \(n \in \mathbb{N}^+\) in \(M\) such that for any \(x\) in \(M\), there exits \(\lambda_1, \ldots, \lambda_n\) in \(A\) with
        \begin{align*}
            x = \lambda_1 m_1 + \cdots + \lambda_n m_n
        \end{align*}
    \end{definition}
\end{defbox}

\begin{thmbox}
    \begin{lemma}
        An \(A\)-module is finitely generated if and only if there exists a surjective \(A\)-module homomorphism
        \begin{align*}
            A^n \longrightarrow M
        \end{align*}
        for some \(n \in \mathbb{N}^+\).
    \end{lemma}
\end{thmbox}

\begin{defbox}
    \begin{definition}
        Let \(M\) be an \(A\)-module. A set \(B \subset M\) is a basis of \(M\) if
        \begin{enumerate}
            \item \(B\) is a generating set for \(M\)
            \item \(B\) is linearly independent
        \end{enumerate}
        A free module is a module with a basis.
    \end{definition}
\end{defbox}

\begin{rembox}
    \begin{remark}
        An \(A\)-module being free does \textbf{not} imply the module being finitely generated. Similary, an \(A\)-module being finitely generated does \textbf{not} imply the module being free.
    \end{remark}
\end{rembox}

\begin{exmbox}
    \begin{example}
        Two examples to illustrate the remark above.
        \begin{enumerate}
            \item As an \(\mathbb{Z}\)-module, \(\mathbb{Z} / 2 \mathbb{Z}\) is finitely generated but is not free.
            \item As an \(\mathbb{Z}\)-module, \(\bigoplus_{\mathbb{N}} \mathbb{Z}\) is free, but is not finitely generated.
        \end{enumerate}
    \end{example}
\end{exmbox}

\begin{proof}
    \begin{enumerate}
        \item \(\{1\}\) is a generating set of \(\mathbb{Z}/2\mathbb{Z}\) since \(1 \cdot 1 = 1\) and \(2 \cdot 1 = 0\). However, \(\{1\}\) and ...
    \end{enumerate}
\end{proof}


% ████████  ██████  ██████  ███████ ██  ██████  ███    ██ 
%    ██    ██    ██ ██   ██ ██      ██ ██    ██ ████   ██ 
%    ██    ██    ██ ██████  ███████ ██ ██    ██ ██ ██  ██ 
%    ██    ██    ██ ██   ██      ██ ██ ██    ██ ██  ██ ██ 
%    ██     ██████  ██   ██ ███████ ██  ██████  ██   ████ 
%
%  █████  ███    ██ ██████  
% ██   ██ ████   ██ ██   ██ 
% ███████ ██ ██  ██ ██   ██ 
% ██   ██ ██  ██ ██ ██   ██ 
% ██   ██ ██   ████ ██████  
%
%  █████  ███    ██ ███    ██ ██ ██   ██ ██ ██       █████  ████████  ██████  ██████  
% ██   ██ ████   ██ ████   ██ ██ ██   ██ ██ ██      ██   ██    ██    ██    ██ ██   ██ 
% ███████ ██ ██  ██ ██ ██  ██ ██ ███████ ██ ██      ███████    ██    ██    ██ ██████  
% ██   ██ ██  ██ ██ ██  ██ ██ ██ ██   ██ ██ ██      ██   ██    ██    ██    ██ ██   ██ 
% ██   ██ ██   ████ ██   ████ ██ ██   ██ ██ ███████ ██   ██    ██     ██████  ██   ██ 




\subsubsection*{Torsion and Annihilator}
\begin{defbox}
    \begin{definition}
        \begin{align*}
            \mathrm{Tor}(M) = \makeset{m \in M}{\text{there is an } a \in A \setminus \{ 0 \} \text{ such that} a \cdot m = 0}
        \end{align*}
    \end{definition}
\end{defbox}

\begin{example}
    \begin{enumerate}
        \item Let \(\mathbb{Z}\) be a module over itself. It is \(\mathrm{Tor}(\mathbb{Z}) = \{0\}\).
        \item Let \(n \in \mathbb{N}^+\) and consider the \(\mathbb{Z}\)-module \(\mathbb{Z}^n\). It is
    \end{enumerate}
\end{example}

\begin{thmbox}
    \begin{lemma}
        If \(M\) is a free \(A\)-module, then it is torsion-free, i.e. \(\mathrm{Tor}(M) = \{0\}\).
    \end{lemma}
\end{thmbox}
\begin{proof}
    Let \(M\) be a free \(A\)-module and fix an element \(m \in M\). Since \(M\) is free, \(m\) may be written as
    \begin{align*}
        m = \sum_{i = 1}^n \lambda_i m_i
    \end{align*}
    where \(\lambda_i \in A\) and \(m_i \in M\) with \(1 \leq i \leq n\). If \(m\) is a torsion element, then there is some \(a \in A\) such that \(am = 0\), thus it is
    \begin{align*}
        0 = am = a \sum_{i = 1}^n \lambda_i m_i = \sum_{i = 1}^n a \lambda_i m_i
    \end{align*}
    But \(m_i\) are linearly independent, therefore \(m = 0\).
\end{proof}

\begin{defbox}
    \begin{definition}[Annihilator]
        
    \end{definition}
\end{defbox}

\begin{defbox}
    \begin{definition}[Radical]
        
    \end{definition}
\end{defbox}

\begin{defbox}
    \begin{definition}[Simple Modules]
        Let \(A\) be a ring. A nonzero \(A\)-module \(M\) is called simple if the only submodules are \(\{0\}\) and \(M\) itself.
    \end{definition}
\end{defbox}

\begin{example}
    If \(M\) is a simple \(A\)-module, then any \(f \in \mathrm{Hom}_A (M, M) \setminus \{0\}\) is an isomorphism.
\end{example}

\begin{proof}
    Fix an \(f \in \mathrm{Hom}_A (M, M) \setminus \{0\}\). Since \(\mathrm{ker}(f)\) is a submodule of \(M\), it must be either \(\{0\}\) or whole \(M\). But \(\mathrm{ker}(f) = M\) would mean that \(f = 0\) which was explicitly excluded, thus \(\mathrm{ker}(f) = \{0\}\). By the isomorphism theorem, we also have \(\mathrm{im}(f) \cong \sfrac{M}{\mathrm{ker}(f)} \cong M\). Therefore, \(f\) is bijective.
\end{proof}

\begin{defbox}
    \begin{definition}[Indecomposable]
        Let \(A\) be a ring. A nonzero \(A\)-module \(M\) is called indecomposable if it cannot be written as a direct sum of two non-zero submodules.
    \end{definition}
\end{defbox}

\begin{thmbox}
    \begin{proposition}
        Every simple module is indecomposable.
    \end{proposition}
\end{thmbox}

\begin{exmbox}
    \begin{example}
        Not all indecomposable modules are simple. For example, \(\mathbb{Z}\) is indecomposable, but is not simple.
    \end{example}
\end{exmbox}

\newpage
\section{Exercises and Notes}

\begin{example}
    Let \(f: M \rightarrow N\) be a surjective homomorphism of two finitely generated \(A\)-modules.

    \begin{enumerate}
        \item If \(N \cong A^n\) is a free \(A\)-module, show that \(M \cong \mathrm{ker}(f) \oplus N\).
        
        \begin{proof}
            Since \(N\) is finitely generated, let \((e_1, \ldots, e_n)\) be a set of generators. 
        \end{proof}
    \end{enumerate}
\end{example}

\begin{example}
    Let \(A\) be a ring, \(\mathfrak{a}\) and \(\mathfrak{b}\) ideals, \(M\) and \(N\) \(A\)-modules. Set
    \begin{align*}
        \Gamma_\mathfrak{a}(M) := \makeset{m \in M}{\mathfrak{a} \subset \sqrt{\mathrm{Ann}(m)}} \text{.}
    \end{align*}
    Prove the following statements.
    \begin{enumerate}
        \item If \(\mathfrak{a} \supset \mathfrak{b}\), then \(\Gamma_\mathfrak{a}(M) \subset \Gamma_\mathfrak{b}(M)\).
        
        \begin{proof}
            The proof is a matter of verification. Let \(m \in \Gamma_\mathfrak{a}(M)\). It is
            \begin{align*}
                m \in \Gamma_\mathfrak{a}(M) & \Rightarrow \mathfrak{a} \subset \sqrt{\mathrm{Ann}(m)} \\
                & \Rightarrow \text{For all } a \in \mathfrak{a} \text{ there is a } n \in \mathbb{N}^+ \text{ such that } a^n \in \mathrm{Ann}(m) \text{.}\\
                & \Rightarrow \text{For all } a \in \mathfrak{a} \text{ there is a } n \in \mathbb{N}^+ \text{ such that } a^n \cdot m = 0 \text{.}
                %                
                \intertext{Since \(\mathfrak{a} \supset \mathfrak{b}\), the last statement is true for all \(a \in \mathfrak{b}\). We have}
                %
                & \Rightarrow \text{For all } a \in \mathfrak{b} \text{ there is a } n \in \mathbb{N}^+ \text{ such that } a^n \cdot m = 0 \text{.} \\
                & \Rightarrow \text{For all } a \in \mathfrak{b} \text{ there is a } n \in \mathbb{N}^+ \text{ such that } a^n \in \mathrm{Ann}(m) \text{.}\\
                & \Rightarrow \mathfrak{b} \subset \sqrt{\mathrm{Ann}(m)} \\
                & \Rightarrow m \in \Gamma_\mathfrak{b}(M)
            \end{align*}
            Thus, \(\Gamma_\mathfrak{a}(M) \subset \Gamma_\mathfrak{b}(M)\).
        \end{proof}

        \item If \(M \subset N\), then \(\Gamma_\mathfrak{a}(M) = \Gamma_\mathfrak{a}(N) \cap M\).
        
        \begin{proof}
            Again, the proof is a matter of verification.

            ``\(\subset\)'': \(M \subset N\) implies \(\Gamma_\mathfrak{a}(M) \subset \Gamma_\mathfrak{a}(N)\). Moreover, it is \(\Gamma_\mathfrak{a}(M) \subset M\). Thus, \(\Gamma_\mathfrak{a}(M) \subset \Gamma_\mathfrak{a}(N) \cap M\).

            ``\(\supset\)'': Let \(m \in \Gamma_\mathfrak{a}(N) \cap M\). It is
            
            \begin{align*}
                m \in \Gamma_\mathfrak{a}(N) \cap M & \Rightarrow \mathfrak{a} \subset \sqrt{\mathrm{Ann}(m)} \text{ and } m \in M \text{.} \\
                & \Rightarrow m \in \Gamma_\mathfrak{a}(M) \text{.}
            \end{align*}

            Hence, \(\Gamma_\mathfrak{a}(N) \cap M \subset \Gamma_\mathfrak{a}(M)\).
        \end{proof}
            \item In general, it is \(\Gamma_\mathfrak{a}(\Gamma_\mathfrak{b}(M)) = \Gamma_{\mathfrak{a} + \mathfrak{b}}(M) = \Gamma_\mathfrak{a}(M) \cap \Gamma_\mathfrak{b}(M)\).
            \item In general, it is \(\Gamma_\mathfrak{a}(M) = \Gamma_{\sqrt{\mathfrak{a}}}(M)\).
            \item If \(\mathfrak{a}\) is finitely generated, then
            \begin{align*}
                \Gamma_\mathfrak{a}(M) = \bigcup_{n \geq 1} \makeset{m \in M}{\mathfrak{a}^n m = 0} \text{.}
            \end{align*}
    \end{enumerate}
\end{example}

\begin{example}
    Let \(A\) be a ring, \(M\) a module, \(x \in \mathrm{Rad}(M)\), and \(m \in M\). If \((1 + x)m = 0\), then \(m = 0\).
\end{example}

\begin{proof}
    By definition of radical of a module, it is
    \begin{align*}
        \mathrm{Rad} \left(\sfrac{A}{\mathrm{Ann}(M)}\right) = \sfrac{\mathrm{Rad}(M)}{\mathrm{Ann}(M)} \text{.}
    \end{align*}
    Thus, if \(x \in \mathrm{Rad}(M)\), then its residue \(x^\prime := x + \mathrm{Ann}(M)\) lies in \(\mathrm{Rad}\left(\sfrac{A}{\mathrm{Ann}(M)}\right)\) which means \(x^\prime\) is nilpotent. SOME THEOREM yields \((1 + x^\prime)\) is an unit in \(\sfrac{A}{\mathrm{Ann}(M)}\).
\end{proof}

\chapter{Tensor Product}
\section{Definition and Theorems}

\begin{defbox}
    \begin{definition}
        Let \(M\) and \(N\) be \(A\)-modules. Their tensor product is a pair \((M \otimes_A N, \theta)\) where
        \begin{enumerate}
            \item \(M \otimes_A N\) is an \(A\)-module.
            \item \(\theta: M \times N \rightarrow M \otimes_A N\) is an \(A\)-bilinear mapping.
        \end{enumerate}
        satisfying the universal property, for every pair \((P, \omega)\) of an \(A\)-module and an \(A\)-bilinear mapping \(\omega: M \times N \rightarrow P\), there exists a unique \(A\)-module homomorphism \(f: M \otimes_A N \rightarrow P\) with \(\omega = f \circ \theta\).
    \end{definition}
\end{defbox}

\begin{example}
    Let \(A\) be an integral domain, and \(\mathfrak{a}\) a nonzero ideal. We may use the universal property of tensor products to show
    \begin{align*}
        \mathfrak{a} \otimes_A \mathrm{Frac}(A) = \mathrm{Frac}(A) \text{.}
    \end{align*}
\end{example}
\begin{proof}
    First, define \(\theta: \mathfrak{a} \times \mathrm{Frac}(A) \rightarrow \mathrm{A}\) as \(\theta(a, x) = ax\). It is easy to see that \(\theta\) is bilinear. If we now can show that \(\mathrm{Frac}(A)\) with \(\theta\) satisfies the universal property, then \(\mathrm{Frac}(A)\) is indeed isomorphic to \(\mathfrak{a} \otimes_R \mathrm{Frac}(A)\).

    Let \(P\) be any \(A\)-module and \(\omega: \mathfrak{a} \times \mathrm{Frac}(A) \rightarrow\) an \(A\)-bilinear map. We construct \(f: \mathrm{Frac}(A) \rightarrow P\) by fixing a nonzero \(z \in \mathfrak{a}\), and setting
    \begin{align*}
        f(y) = \omega(z, y/z) \text{.}
    \end{align*}
    Now, we want to show \(f\) satisfies \(\omega = f \circ \theta\) and that such \(f\) is unique.

    It is
    \begin{align*}
        \omega(x, y) &= z \cdot \frac{1}{z} \cdot \omega(x, y) \\
        &= \omega(z, \frac{xy}{z}) \\
        &= f(xy) \\
        &= f(\theta(x, y))
    \end{align*}
    thus, we have \(\omega = f \circ \theta\).

    Moreover, since \(\theta\) is surjective, the definition of \(f\) implies that \(f\) must be unique.
\end{proof}

\begin{example}
    Let \(A\) be a ring, \(\mathfrak{a}\) an ideal of \(A\), and \(M\) an \(A\)-module. Then it is \(M \otimes_A A / \mathfrak{a} \cong M / \mathfrak{a}M\).
\end{example}
\begin{proof}
    As before, we begin by defining
    \begin{align*}
        \theta: M \times A / \mathfrak{a} \rightarrow M / \mathfrak{a}M, \quad (m, a + \mathfrak{a}) \mapsto \theta(m, a + \mathfrak{a}) := am + \mathfrak{a}M \text{.}
    \end{align*}
    Clearly, \(\theta\) is bilinear. Consider an \(A\)-module \(P\) with an \(A\)-bilinear map
    \begin{align*}
        \omega: M \times A / \mathfrak{a} \rightarrow P \text{.}
    \end{align*}
    Now, let
    \begin{align*}
        f: M / \mathfrak{a}M \rightarrow P
    \end{align*}
    with \(\omega = f \circ \theta\). It is
    \begin{align*}
        f(m + \mathfrak{a}M) = f(\theta(m + \mathfrak{a}M, 1)) = \omega(m, 1 + \mathfrak{a}M) \text{.}
    \end{align*}
\end{proof}

\begin{defbox}
    \begin{definition}
        Let \(M\) and \(N\) be \(A\)-modules. Their tensor product is the pair \((M \otimes_A N, \theta)\), where
        \begin{enumerate}
            \item \(M \otimes_A N\) is the quotient of the free \(A\)-module \(A^{M \times N}\) on the direct product \(M \times N\), by the submodule generated by the set of elements of the form:
            \begin{align*}
                (\lambda m_1 + m_2, n) - \lambda (m_1, n) - (m_2, n) \\
                (m, \lambda n_1 + n_2) - \lambda (m, n_1) - (m, n_2)
            \end{align*}
            for \(m, m_1, m_2 \in M\); \(n, n_1, n_2 \in N\); and \(\lambda \in A\), where we denote \((m, n)\) for its image under the canonical mapping \(M \times N \rightarrow A^{(M \times N)}\).
            \item \(\theta: M \times N \rightarrow M \otimes_A N\) is the composition of the canonical mapping \(M \times N \rightarrow A^{(M \times N)}\) with the quotient module homomorphism \(A^{(M \times N)} \rightarrow M \otimes_A N\).
        \end{enumerate}
    \end{definition}
\end{defbox}

\begin{exmbox}
    \begin{example}
        It is \(\mathbb{Z}/2\mathbb{Z} \otimes \mathbb{Z}/3\mathbb{Z} = 0\).
    \end{example}
\end{exmbox}
\begin{proof}
    Let's show this in multiple concrete ways.
    \newline
    \textbf{Method 1:}
    I want to do this conretely. First, we have
    \begin{align*}
        \mathbb{Z}/2\mathbb{Z} \times \mathbb{Z}/3\mathbb{Z} = \set{(0, 0); (0, 1);, (0, 2); (1, 0); (1, 1); (1, 2)} \text{.}
    \end{align*}
    Thus, the elements of \(\mathbb{Z}^{(\mathbb{Z}/2\mathbb{Z} \times \mathbb{Z}/3\mathbb{Z})}\) are in the form
    \begin{align*}
        (x_{(0, 0)}, x_{(0, 1)}, x_{(0, 2)}, x_{(1, 0)}, x_{(1, 1)}, x_{(1, 2)})
    \end{align*}
    where \(x_{(i, j)} \in \mathbb{Z}\) with \(i \in \{0, 1\}\) and \(j \in \{0, 1, 2\}\).

    Now, we want to find the submodule generated by the rules in the definition.

    \begin{enumerate}
        \item Set \(m_1 = m_2 = n = \lambda = 0\), then
        \begin{align*}
            (0 \cdot 0 + 0, 0) + 0 \cdot (0, 0) - (0, 0) = (0, 0) = 1 \cdot (0, 0) \rightarrow (1, 0, 0, 0, 0, 0) \text{.}
        \end{align*}
        \item Set \(m = n_2 = 0\), \(n_1 = 1\), and \(\lambda = 2\), then
        \begin{align*}
            (0, 2 \cdot 1 + 0) - 2 \cdot (0, 1) - (0, 0) &= (0, 2) - (2 \cdot 0, 1) \\
            &= (0, 2) - (0, 1) \\
            &= (0, 1) \\
            &= 1 \cdot (0, 1) \\
            &\rightarrow (0, 1, 0, 0, 0, 0)
        \end{align*}
        \item I think the rest is clear for now.
    \end{enumerate}
    We may conclude that the submodule generated by the rules defined is the whole module, thus \(\mathbb{Z}/2\mathbb{Z} \otimes \mathbb{Z}/3\mathbb{Z} = 0\).
    %
    \newline
    \textbf{Method 2:}
    https://www.math.brown.edu/reschwar/M153/tensor.pdf
\end{proof}


\begin{thmbox}
    \begin{proposition}
        Let \(A\) be a ring, and \(M, N\) and \(P\) be \(A\)-modules.
        \begin{enumerate}
            \item (identity) \(A \otimes_A M = M\).
            \item (commutative law) \(M \otimes_A N = N \otimes_A M\).
        \end{enumerate}
    \end{proposition}
\end{thmbox}
\begin{proof}
    As in the proposition, let \(A\) be a ring, and \(M, N\) and \(P\) be \(A\)-modules.
    \begin{enumerate}
        \item Define \(\beta: A \times M \rightarrow M\) by \(\beta(x, m) := xm\). Clearly, \(\beta\) is bilinear.
    \end{enumerate}
\end{proof}


\newpage
\section{Exercises and Notes}

\begin{example}
    Let \(A \rightarrow B \rightarrow C\) be ring homomorphisms and \(M\) and \(N\) be \(A\)-modules. Show the following.
    \begin{enumerate}
        \item \((M \otimes_A B) \otimes_B C \cong M \otimes_A C\)
        
        \begin{proof}
            It is
            \begin{align*}
                (M \otimes_A B) \otimes_B C & \cong M \otimes_A (B \otimes_B C) \\
                & \cong M \otimes_A C
            \end{align*}
        \end{proof}

        \item \((M \otimes_A N) \otimes_A B \cong (M \otimes_A B) \otimes_B (N \otimes_A B)\)
        
        \begin{proof}
            trivial
        \end{proof}
    \end{enumerate}
\end{example}

\begin{example}
    Let \(A\) be a ring.
    \begin{enumerate}
        \item If \(M, N\) are \(A\)-modules, then \(\mathrm{Hom}_A(M, N)\) may be viewed as an \(A\)-module via
        \begin{align*}
            a \cdot \varphi := (m \mapsto a \cdot \varphi(m))
        \end{align*}
        for \(a \in A\) and \(\varphi \in \mathrm{Hom}_A(M, N)\).

        \begin{proof}
            this is trivial
        \end{proof}
        \item If \(M, N, L\) are \(A\)-modules, then there exists a natural isomorphism of \(A\)-modules
        \begin{align*}
            \mathrm{Hom}_A(L \otimes_A M, N) \cong \mathrm{Hom}_A(L, \mathrm{Hom}_A(M, N))
        \end{align*}
    \end{enumerate}
\end{example}

\begin{example}
    Let \(A\) be a ring, \(\mathfrak{a}\) an ideal of \(A\), and \(M\) an \(A\)-module.
    \begin{enumerate}
        \item Show that \(M / \mathfrak{a} M \cong M \otimes_A A / \mathfrak{a}\).
        \begin{proof}
            Define \(\varphi: M \otimes_A A / \mathfrak{a} \rightarrow M / \mathfrak{a}M \) by
            \begin{align*}
                m \otimes_A \overline{x} \mapsto x \cdot m + \mathfrak{a}M \text{.}
            \end{align*}
            \(\varphi\) is an homomorphism because
            \begin{enumerate}
                \item \(\varphi((m_1 \otimes_A \overline{x_1}) + (m_2 \otimes_A \overline{x_2})) = \)
            \end{enumerate}
        \end{proof}
    \end{enumerate}
\end{example}

\chapter{Nakayama's Lemma}

\begin{thmbox}
    \begin{proposition}
        Let \(M\) be a finitely generated \(A\)-module, and \(\mathfrak{a}\) an ideal of \(A\). Then, \(\mathfrak{a}M = M\) if and only if there exists \(a \in \mathfrak{a}\) such that \((1 + a)M = 0\).
    \end{proposition}
\end{thmbox}
\begin{proof}
    ``\(\Rightarrow\)'': Let \(\mathfrak{a}M = M\), so for all \(a \in \mathfrak{a}\) and \(m, m^\prime \in M\), it is \(am = m^\prime\), in particular, we have \(-am = m\). Rewriting the equation yields \(0 = am + m = (1 + a) m\). Therefore, it is \((1 + a)M = 0\).

    ``\(\Leftarrow\)'': On the other hand, if there is an \(a \in \mathfrak{a}\) such that \((1 + a)M = 0\), then for all \(m \in M\) it is \(0 = (1 + a)m = m + am\) and rewriting it gives \(m = -am\). So any \(m\) is contained in \(\mathfrak{a}M\), i.e. \(M \subset \mathfrak{a}M\). Trivially, it is also \(M \subset \mathfrak{a}M\), hence we have \(\mathfrak{a}M = M\).
\end{proof}

\begin{thmbox}
    \begin{theorem}
        Let \(M\) be a finitely generated \(A\)-module. If there is an ideal \(\mathfrak{a}\) in \(A\) with \(\mathfrak{a} \in \mathrm{Jac}(A)\) such that \(\mathfrak{a}M = M\), then \(M = 0\).
    \end{theorem}
\end{thmbox}
\begin{proof}
    
\end{proof}

\begin{thmbox}
    \begin{theorem}
        Let \(A\) be a local ring, \(\mathfrak{m}\) the maximal ideal of \(A\), and \(k = A / \mathfrak{m}\), and \(M\) a finitely generated \(A\)-module. Then we have the following.
        \begin{enumerate}
            \item For all submodules \(N\) of \(M\) with \(M = N + \mathfrak{m}M\) it is \(N = M\).
        \end{enumerate}
    \end{theorem}
\end{thmbox}

\chapter{Exact Sequences}
\section{Definition and Theorems}

\begin{defbox}
    \begin{definition}
        Exact at, exact sequence, short exact sequence    
    \end{definition}
\end{defbox}

\begin{exmbox}
    \begin{example}
        Let \(M\) and \(N\) be \(A\)-modules. Then, the sequence
        \begin{align*}
            0 \rightarrow M \rightarrow M \oplus N \rightarrow N \rightarrow 0
        \end{align*}
        is short exact.
    \end{example}
\end{exmbox}

\begin{thmbox}
    \begin{lemma}
        If \(0 \rightarrow M \rightarrow N \rightarrow P \rightarrow 0\) is exact, and \(M\) and \(P\) are finitely presented, then \(N\) is finitely presented.
    \end{lemma}
\end{thmbox}

\begin{proof}
    
\end{proof}

\begin{thmbox}
    \begin{proposition}
        Let \(M\) be an \(A\)-module, \(m_\lambda\) with \(\lambda \in \Lambda\) a set of generators. Then there is an exact sequence \(0 \rightarrow K \rightarrow A^{\oplus \Lambda} \rightarrow M \rightarrow 0\)
    \end{proposition}
\end{thmbox}



\section{Notes and Exercises}




\chapter{Noetherian Modules}

\begin{defbox}
    \begin{definition}
        An \(A\)-module \(M\) is called Noetherian if one of the following equivalent conditions hold.
        \begin{enumerate}
            \item Its submodules satisfies the asending chain condition, i.e. MISSING.
            \item All submodules of \(M\) are finitely generated.
        \end{enumerate}
    \end{definition}
\end{defbox}

\begin{proof}
    ``\(\Rightarrow\)'': Let \(M\) be an \(A\)-module that satisfies the ascending chain condition and assume a submodule \(N\) is not finitely generated. In this case, we may construct a chain of submodules
    \begin{align*}
        N_1 \subset N_2 \subset \cdots N_i \subset \cdots
    \end{align*}
    where \(N_i = (n_1, n_2, \ldots, n_{i-1})\) with \(n_i \in N\) and \(n_i \not\in N_i\) for all \(i \in \mathbb{N}^+\). This chain never stabilizes, thus \(N\) must be finitely generated.

    \noindent``\(\Leftarrow\)'':
\end{proof}

\begin{thmbox}
    \begin{lemma}
        Let \(0 \rightarrow M \rightarrow N \rightarrow P \rightarrow 0\) be an exact sequence of \(A\)-modules. Then \(N\) is Noetherian if and only if \(M\) and \(P\) are Noetherian.
    \end{lemma}
\end{thmbox}
\begin{proof}
    Let \(0 \rightarrow M \rightarrow N \rightarrow P \rightarrow 0\) be an exact sequence of \(A\)-modules.

    \noindent ``\(\Rightarrow\)'': Let \(N\) be Noetherian.
    \begin{enumerate}
        \item We show that \(M\) is Noetherian by verifying all its submodules are finitely generated. Let \(M^\prime\) be a submodule of \(M\). In that case, \(\alpha(M^\prime)\) is a submodule of \(N\) and thus finitely generated. \(\alpha\) restricted 
        \item We show that \(P\) is Noetherian by verifying all its submodules are finitely generated. Let \(P^\prime\) be a submodule of \(P\). Since \(\beta\) is surjective, we have \(P^\prime = \beta \left(\beta^{-1}(P^\prime)\right)\). \(\beta^{-1}(P^\prime)\) is a submodule of \(N\) and it is finitely generated because \(N\) is Noetherian.
    \end{enumerate}
\end{proof}

\begin{thmbox}
    \begin{proposition}
        \textit{The property Noetherian is stable under intersection, direct sum,addition, and localization.} Let \(M\) be an \(A\)-module, \(N_1\) and \(N_2\) submodules of \(M\).
        \begin{enumerate}
            \item If \(N_1\) and \(N_1\) are Noetherian, so is \(N_1 \cap N_2\), \(N_1 \oplus N_2\), and \(N_1 + N_2\).
        \end{enumerate}
    \end{proposition}
\end{thmbox}
\begin{proof}
    \begin{enumerate}
        \item Since all submodules of a Noetherian module is again Noetherian, \(N_1 \cap N_2\) is Noetherian because it is a submodule of \(M\) which is Noetherian.
        \item Consider the sequence \(0 \rightarrow N_1 \rightarrow N_1 \oplus N_2 \rightarrow N_2 \rightarrow 0\).
        \item 
    \end{enumerate}
\end{proof}

\begin{exmbox}
    \begin{example}
        Let \(M\) be an \(A\)-module, and \(N_1\) and \(N_2\) submodules of \(M\). In general, \(N_1 \otimes N_2\) is not Noetherian.
    \end{example}
\end{exmbox}

\chapter{Artinian Modules}
\section{Definition and Theorems}
\begin{defbox}
    \begin{definition}[Artinian Module]
        
    \end{definition}
\end{defbox}


\begin{example}[Examples of Artinian Modules]
    \begin{enumerate}
        \item For \(n \in \mathbb{N}^+\), \(\mathbb{Z} / n \mathbb{Z}\) is Artinian.
    \end{enumerate}
\end{example}

\begin{example}[Counterexamples of Artinian Modules]
    \begin{enumerate}
        \item \(\mathbb{Z}\) is not Artinian.
    \end{enumerate}
\end{example}

\begin{thmbox}
    \begin{lemma}
        Let \(0 \rightarrow M \rightarrow N \rightarrow P \rightarrow 0\) be an exact sequence of \(A\)-modules. Then \(N\) is Artinian if and only if \(M\) and \(P\) are Artinian.
    \end{lemma}
\end{thmbox}

\begin{thmbox}
    \begin{proposition}
        The property of Artinian is stable under intersection, direct sum, addition, localization, 
    \end{proposition}
\end{thmbox}


\part*{Unorganized}

\begin{example}
    Let \(A\) be a local ring with maximal ideal \(\mathfrak{m}\).
    \begin{enumerate}
        \item What do the simple \(A\)-module look like?
        
        \begin{proof}
            Let \(M\) be a simple \(A\)-module. Since \(M\) is simple, the only proper submodule is the zero-module.
        \end{proof}
    \end{enumerate}
\end{example}

\textbf{Length}

\begin{example}
    Let \(M\) be an \(A\)-module.
    \begin{enumerate}
        \item If \(M\) is simple, then any nonzero element \(m \in M\) generates M.
        \begin{proof}
            Fix an element \(m \in M\) and assume \(m\) does not generate whole \(M\). In that case, there must be a \(m^\prime \in M\) such that \(m \neq \lambda m^\prime\) for all \(\lambda \in A\). Then, \((m)\) is non-zero, but also not whole \(M\) which is a contradiction.
        \end{proof}
        \item \(M\) is simple if and only if \(M \cong A / \mathfrak{m}\) for some maximal ideal \(\mathfrak{m}\), and if so, then \(\mathfrak{m} = Ann(M)\).
        \begin{proof}
            We first show that \(M\) is simple if and only if \(M \cong A/\mathfrak{m}\) for some maximal ideal \(\mathfrak{m}\).
            ``\(\Rightarrow\)'': Let \(M\) be simple. By the statement above, \(M\) is cyclic.
        \end{proof}
       \end{enumerate}
\end{example}


\begin{example}
    Let \(k\) be a field. Is \(X = \mathrm{Spec}(k[X, Y] / (xy - 1))\) with the Zariski-topology connected?
\end{example}


\begin{example}
    If \(A_\mathfrak{p}\) is reduced at all \(\mathfrak{p} \in \mathrm{Spec}(A)\), then \(A\) is reduced.
\end{example}
\begin{proof}
    THIS IS A WRONG PROOF!

    Denote the canonic \(\varphi_\mathfrak{p}: A \rightarrow A_\mathfrak{p}\). Assume \(x \in A\) with \(x^n = 0\). It is
    \begin{align*}
        0 = \varphi(0) = \varphi(x^n) = (\varphi(x))^n
    \end{align*}
    but since \(A_\mathfrak{p}\) is reduced, conclude \(\varphi(x) = 0\), so \(x = 0\).

    The issue with this proof is that for example \(\varphi(x) \cdot \varphi(x)^2 = 0\) because \(\varphi(x)\) and \(\varphi(x)^2\) are zero divisors.
\end{proof}

\begin{thmbox}
    \begin{proposition}
        Let \(A\) be a ring. Then, the following are equivalent.
        \begin{enumerate}
            \item \(A\) is reduced.
            \item \(A_\mathfrak{p}\) is reduced for all prime ideals \(\mathfrak{p} \in \mathrm{Spec}(A)\).
            \item \(A_\mathfrak{m}\) is reduced for all maximal ideals \(\mathfrak{m} \in \mathrm{Spm}(A)\).
        \end{enumerate}
    \end{proposition}
\end{thmbox}
\begin{proof}
    ``2 \(\Rightarrow\) 1'': Assume \(x \in A\) is nilpotent and nonzero.
\end{proof}

\chapter{Length}
\subsection{Definition and Theorems}
\begin{defbox}
    \begin{definition}[Simple Modules]
        
    \end{definition}
\end{defbox}
\begin{defbox}
    \begin{definition}
        Let \(M\) be an \(A\)-{\color{mathobj}module}. We call a {\color{mathobj}chain} of {\color{mathobj}submodules}
        \begin{align*}
            M = M_0 \supset M_1 \supset \cdots \supset M_n = 0
        \end{align*}
        a {\color{maththen}composition series} of {\color{maththen}length} \(n\) if \textbf{each} successive {\color{mathobj}quotient} \(M_{i-1} / M_i\) is {\color{mathif}simple}.

        We define the {\color{maththen}length} \(l(M)\) to be the infimum of all those length, i.e.
        \begin{align*}
            l(M) := \mathrm{inf} \makeset{n}{M \text{ has a composition series of length } n} \text{.}
        \end{align*}

        By convention, if \(M\) has no composition series, then \(l(M) := \inf\).
    \end{definition}
\end{defbox}


\end{document}