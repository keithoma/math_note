\documentclass[a4paper]{book}
\title{Commutative Ring Theory}
\author{Kei Thoma}


% ---------------------------------------------------------------------
% P A C K A G E S
% ---------------------------------------------------------------------

% typography and formatting
\usepackage[english]{babel}
\usepackage[utf8]{inputenc}
\usepackage{geometry}
\usepackage{exsheets}
\usepackage{environ}
\usepackage{graphicx}
\usepackage{cutwin}
\usepackage{pifont}

% mathematics
\usepackage{xfrac}  
\usepackage{amsthm} % for theorems, and definitions
\usepackage{amssymb}
\usepackage{amsmath}
\usepackage{textcomp}
\usepackage{mathtools}
% \usepackage{MnSymbol} % for \cupdot

% extra
\usepackage{xcolor}
\usepackage{tikz}

% ---------------------------------------------------------------------
% S E T T I N G
% ---------------------------------------------------------------------

%maybe delete later, for colorbox
\usepackage{tcolorbox}
\newtcolorbox{defbox}{colback=blue!5!white,colframe=blue!75!black}
\newtcolorbox{defboxlight}{colback=cyan!5!white,colframe=cyan!75!black}
\newtcolorbox{thmbox}{colback=orange!5!white,colframe=orange!75!black}
\newtcolorbox{rembox}{colback=purple!5!white,colframe=purple!75!black}
\newtcolorbox{exmbox}{colback=gray!5!white,colframe=gray!75!black}
\newtcolorbox{intbox}{colback=violet!5!white,colframe=violet!75!black}

% typography and formatting
\geometry{margin=2cm, paperheight=30cm}

\SetupExSheets{
  counter-format = ch.qu,
  counter-within = chapter,
  question/print = true,
  solution/print = true,
}

% mathematics
\newcounter{global}

\theoremstyle{definition}
\newtheorem{definition}{Definition}[]
\newtheorem{example}{Example}[definition]

\newtheorem{theorem}[definition]{Theorem}
\newtheorem{corollary}{Corollary}
\newtheorem{lemma}[definition]{Lemma}
\newtheorem{proposition}[definition]{Proposition}

\newtheorem*{remark}{Remark}
\newtheorem*{intuition}{Intuition}

% extra
\definecolor{mathif}{HTML}{0000A0} % for conditions
\definecolor{maththen}{HTML}{CC5500} % for consequences
\definecolor{mathrem}{HTML}{8b008b} % for notes
\definecolor{mathobj}{HTML}{008800}

\usetikzlibrary{positioning}
\usetikzlibrary{shapes.geometric, arrows}

% ---------------------------------------------------------------------
% C O M M A N D S
% ---------------------------------------------------------------------

\newcommand{\norm}[1]{\left\lVert#1\right\rVert}
\newcommand{\rank}{\text{rank}}
\newcommand{\Vol}{\text{Vol}}

\newcommand{\set}[1]{\left\{\, #1 \,\right\}}
\newcommand{\makeset}[2]{\left\{\, #1 \mid #2 \,\right\}}

\newcommand*\diff{\mathop{}\!\mathrm{d}}
\newcommand*\Diff{\mathop{}\!\mathrm{D}}

\newcommand\restr[2]{{% we make the whole thing an ordinary symbol
  \left.\kern-\nulldelimiterspace % automatically resize the bar with \right
  #1 % the function
  \vphantom{\big|} % pretend it's a little taller at normal size
  \right|_{#2} % this is the delimiter
  }}

% ---------------------------------------------------------------------
% R E N D E R
% ---------------------------------------------------------------------

\newif\ifshowproof
\showprooftrue

\NewEnviron{Proof}{%
    \ifshowproof%
        \begin{proof}%
            \BODY
        \end{proof}%
    \fi%
}%

\begin{document}
\section{Exact Sequence}
\begin{defbox}
    \begin{definition}
        
    \end{definition}
\end{defbox}
\begin{thmbox}
    \begin{theorem}
        \begin{align*}
            0 \longrightarrow M' \longrightarrow M \longrightarrow M'' \longrightarrow 0
        \end{align*}
        \(M / M' \cong M''\)
    \end{theorem}
\end{thmbox}
\begin{exmbox}
    \begin{example}
        \begin{enumerate}
            \item
            \begin{align*}
                0 \longrightarrow N \longrightarrow M \longrightarrow M / N \longrightarrow 0
            \end{align*}
            \begin{align*}
                0 \longrightarrow M' \longrightarrow M' \oplus M'' \longrightarrow M'' \longrightarrow 0
            \end{align*}
        \end{enumerate}
    \end{example}
\end{exmbox}
\begin{defbox}
    \begin{definition}
        Split
    \end{definition}
\end{defbox}
\begin{enumerate}
    \item \(g \circ s = \text{id}_{M''}\)
    \item \(s\) is injective
    \item this does not imply that \(M\) and \(M''\) are isomorphic, because for that we also need \(s \circ g = \text{id}_{M}\)
    \item \(g\) admits section is a very good naming, because basically it means that \(M''\) lies in \(M\)
    \item \(M''\) may be viewed as a submodule of \(M\)
    \item since \(f\) is injective, we may see \(M'\) as a submodule of \(M\)
    \item \(M' \cap M'' = 0 \iff f(M) \cap s(M'') = 0 \iff \text{ker} (g) \cap s(M'') = 0\)
    \item And the last equation is true because if \(x \in \text{ker}(g) \cap s(M'')\), then \(g(x) = 0\) and \(x = s(m'')\), so \(g(s(m'')) = m'' = 0\), putting it back together \(x = s(m'') = s(0) = 0\).
    \item \(M = M' + M'' \iff M = f(M) + s(M'') \iff M = \text{ker}(g) + s(M'')\)
    \item I'm not sure about this, but w/E
    \item thus: \(M = M' \oplus M''\)
\end{enumerate}

\begin{defbox}
    \begin{definition}
        \begin{align*}
            \text{coker}(f) = M / \text{im}(f)
        \end{align*}
    \end{definition}
\end{defbox}

\begin{exmbox}
    \begin{example}
        \begin{align*}
            0 \longrightarrow \text{ker}f \longrightarrow M' \longrightarrow M \longrightarrow \text{coker} f \longrightarrow 0
        \end{align*}
    \end{example}
\end{exmbox}

\newpage
\section{Snake Lemma}


\newpage

\end{document}