\documentclass[a4paper]{book}
\title{Commutative Ring Theory}
\author{Kei Thoma}


% ---------------------------------------------------------------------
% P A C K A G E S
% ---------------------------------------------------------------------

% typography and formatting
\usepackage[english]{babel}
\usepackage[utf8]{inputenc}
\usepackage{geometry}
\usepackage{exsheets}
\usepackage{environ}
\usepackage{graphicx}
\usepackage{cutwin}
\usepackage{pifont}

% mathematics
\usepackage{xfrac}  
\usepackage{amsthm} % for theorems, and definitions
\usepackage{amssymb}
\usepackage{amsmath}
\usepackage{textcomp}
\usepackage{mathtools}
% \usepackage{MnSymbol} % for \cupdot

% extra
\usepackage{xcolor}
\usepackage{tikz}

% ---------------------------------------------------------------------
% S E T T I N G
% ---------------------------------------------------------------------

%maybe delete later, for colorbox
\usepackage{tcolorbox}
\newtcolorbox{defbox}{colback=blue!5!white,colframe=blue!75!black}
\newtcolorbox{defboxlight}{colback=cyan!5!white,colframe=cyan!75!black}
\newtcolorbox{thmbox}{colback=orange!5!white,colframe=orange!75!black}
\newtcolorbox{rembox}{colback=purple!5!white,colframe=purple!75!black}
\newtcolorbox{exmbox}{colback=gray!5!white,colframe=gray!75!black}
\newtcolorbox{intbox}{colback=violet!5!white,colframe=violet!75!black}

% typography and formatting
\geometry{margin=2cm, paperheight=60cm}

\SetupExSheets{
  counter-format = ch.qu,
  counter-within = chapter,
  question/print = true,
  solution/print = true,
}

% mathematics
\newcounter{global}

\theoremstyle{definition}
\newtheorem{definition}{Definition}[]
\newtheorem{example}{Example}[definition]

\newtheorem{theorem}[definition]{Theorem}
\newtheorem{corollary}{Corollary}
\newtheorem{lemma}[definition]{Lemma}
\newtheorem{proposition}[definition]{Proposition}

\newtheorem*{remark}{Remark}
\newtheorem*{intuition}{Intuition}

% extra
\definecolor{mathif}{HTML}{0000A0} % for conditions
\definecolor{maththen}{HTML}{CC5500} % for consequences
\definecolor{mathrem}{HTML}{8b008b} % for notes
\definecolor{mathobj}{HTML}{008800}

\usetikzlibrary{positioning}
\usetikzlibrary{shapes.geometric, arrows}

% ---------------------------------------------------------------------
% C O M M A N D S
% ---------------------------------------------------------------------

\newcommand{\norm}[1]{\left\lVert#1\right\rVert}
\newcommand{\rank}{\text{rank}}
\newcommand{\Vol}{\text{Vol}}

\newcommand{\set}[1]{\left\{\, #1 \,\right\}}
\newcommand{\makeset}[2]{\left\{\, #1 \mid #2 \,\right\}}

\newcommand*\diff{\mathop{}\!\mathrm{d}}
\newcommand*\Diff{\mathop{}\!\mathrm{D}}

\newcommand\restr[2]{{% we make the whole thing an ordinary symbol
  \left.\kern-\nulldelimiterspace % automatically resize the bar with \right
  #1 % the function
  \vphantom{\big|} % pretend it's a little taller at normal size
  \right|_{#2} % this is the delimiter
  }}

% ---------------------------------------------------------------------
% R E N D E R
% ---------------------------------------------------------------------

\newif\ifshowproof
\showprooftrue

\NewEnviron{Proof}{%
    \ifshowproof%
        \begin{proof}%
            \BODY
        \end{proof}%
    \fi%
}%



























\begin{document}
\chapter{Noetherian Rings}
\section*{Cheat Sheet}
\begin{defbox}
    \begin{definition}
        A ring \(R\) is called Noetherian if it fulfills one of the following equivalent conditions.
        \begin{enumerate}
            \item Each ideal \(\mathfrak{a}\) of \(R\) is finitely generated.
            \item Any infinite increasing sequence of ideals \(\mathfrak{a}_1 \subset \mathfrak{a}_2 \subset \mathfrak{a}_3 \subset \cdots\) in \(R\) eventually stabilized, i.e. \(\mathfrak{a}_k = \mathfrak{a}_{k + 1}\) for \(k\) large enough.
            \item Every nonempty collection \(S\) of ideals of \(R\) contains a maximal element with respect to inclusion.
        \end{enumerate}
    \end{definition}
\end{defbox}





\begin{thmbox}
    \begin{theorem}
        Let \(R\) be a Noetherian ring. Each surjective ring homomorphism \(\varphi: R \longrightarrow R\) is injective, and thus an isomorphism.
    \end{theorem}
\end{thmbox}





\begin{thmbox}
    \begin{theorem}
        If \(R\) is a Noetherian integral domain, then every nonzero nonunit in \(R\) can be factored into irreducibles.
    \end{theorem}
\end{thmbox}

\begin{thmbox}
    \begin{theorem}
        \begin{enumerate}
            \item \(R / I\) is Noetherian.
            \item \(R\) Noetherian \(\iff\) polynomial ring with finite variables is Noetherian \(\iff\) formal power series with finite variables is Noetherian
            \item localization is Noetherian
        \end{enumerate}
    \end{theorem}
\end{thmbox}

\begin{thmbox}
    \begin{theorem}
        In a nonzero Noetherian ring, each proper ideal is an intersection of finitely many primary ideals.
    \end{theorem}
\end{thmbox}


\newpage
\section{Introduction}
\begin{defbox}
    \begin{definition}
        A ring \(R\) is called Noetherian if it fulfills one of the following equivalent conditions.
        \begin{enumerate}
            \item Each ideal \(\mathfrak{a}\) of \(R\) is finitely generated.
            \item Any infinite increasing sequence of ideals \(\mathfrak{a}_1 \subset \mathfrak{a}_2 \subset \mathfrak{a}_3 \subset \cdots\) in \(R\) eventually stabilized, i.e. \(\mathfrak{a}_k = \mathfrak{a}_{k + 1}\) for \(k\) large enough.
            \item Every nonempty collection \(S\) of ideals of \(R\) contains a maximal element with respect to inclusion.
        \end{enumerate}
    \end{definition}
\end{defbox}
\begin{exmbox}
    \begin{example}
        PIDs and thus fields are Noetherian rings.
    \end{example}
\end{exmbox}
\begin{rembox}
    \begin{intuition}
        Noetherian rings can be regarded as a good generalization of PIDs: the property of all ideals being singly generated is often not preserved under common ring-theoretic constructions (\(\mathbb{Z}\) is a PID but \(\mathbb{Z}[X]\) is not), but the property of all ideals being finitely generated does remain valid under many constructions of new rings from old rings. For example, every quadratic ring \(\mathbb{Z}[\sqrt{d}]\) is Noetherian; many of these rings are not PIDs.
    \end{intuition}
\end{rembox}
\begin{exmbox}
    \begin{example}
        The polynomial ring \(K[X_1, X_2, \ldots]\) in infinitely many variables over a field \(K\) is not Noetherian.
    \end{example}
\end{exmbox}
\begin{rembox}
    \begin{remark}
        The third condition shows a Noetherian ring \(R\) other than the zero ring has a maximal ideal and every proper ideal \(\mathfrak{a}\) in a Noetherian ring \(R\) is contained in a maximal ideal. This does not need Zorn's lemma, which is used to show maximal ideals exist in arbitary rings.
    \end{remark}
\end{rembox}
\begin{thmbox}
    \begin{theorem}
        Let \(R\) be a Noetherian ring. Each surjective ring homomorphism \(\varphi: R \longrightarrow R\) is injective, and thus an isomorphism.
    \end{theorem}
\end{thmbox}
\begin{proof}
    Let \(\varphi: R \longrightarrow R\) be surjective.
    
    The trick is to see that kernel of \(\varphi\) is an ideal and that a \(n\)-fold composition of \(\varphi\) give exactly the ideals we want to study.

    We have the following sequence of increasing ideals
    \begin{align*}
        \text{ker}(\varphi) \subset \text{ker}(\varphi^2) \subset \text{ker}(\varphi^3) \subset \cdots
    \end{align*}
    This is valid because: Let \(x \in \text{ker}(\varphi^n)\), then \(\varphi^n(x) = 0\) and that means \(\varphi^{n+1}(x) = \varphi^n(\varphi(x)) = \varphi^{n}(0) = 0\), thus \(x \in \text{ker}(\varphi^{n+1})\).

    Since \(R\) is Noetherian, this sequence of ideals becomes stationary, hence \(\text{ker}(\varphi^{n}) = \text{ker}(\varphi^{n+1})\) for \(n\) large enough.

    Let \(x \in \text{ker}(\varphi)\), then \(\varphi(x) = 0\). Since \(\varphi\) is surjective, \(\varphi^n\) is surjective, so \(x = \varphi^n(x')\) for some \(x' \in R\).

    That means \(0 = \varphi(x) = \varphi(\varphi^{n}(x')) = \varphi^{n+1}(x')\).

    So \(x' \in \text{ker}(\varphi^{n+1}) = \text{ker}(\varphi^{n})\). Thus \(x = \varphi^{n}(x') = 0\). So \(\text{ker}(\varphi) = \{0\}\).
\end{proof}

\begin{exmbox}
    \begin{example}
        \(\mathbb{R}[X]\) is a Noetherian ring and the map \(f: \mathbb{R}[X] \longrightarrow \mathbb{R}[X], f(X) \mapsto f(X^2)\) is an injective ring homomorphism that is not surjective.
    \end{example}
\end{exmbox}

\begin{thmbox}
    \begin{theorem}
        If \(R\) is a Noetherian integral domain, then every nonzero nonunit in \(R\) can be factored into irreducibles.
    \end{theorem}
\end{thmbox}
\begin{rembox}
    \begin{remark}
        \begin{enumerate}
            \item If \(R\) is a field, then there are no nonzero units and the theorem is vacuously true.
            \item This theorem is not saying that a Noetherian integral domain has a unique factorization. Many Noetherian integral domains do not have unique factorization.
            \item If an integral domain \(R\) contains a nonzero nonunit \(x\) that has no irreducible factorization, then \(R\) can't be Noetherian.
        \end{enumerate}
    \end{remark}
\end{rembox}








\newpage
\section{Ring Operations}
Which operations keep Noetherian?
\begin{enumerate}
    \item subring
    \item quotient
    \item polynomial
    \item fractions
    \item direct product
    the ideals in \(A \times B\) corresponds to ideals in \(A\) and \(B\) if the ring is infinite
    \item 
\end{enumerate}


\begin{exmbox}
    \begin{example}
        \textit{A subring of a Noetherian ring need not be Noetherian.} Take any non-Noetherian ring. It is a subring of its field of fractions which is trivially Noetherian.
    \end{example}
\end{exmbox}
\begin{example}
    \begin{enumerate}
        \item The polynomial ring \(K[X_1, X_2, \ldots]\) with infinitely many variables is not Noetherian, but \(\text{Frac}(K[X_1, X_2, \ldots])\) is Noetherian.
        \item \(K[X, Y]\) is Noetherian, but the subring \(\makeset{XY^k}{k \in \mathbb{N}^+}\) is not.
    \end{enumerate}
\end{example}

\begin{thmbox}
    \begin{theorem}
        If \(R\) is a Noetherian ring, then so is \(R / I\) for any ideal \(I\) in \(R\).
    \end{theorem}
\end{thmbox}
\begin{proof}
    There are at least two ways.

    First is direct by using the fact that \(\pi^{-1}(\mathfrak{a})\) is an ideal in \(R\) that is finitely generated.

    The other is using exact sequences.
\end{proof}

\begin{thmbox}
    \begin{theorem}
        \(R\) is a Noetherian ring if and only if \(R[X]\) is Noetherian if and only if \(R[X_1, X_2]\) is Noetherian.
    \end{theorem}
\end{thmbox}

\begin{exmbox}
    \begin{example}
        \(\mathbb{R}[X_1, X_2, \ldots]\) is not Noetherian.
    \end{example}
\end{exmbox}

\begin{thmbox}
    \begin{theorem}
        For a ring \(A\) its polynomial ring \(A[X_1, X_2, \ldots]\) is not Noetherian.
    \end{theorem}
\end{thmbox}
\begin{proof}
    The chain of ideals \((X_1) \subset (X_1, X_2) \subset \cdots\) never becomes stationary.
\end{proof}

\begin{thmbox}
    \begin{theorem}
        If \(R\) is Noetherian, then so is the formal power series \(R[[X]]\) in (finitely many variables)
    \end{theorem}
\end{thmbox}
What about the converse?
What about infinite number of variables?

\begin{proof}
    
\end{proof}



\begin{thmbox}
    \begin{theorem}
        If \(A\) is Noetherian, so is its ring of fractions \(S^{-1}A\).
    \end{theorem}
\end{thmbox}
\begin{proof}
    \begin{enumerate}
        \item Let \(A\) be a Noetherian ring.
        \item Let \(S\) be a multiplicative set in \(A\).
        \item Let \(S^{-1}A\) be the localization of \(A\) at \(S\) with \(\tau: A \longrightarrow S^{-1}A\) the canonical homomorphism.
        \item Fix an ideal \(\mathfrak{a}\) in \(S^{-1}A\). \textit{(We will show \(\mathfrak{a}\) is finitely generated.)}
        \item By ideal correspodence, \(\tau^{-1}(\mathfrak{a})\) is an ideal in \(A\).
        \item Since \(A\) is Noetherian, \(\tau^{-1}(\mathfrak{a})\) is finitely generated.
        \item Let \(x_1, \ldots x_n \in A\) generate \(\tau^{-1}(\mathfrak{a})\).
        \item Let \(\frac{x}{s} \in \mathfrak{a}\).
    \end{enumerate}
\end{proof}

What about the converse?

\begin{example}
    The quadratic ring \(\mathbb{Z}[\sqrt{d}]\) for a nonsquare integer \(d \in \mathbb{Z}\) is Noetherian.
\end{example}
\begin{proof}
    \begin{enumerate}
        \item It is \(\mathbb{Z}[\sqrt{d}] \cong \mathbb{Z}[X] / (X^2 - d)\).
        \item \(\mathbb{Z}[X]\) is Noetherian.
        \item Thus \(\mathbb{Z}[X] / (X^2 - d)\) is Noetherian.
        \item By the isomorphism, \(\mathbb{Z}[\sqrt{d}]\) is Noetherian.
    \end{enumerate}
\end{proof}






\newpage
\subsection{Corollaries}
\begin{thmbox}
    \begin{theorem}
        Let \(\varphi: R \longrightarrow R'\) be a ring homomorphism. If \(R\) is Noetherian, then so is \(\text{im}(\varphi)\).
    \end{theorem}
\end{thmbox}
\begin{proof}
    \begin{enumerate}
        \item Let \(R\) be a Noetherian ring.
        \item Let \(\varphi: R \longrightarrow R'\) be a ring homomorphism.
        \item By the isomorphism theorem, we have the isomorphism \(R / \text{ker}(\varphi) \cong \text{im}(\varphi).\)
        \item Since \(R\) is Noetherian, \(R / \text{ker}(\varphi)\) is as well.
        \item By the isomorphism given in (3), \(\text{im}(\varphi)\) is Noetherian.
    \end{enumerate}
\end{proof}





\newpage
\section{Primary Decomposition}
\subsection{Primary Ideals}
\begin{defbox}
    \begin{definition}
        A proper ideal \(\mathfrak{q}\) of a ring \(A\) is called primary if one of the following equivalent conditions are fulfilled.
        \begin{enumerate}
            \item For all \(ab \in \mathfrak{q}\), it is either \(a \in \mathfrak{q}\) or \(b^n \in \mathfrak{q}\) for some \(\mathbb{N}^+\).
            \item The zero divisors in \(A / \mathfrak{q}\) is nilpotent.
        \end{enumerate}
    \end{definition}
\end{defbox}




\begin{example}
    In the ring of integers \(\mathbb{Z}\), the primary ideals are \(0\) and \((p)^k\) for prime numbers \(p\) and \(k \in \mathbb{N}^+\).

    More concretely, consider \((6)\). This is not a primary ideal because \(6 = 2 \cdot 3\), but \(6\) neither contains \(2\), \(3\) nor the powers of them.

    On the other hand, \((2)^3 = (8)\) is primary. For example, the decomposition \(8 = 2 \cdot 4\)

    Concretely, consider \((2)^3 = (8)\).

    On the other hand, \((6)\) is not primary because \(6 = 2 \cdot 3\), but
\end{example}


\begin{thmbox}
    \begin{theorem}
        Let \(\mathfrak{q}\) be an ideal in \(\mathbb{Z}\). Then \(\mathfrak{q}\) is primary if and only if \(\mathfrak{q} = (0)\) or \(\mathfrak{q} = (p^n)\) for some prime number \(p\) and some \(\mathbb{N}^+\).
    \end{theorem}
\end{thmbox}
\begin{proof}
    ``\(\Rightarrow\)'':
    \begin{enumerate}
        \item Let \(\mathfrak{q}\) be a primary ideal.
        \item Fix \(x \in \mathfrak{q}\).
        \item \(x = p_1^{k_1} \cdots p_n^{k_n}\).
        \item By induction, we must find \(p_i^{k_i} \in \mathfrak{q}\)
        \item Thus \(\mathfrak{q} = (p_i^n)\)
    \end{enumerate}

    ``\(\Leftarrow\)'': trivial
\end{proof}

\begin{example}
    Let \(K\) be a field and \(A = K[X, Y, Z] / (XY - Z^2)\). \(\mathfrak{a} = (X, Z)\).

    \((X, Z)^2\) is not primary in \(A\)

    \((X, Z)^2 / (XY - Z^2) \cong (X^2, XZ, Z^2) / (XY - Z^2)\)

    \(XZ \in (X, Z)^2\)
\end{example}


\begin{thmbox}
    \begin{theorem}
        Let \(\mathfrak{q}\) be a primary ideal. Then \(\sqrt{\mathfrak{q}}\) is a prime ideal in \(R\). It is the unique smallest prime ideal containing \(\mathfrak{q}\).
    \end{theorem}
\end{thmbox}






\begin{thmbox}
    \begin{theorem}
        All prime ideals are primary.
    \end{theorem}
\end{thmbox}
\begin{example}
    Clearly, the converse is false.
\end{example}






\begin{intuition}
    We can think of primary ideals as generalization of prime powers.
\end{intuition}
\begin{example}
    Power of prime ideal need not be a primary ideal.
\end{example}


\begin{exmbox}
    \begin{example}
        \textit{\(\mathfrak{m}\)-primary ideals need not be powers of \(\mathfrak{m}\).}

        Consider the ring \(A = K[X, Y]\) and the ideal \(\mathfrak{q} = (X, Y^2)\).

        \textit{\(X, Y^2\) is primary.} 
    \end{example}
\end{exmbox}





\newpage
\subsection{Primary Decomposition}


\begin{defbox}
    \begin{definition}
        An ideal \(\mathfrak{a}\) is called reducible if \(\mathfrak{a} = \mathfrak{b}_1 \cap \mathfrak{b}_2\) for ideals \(\mathfrak{b}\) and \(\mathfrak{b}'\) strictly containing \(\mathfrak{a}\).

        An irreducible ideal is proper and not reducible.
    \end{definition}
\end{defbox}

\begin{example}
    \((6) \cap (9) = (18)\)
\end{example}



\begin{thmbox}
    \begin{theorem}
        In a nonzero Noetherian ring, each proper ideal is an intersection of finitely many irreducible ideals.
    \end{theorem}
\end{thmbox}

\begin{thmbox}
    \begin{theorem}
        In a nonzero Noetherian ring, each irreducible ideal is a primary ideal.
    \end{theorem}
\end{thmbox}
What about the converse?


\begin{thmbox}
    \begin{theorem}
        In a nonzero Noetherian ring, each proper ideal is an intersection of finitely many primary ideals.
    \end{theorem}
\end{thmbox}
What about the converse?

Steps to prove the result above:
1. 


\begin{example}
    Consider \(K[X, Y, Z]\). Ideals \((X, Y)\), \((X, Z)\), and \((X, Y, Z)\).

    \begin{align}
        (X, Y) \cdot (X, Z) = (X, Y) \cap (X, Z) \cap (X, Y, Z)^2
    \end{align}
\end{example}


\begin{thmbox}
    \begin{theorem}
        Let \(\mathfrak{q}\) be an primary ideal and \(\mathfrak{p} = \sqrt{\mathfrak{q}}\) be the corresponding prime ideal. Then for any \(x \in R\)
        \begin{enumerate}
            \item \((\mathfrak{q} : x) = R\) if \(x \in \mathfrak{q}\). I think this is true for any \(\mathfrak{q}\).
        \end{enumerate}
    \end{theorem}
\end{thmbox}
\begin{proof}
    \begin{enumerate}
        \item 
    \end{enumerate}
\end{proof}


\begin{thmbox}
    \begin{theorem}
        For a decomposable ideal \(\mathfrak{a} \subset R\), let
        \begin{align*}
            \mathfrak{a} = \bigcap_{i=1}^r \mathfrak{q}_i
        \end{align*}
    \end{theorem}
    be a minimal primary decomposition. 
\end{thmbox}




\newpage
\section{Noetherian Modules}

\begin{exmbox}
    \begin{example}
        Let \(R = \mathbb{R}[X_1, X_2, \ldots]\). Let \(I = (X_1, X_2, \ldots)\) be an ideal. \(I\) is the set of polynomials with constant term \(0\). \(I\) is not finitely generated.
    \end{example}
\end{exmbox}
\begin{proof}
    Let \(\mathfrak{a}\)
\end{proof}

\begin{defbox}
    \begin{definition}
        An \(R\)-module \(M\) is called
        \begin{enumerate}
            \item Every ascending chain
            \begin{align*}
                M_1 \subset M_2 \subset \cdots \subset M_i \subset M_{i + 1} \subset \cdots \subset M
            \end{align*}
            of submodules in \(M\) becomes stationary, i.e. there is an index \(i_0 \in \mathbb{N}\) such that \(M_{i_0} = M_i\) for all \(i \geq i_0\).
            \item All submodules of \(M\) are finitely generated.
        \end{enumerate}
    \end{definition}
\end{defbox}

\begin{thmbox}
    \begin{theorem}
        If \(M\) is a Noetherian \(R\)-module then every submodule of \(M\) is Noetherian.
    \end{theorem}
\end{thmbox}
\begin{proof}
    This follows immediately from the definition of Noetherian modules, because a submodule of a submodule is again a submodule.
    \begin{enumerate}
        \item Let \(M\) be a Noetherian \(R\)-module.
        \item Fix a submodule \(N\) of \(M\).
        \item Fix a submodule \(L\) of \(N\).
        \item Since \(L\) is also a submodule of \(M\) and \(M\) is Noetherian, \(L\) is finitely generated.
        \item Thus all submodules of \(N\) are finitely generated and \(N\) is Noetherian.
    \end{enumerate}
\end{proof}
\begin{thmbox}
    \begin{theorem}
        If \(M\) is a Noetherian \(R\)-module then every quotient module \(M / N\) is Noetherian.
    \end{theorem}
\end{thmbox}
\begin{thmbox}
    \begin{theorem}
        Let \(M\) be an \(R\)-module and \(N\) be a submodule. Then \(M\) is Noetherian if and only if \(N\) and \(M / N\) are Noetherian.
    \end{theorem}
\end{thmbox}
\begin{proof}
    ``\(\Rightarrow\)'': see above

    ``\(\Leftarrow\)'':
    \begin{enumerate}
        \item Let \(M\) be an \(R\)-module.
        \item Let \(N\) be a Noetherian submodule in \(M\) such that \(M / N\) is Noetherian.
        \item Fix a submodule \(L\) of \(M\). \textit{(We will show that \(L\) is finitely generated.)}
        \item The image \(\pi(L)\) in \(M / N\) is finitely generated because \(\pi(L)\) is a submodule in \(M / N\) and \(M / N\) is Noetherian.
        \item Let \(x_1, \ldots, x_n \in L\) such that \(\pi(x_1), \ldots \pi(x_n)\) generate \(\pi(L)\).
        \item The intersection \(L \cap N\) is finitely generated because \(L \cap N\) is a submodule in \(N\) and \(N\) is Noetherian.
        \item Let \(y_1, \ldots, y_m \in L\) generate \(L \cap N\).
        \item Fix an element \(x \in L\). \textit{(We will show that \(x\) is a linear combination of geneators in \(L\).)}
        \item We have \(\pi(x) = r_1 \pi(x_1) + \cdots + r_n \pi(x_n) = \sum_{i = 1}^n r_i \pi(x_i)\) for some \(r_1, \ldots, r_n \in R\).
        \item \(\pi(x) - \sum_{i = 1}^n r_i \pi(x_i) = 0\) thus \(x - \sum_{i=1}^n r_i x_i\) lies in \(N\).
        \item Since \(x \in L\) and \(\sum_{i = 1}^n r_i x_i \in L\), their difference \(x - \sum_{i = 1}^n r_i x_i\) lies also in \(L\).
        \item Hence \(x - \sum_{i = 1}^n r_i x_i \in L \cap N\).
        \item Therefore \(x - \sum_{i = 1}^n r_i x_i = \sum_{i=1}^m s_i y_i\).
        \item Thus \(x = \sum_{i = 1}^n r_i x_i + \sum_{i=1}^m s_i y_i\)
    \end{enumerate}
\end{proof}


\newpage
\section{More Noether}
\begin{thmbox}
    \begin{theorem}
        Let
        \begin{align*}
            0 \longrightarrow M' \longrightarrow M \longrightarrow M'' \longrightarrow 0
        \end{align*}
        be an exact sequence of \(R\)-modules. Then \(M\) is Noetherian if and only if \(M'\) and \(M''\) are Noetherian.
    \end{theorem}
\end{thmbox}
\begin{proof}
    ``\(\Rightarrow\)'':
    \begin{enumerate}
        \item Let \(M\) be Noetherian.
        \item By exactness, \(M \rightarrow M''\) is surjective.
        \item Thus \(M''\) is Noetherian.
        \item By exactness, \(M' \rightarrow M\) is injective.
        \item Thus \(M'\) can be regarded as a submodule of \(M\).
        \item Hence \(M'\) is Noetherian.
    \end{enumerate}
    ``\(\Leftarrow\)'':
    \begin{enumerate}
        \item Let \(M'\) be Noetherian.
        \item Let \(M''\) be Noetherian.
        \item Fix a submodule \(N\) of \(M\).
        \item The image \(g(L)\) is finitely generated.
        \item The intersection \(L \cap f(M')\) is finitely generated.
        \item Fix an element \(x \in L\).
        \item \(g(x) = r_1 g(x_1) + \cdots + r_n g(x_n)\)
        \item 
    \end{enumerate}
\end{proof}

\begin{thmbox}
    \begin{theorem}
        Let \(M_1, \ldots, M_n\) be Noetherian \(A\)-modules. Then their direct sum \(\bigoplus_{i=1}^n M_i\) is Noetherian.
    \end{theorem}
\end{thmbox}
\begin{proof}
    It is enough to show \(M_1 \oplus M_2\) is Noetherian. We have the exact sequence
    \begin{align}
        0 \longrightarrow M_1 \longrightarrow M_1 \oplus M_2 \longrightarrow M_2 \longrightarrow 0
    \end{align}
\end{proof}

\end{document}