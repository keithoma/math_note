\documentclass[a4paper]{book}
\title{Commutative Ring Theory}
\author{Kei Thoma}


% ---------------------------------------------------------------------
% P A C K A G E S
% ---------------------------------------------------------------------

% typography and formatting
\usepackage[english]{babel}
\usepackage[utf8]{inputenc}
\usepackage{geometry}
\usepackage{exsheets}
\usepackage{environ}
\usepackage{graphicx}
\usepackage{cutwin}
\usepackage{pifont}

% mathematics
\usepackage{xfrac}  
\usepackage{amsthm} % for theorems, and definitions
\usepackage{amssymb}
\usepackage{amsmath}
\usepackage{textcomp}
\usepackage{mathtools}
% \usepackage{MnSymbol} % for \cupdot

% extra
\usepackage{xcolor}
\usepackage{tikz}

% ---------------------------------------------------------------------
% S E T T I N G
% ---------------------------------------------------------------------

%maybe delete later, for colorbox
\usepackage{tcolorbox}
\newtcolorbox{defbox}{colback=blue!5!white,colframe=blue!75!black}
\newtcolorbox{defboxlight}{colback=cyan!5!white,colframe=cyan!75!black}
\newtcolorbox{thmbox}{colback=orange!5!white,colframe=orange!75!black}
\newtcolorbox{rembox}{colback=purple!5!white,colframe=purple!75!black}
\newtcolorbox{exmbox}{colback=gray!5!white,colframe=gray!75!black}
\newtcolorbox{intbox}{colback=violet!5!white,colframe=violet!75!black}

% typography and formatting
\geometry{margin=2cm, paperheight=60cm}

\SetupExSheets{
  counter-format = ch.qu,
  counter-within = chapter,
  question/print = true,
  solution/print = true,
}

% mathematics
\newcounter{global}

\theoremstyle{definition}
\newtheorem{definition}{Definition}[]
\newtheorem{example}{Example}[definition]

\newtheorem{theorem}[definition]{Theorem}
\newtheorem{corollary}{Corollary}
\newtheorem{lemma}[definition]{Lemma}
\newtheorem{proposition}[definition]{Proposition}

\newtheorem*{remark}{Remark}
\newtheorem*{intuition}{Intuition}

% extra
\definecolor{mathif}{HTML}{0000A0} % for conditions
\definecolor{maththen}{HTML}{CC5500} % for consequences
\definecolor{mathrem}{HTML}{8b008b} % for notes
\definecolor{mathobj}{HTML}{008800}

\usetikzlibrary{positioning}
\usetikzlibrary{shapes.geometric, arrows}

% ---------------------------------------------------------------------
% C O M M A N D S
% ---------------------------------------------------------------------

\newcommand{\norm}[1]{\left\lVert#1\right\rVert}
\newcommand{\rank}{\text{rank}}
\newcommand{\Vol}{\text{Vol}}

\newcommand{\set}[1]{\left\{\, #1 \,\right\}}
\newcommand{\makeset}[2]{\left\{\, #1 \mid #2 \,\right\}}

\newcommand*\diff{\mathop{}\!\mathrm{d}}
\newcommand*\Diff{\mathop{}\!\mathrm{D}}

\newcommand\restr[2]{{% we make the whole thing an ordinary symbol
  \left.\kern-\nulldelimiterspace % automatically resize the bar with \right
  #1 % the function
  \vphantom{\big|} % pretend it's a little taller at normal size
  \right|_{#2} % this is the delimiter
  }}

% ---------------------------------------------------------------------
% R E N D E R
% ---------------------------------------------------------------------

\newif\ifshowproof
\showprooftrue

\NewEnviron{Proof}{%
    \ifshowproof%
        \begin{proof}%
            \BODY
        \end{proof}%
    \fi%
}%


\begin{document}
\begin{thmbox}
    \begin{theorem}
        Let \(M\) be a finitely generated \(R\)-module and \(I \subset \text{Jac}(R)\) an ideal such that \(IM = M\). Then \(M = 0\).
    \end{theorem}
\end{thmbox}
\begin{proof}
    We prove the statement through contradiction.
    \begin{enumerate}
        \item Assume \(M \neq 0\).
        \item Since \(M\) is finitely generated, it has a generating system \(m_1, \ldots, m_n \in M\). Assert that this generating system is minimal.
        \item With \(IM = M\), there is an equation \begin{align*}
            m_1 = a_1 m_1 + \cdots a_n m_n
        \end{align*}
        for some \(a_1, \ldots, a_n \in I\).
        \item We may rewrite \begin{align*}
            m_1 = a_1 m_1 + \cdots a_n m_n &\iff m_1 - a_1 m_1 = a_2 m_2 + \cdots + a_n m_n \\
            &\iff (1 - a_1) m_1 = a_2 m_2 + \cdots + a_n m_n
        \end{align*}
        \item Since \(a_1 \in I \subset \text{Jac}(A)\), \(1 - a_1\) is a unit in \(R\).
        \item We arrive at a contradiction with the minimality of the genrating system.
    \end{enumerate}
\end{proof}




\begin{thmbox}
    \begin{theorem}
        Let \(M\) be a finitely generated \(R\)-module, \(I \subset \text{Jac}(R)\) an ideal, and \(N\) a submodule of \(M\) such that \(M = N + IM\). Then \(M = N\).
    \end{theorem}
\end{thmbox}
\begin{proof}
    \begin{enumerate}
        \item \(M = N + IM\) implies \(M/N = (N + IM) / N = 0 + I (M/N) = I(M/N)\).
        \item \(M/N\) is finitely generated.
        \item Applying Nakayama yields
        \item \(M/N = 0\)
        \item Thus \(M = N\).
    \end{enumerate}
\end{proof}



\begin{thmbox}
    \begin{theorem}
        Let \(R\) be a local ring with maximal ideal \(\mathfrak{m}\) and \(M\) a finitely generated \(R\)-module. 
        \begin{enumerate}
            \item Then \(M/\mathfrak{m}M\) is a vector space over the field \(R/\mathfrak{m}\).
            \item If \(x_1, \ldots, x_n \in M\) are elements such that \(x_1, \ldots, x_n \in M/\mathfrak{m}M\) generates this vector space, then \(M = \sum_{i=1}^n Rx_i\)
        \end{enumerate}
    \end{theorem}
\end{thmbox}


\begin{thmbox}
    \begin{theorem}
        Let \(I\) be an nilpotent ideal in \(R\) and \(M\) be an \(R\)-module such that \(IM = M\). Then \(M = 0\).
    \end{theorem}
\end{thmbox}
\begin{proof}
    \begin{enumerate}
        \item \(IM = M\) implies \(I \cdot IM = IM = M\).
        \item Induction yields \(I^n M = M = 0\).
    \end{enumerate}
\end{proof}


\end{document}