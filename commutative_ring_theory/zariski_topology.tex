\documentclass[a4paper]{book}
\title{Commutative Ring Theory}
\author{Kei Thoma}


% ---------------------------------------------------------------------
% P A C K A G E S
% ---------------------------------------------------------------------

% typography and formatting
\usepackage[english]{babel}
\usepackage[utf8]{inputenc}
\usepackage{geometry}
\usepackage{exsheets}
\usepackage{environ}
\usepackage{graphicx}
\usepackage{cutwin}
\usepackage{pifont}

% mathematics
\usepackage{xfrac}  
\usepackage{amsthm} % for theorems, and definitions
\usepackage{amssymb}
\usepackage{amsmath}
\usepackage{textcomp}
\usepackage{mathtools}
% \usepackage{MnSymbol} % for \cupdot

% extra
\usepackage{xcolor}
\usepackage{tikz}

% ---------------------------------------------------------------------
% S E T T I N G
% ---------------------------------------------------------------------

%maybe delete later, for colorbox
\usepackage{tcolorbox}
\newtcolorbox{defbox}{colback=blue!5!white,colframe=blue!75!black}
\newtcolorbox{defboxlight}{colback=cyan!5!white,colframe=cyan!75!black}
\newtcolorbox{thmbox}{colback=orange!5!white,colframe=orange!75!black}
\newtcolorbox{rembox}{colback=purple!5!white,colframe=purple!75!black}
\newtcolorbox{exmbox}{colback=gray!5!white,colframe=gray!75!black}
\newtcolorbox{intbox}{colback=violet!5!white,colframe=violet!75!black}

% typography and formatting
\geometry{margin=2cm, paperheight=60cm}

\SetupExSheets{
  counter-format = ch.qu,
  counter-within = chapter,
  question/print = true,
  solution/print = true,
}

% mathematics
\newcounter{global}

\theoremstyle{definition}
\newtheorem{definition}{Definition}[]
\newtheorem{example}{Example}[definition]

\newtheorem{theorem}[definition]{Theorem}
\newtheorem{corollary}{Corollary}
\newtheorem{lemma}[definition]{Lemma}
\newtheorem{proposition}[definition]{Proposition}

\newtheorem*{remark}{Remark}
\newtheorem*{intuition}{Intuition}

% extra
\definecolor{mathif}{HTML}{0000A0} % for conditions
\definecolor{maththen}{HTML}{CC5500} % for consequences
\definecolor{mathrem}{HTML}{8b008b} % for notes
\definecolor{mathobj}{HTML}{008800}

\usetikzlibrary{positioning}
\usetikzlibrary{shapes.geometric, arrows}

% ---------------------------------------------------------------------
% C O M M A N D S
% ---------------------------------------------------------------------

\newcommand{\norm}[1]{\left\lVert#1\right\rVert}
\newcommand{\rank}{\text{rank}}
\newcommand{\Vol}{\text{Vol}}

\newcommand{\set}[1]{\left\{\, #1 \,\right\}}
\newcommand{\makeset}[2]{\left\{\, #1 \mid #2 \,\right\}}

\newcommand*\diff{\mathop{}\!\mathrm{d}}
\newcommand*\Diff{\mathop{}\!\mathrm{D}}

\newcommand\restr[2]{{% we make the whole thing an ordinary symbol
  \left.\kern-\nulldelimiterspace % automatically resize the bar with \right
  #1 % the function
  \vphantom{\big|} % pretend it's a little taller at normal size
  \right|_{#2} % this is the delimiter
  }}

% ---------------------------------------------------------------------
% R E N D E R
% ---------------------------------------------------------------------

\newif\ifshowproof
\showprooftrue

\NewEnviron{Proof}{%
    \ifshowproof%
        \begin{proof}%
            \BODY
        \end{proof}%
    \fi%
}%


\begin{document}

\begin{defbox}
    \begin{definition}
        Let \(\mathfrak{a}\) be an ideal in \(R\). The variety \(\boldsymbol{V}(\mathfrak{a})\) of \(\mathfrak{a}\) is the subset of \(\text{Spec}(R)\) consisting of all prime ideals that contain \(\mathfrak{a}\), i.e.
        \begin{align*}
            \boldsymbol{V}(\mathfrak{a}) := \makeset{\mathfrak{p} \in \text{Spec}(R)}{\mathfrak{a} \subset \mathfrak{p}} \text{.}
        \end{align*}
    \end{definition}
\end{defbox}


\begin{defbox}
    \begin{definition}
        Given an \(f \in R\), we define the distinguished or basic set
        \begin{align*}
            D(f) := \text{Spec}(R) \setminus V(f)
        \end{align*}
    \end{definition}
\end{defbox}

\begin{thmbox}
    \begin{theorem}
        For \(f, g \in R\), we have:
        \begin{enumerate}
            \item \(D(f) \cap D(g) = D(f \cdot g)\).
            \item \(D(f) = \emptyset \iff f\) is nilpotent.
            \item \(D(f) = \text{Spec}(A) \iff f \in A^\times\)
        \end{enumerate}
    \end{theorem}
\end{thmbox}

\begin{proof}
    \begin{enumerate}
        \item Let \(\mathfrak{p} \in D(f) \cap D(g)\).
        \item We have \begin{align*}
            \mathfrak{p} \in D(f) \cap D(g) &\iff \mathfrak{p} \in (\text{Spec}(R) \setminus V(f)) \cap (\text{Spec}(R) \setminus V(g)) \\
            &\iff \mathfrak{p} \in \text{Spec}(R) \setminus V(f) \text{ and } \mathfrak{p} \in \text{Spec}(R) \setminus V(g) \\
            &\iff \mathfrak{p} \not\in V(f) \text{ and } \mathfrak{p} \not\in V(g) \\
            &\iff \mathfrak{p} \not\in \makeset{\mathfrak{p} \in \text{Spec}(R)}{(f) \subset \mathfrak{p}} \text{ and } \mathfrak{p} \not\in \makeset{\mathfrak{p} \in \text{Spec}(R)}{(g) \subset \mathfrak{p}} \\
            &\iff (f) \not\subset \mathfrak{p} \text{ and } (g) \not\subset \mathfrak{p} \\
            &\iff (f)(g) \not\subset \mathfrak{p}
        \end{align*}
    \end{enumerate}

    2.

    \begin{align*}
        D(f) = \emptyset &\iff \text{Spec}(R) \setminus V(f) = \emptyset\\
        &\iff \text{Spec}(R) = V(f) \\
        &\iff \text{All prime ideals \(\mathfrak{p}\) contain \((f)\)}. \\
        &\iff (f) \subset \bigcap_{\mathfrak{p} \in \text{Spec}(R)}\mathfrak{p} = \text{Nil}(R)
    \end{align*}
\end{proof}


\begin{thmbox}
    \begin{theorem}
        The set \(\{\mathfrak{p}\}\) is closed in \(\text{Spec}(R)\) if and only if \(\mathfrak{p}\) is a maximal ideal.
    \end{theorem}
\end{thmbox}
\begin{proof}
    \begin{enumerate}
        \item Let \(\{\mathfrak{p}\}\) be a closed subset in \(\text{Spec}(R)\).
        \item By definition, \(\{\mathfrak{p}\} = V(\mathfrak{a})\) for some ideal \(\mathfrak{a}\) in \(R\).
        \item If \(\mathfrak{p}\) is not maximal, it is contained in a maximal ideal \(\mathfrak{m}\).
        \item Thus \(V(\mathfrak{a}) \supset \{\mathfrak{p}, \mathfrak{m}\}\)
    \end{enumerate}

    \begin{enumerate}
        \item trivial
    \end{enumerate}
\end{proof}


\begin{thmbox}
    \begin{theorem}
        The closure of \(\{\mathfrak{p}\}\) is \(V(\mathfrak{p})\).
    \end{theorem}
\end{thmbox}
\begin{proof}
    \(\overline{\{\mathfrak{p}\}} \subset V(\mathfrak{p})\)
    \begin{enumerate}
        \item trivial since \(\mathfrak{p} \subset V(\mathfrak{p})\)
    \end{enumerate}
    \(V(\mathfrak{p}) \subset \overline{\{\mathfrak{p}\}}\)
    \begin{enumerate}
        \item Let \(\mathfrak{p}' \in V(\mathfrak{p}) = \makeset{\mathfrak{p}' \in \text{Spec}(R)}{\mathfrak{p} \subset \mathfrak{p}'}\).
        \item So \(\mathfrak{p} \subset \mathfrak{p}'\).
        \item I think the idea is simply \(\overline{\{\mathfrak{p}\}} = V(\mathfrak{a})\)
    \end{enumerate}
\end{proof}

Is \(\text{Spec}(R)\) a Kolmogorov-space?

Let \(\mathfrak{p}_1, \mathfrak{p}_2 \in \text{Spec}(R)\) with \(\mathfrak{p}_1 \neq \mathfrak{p}_2\).


\begin{theorem}
    \(X\) is irreducible if and only if every pair of non-empty open sets in \(X\) has a non-empty intersection.
\end{theorem}
\begin{proof}
    \begin{enumerate}
        \item Let \(X\) be irreducible, so for any closed subsets \(X_1\) and \(X_2\) with \(X =  X_1 \cup X_2\) implies \(X = X_1\) or \(X = X_2\).
        \item Let \(A\) and \(B\) be two non-empty open subsets.
        \item Then \(X \setminus A\) and \(X \setminus B\) is closed.
        \item We have \((X \setminus A) \cup (X \setminus B) = X \setminus (A \cap B)\)
        \item Assume \(A \cap B = \emptyset\).
        \item contradiction.
    \end{enumerate}
-----------------
    \begin{enumerate}
        \item Every pair of non-empty open sets in \(X\) has a non-empty intersection.
        \item Assume \(X\) is reducible, i.e. \(X = X_1 \cup X_2\)
        \item \(X \setminus X_1\) and \(X \setminus X_2\) are open.
        \item \((X \setminus X_1) \cap (X \setminus X_2) = X \setminus (X_1 \cup X_2) = \emptyset\)
        \item contradiction
    \end{enumerate}
\end{proof}

\begin{theorem}
    \(X\) is irreducible if and only if every non-empty open subset of \(X\) is dense in \(X\).
\end{theorem}
\begin{proof}
    \begin{enumerate}
        \item Let \(X\) be irreducible.
        \item Fix an open subset \(A\).
        \item \(X \setminus A\) is closed.
        \item \(\text{cl}(A)\) is closed.
        \item \(X = (X \setminus A) \cup \text{cl}(A)\), so \(X = \text{cl}(A)\).
    \end{enumerate}
\end{proof}

\begin{theorem}
    \(\text{Spec}(R)\) is irreducible if and only if the nilradical of \(A\) is a prime ideal.
\end{theorem}
\begin{proof}
    \begin{enumerate}
        \item Let \(\text{Spec}(R)\) be irreducible.
        \item Consider \(\bigcap_{\mathfrak{p} \in \text{Spec}(R)} \mathfrak{p}\)
        \item \(\bigcap_{\mathfrak{p} \in \text{Spec}(R)} \mathfrak{p} = \bigcap_{\mathfrak{m}} V(\mathfrak{m})\)
        \item \(X \setminus \bigcap_{\mathfrak{m}} V(\mathfrak{m}) = \bigcup_{\mathfrak{m}}(X \setminus V(\mathfrak{m}))\)
        \item \(V(\mathfrak{m}) = \emptyset\)
    \end{enumerate}
\end{proof}


\end{document}