\documentclass[a4paper]{book}
\title{Commutative Ring Theory}
\author{Kei Thoma}


% ---------------------------------------------------------------------
% P A C K A G E S
% ---------------------------------------------------------------------

% typography and formatting
\usepackage[english]{babel}
\usepackage[utf8]{inputenc}
\usepackage{geometry}
\usepackage{exsheets}
\usepackage{environ}
\usepackage{graphicx}
\usepackage{cutwin}
\usepackage{pifont}

% mathematics
\usepackage{xfrac}  
\usepackage{amsthm} % for theorems, and definitions
\usepackage{amssymb}
\usepackage{amsmath}
\usepackage{textcomp}
\usepackage{mathtools}
% \usepackage{MnSymbol} % for \cupdot

% extra
\usepackage{xcolor}
\usepackage{tikz}

% ---------------------------------------------------------------------
% S E T T I N G
% ---------------------------------------------------------------------

%maybe delete later, for colorbox
\usepackage{tcolorbox}
\newtcolorbox{defbox}{colback=blue!5!white,colframe=blue!75!black}
\newtcolorbox{defboxlight}{colback=cyan!5!white,colframe=cyan!75!black}
\newtcolorbox{thmbox}{colback=orange!5!white,colframe=orange!75!black}
\newtcolorbox{rembox}{colback=purple!5!white,colframe=purple!75!black}
\newtcolorbox{exmbox}{colback=gray!5!white,colframe=gray!75!black}
\newtcolorbox{intbox}{colback=violet!5!white,colframe=violet!75!black}

% typography and formatting
\geometry{margin=2cm, paperheight=60cm}

\SetupExSheets{
  counter-format = ch.qu,
  counter-within = chapter,
  question/print = true,
  solution/print = true,
}

% mathematics
\newcounter{global}

\theoremstyle{definition}
\newtheorem{definition}{Definition}[]
\newtheorem{example}{Example}[definition]

\newtheorem{theorem}[definition]{Theorem}
\newtheorem{corollary}{Corollary}
\newtheorem{lemma}[definition]{Lemma}
\newtheorem{proposition}[definition]{Proposition}

\newtheorem*{remark}{Remark}
\newtheorem*{intuition}{Intuition}

% extra
\definecolor{mathif}{HTML}{0000A0} % for conditions
\definecolor{maththen}{HTML}{CC5500} % for consequences
\definecolor{mathrem}{HTML}{8b008b} % for notes
\definecolor{mathobj}{HTML}{008800}

\usetikzlibrary{positioning}
\usetikzlibrary{shapes.geometric, arrows}

% ---------------------------------------------------------------------
% C O M M A N D S
% ---------------------------------------------------------------------

\newcommand{\norm}[1]{\left\lVert#1\right\rVert}
\newcommand{\rank}{\text{rank}}
\newcommand{\Vol}{\text{Vol}}

\newcommand{\set}[1]{\left\{\, #1 \,\right\}}
\newcommand{\makeset}[2]{\left\{\, #1 \mid #2 \,\right\}}

\newcommand*\diff{\mathop{}\!\mathrm{d}}
\newcommand*\Diff{\mathop{}\!\mathrm{D}}

\newcommand\restr[2]{{% we make the whole thing an ordinary symbol
  \left.\kern-\nulldelimiterspace % automatically resize the bar with \right
  #1 % the function
  \vphantom{\big|} % pretend it's a little taller at normal size
  \right|_{#2} % this is the delimiter
  }}

% ---------------------------------------------------------------------
% R E N D E R
% ---------------------------------------------------------------------

\newif\ifshowproof
\showprooftrue

\NewEnviron{Proof}{%
    \ifshowproof%
        \begin{proof}%
            \BODY
        \end{proof}%
    \fi%
}%

\begin{document}




\section{Radical of an Ideal}
\begin{example}
  \(\sqrt{(X^2Y, XY^2)} = (XY)\)
\end{example}
\begin{proof}
  \begin{enumerate}
    \item Let \(x \in \sqrt{(X^2Y, XY^2)}\).
    \item Assume \(x \not\in (XY)\).
  \end{enumerate}

  ------------------------------------

  \begin{enumerate}
    \item We want to show \(XY \in \sqrt{(X^2Y, XY^2)}\).
    \item \(\iff (XY)^2 \in (X^2Y, XY^2)\)
    \item And this is true because we have \(X^2Y^2 = X^2Y \cdot Y\)
  \end{enumerate}

  ---------------------------------

  \begin{enumerate}
    \item We want to show \(X \in \sqrt{(X^2Y, XY^2)}\).
    \item \(\iff X^2 \)
  \end{enumerate}

  Apparantly there are no easy way to

  Maybe another way.
  \begin{enumerate}
    \item What are the nilpotent elements of \(K[X, Y] / (X^2Y, XY^2)\)?
    \item 
  \end{enumerate}

  This is basically asking the same question ...
\end{proof}

\section{Ideal Quotient}
\begin{defbox}
  \begin{definition}
    Let \(I\) be an ideal in \(R\) and \(J\) an element in \(R\). The ideal quotient is
    \begin{align*}
      (I : J) = \makeset{r \in R}{rJ \subset I}
    \end{align*}
    Moreover
    \begin{align*}
      (I : x) = \makeset{r \in R}{rx \subset I}
    \end{align*}
  \end{definition}
\end{defbox}




\begin{thmbox}
  \begin{theorem}
    The ideal quotient are ideals.
  \end{theorem}
\end{thmbox}
\begin{proof}
  Let \(x \in (I : J)\) and \(r \in R\). Then \(xJ \subset I\) by definition. We also have \(rxJ \subset xJ \subset I\).

  Let \(x, y \in (I : J)\). Then \((x + y)J = xJ + yJ \subset I\).
\end{proof}


\begin{example}
  \begin{enumerate}
    \item \(((2) : 3) = \makeset{r \in \mathbb{Z}}{3r \in (2)} \Rightarrow ((2) : 3) = (2)\).
    \item \(((6) : 3) = \makeset{r \in \mathbb{Z}}{3r \in (6)} \Rightarrow ((6) : 3) = (2)\).
    \item \(((8) : 2) = \makeset{r \in \mathbb{Z}}{2r \in (8)} \Rightarrow ((8) : 2) = (4)\).
  \end{enumerate}
\end{example}


\begin{example}
  Let \(I = (X^2Y, XY^2)\) be an ideal in the polynomial ring \(R = K[X, Y]\) over a field \(K\).
  \begin{enumerate}
    \item \(((X^2Y, XY^2) : XY) = \makeset{r \in R}{r \cdot XY \subset (X^2Y, XY^2)} = (X, Y) \Rightarrow \sqrt{(X, Y)} = (X, Y)\).
    \item \(((X^2Y, XY^2) : (X^2, XY)) = (Y) \Rightarrow \sqrt{(Y)} = (Y)\).
    \item \(((X^2Y, XY^2) : (XY, Y^2)) = (X) \Rightarrow \sqrt{(X)} = (X)\).
  \end{enumerate}
  It is \(\sqrt{(X^2Y, XY^2)} = (XY)\).


\end{example}

\begin{theorem}
  
\end{theorem}




\end{document}