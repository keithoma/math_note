\documentclass[a4paper]{book}
\title{Commutative Ring Theory}
\author{Kei Thoma}


% ---------------------------------------------------------------------
% P A C K A G E S
% ---------------------------------------------------------------------

% typography and formatting
\usepackage[english]{babel}
\usepackage[utf8]{inputenc}
\usepackage{geometry}
\usepackage{exsheets}
\usepackage{environ}
\usepackage{graphicx}
\usepackage{cutwin}
\usepackage{pifont}

% mathematics
\usepackage{xfrac}  
\usepackage{amsthm} % for theorems, and definitions
\usepackage{amssymb}
\usepackage{amsmath}
\usepackage{textcomp}
\usepackage{mathtools}
% \usepackage{MnSymbol} % for \cupdot

% extra
\usepackage{xcolor}
\usepackage{tikz}

% ---------------------------------------------------------------------
% S E T T I N G
% ---------------------------------------------------------------------

%maybe delete later, for colorbox
\usepackage{tcolorbox}
\newtcolorbox{defbox}{colback=blue!5!white,colframe=blue!75!black}
\newtcolorbox{defboxlight}{colback=cyan!5!white,colframe=cyan!75!black}
\newtcolorbox{thmbox}{colback=orange!5!white,colframe=orange!75!black}
\newtcolorbox{rembox}{colback=purple!5!white,colframe=purple!75!black}
\newtcolorbox{exmbox}{colback=gray!5!white,colframe=gray!75!black}
\newtcolorbox{intbox}{colback=violet!5!white,colframe=violet!75!black}

% typography and formatting
\geometry{margin=3cm}

\SetupExSheets{
  counter-format = ch.qu,
  counter-within = chapter,
  question/print = true,
  solution/print = true,
}

% mathematics
\newcounter{global}

\theoremstyle{definition}
\newtheorem{definition}{Definition}[]
\newtheorem{example}{Example}[definition]

\newtheorem{theorem}[definition]{Theorem}
\newtheorem{corollary}{Corollary}
\newtheorem{lemma}[definition]{Lemma}
\newtheorem{proposition}[definition]{Proposition}

\newtheorem*{remark}{Remark}
\newtheorem*{intuition}{Intuition}

% extra
\definecolor{mathif}{HTML}{0000A0} % for conditions
\definecolor{maththen}{HTML}{CC5500} % for consequences
\definecolor{mathrem}{HTML}{8b008b} % for notes
\definecolor{mathobj}{HTML}{008800}

\usetikzlibrary{positioning}
\usetikzlibrary{shapes.geometric, arrows}

% ---------------------------------------------------------------------
% C O M M A N D S
% ---------------------------------------------------------------------

\newcommand{\norm}[1]{\left\lVert#1\right\rVert}
\newcommand{\rank}{\text{rank}}
\newcommand{\Vol}{\text{Vol}}

\newcommand{\set}[1]{\left\{\, #1 \,\right\}}
\newcommand{\makeset}[2]{\left\{\, #1 \mid #2 \,\right\}}

\newcommand*\diff{\mathop{}\!\mathrm{d}}
\newcommand*\Diff{\mathop{}\!\mathrm{D}}

\newcommand\restr[2]{{% we make the whole thing an ordinary symbol
  \left.\kern-\nulldelimiterspace % automatically resize the bar with \right
  #1 % the function
  \vphantom{\big|} % pretend it's a little taller at normal size
  \right|_{#2} % this is the delimiter
  }}

% ---------------------------------------------------------------------
% R E N D E R
% ---------------------------------------------------------------------

\newif\ifshowproof
\showprooftrue

\NewEnviron{Proof}{%
    \ifshowproof%
        \begin{proof}%
            \BODY
        \end{proof}%
    \fi%
}%

\begin{document}
\section{Localization}
\begin{defbox}
    \begin{definition}
        A multiplicative set is a subset \(S\) of a ring \(R\) such that the following two conditions hold.
        \begin{enumerate}
            \item \(S\) contains the identity, i.e. \(1 \in S\).
            \item For any two elements \(x \in S\) and \(y \in S\) it is their prodcut is contained in \(S\), i.e. \(xy \in S\).
        \end{enumerate}
    \end{definition}
\end{defbox}

\begin{exmbox}
    \begin{example}
        In any ring \(R\), the subsets \(\{1\}\), and \(R\) are multiplicative sets.
    \end{example}
\end{exmbox}

\begin{example}
    Consider the ring of integers \(\mathbb{Z}\). Some of the multiplicative sets in \(\mathbb{Z}\) includes
    \begin{itemize}
        \item \(\{1, 2, 4, 8, \ldots\} = \makeset{2^n}{n \in \mathbb{N}_0}\)
        \item \(\{1, 3, 9, 27, \ldots\} = \makeset{3^n}{n \in \mathbb{N}_0}\)
        \item \(\{1, 2, 3, 4, 6, 8, 9, 12, 16, 18, \ldots\} = \makeset{2^n \cdot 3^m}{n, m \in \mathbb{N}_0}\)
        \item \(\{1, 4, 16, 64, \ldots\} = \makeset{4^n}{n \in \mathbb{N}}\)
        \item \(\{1, -2, 4, -8, \ldots\} = \makeset{(-2)^n}{n \in \mathbb{N}}\)
        \item \(\{1, -1, 2, -2, 4, -4, 8, -8, \ldots\} = \makeset{\pm 2^n}{n \in \mathbb{N}}\)
    \end{itemize}
\end{example}

Observation: Given some elements of the ring, we may generate the multiplicative set.

\begin{example}
    \begin{itemize}
        \item In \(\mathbb{Z} / 2 \mathbb{Z}\), the only multiplicative sets are \(\{1\}\) and \(\mathbb{Z} / 2 \mathbb{Z}\)
        

        \item In \(\mathbb{Z} / 3 \mathbb{Z}\), the multiplicative sets are \(\{1\}\), \(\{1, 0\}\), \(\{1, 2\}\) and \(\mathbb{Z} / 2 \mathbb{Z}\)
        
        
        \item In \(\mathbb{Z} / 4 \mathbb{Z}\), the multiplicative sets are \(\{1\}\), \(\{1, 0\}\), \(\{1, 2, 0\}\), \(\{1, 3\}\), and \(\mathbb{Z} / 4 \mathbb{Z}\)
        
        
        \item In \(\mathbb{Z} / 5 \mathbb{Z}\), the multiplicative sets are \(\{1\}\), \(\{1, 0\}\), and \(\{0, 1, 2, 3, 4\}\)
        \begin{enumerate}
            \item \(S(\{1, 2\}) = \{1, 2, 3, 4\}\)
            \item \(S(\{1, 3\}) = \{1, 2, 3, 4\}\)
            \item \(S(\{1, 4\}) = \{1, 4\}\)
        \end{enumerate}

        \item In \(\mathbb{Z} / 6 \mathbb{Z}\), the multiplicative sets are \(\{1\}\), \(\{1, 0\}\)
        \begin{itemize}
            \item \(S(\{1, 2\}) = \{1, 2, 4\}\)
            \item \(S(\{1, 3\}) = \{1, 3\}\)
            \item \(S(\{1, 4\}) = \{1, 4\}\)
            \item \(S(\{1, 5\}) = \{1, 5\}\)
            \item \(S(\{1, 2, 3\}) = \{0, 1, 2, 3, 4\}\)
            \item \(S(\{1, 2, 5\}) = \{1, 2, 4, 5\}\)
            \item \(S(\{1, 3, 4\}) = \{0, 1, 3, 4\}\)
            \item \(S(\{1, 3, 5\}) = \{1, 3, 5\}\)
            \item \(S(\{1, 4, 5\}) = \{1, 2, 4, 5\}\)
        \end{itemize}

        \item In \(\mathbb{Z} / 7 \mathbb{Z}\), the multiplicative sets are
        \begin{itemize}
            \item \(S(\{1, 2\}) = \{1, 2, 4\}\)
            \item \(S(\{1, 3\}) = \{1, 2, 3, 4, 5, 6\}\)
            \item \(S(\{1, 4\}) = \{1, 2, 4\}\)
            \item \(S(\{1, 5\}) = \{1, 2, 3, 4, 5, 6\}\)
            \item \(S(\{1, 6\}) = \{1, 6\}\)
            \item \(S(\{1, 2, 3\}) = \{1, 2, 3, 4, 5, 6\}\)
            \item \(S(\{1, 2, 5\}) = \{1, 2, 3, 4, 5, 6\}\)
            \item \(S(\{1, 2, 6\}) = \{1, 2, 3, 4, 5, 6\}\)
            \item \(S(\{1, 3, 4\}) = \{1, 2, 3, 4, 5, 6\}\)
            \item \(S(\{1, 3, 5\}) = \{1, 2, 3, 4, 5, 6\}\)
            \item \(S(\{1, 3, 6\}) = \{1, 2, 3, 4, 5, 6\}\)
            \item \(S(\{1, 4, 5\}) = \{1, 2, 3, 4, 5, 6\}\)
            \item \(S(\{1, 4, 6\}) = \{1, 2, 3, 4, 5, 6\}\)
            \item \(S(\{1, 5, 6\}) = \{1, 2, 3, 4, 5, 6\}\)
        \end{itemize}

        \item In \(\mathbb{Z} / 12 \mathbb{Z}\)
        \begin{itemize}
            \item \(S(\{1, 2\}) = \{1, 2, 4, 8\}\)
            \item \(S(\{1, 3\}) = \{1, 3, 9\}\)
            \item \(S(\{1, 4\}) = \{1, 4\}\)
        \end{itemize}
    \end{itemize}
\end{example}

Well, I wrote a python script for this

\begin{rembox}
    \begin{remark}
        The statement: If \(S\) is a multiplicative set with \(0\), then \(S \setminus \{0\}\) is a multiplicative set is false.

        The converse is (probably) true: If \(S\) is multiplicative, then so is \(S \cup \{0\}\).
    \end{remark}
\end{rembox}


\begin{rembox}
    \begin{remark}
        A multiplicative set need not to have a finite generating set.
    \end{remark}
\end{rembox}

Question: Do multiplicative sets have countably finite generating sets?

Answer: Most certainly it isn't. There is no way that \(R\) has a countable set of elements that generate \(R\) just through multiplication.

\begin{thmbox}
    \begin{theorem}
        An ideal \(\mathfrak{p}\) of a ring \(R\) is prime if and only if its complement \(R \setminus \mathfrak{p}\) is multiplicativly closed.
    \end{theorem}
\end{thmbox}

\begin{rembox}
    \begin{remark}
        Not all multiplicative sets are the complements of a prime ideal. The above theorem gives an alternative definition of prime ideals but does not function as a definition of multiplicative sets.
    \end{remark}
\end{rembox}

\begin{proof}
    Let \(\mathfrak{p}\) be an ideal of a ring \(R\).

    \noindent``\(\Rightarrow\)'': Suppose \(\mathfrak{p}\) is prime. Since \(\mathfrak{p}\) does not contain \(1\), its complement \(R \setminus \mathfrak{p}\) must contain \(1\). Fix two elements \(x \in R \setminus \mathfrak{p}\) and \(y \in R \setminus \mathfrak{p}\) and assume \(xy \not\in R \setminus \mathfrak{p}\). Then, \(xy \in \mathfrak{p}\), but then \(x \in \mathfrak{p}\) or \(y \in \mathfrak{p}\) both of which are contradictions.

    \noindent ``\(\Leftarrow\)'': Suppose \(R \setminus \mathfrak{p}\) is multiplicativly closed. 
\end{proof}

\begin{example}
    Consider the ring of integers \(\mathbb{Z}\). Some of the multiplicative sets in \(\mathbb{Z}\) includes
    \begin{itemize}
        \item \(\mathbb{Z} \setminus (2) = \{\ldots, -5, -3, -1, 1, 3, 5, \ldots\}\) this corresponds to the high school math observation that odd times odd is odd
    \end{itemize}
\end{example}

\begin{thmbox}
    \begin{theorem}
        If \(\mathfrak{a}\) is an ideal in \(R\), then \(1 + \mathfrak{a}\) is a multiplicative set.
    \end{theorem}
\end{thmbox}

What about the converse? Let \(1 + \mathfrak{a}\) be multiplicative. 

\newpage
\section{Localization}
\begin{defbox}
    \begin{definition}
        
    \end{definition}
\end{defbox}

\begin{example}
    \begin{itemize}
        \item Localization of \(\mathbb{Z}\) at \(S:= \{1, 2, 4, 8, \ldots\}\)
        \begin{itemize}
            \item some elements: \(\frac{1}{2}, \frac{1}{4}, \frac{1}{8}\)
            \item it is \(\frac{1}{2} = \frac{2}{4}\) because \(1 \cdot 4 - 2 \cdot 2 = 0\)
        \end{itemize}
        \item Localization of \(\mathbb{Z}\) at \(S:= \{1, -2, 4, -8, \ldots\}\)
        \item some elements: \(\frac{-1}{-2}\), \(\frac{2}{4}\)
        \item it is \(\frac{-1}{-2} = \frac{2}{4}\) because \(-1 \cdot 4 = -2 \cdot 2\)
    \end{itemize}
\end{example}

\begin{defbox}
    \begin{definition}
        \begin{align*}
            \tau: R \longrightarrow S^{-1}R, x \mapsto \frac{x}{1}
        \end{align*}
    \end{definition}
\end{defbox}

\begin{rembox}
    \begin{remark}
        \(\tau\) is not injective
    \end{remark}
\end{rembox}
Why??

\begin{thmbox}
    \begin{theorem}
        For \(\tau: R \longrightarrow S^{-1}R\) it is true:
        \begin{enumerate}
            \item \(\text{ker}\tau = \makeset{x \in R}{sx = 0 \text{ for some } x \in S}\). This means that the kernel is a subset of the zero divisor
            \item For all \(s \in S\), we have \(\tau(s)\) is a unit in the Localization.
            \item \(S^{-1}R \neq 0\) iff \(0 \not\in S\)
            \item if \(S\) contains only units, \(\tau\) bijective
        \end{enumerate}
    \end{theorem}
\end{thmbox}
\begin{proof}
    \begin{enumerate}
        \item Let \(x \in \text{ker}(\tau)\), then \(\tau(x) = \frac{x}{1} \sim \frac{0}{1}\) \(\iff\) \(xs = 0\)
        \item Let \(s \in S\), then \(\tau(s) = \frac{s}{1}\). We want to find \(\frac{p}{q}\) such that \(\frac{sp}{q} \sim \frac{1}{1}\) so \((sp - q)t = 0 \iff spt - qt = 0\)
        
        Yeah, so simply:

        Let \(s \in S\). The inverse of \(\tau(s) = \frac{s}{1}\) is \(\frac{1}{s}\) as \(\frac{s}{s} \sim \frac{1}{1}\)

        \item \(0 \in S\), then by definition.
    \end{enumerate}
\end{proof}

\newpage
\section{Localization of Modules}
\begin{defbox}
    \begin{definition}
        
    \end{definition}
\end{defbox}

\begin{thmbox}
    \begin{theorem}
        \begin{enumerate}
            \item For every \(R\)-module \(M\), there is a canonical isomorphism
            \begin{align*}
                M \otimes_R R_S \overset{\sim}{\longrightarrow} M_S, (x \otimes \frac{a}{s}) \mapsto \frac{ax}{s}
            \end{align*}
        \end{enumerate}
    \end{theorem}
\end{thmbox}
\begin{proof}
    \textbf{}
\end{proof}

\end{document}