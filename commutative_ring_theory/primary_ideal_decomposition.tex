\documentclass[a4paper]{book}
\title{Commutative Ring Theory}
\author{Kei Thoma}


% ---------------------------------------------------------------------
% P A C K A G E S
% ---------------------------------------------------------------------

% typography and formatting
\usepackage[english]{babel}
\usepackage[utf8]{inputenc}
\usepackage{geometry}
\usepackage{exsheets}
\usepackage{environ}
\usepackage{graphicx}
\usepackage{cutwin}
\usepackage{pifont}

% mathematics
\usepackage{xfrac}  
\usepackage{amsthm} % for theorems, and definitions
\usepackage{amssymb}
\usepackage{amsmath}
\usepackage{textcomp}
\usepackage{mathtools}
% \usepackage{MnSymbol} % for \cupdot

% extra
\usepackage{xcolor}
\usepackage{tikz}

% ---------------------------------------------------------------------
% S E T T I N G
% ---------------------------------------------------------------------

%maybe delete later, for colorbox
\usepackage{tcolorbox}
\newtcolorbox{defbox}{colback=blue!5!white,colframe=blue!75!black}
\newtcolorbox{defboxlight}{colback=cyan!5!white,colframe=cyan!75!black}
\newtcolorbox{thmbox}{colback=orange!5!white,colframe=orange!75!black}
\newtcolorbox{rembox}{colback=purple!5!white,colframe=purple!75!black}
\newtcolorbox{exmbox}{colback=gray!5!white,colframe=gray!75!black}
\newtcolorbox{intbox}{colback=violet!5!white,colframe=violet!75!black}

% typography and formatting
\geometry{margin=2cm, paperheight=60cm}

\SetupExSheets{
  counter-format = ch.qu,
  counter-within = chapter,
  question/print = true,
  solution/print = true,
}

% mathematics
\newcounter{global}

\theoremstyle{definition}
\newtheorem{definition}{Definition}[]
\newtheorem{example}{Example}[definition]
\newtheorem{exercise}{Exercise}[definition]

\newtheorem{theorem}[definition]{Theorem}
\newtheorem{corollary}{Corollary}
\newtheorem{lemma}[definition]{Lemma}
\newtheorem{proposition}[definition]{Proposition}

\newtheorem*{remark}{Remark}
\newtheorem*{intuition}{Intuition}

% extra
\definecolor{mathif}{HTML}{0000A0} % for conditions
\definecolor{maththen}{HTML}{CC5500} % for consequences
\definecolor{mathrem}{HTML}{8b008b} % for notes
\definecolor{mathobj}{HTML}{008800}

\usetikzlibrary{positioning}
\usetikzlibrary{shapes.geometric, arrows}

% ---------------------------------------------------------------------
% C O M M A N D S
% ---------------------------------------------------------------------

\newcommand{\norm}[1]{\left\lVert#1\right\rVert}
\newcommand{\rank}{\text{rank}}
\newcommand{\Vol}{\text{Vol}}

\newcommand{\set}[1]{\left\{\, #1 \,\right\}}
\newcommand{\makeset}[2]{\left\{\, #1 \mid #2 \,\right\}}

\newcommand*\diff{\mathop{}\!\mathrm{d}}
\newcommand*\Diff{\mathop{}\!\mathrm{D}}

\newcommand\restr[2]{{% we make the whole thing an ordinary symbol
  \left.\kern-\nulldelimiterspace % automatically resize the bar with \right
  #1 % the function
  \vphantom{\big|} % pretend it's a little taller at normal size
  \right|_{#2} % this is the delimiter
  }}

% ---------------------------------------------------------------------
% R E N D E R
% ---------------------------------------------------------------------

\newif\ifshowproof
\showprooftrue

\NewEnviron{Proof}{%
    \ifshowproof%
        \begin{proof}%
            \BODY
        \end{proof}%
    \fi%
}%

\begin{document}

\begin{enumerate}
    \item We want to find the decomposition of \(I = \bigcap_{i=1}^n \mathfrak{q}_i\).
    \item It is \(\sqrt{\mathfrak{q}_i} = \sqrt{(I : x)}\) for some \(x \in R\).
    \item What we know though is if \(x \not\in \sqrt{I}\) then \((I : x) = I\) which doesn't help.
    \item
\end{enumerate}

\begin{example}
    \(I = (X^2Y, XY^2)\).
    \begin{enumerate}
        \item \(\sqrt{(X^2Y, XY^2)} = (XY, X^2, Y^2)\)
        \item \(X^2 Y^2 - XY^2 = (X^2 - X) Y^2\)
    \end{enumerate}
\end{example}


\begin{thmbox}
    \begin{theorem}
        In a Noetherian ring, each ideal has a minimal primary decomposition.
    \end{theorem}
\end{thmbox}

\noindent\textbf{Every irreducible ideal is primary.}

\begin{enumerate}
    \item 
\end{enumerate}

\begin{thmbox}
    \begin{theorem}
        Let
        \begin{align*}
            I = \bigcap_{i=1}^n \mathfrak{q}_i
        \end{align*}
        Then \(\sqrt{\mathfrak{q}_i} = (I : x)\) for some \(x\)
    \end{theorem}
\end{thmbox}


--------------------

\(I = (X^2Y, XY^2)\)

\begin{enumerate}
    \item \((X^2Y, XY^2) = (X^2, XY^2) \cap (Y, XY^2) = (X^2, XY^2) \cap (Y)\)
    \item \((X^2, XY^2) \cap (Y) = (X^2, X) \cap (X^2, Y^2) \cap (Y) = (X) \cap (X^2, Y^2) \cap (Y)\)
\end{enumerate}


\begin{theorem}
    
\end{theorem}
\begin{proof}
    \begin{enumerate}
        \item Let \(\{u_1, \ldots, u_r\}\) generate \(I\).
        \item If \(u_1\) is not a pure power, we can write \(u_1 = vw\) where \(v\) and \(w\) are coprime monomials.
        \item We claim: \(I = (v, u_2, \ldots, u_r) \cap (w, u_2, \ldots, u_r)\).
    \end{enumerate}
\end{proof}

The associated primes are \(\{(X),(Y)\}\). The embedded primes are \(\{(X, Y)\}\)


\begin{enumerate}
    \item \((X^2, XY, XZ) = (X^2, XY, XZ) \cap (X^2, XY, Z) = (X) \cap (X^2, X, Z) \cap (X^2, Y, Z)\)
\end{enumerate}


\((X^3Y, XY^4) = (X^3, XY^4) \cap (Y, XY^4) = (X) \cap (X^3, Y^4) \cap (Y, X) \cap (Y) = (X) \cap (X^3, Y^4) \cap (Y)\)


\((X^2Z, YZ, Z - XY)\)


\begin{exercise}
    If an ideal \(\mathfrak{a}\) has a primary decomposition, then \(\text{Spec}(A / \mathfrak{a})\) has only finitely many irreducible components.
\end{exercise}
\begin{proof}[Solution]
    Let \(\mathfrak{a}\) be a decomposable ideal in a ring \(A\). Denote the set of prime ideals that contain \(\mathfrak{a}\) by
    \begin{align*}
        V(\mathfrak{a}) := \makeset{\mathfrak{p} \in \text{Spec}(A)}{\mathfrak{a} \subset \mathfrak{p}} \subset \text{Spec}(A)
    \end{align*}
    and denote the canonical projection to the quotient ring by \(\pi: A \longrightarrow A / \mathfrak{a}\).
    \begin{enumerate}
        \item Since \(\mathfrak{a}\) is decomposable, by proposition 4.6., the isolated primes of \(\mathfrak{a}\) are precisely the minimal elemenets of \(S_\mathfrak{a}\).
        \item The prime ideals of the quotient ring \(A / \mathfrak{a}\) are precisely the images of prime ideals that contain the ideal \(\mathfrak{a}\) of \(A\) under the canonical projection, i.e. if \(\overline{\mathfrak{p}} \subset A / \mathfrak{a}\) is prime, then \(\mathfrak{p} = \pi^{-1}(\overline{\mathfrak{p}})\) is prime in \(A\) and contains \(\mathfrak{a}\), and if \(\mathfrak{p} \subset A\) is a prime ideal that contains \(\mathfrak{a}\), then \(\overline{\mathfrak{p}} = \pi(\mathfrak{p})\) is prime in \(A / \mathfrak{a}\).
        
        Or state it differently, there is a one-to-one correspodence between the sets
        \begin{align*}
            V(\mathfrak{a}) = \makeset{\mathfrak{p} \subset \text{Spec}(A)}{\mathfrak{a} \subset \mathfrak{p}} \longleftrightarrow \text{Spec}(A / \mathfrak{a}) \text{.}
        \end{align*}
        \begin{enumerate}
            \item By proposition 1.1., the ideals in \(A / \mathfrak{a}\) corresponds to an ideal in \(A\) that contains \(\mathfrak{a}\), i.e. if \(\overline{\mathfrak{b}} \in A / \mathfrak{a}\) is an ideal, then \(\mathfrak{b} = \pi^{-1}(\overline{\mathfrak{b}})\) is an ideal in \(A\) that contains \(\mathfrak{a}\), and if \(\mathfrak{b} \in A\) is an ideal that contains \(\mathfrak{a}\), then \(\overline{\mathfrak{b}} = \pi(\mathfrak{b})\) is an ideal in \(A / \mathfrak{a}\).
            \item If \(\overline{\mathfrak{p}} \in A / \mathfrak{a}\) is a prime ideal, then \(\mathfrak{p} = \pi^{-1}(\overline{\mathfrak{p}})\) is a prime ideal in \(A\) because preimages preserves prime ideals.
            \item If \(\mathfrak{p} \subset A\) is a prime ideal, then \(\overline{\mathfrak{p}} = \pi(\mathfrak{p})\) is a prime ideal in \(A / \mathfrak{a}\).
            \begin{enumerate}
                \item Let \(\overline{x} \cdot \overline{y} \in \overline{\mathfrak{p}}\).
                \item Since \(\pi\) is surjective, there are \(x\) and \(y\) in \(A\) such that \(\pi(x) = \overline{x}\) and \(\pi(y) = \overline{y}\).
                \item Moreover, since \(\pi\) is a ring homomorphism, we have \(\overline{x} \cdot \overline{y} = \pi(x) \cdot \pi (y) = \pi(xy)\).
                \item Since \(\overline{x} \cdot \overline{y} \in \overline{\mathfrak{p}}\) and since 
            \end{enumerate}
        \end{enumerate}
        \item Thus, by 1. and 2., the isolated primes of \(\mathfrak{a}\) corresponds to minimal elements of \(\text{Spec}(A / \mathfrak{a})\). Since the number of associated primes is finite, the number of isolated primes and hence the number of minimal elements of \(\text{Spec}(A / \mathfrak{a})\) must be finite as well.
        \item Bla bla bla, finite minimal elements, finite irreducible components.
    \end{enumerate}
\end{proof}


\begin{exercise}[Atiyah \& MacDonald 4.2., Bosch 2.1.]
    If \(\mathfrak{a} = \sqrt{\mathfrak{a}}\), then \(\mathfrak{a}\) has no embedded prime ideals.
\end{exercise}
\begin{proof}[Solution]
    Let \(\mathfrak{a}\) be an ideal in a ring \(A\).
    \begin{enumerate}
        \item If \(\mathfrak{a}\) is not decomposable, then the statement is trivially\footnote{While it is always aluring to use the word ``vacuously'', I don't think this is a case of a vacuous truth. If \(\mathfrak{a}\) is not decomposable, then the set of associated ideals of \(\mathfrak{a}\) is empty and thus embedded primes which a subset of associated ideals is also empty. It fulfills the statement by definition and not because there is nothing to check.} true. Thus, consider an ideal \(\mathfrak{a}\) that is decomposable and denote one of its minimal primary decomposition by
        \begin{align*}
            \mathfrak{a} = \bigcap_{i = 1}^n \mathfrak{q}_i \text{.}
        \end{align*}
        \item Taking the radical on both sides yields
        \begin{alignat*}{3}
            \sqrt{\mathfrak{a}} &= \sqrt{\bigcap_{i = 1}^n \mathfrak{q}_i} \\
            &= \bigcap_{i = 1}^n \sqrt{\mathfrak{q}_i} & \qquad \text{bla bla}
        \end{alignat*}
        Since \(\mathfrak{a} = \sqrt{\mathfrak{a}}\), the last expression gives
        \begin{align*}
            \mathfrak{a} = \bigcap_{i=1}^n \sqrt{\mathfrak{q}_i}
        \end{align*}
        which is a primary decomposition of \(\mathfrak{a}\).
        \item Assume \(\sqrt{\mathfrak{q}_i} \subset \sqrt{\mathfrak{q}_j}\) for some \(j \neq i\). In that case, we have a primary decomposition
        \begin{align*}
            \mathfrak{a} = \bigcap_{\substack{i = 1 \\ i \neq j}}^n \sqrt{\mathfrak{q}_i}
        \end{align*}
        which has less primary components than the thing above which is a contradiction.

        I think I can write this much better.
    \end{enumerate}
\end{proof}

Additional Bosch: Is the converse true?


\begin{exercise}[Atiyah \& MacDonald 4.3]
    If \(A\) is absolutely flat, every prime ideal is maximal.
\end{exercise}
\begin{proof}[Hints]
    \begin{itemize}
        \item Exercise 2.28 is crucial.
    \end{itemize}
\end{proof}
\begin{proof}[Solution]
    Let \(A\) be an absolutely flat ring and fix a prime ideal \(\mathfrak{p}\) in \(A\). Our goal is to show that \(A / \mathfrak{p}\) is a field. For that endeavour, fix an element \(\overline{x} \in A / \mathfrak{p}\) with \(\overline{x} \neq \overline{0}\). We will show \(\overline{x}\) is invertible.
    \begin{enumerate}
        \item By exercise 2.28., if \(A\) is absolutely flat, then so is \(A / \mathfrak{p}\). Furthermore, exercise 2.28. says in any absolutely flat ring, all non-units are zero-divisors.
        \item Suppose there exists a non-zero non-unit \(\overline{x}\) in \(A / \mathfrak{p}\). By 1., \(\overline{x} \in A / \mathfrak{p}\) must be a non-zero zero-divisor. Thus, there is some non-zero \(\overline{y} \in A / \mathfrak{p}\) such that \(\overline{x} \cdot \overline{y} = \overline{0}\).
        \item Now, \(\overline{x} \cdot \overline{y} = \overline{0} \in A / \mathfrak{p}\) implies \(x \cdot y \in \mathfrak{p}\). By the definition of prime ideals, we hence have \(x \in \mathfrak{p}\) or \(y \in \mathfrak{p}\), but this is equivalent to saying \(\overline{x} = \overline{0}\) or \(\overline{y} = \overline{0}\) which were both excluded. We arrived at a contradiction. There are no non-zero non-units in \(A / \mathfrak{p}\).
    \end{enumerate}
\end{proof}

\begin{exercise}[Atiyah \& MacDonald 4.4]
    In the polynomial ring \(\mathbb{Z}[X]\), the ideal \(\mathfrak{m} = (2, X)\) is maximal and the ideal \(\mathfrak{q} = (4, X)\) is \(\mathfrak{m}\)-primary, but is not the power of \(\mathfrak{m}\).
\end{exercise}
\begin{proof}
    ``The ideal \(\mathfrak{m} = (2, X)\) is maximal in the polynomial ring \(\mathbb{Z}[X]\).''
    \begin{enumerate}
        \item We have the isomorphism \(\mathbb{Z}[X] / (2, X) \cong \mathbb{Z} / 2 \mathbb{Z}\).
        \begin{enumerate}
            \item Consider the map
            \begin{align*}
                \varphi: \mathbb{Z}[X] &\longrightarrow \mathbb{Z} / 2 \mathbb{Z} \\
                P(X) &\mapsto P(0) \mod{2 \mathbb{Z}} \text{.}
            \end{align*}
            \item The map \(\varphi\) is a homomorphism because it is the composition of substitution homomorphism \(\mathbb{Z}[X] \longrightarrow \mathbb{Z}\) and the canonical projection \(\mathbb{Z} \longrightarrow \mathbb{Z} / 2 \mathbb{Z}\).
            \item Moreover, the map \(\varphi\) is surjective because it is the composition of two surjective maps. Thus, \(\text{im}(\varphi) = \mathbb{Z} / 2 \mathbb{Z}\).
            \item We also have \(\text{ker}(\varphi) = (2, X)\).
            \begin{enumerate}
                \item Let \(P \in \text{ker}(\varphi)\). Then, \(\varphi(P) = P(0) \mod{2\mathbb{Z}}\).
                \item zzzzzzzzzzzzzzzzzzzzzzzzz
            \end{enumerate}
            \item Thus, by the isomorphism theorem, we have \(\mathbb{Z}[X] / (2, X) \cong \mathbb{Z} / 2 \mathbb{Z}\).
        \end{enumerate}
        \item Since \(\mathbb{Z} / 2 \mathbb{Z}\) is a field, \((2, X)\) is a maximal ideal.
    \end{enumerate}

    \noindent``The ideal \(\mathfrak{q} = (4, X)\) is \(\mathfrak{m}\)-primary.''
    \begin{enumerate}
        \item We simply have \(\sqrt{(4, X)} = \sqrt{\sqrt{4} + \sqrt{X}}= \sqrt{(2) + (X)} = \sqrt{(2, X)}\).
        \item Thus, \((4, X)\) is \(\mathfrak{m}\)-primary.
    \end{enumerate}

    \noindent``The ideal \(\mathfrak{m} = (4, X)\) is not a power of \(\mathfrak{m}\).''
    \begin{enumerate}
        \item We have the chain of strict inclusion
        \begin{align*}
            \mathfrak{m} \supsetneq \mathfrak{m}^2 \supsetneq \cdots \supsetneq \mathfrak{m}^k \supsetneq \cdots
        \end{align*}
        \item \(\mathfrak{m}^2 = (4, 2X, X^2)\).
        \item So \((4, X) \subset (2, X)\) but \((4, X) \not\subset (4, 2X, X^2)\)
        \item Thus it cannot be a power of \(\mathfrak{m}\).
    \end{enumerate}
\end{proof}


\begin{exercise}[Atiyah \& MacDonald 4.5]
    In the polynomial ring \(K[X, Y, Z]\) where \(K\) is a field, let \(\mathfrak{p}_1 = (X, Y)\), \(\mathfrak{p}_2 = (X, Z)\), and \(\mathfrak{m} = (X, Y, Z)\). Then, \(\mathfrak{p}_1\) and \(\mathfrak{p}_2\) are prime and \(\mathfrak{m}\) is maximal. Let \(\mathfrak{a} = \mathfrak{p}_1 \mathfrak{p}_2\). Show that \(\mathfrak{a} = \mathfrak{p}_1 \cap \mathfrak{p}_2 \cap \mathfrak{m}^2\) is a reduced primary decomposition of \(\mathfrak{a}\). Which components are isolated and which are embedded?
\end{exercise}
``reduced primary decomposition''
\begin{enumerate}
    \item \(\mathfrak{a} = (X, Y) (X, Z) = (X^2, XY, XZ, YZ)\).
    \begin{enumerate}
        \item \(X^2 \in (X, Y) \cap (X, Z) \cap (X, Y, Z)^2\)
        \item \(XY \in\) bla bla
    \end{enumerate}
    \item The representation is \(\mathfrak{a} = \mathfrak{p}_1 \cap \mathfrak{p}_2 \cap \mathfrak{m}^2\) a primary decomposition because \(\mathfrak{p}_1\) and \(\mathfrak{p}_2\) are prime, hence primary and \(\mathfrak{m}^2\) is \(\mathfrak{m}\)-primary.
    \item Clearly, the radicals of the ideals are distinct.
    \item as
    \begin{enumerate}
        \item \((X, Y) \not\supset (X, Z) \cap (X, Y, Z)^2 = (X, Z) \cap (X^2, XY, XZ, Y^2, YZ, Z^2) = (X^2, XY, XZ, YZ, Z^2)\)
        \item \((X, Y, Z)^2 \not\supset (X, Z) \cap (X, Y) = (X, YZ)\)
    \end{enumerate}
\end{enumerate}

\begin{exercise}[Atiyah \& MacDonald 4.7]
    Let \(A\) be a ring and \(A[X]\) be a polynomial ring. \(\mathfrak{a}[X]\) is the extension of \(\mathfrak{a}\) of \(\mathfrak{a}\) to \(A[X]\).
\end{exercise}
\begin{proof}
    
\end{proof}





\begin{exercise}[Bosch 2.2.]
    Let \(\mathfrak{p} \subset A\) be a prime ideal and assume that \(A\) is Noetherian. Show that the \(n\)-th power \(\mathfrak{p}^n\) is the smallest \(\mathfrak{p}\)-primary ideal containing \(\mathfrak{p}^n\). Can we expect that there is a smallest primary ideal containing \(\mathfrak{p}^n\)?
\end{exercise}

\end{document}