\documentclass[a4paper]{book}
\title{Commutative Ring Theory}
\author{Kei Thoma}


% ---------------------------------------------------------------------
% P A C K A G E S
% ---------------------------------------------------------------------

% typography and formatting
\usepackage[english]{babel}
\usepackage[utf8]{inputenc}
\usepackage{geometry}
\usepackage{exsheets}
\usepackage{environ}
\usepackage{graphicx}
\usepackage{cutwin}
\usepackage{pifont}

% mathematics
\usepackage{xfrac}  
\usepackage{amsthm} % for theorems, and definitions
\usepackage{amssymb}
\usepackage{amsmath}
\usepackage{textcomp}
\usepackage{mathtools}
% \usepackage{MnSymbol} % for \cupdot

% extra
\usepackage{xcolor}
\usepackage{tikz}

% ---------------------------------------------------------------------
% S E T T I N G
% ---------------------------------------------------------------------

%maybe delete later, for colorbox
\usepackage{tcolorbox}
\newtcolorbox{defbox}{colback=blue!5!white,colframe=blue!75!black}
\newtcolorbox{defboxlight}{colback=cyan!5!white,colframe=cyan!75!black}
\newtcolorbox{thmbox}{colback=orange!5!white,colframe=orange!75!black}
\newtcolorbox{rembox}{colback=purple!5!white,colframe=purple!75!black}
\newtcolorbox{exmbox}{colback=gray!5!white,colframe=gray!75!black}
\newtcolorbox{intbox}{colback=violet!5!white,colframe=violet!75!black}

% typography and formatting
\geometry{margin=2cm, paperheight=60cm}

\SetupExSheets{
  counter-format = ch.qu,
  counter-within = chapter,
  question/print = true,
  solution/print = true,
}

% mathematics
\newcounter{global}

\theoremstyle{definition}
\newtheorem{definition}{Definition}[]
\newtheorem{example}{Example}[definition]

\newtheorem{theorem}[definition]{Theorem}
\newtheorem{corollary}{Corollary}
\newtheorem{lemma}[definition]{Lemma}
\newtheorem{proposition}[definition]{Proposition}

\newtheorem*{remark}{Remark}
\newtheorem*{intuition}{Intuition}

% extra
\definecolor{mathif}{HTML}{0000A0} % for conditions
\definecolor{maththen}{HTML}{CC5500} % for consequences
\definecolor{mathrem}{HTML}{8b008b} % for notes
\definecolor{mathobj}{HTML}{008800}

\usetikzlibrary{positioning}
\usetikzlibrary{shapes.geometric, arrows}

% ---------------------------------------------------------------------
% C O M M A N D S
% ---------------------------------------------------------------------

\newcommand{\norm}[1]{\left\lVert#1\right\rVert}
\newcommand{\rank}{\text{rank}}
\newcommand{\Vol}{\text{Vol}}

\newcommand{\set}[1]{\left\{\, #1 \,\right\}}
\newcommand{\makeset}[2]{\left\{\, #1 \mid #2 \,\right\}}

\newcommand*\diff{\mathop{}\!\mathrm{d}}
\newcommand*\Diff{\mathop{}\!\mathrm{D}}

\newcommand\restr[2]{{% we make the whole thing an ordinary symbol
  \left.\kern-\nulldelimiterspace % automatically resize the bar with \right
  #1 % the function
  \vphantom{\big|} % pretend it's a little taller at normal size
  \right|_{#2} % this is the delimiter
  }}

% ---------------------------------------------------------------------
% R E N D E R
% ---------------------------------------------------------------------

\newif\ifshowproof
\showprooftrue

\NewEnviron{Proof}{%
    \ifshowproof%
        \begin{proof}%
            \BODY
        \end{proof}%
    \fi%
}%

\begin{document}
\begin{defbox}
    \begin{definition}
        Let \(B\) a ring, \(A\) a subring of \(B\).
        \begin{enumerate}
            \item An element \(x\) of \(B\) is said to be integral over \(A\) if \(x\) is a root of a monic polynomial with coefficients in \(A\), that is if \(x\) satisfies an equation of the form
            \begin{align*}
                x^n + a_1 x^{n-1} + \cdots + a_n = 0
            \end{align*}
            where \(a_i\) are the elements of \(A\).
            \item The set of elements of \(B\) that are integral over \(A\) is called the integral closure of \(A\) in \(B\). The integral closure is, itself, a subring of \(B\) and contais \(A\).
            \item If each element of \(A\) is integral over itself, then \(A\) is said to be integrally closed in \(B\).
            \item If the integral closure of \(A\) in \(B\) is whole \(B\), we say the ring \(B\) is integral over \(A\).
            \item An integral domain is said to be integrally closed (without qualification) if it is integrally closed in its field of fractions.
            \item An element of \(B\) is said to be integral over \(\mathfrak{a}\) if it satisfies an equation of integral dependence over \(A\) in which all the coefficients lie in \(\mathfrak{a}\).
            \item The integral closure of \(\mathfrak{a}\) in \(B\) is the set of all elements of \(B\) which are integral over \(\mathfrak{a}\).
        \end{enumerate}
    \end{definition}
\end{defbox}

\begin{thmbox}
    \begin{theorem}
        The following are equivalent:
        \begin{enumerate}
            \item \(x \in B\) is integral over \(A\)
            \item \(A[x]\) is a finitely generated \(A\)-module
            \item \(A[x]\) is contained in a subring \(C\) of \(B\) such that \(C\) is a finitely generated \(A\)-module
            \item There exsists a faithful \(A[x]\)-module \(M\) which is finitely generated as an \(A\)-module
        \end{enumerate}
    \end{theorem}
\end{thmbox}
\begin{proof}
    \begin{enumerate}
        \item 
    \end{enumerate}
\end{proof}

\begin{thmbox}
    \begin{theorem}
        Let \(x_i\) be elements of \(B\), each integral over \(A\). Then the ring \(A[x_1, \ldots, x_n]\) is finitely generated \(A\)-module.
    \end{theorem}
\end{thmbox}

\begin{thmbox}
    \begin{theorem}
        The set \(C\) of elements of \(B\) which are integral over \(A\) is a subring of \(B\) containing \(A\).
    \end{theorem}
\end{thmbox}

\begin{defbox}
    \begin{definition}
        \begin{enumerate}
            \item The ring \(C\) is called the integral closure of \(A\) in \(B\).
            \item If \(C = A\), then \(A\) is said to be integrally closed in \(B\).
            \item If \(C = B\) the ring \(B\) is said to be integral over \(A\).
        \end{enumerate}
    \end{definition}
\end{defbox}

\begin{thmbox}
    \begin{theorem}
        If \(A \subset B \subset C\) are rings and if \(B\) is integral over \(A\), and \(C\) is integral over \(B\), then \(C\) is integral over \(A\).
    \end{theorem}
\end{thmbox}

\begin{thmbox}
    \begin{theorem}
        Let \(A \subset B\) be rings and let \(C\) be the integral closure of \(A\) in \(B\). Then \(C\) is integrally closed in \(B\).
    \end{theorem}
\end{thmbox}


\begin{thmbox}
    \begin{theorem}
        Let \(A \subset B\) be rings, \(B\) integral over \(A\).
        \begin{enumerate}
            \item If \(\mathfrak{b}\) is an ideal of \(B\) and \(\mathfrak{a} = \mathfrak{b}^c = A \cap \mathfrak{b}\) then \(B / \mathfrak{b}\) is integral over \(A / \mathfrak{a}\).
            \item If \(S\) is a multiplicatively closed subset of \(A\), then \(S^{-1}B\) is integral over \(S^{-1}A\).
        \end{enumerate}
    \end{theorem}
\end{thmbox}

\begin{thmbox}
    \begin{theorem}
        Let \(A \subset B\) be integral domains, \(B\) integral over \(A\). Then \(B\) is a field if and only if \(A\) is a field.
    \end{theorem}
\end{thmbox}

\begin{thmbox}
    \begin{theorem}
        Let \(A \subset B\) be rings, \(B\) integral over \(A\), let \(\mathfrak{q}\) be a prime ideal of \(B\) and let \(\mathfrak{p} = \mathfrak{q}^c = \mathfrak{q} \cap A\). Then \(\mathfrak{q}\) is maximal if and only if \(\mathfrak{p}\) is maximal.
    \end{theorem}
\end{thmbox}

\begin{thmbox}
    \begin{theorem}
        Let \(A \subset B\) be rings, \(B\) integral over \(A\), let \(\mathfrak{q}, \mathfrak{q}'\) be prime ideals of \(B\) such that \(\mathfrak{q} \subset \mathfrak{q}'\) and \(\mathfrak{q}^c = \mathfrak{q}'^c\). Then \(\mathfrak{q} = \mathfrak{q}'\).
    \end{theorem}
\end{thmbox}

\begin{thmbox}
    \begin{theorem}
        Let \(A \subset B\) be rings, \(B\) integral over \(A\), and let \(\mathfrak{p}\) be a prime ideal of \(A\). Then there exists a prime ideal \(\mathfrak{q}\) of \(B\) such that \(\mathfrak{q} \cap A = \mathfrak{p}\).
    \end{theorem}
\end{thmbox}



\begin{thmbox}
    \begin{theorem}
        Let \(A \subset B\) be rings, \(B\) integral over \(B\), let \(\mathfrak{p}_1 \subset \cdots \subset \mathfrak{p}_n\) be a chain of prime ideals of \(A\) and \(\mathfrak{q} \subset \cdots \subset \mathfrak{q}_m\) \((m < n)\) a chain of prime ideals of \(B\) such that \(\mathfrak{q}_i \cap A = \mathfrak{p}_i\) \((1 \leq i \leq m)\). Then the chain \(\mathfrak{q}_1 \subset \cdots \subset \mathfrak{q}_m\) can be extended to a chain \(\mathfrak{q}_1 \subset \cdots \subset \mathfrak{q}_n\) such that \(\mathfrak{q}_i \cap A = \mathfrak{p}_i\) for \((1 \leq i \leq n)\).
    \end{theorem}
\end{thmbox}
\end{document}