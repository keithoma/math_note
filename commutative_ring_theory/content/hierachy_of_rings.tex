\chapter{Hierarchy of Rings}

\newpage
\section{Integral Domains}
\subsection*{Definitions and Theorems}
\begin{defbox}
    \begin{definition}[Integral Domains]
        An integral domain \(A\) is a nonzero ring satisfying the following equivalent conditions.
        \begin{enumerate}
            \item The product of two nonzero elements is nonzero, i.e. for all \(a\) and \(b\) in \(A\) it is \(ab \neq 0\).
            \item The zero ideal \((0)\) is a prime ideal.
            \item Every nonzero element is cancellable under multiplication, i.e. \(ab = ac\) implies \(b = c\).
        \end{enumerate}
    \end{definition}
\end{defbox}

\begin{thmbox}
    \begin{lemma}
        Let \(A\) be a ring and \(\mathfrak{p}\) an ideal. Then, \(\mathfrak{p}\) is a prime ideal if and only if \(A / \mathfrak{p}\) is an integral domain.
    \end{lemma}
\end{thmbox}


\subsection*{Notes}

\newpage
\section{Unique Factorization Domains}
\subsection*{Definitions and Theorems}
\subsection*{Notes}

\newpage
\section{Principal Ideal Domains}
\subsection*{Definitions and Theorems}

\begin{defbox}
    \begin{definition}[Principal Ideal Domains]
        A principal ideal domain is an integral domain in which every ideal is principal.
    \end{definition}
\end{defbox}

\begin{thmbox}
    \begin{lemma}
        In a principal ideal domain, all nonzero prime ideals are maximal and are generated by a prime element, i.e. if \(A\) is a principal ideal domain, then
        \begin{align*}
            \mathrm{Spec}(A) = \mathrm{Spm}(A) \cup \{(0)\} = \makeset{(p)}{p \text{ is a prime element in } A} \text{.}
        \end{align*}
    \end{lemma}
\end{thmbox}

\begin{thmbox}
    \begin{lemma}
        Let \(A\) be a principal ideal domain and \(\mathfrak{a}\) be an ideal in \(A\). The quotient \(A/\mathfrak{a}\) is a principal ideal ring.
    \end{lemma}
\end{thmbox}

\begin{rembox}
    \begin{remark}
        In the above lemma, the quotient \(A / \mathfrak{a}\) need not be an principal ideal domain because \(A / \mathfrak{a}\) is not even be an integral domain if \(\mathfrak{a}\) is not a prime ideal.
    \end{remark}
\end{rembox}

\begin{exmbox}
    \begin{example}
        \(\mathbb{Z}/6\mathbb{Z}\) is a principal ideal ring, but not a principal ideal domain.
    \end{example}
\end{exmbox}

\begin{thmbox}
    \begin{proposition}
        Let \(A\) be a principal ideal domain and \((x)\) an ideal in \(A\). The proper ideals in \(A / (x)\) are in the form \((a)\) where \(a \, | \, x\). 
    \end{proposition}
\end{thmbox}

\subsection*{Notes}

\newpage
\section{Euclidean Domains}
\subsection*{Definitions and Theorems}
\subsection*{Notes}