\chapter{Ideals}

\begin{defbox}
    \begin{definition}[Ideal]
        Let \(A\) be a ring. A subset \(\mathfrak{a} \subset A\) is called an ideal if it satisfies the following two conditions.
        \begin{enumerate}
            \item \((\mathfrak{a}, +)\) is a subgroup of \((A, +)\).
            \item For every \(r \in A\) and every \(x \in \mathfrak{a}\), it is \(rx \in \mathfrak{a}\).
        \end{enumerate}
        \
        Given a subset \(S \subset A\), by the ideal \((S)\) that \(S\) generates, we mean the smallest ideal containing \(S\). If an ideal is generated by a subset \(S \subset A\), then the elements of this subset are called generators.

        \
        An ideal that is generated by a single element is called principal.

        \
        If an ideal \(\mathfrak{a}\) is not the whole ring \(A\), then the ideal is called proper.
    \end{definition}
\end{defbox}

\begin{defbox}
    \begin{definition}[Ideal Operation]
        Let \(\mathfrak{a}\) and \(\mathfrak{b}\) be ideals of a ring \(A\).
        \begin{enumerate}
            \item The sum of two ideals \(\mathfrak{a}\) and \(\mathfrak{b}\) is defined by
            \begin{align*}
                \mathfrak{a} + \mathfrak{b} = \makeset{a + b}{a \in \mathfrak{a} \text{ and } b \in \mathfrak{b}} = (\mathfrak{a}, \mathfrak{b})
            \end{align*}
            which is again an ideal. It is the smallest ideal in \(A\) that contains \(\mathfrak{a}\) and \(\mathfrak{b}\).
            \item The product of an ideal
            \item The intersection of
            \item The radical of an ideal \(\mathfrak{a}\) is defined by
            \begin{align*}
                \sqrt{\mathfrak{a}} = \makeset{x \in A}{x^n \in \mathfrak{a} \text{ for some } n \in \mathbb{N}^+}
            \end{align*}
            which is again an ideal.
            \item The transporter
        \end{enumerate}
    \end{definition}
\end{defbox}

\begin{proof}
    We verify the statements made in the definition.
    \begin{enumerate}
        \item \begin{enumerate}
            \item ``\(\mathfrak{a} + \mathfrak{b}\) is an ideal.'':
        \end{enumerate}
    \end{enumerate}
\end{proof}

\begin{exmbox}
    \begin{example}
        The union of two ideals is \textbf{not} an ideal in general. Consider \((2)\) and \((3)\) in \(\mathbb{Z}\). If \((2) \cup (3)\) was an ideal, then \(3 - 2 = 1\) would be contained in \((2) \cup (3)\). But \(1 \not\in (2)\) and \(1 \not\in (3)\), thus \(1 \not\in (2) \cup (3)\).
    \end{example}
\end{exmbox}

\begin{thmbox}
    \begin{proposition}
        Let \(\mathfrak{a}\) be an ideal of \(A\).
        \begin{enumerate}
            \item \(\mathfrak{a} = A\) if and only if \(1 \in \mathfrak{a}\) if and only if \(\mathfrak{a}\) contains an unit.
            \item \(\mathfrak{a}^2 \subset \mathfrak{a}\).
            \item \(\mathfrak{a} \cdot \mathfrak{b} \subset \mathfrak{a} \cap \mathfrak{b} \subset \mathfrak{a} + \mathfrak{b}\).
            \item \(\mathfrak{a} \subset \mathfrak{a} + \mathfrak{b}\) and \(\mathfrak{b} \subset \mathfrak{a} + \mathfrak{b}\).
        \end{enumerate}
    \end{proposition}
\end{thmbox}

\begin{thmbox}
    \begin{proposition}
        Let \(\mathfrak{a}\) and \(\mathfrak{b}\) be two ideals of a ring \(A\).
        \begin{enumerate}
            \item \(\mathfrak{a} \subset \sqrt{\mathfrak{a}}\).
            \item \(\sqrt{\sqrt{\mathfrak{a}}} = \sqrt{\mathfrak{a}}\).
            \item If \(\mathfrak{a} \subset \mathfrak{b}\), then \(\sqrt{\mathfrak{a}} \subset \sqrt{\mathfrak{b}}\).
            \item \(\sqrt{\mathfrak{a}} = A\) if and only if \(\mathfrak{a} = A\).
            \item \(\sqrt{\mathfrak{a} \cdot \mathfrak{b}} = \sqrt{\mathfrak{a} \cap \mathfrak{b}} = \sqrt{\mathfrak{a}} \cap \sqrt{\mathfrak{b}}\).
            \item \(\sqrt{\mathfrak{a} + \mathfrak{b}} = \sqrt{\sqrt{\mathfrak{a}} + \sqrt{\mathfrak{b}}}\).
            \item If \(\mathfrak{a} = \mathfrak{p}^n\) for some prime ideal \(\mathfrak{p}\) and \(n \in \mathbb{N}^+\), then \(\sqrt{\mathfrak{a}} = \mathfrak{p}\).
        \end{enumerate}
    \end{proposition}
\end{thmbox}

\begin{proof}
    We verify each statement.
    \begin{enumerate}
        \item Let \(x \in \mathfrak{a}\), then trivially, \(x^1 \in \mathfrak{a}\), so \(x \in \sqrt{\mathfrak{a}}\).
        


        \item Since \(\sqrt{\sqrt{\mathfrak{a}}} \supset \sqrt{\mathfrak{a}}\) from above, it suffices to verify the other inclusion. Let \(x \in \sqrt{\sqrt{\mathfrak{a}}}\), then \(x^n \in \sqrt{\mathfrak{a}}\) and in turn, \(\left(x^n\right)^m \in \mathfrak{a}\). Thus, \(x^{nm} \in \mathfrak{a}\), therefore, \(x \in \sqrt{\mathfrak{a}}\).
        
        
        
        \item Suppose \(\mathfrak{a} \subset \mathfrak{b}\) and let \(x \in \sqrt{\mathfrak{a}}\). Then, \(x^n \in \mathfrak{a}\) for some \(n \in \mathbb{N}^+\), thus \(x^n \in \mathfrak{b}\). It follows that \(x \in \sqrt{\mathfrak{b}}\).
        


        \item ``\(\Rightarrow\)'': Let \(\sqrt{\mathfrak{a}} = A\), then for all \(x \in A\), we have that \(x^n \in \mathfrak{a}\) for some \(n \in \mathbb{N}^+\). In particular, \(1^n \in \mathfrak{a}\), but \(1^n = 1\) for all \(n \in \mathbb{N}^+\). Thus, \(\mathfrak{a} = A\).
        
        ``\(\Leftarrow\)'': On the other hand, let \(\mathfrak{a} = A\). In general, it is \(\mathfrak{a} \subset \sqrt{\mathfrak{a}}\), therefore \(A \subset \sqrt{\mathfrak{a}}\) which immediately yields the desired equality \(A = \sqrt{\mathfrak{a}}\).



        \item ``\(\sqrt{\mathfrak{a} \cdot \mathfrak{b}} \subset \sqrt{\mathfrak{a} \cap \mathfrak{b}}\)'': If \(x \in \sqrt{\mathfrak{a} \cdot \mathfrak{b}}\), then \(x^n \in \mathfrak{a} \cdot \mathfrak{b}\) for some \(n \in \mathbb{N}^+\). Since \(\mathfrak{a} \cdot \mathfrak{b} \subset \mathfrak{a} \cap \mathfrak{b}\), we have \(x^n \in \mathfrak{a} \cap \mathfrak{b}\), and it follows that \(x \in \sqrt{\mathfrak{a} \cap \mathfrak{b}}\).
        
        ``\(\sqrt{\mathfrak{a} \cdot \mathfrak{b}} \supset \sqrt{\mathfrak{a} \cap \mathfrak{b}}\)'': Let \(x \in \sqrt{\mathfrak{a} \cap \mathfrak{b}}\), then \(x^n \in \mathfrak{a} \cap \mathfrak{b}\) for some \(n \in \mathbb{N}^+\). Hence it is \(x^n \in \mathfrak{a}\) and \(x^n \in \mathfrak{b}\), therefore \(x^n \cdot x^n = x^{2n} \in \mathfrak{a} \cdot \mathfrak{b}\). Conclude \(x \in \sqrt{\mathfrak{a} \cdot \mathfrak{b}}\).

        ``\(\sqrt{\mathfrak{a} \cap \mathfrak{b}} \subset \sqrt{\mathfrak{a}} \cap \sqrt{\mathfrak{b}}\)'': If \(x \in \sqrt{\mathfrak{a} \cap \mathfrak{b}}\), then \(x^n \in \mathfrak{a} \cap \mathfrak{b}\), thus \(x^n \in \mathfrak{a}\) and \(x^n \in \mathfrak{b}\). We may write \(x \in \sqrt{\mathfrak{a}}\) and \(x \in \sqrt{\mathfrak{b}}\), therefore \(x \in \sqrt{\mathfrak{a}} \cap \sqrt{\mathfrak{b}}\).

        ``\(\sqrt{\mathfrak{a} \cap \mathfrak{b}} \supset \sqrt{\mathfrak{a}} \cap \sqrt{\mathfrak{b}}\)'': Finally, let \(x \in \sqrt{\mathfrak{a}} \cap \sqrt{\mathfrak{b}}\). Then, \(x \sqrt{\mathfrak{a}}\) and \(x \sqrt{\mathfrak{b}}\), so \(x^n \in \mathfrak{a}\) and \(x^m \in \mathfrak{b}\) for some \(n, m \in \mathbb{N}^+\). Say \(n \geq m\), then \(x^n \in \mathfrak{b}\). This yields \(x^n \in \mathfrak{a} \cap \mathfrak{b}\), thus \(x \in \sqrt{\mathfrak{a} \cap \mathfrak{b}}\).



        \item ``\(\sqrt{\mathfrak{a} + \mathfrak{b}} \subset \sqrt{\sqrt{\mathfrak{a}} + \sqrt{\mathfrak{b}}}\)'': Let \(x \in \sqrt{\mathfrak{a} + \mathfrak{b}}\), then \(x^n \in \mathfrak{a} + \mathfrak{b}\) for some \(n \in \mathbb{N}^+\). By definition of sum of ideals, we have that \(x^n = a + b\) for some \(a \in \mathfrak{a}\) and \(b \in \mathfrak{b}\). Since \(\mathfrak{a} \subset \sqrt{\mathfrak{a}}\) and \(\mathfrak{b} \subset \sqrt{\mathfrak{b}}\), we have \(x^n \in \sqrt{\mathfrak{a}} + \sqrt{\mathfrak{b}}\), thus \(x \in \sqrt{\sqrt{\mathfrak{a}} + \sqrt{\mathfrak{b}}}\).
        
        ``\(\sqrt{\mathfrak{a} + \mathfrak{b}} \supset \sqrt{\sqrt{\mathfrak{a}} + \sqrt{\mathfrak{b}}}\)'': Now let \(x \in \sqrt{\sqrt{\mathfrak{a}} + \sqrt{\mathfrak{b}}}\), then \(x^n \in \sqrt{\mathfrak{a}} + \sqrt{\mathfrak{b}}\) for some \(n \in \mathbb{N}^+\). Hence there exists \(a \in \sqrt{\mathfrak{a}}\) and \(b \in \sqrt{\mathfrak{b}}\) such that \(x^n = a + b\). We have that \(a^p \in \mathfrak{a}\) and \(b^q \in \mathfrak{b}\) for some \(p, q \in \mathbb{N}^+\). Consider
        \begin{align*}
            \left(x^n\right)^{(p + q -1)} &= (a + b)^{(p + q -1)} \\ 
            &= \sum_{k=0}^{p + q - 1} \binom{p + q - 1}{k} a^k \cdot b^{p + q - 1 - k} \text{.}
        \end{align*}
        For each \(k \in \set{0, 1, \ldots, p + q -1}\), we have \(a^k \in \mathfrak{a}\) or \(b^{p + q - 1} \in \mathfrak{b}\). Thus, the whole sum lies in \(\mathfrak{a} + \mathfrak{b}\) or in other words \(x^{n(p + q - 1)} \in \mathfrak{a} + \mathfrak{b}\). Conclude \(x \in \sqrt{\mathfrak{a} + \mathfrak{b}}\).



        \item ``\(\sqrt{\mathfrak{a}} \subset \mathfrak{p}\)'': Let \(x \in \sqrt{\mathfrak{a}}\), then \(x^m \in \mathfrak{a}\) for some \(m \in \mathbb{N}^+\). Because \(\mathfrak{a} = \mathfrak{p}^n\), we have \(x^m \in \mathfrak{p}^n\). We also have \(\mathfrak{p}^n \subset \mathfrak{p}\), thus \(x^m \in \mathfrak{p}\) and since \(\mathfrak{p}\) is prime, \(x \in \mathfrak{p}\).
        
        ``\(\sqrt{\mathfrak{a}} \supset \mathfrak{p}\)'': On the other hand, if \(x \in \mathfrak{p}\), then \(x^n \in \mathfrak{p}^n = \mathfrak{a}\), therefore \(x \in \sqrt{\mathfrak{a}}\).
    \end{enumerate}
\end{proof}

\begin{thmbox}
    \begin{proposition}
        \begin{enumerate}
            \item \(\mathfrak{a} \subset (\mathfrak{a} : \mathfrak{b})\).
        \end{enumerate}
    \end{proposition}
\end{thmbox}

\begin{exmbox}
    \begin{example}
        Does \(\sqrt{\mathfrak{a}^2} = \mathfrak{a}\) hold?
    \end{example}
\end{exmbox}


\begin{thmbox}
    \begin{proposition}
        Let \(A_1, \ldots, A_n\) be rings for \(n \in \mathbb{N}^+\) and denote \(A := A_1 \times \cdots \times A_n\). The ideals in \(A\) are exactly in the form \(\mathfrak{a}_1 \times \cdots \times \mathfrak{a}_n\) where \(\mathfrak{a}_i\) is an ideal in \(A_i\) for \(1 \leq i \leq n\), i.e.
        \begin{align*}
            \set{\text{ideals in } A} = \prod_{i = 1}^n \set{\text{ideals in } A_i}
        \end{align*}
    \end{proposition}
\end{thmbox}