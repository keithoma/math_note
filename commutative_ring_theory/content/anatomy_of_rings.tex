
\chapter{Anatomy of Rings}

\subsubsection*{Zero Divisor}
\begin{defbox}
    \begin{definition}[Zero Divisor]
        An element \(a\) of a ring \(A\) is called a zero divisor if one of the following equivalent conditions hold.
        \begin{enumerate}
            \item There exists a nonzero \(x \in A\) such that \(ax = 0\).
            \item The map \(A \rightarrow A\) that sends \(x\) to \(ax\) is not injective.
        \end{enumerate} 
    \end{definition}
\end{defbox}

\subsubsection*{Group of Units}
\begin{defbox}
    \begin{definition}[Group of Units]
        
    \end{definition}
\end{defbox}

\subsubsection*{Nilpotent Elements}
\begin{defbox}
    \begin{definition}[Nilpotent Element and Nilradical]
        An element \(x\) of a ring \(A\) is called nilpotent if there exists some positive integer \(n \in \mathbb{N}^+\), called the index or the degree, such that \(x^n = 0\).

        The set of all nilpotent elements is called the nilradical of the ring and is denoted by \(\mathrm{Nil}(A)\).
    \end{definition}
\end{defbox}

\begin{defbox}
    \begin{definition}[Reduced Ring]
        A ring \(A\) is called reduced ring if it has no non-zero nilpotent elements.
    \end{definition}
\end{defbox}

\begin{thmbox}
    \begin{proposition}
        Let \(A\) and \(B\) be two rings and \(A^\prime \subset A\) be a subring of \(A\).
        \begin{enumerate}
            \item If \(A\) is reduced, then \(A^\prime\) is also reduced.
            \item If \(A\) and \(B\) are reduced, then \(A \times B\) is also reduced.
            
            (XXX DOES THIS ALSO HOLD FOR ARBITARY MANY PRODUCTS?)
        \end{enumerate}
    \end{proposition}
\end{thmbox}

\subsubsection*{Irreducible and Prime Elements}

\begin{defbox}
    \begin{definition}[Irreducible Element]
        An element \(a\) of an integral domain \(A\) is a nonzero element that is
        \begin{enumerate}
            \item not invertible, i.e. \(a\) is not a unit, and
            \item is not a product of two non-invertible elements.
        \end{enumerate}
        REWRITE THIS DEFINITION
    \end{definition}
\end{defbox}

\begin{defbox}
    \begin{definition}[Prime Element]
        A non-zero non-unit element \(a\) of a ring \(A\) is called prime if whenever \(a \, |  \, bc\) for some \(b\) and \(c\) in \(A\), then it implies \(a \, | \, b\) or \(a \, | \, c\).
    \end{definition}
\end{defbox}

\begin{thmbox}
    \begin{proposition}
        In an integral domain, every prime element is irreducible.
    \end{proposition}
\end{thmbox}

\begin{exmbox}
    \begin{example}
        The converse of the above proposition is not true in general.
    \end{example}
\end{exmbox}

\section{Exercises and Notes}

\begin{example}
    Let \(K\) be a field and \(A = \sfrac{K[X, Y]}{(X - XY^2, Y^3)}\).
    \begin{enumerate}
        \item Compute the nilradical \(\mathrm{Nil}(A)\).
        \begin{proof}[Solution]
            Denote \((X - XY^2, Y^3) =: \mathfrak{a}\).
            \begin{align*}
                X + \mathfrak{a} &= XY^2 + \mathfrak{a} && \text{because } X - XY^2 \Rightarrow X \sim XY^2 \text{.} \\
                &= XY^2 Y^2 + \mathfrak{a} && \text{because } XY^2 - XY^2Y^2 = Y^2 (X - XY^2) = 0 \Rightarrow XY^2 \sim XY^2Y^2 \\
                &= XY \cdot Y^3 + \mathfrak{a} \\
                &= XY \cdot 0 + \mathfrak{a} \\
                &= 0 + \mathfrak{a} \text{.}
            \end{align*}
            Thus, \(X \in (X - XY^2, Y^3)\). We have therefore the isomorphism \(\sfrac{K[X, Y]}{(X-XY^2, Y^3)} \simeq \sfrac{K[Y]}{(Y^3)}\). [I WANT A ELEGANT REASON FOR THIS. PROBABLY ISOMORPHISM THEOREM.]

            Clearly, \(Y \in \mathrm{Nil}(A)\) or in other words \((Y) \subset \mathrm{Nil}(A)\). But we also have that \(\sfrac{K[Y]}{(Y)} = K\) which is a field, therefore \((Y)\) is a maximal ideal. Because \(1 \not\in \mathrm{Nil}(A)\) conclude \(\mathrm{Nil}(A) = (Y)\).
        \end{proof}
    \end{enumerate}
\end{example}
