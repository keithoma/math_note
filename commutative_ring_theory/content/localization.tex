\chapter{Localization}

\subsection*{Definition and Theorems}
\subsubsection{Multiplicative Subsets}


\begin{defbox}
    \begin{definition}[Multiplicative Subset]
        A subset \(S\) of a ring \(A\) is called a multiplicative subset if the following conditions hold.
        \begin{enumerate}
            \item \(1 \in S\).
            \item For all \(x, y \in S\) it is \(xy \in S\).
        \end{enumerate}
    \end{definition}
\end{defbox}

\begin{exmbox}
    \begin{example}
        Let \(A\) be a ring. Trivially, the following subsets of \(A\).are multiplicative subsets.
        \begin{enumerate}
            \item \(A\) itself is a multiplicative subset.
            \item \(\{1\}\) is a multiplicative subset.
            \item \(\{0, 1\}\) is a multiplicative subset.
        \end{enumerate}
    \end{example}
\end{exmbox}

\begin{exmbox}
    \begin{example}
        Let \(A\) be a ring. Important examples of a multiplicative subset include the following.
        \begin{enumerate}
            \item The set of units \(A^\times\) is a multiplicative subset.
            \item The set of non-zero-divisors \(A \setminus \mathrm{ZD}(A)\) is a multiplicative subset.
        \end{enumerate}
    \end{example}
\end{exmbox}

\begin{proof}
    Let \(A\) be a ring.
    \begin{enumerate}
        \item We show \(A^\times\) is a multiplicative subset. Clearly, \(1\) is a unit and thus lies in \(A^\times\). Let \(x\) and \(y\) be units in \(A\), then there are some \(x^{-1}\) and \(y^{-1}\) in \(A\) with \(x \cdot x^{-1} = 1\) and \(y \cdot y^{-1}\). Then, \(xy \cdot x^{-1} \cdot y^{-1} = x x^{-1} \cdot y y^{-1} = 1\), so \(xy\) is a unit and \(A^\times\) is multiplicatively closed.
    \end{enumerate}
\end{proof}

\begin{example}
    Let \(A\) be a ring. Other examples of multiplicative subsets are the following.
    \begin{enumerate}
        \item Let \(S\) be a multiplicative subset. Then, \(S \cup \{0\}\) is also multiplicative subset.
        \item For any element \(x \in A\), the set generated by its power \(\set{1, x, x^2, x^3, \ldots}\) is a multiplicative subset.
        \item For any ideal \(\mathfrak{a} \subset A\), the set \(1 + \mathfrak{a}\) is a multiplicative subset.
    \end{enumerate}
\end{example}

\begin{thmbox}
    \begin{lemma}
        An ideal \(\mathfrak{p}\) of a ring \(A\) is prime if and only if its complement \(A \setminus \mathfrak{p}\) is a multiplicative subset.
    \end{lemma}
\end{thmbox}
\begin{proof}
    Let \(A\) be a ring and \(\mathfrak{p}\) be an ideal in \(A\).

    ``\(\Rightarrow\)'': Suppose \(\mathfrak{p}\) is prime. By definition, \(1 \not\in \mathfrak{p}\), hence \(1\) lies in the complement \(A \setminus \mathfrak{p}\). Now let \(x,y \in A \setminus \mathfrak{p}\) and assume \(xy \not\in A \setminus \mathfrak{p}\). In this case, \(xy \in \mathfrak{p}\) and since \(\mathfrak{p}\) is prime, we must have \(x \in \mathfrak{p}\) or \(x \in \mathfrak{p}\) both of which are contradictions.

    ``\(\Leftarrow\)'': On the other hand, let \(A \setminus \mathfrak{p}\) be a multiplicative subset. Fix a \(xy \in \mathfrak{p}\) and assume \(x, y \not\in \mathfrak{p}\). We have that \(x, y \in A \setminus \mathfrak{p}\) and since \(A \setminus \mathfrak{p}\) is a multiplicative subset, it is \(xy \in A \setminus \mathfrak{p}\). This implies \(xy \not\in \mathfrak{p}\) which is a contradiction.
\end{proof}

\begin{rembox}
    \begin{remark}
        The lemma does not imply that any complement of a multiplicative subset is a prime ideal. Only if the complement of a multiplicative subset is already an ideal it is prime. Thus, constructing multiplicative subsets through complements of primitive ideals are not exhaustive.
    \end{remark}
\end{rembox}
\begin{exmbox}
    \begin{example}
        Consider \(\mathbb{Z}\) and the multiplicative subset \(\{1\}\). The complement \(\mathbb{Z} \setminus \{1\}\) is not an ideal.
    \end{example}
\end{exmbox}

\begin{thmbox}
    \begin{proposition}
        intersection is again multiplicative

        cartesian product?
    \end{proposition}
\end{thmbox}
\begin{exmbox}
    \begin{example}
        subsets?
        unions
        symmetric difference
    \end{example}
\end{exmbox}

\subsubsection{Localization}
\begin{defbox}
    \begin{definition}[Localization]
        \(S^{-1}A\) is again a ring.
    \end{definition}
\end{defbox}

\begin{thmbox}
    \begin{lemma}[Universal Property of Localization]
        Let \(A\) and \(B\) be two rings, \(S\) a multiplicative subset of \(A\), and \(f: A \rightarrow B\) a ring homomorphism that maps every element of \(S\) to a unit in \(B\). In this case, there exists a unique ring homomorphism \(g: S^{-1}A \rightarrow B\) such that \(f = g \circ \varphi\).
    \end{lemma}
\end{thmbox}

\begin{thmbox}
    \begin{lemma}
        Let \(A\) be a ring and \(S\) a multiplicative subset, then the following are equivalent.
        \begin{enumerate}
            \item \(S^{-1}A = 0\).
            \item \(S\) contains a nilpotent element.
            \item \(0 \in S\).
        \end{enumerate}
    \end{lemma}
\end{thmbox}
\begin{proof}
    ``\(1. \Rightarrow 2.\)'': Let \(S^{-1}A = 0\), then for all \(x \in A\) and \(s \in S\) it is \((x, s) \sim (0, 1)\), thus \(x \cdot u = 0\) for some \(u \in S\). In particular, this holds for \(x = 1\), therefore \(1 \cdot u = 0\). Since a unit can never be a zero divisor, we must have \(u = 0\) which is nilpotent and lies in \(S\).

    ``\(1. \Leftarrow 2.\)'': On the other hand, let \(x \in S\) be nilpotent, i.e. \(x^n = 0\) for some \(n \in \mathbb{N}^+\). Because \(S\) is multiplicatively closed \(x^n = 0\) lies in \(S\). Fix an element \((y, s) \in S^{-1}A\), then \(y \cdot 1 \cdot 0 = 0 \cdot s \cdot 0\). Hence \((y, s) \sim (0, 1)\) and we have \(S^{-1}A = 0\).

    ``\(2. \Rightarrow 3.\)'': Again, let \(x \in S\) be nilpotent, thus \(x^n = 0\) for some \(n \in \mathbb{N}^+\). \(S\) is multiplicatively closed and we have \(x^n = 0 \in S\).

    ``\(2. \Leftarrow 3.\)'': If \(0 \in S\), then \(S\) simply contains a nilpotent element because \(0\) is nilpotent.
\end{proof}



\begin{example}
    Some concrete examples of localization include the following.
    \begin{enumerate}
        \item 
    \end{enumerate}
\end{example}


\begin{thmbox}
    \begin{proposition}
        Let \(A\) be a ring. \(A\) is reduced if and only if all its localizations \(A_\mathfrak{p}\) at \(\mathfrak{p} \in \mathrm{Spec} \, A\) is reduced.
    \end{proposition}
\end{thmbox}

\begin{proof}
    ``\(\Rightarrow\)'': We prove the statement by contrapositive. Let \(A_\mathfrak{p}\) be not reduced for all \(\mathfrak{p} \in \mathrm{Spec} \, A\). Thus, in all \(A_\mathfrak{p}\), there is an element, say \(x / s\) that is nilpotent and not zero, i.e. \((x / s)^n = 0\) for some \(n \in \mathbb{N}^+\). By the definition of localization, we get \(x^n \cdot u = 0\) for some \(u \in A \setminus \mathfrak{p}\). Now, \(u \in A \setminus \mathfrak{p}\) cannot be zero, because if it was, \(A_\mathfrak{p} = 0\) which is reduced. Thus, \(x\) is nilpotent and \(A\) is not reduced.
\end{proof}








%%%%%%%%%%%%%%%%%%%%%%%%%%%%%
\subsubsection*{Interactions}
\begin{thmbox}
    \begin{proposition}
        Let \(A\) be a ring and \(S \subset A\) be a multiplicative subset that does not contain \(0\).
        
        \begin{enumerate}
            \item \(A\) is an integral domain if and only if \(S^{-1}A\) is an integral domain.
            \item \(A\) is a unique factorization domain if and only if \(S^{-1}A\) is a unique factorization domain.
        \end{enumerate}
    \end{proposition}
\end{thmbox}



\begin{proof}
    ``\(\Rightarrow\)'': Let \(A\) be an integral domain. Since \(S\) does not contain \(0\), the localization \(S^{-1}A\) is a nonzero ring (see EXAMPLE). Let \((x, s) \in S^{-1}A \setminus \{0\}\) be a nonzero element and suppose there is a \((y, t) \in S^{-1}A\) with \((x, s) \cdot (y, t) = 0\). It is \((xy, st) = (0, 1)\) and thus \(xy \cdot u = 0\) for some \(u \in S\). Because \(x\) was nonzero and \(S\) does not contain \(0\) we must have \(y = 0\). Hence \(S^{-1}A\) is an integral domain.

    ``\(\Leftarrow\)'': On the other hand, let \(S^{-1}A\) be an integral domain. JUST USE THE CANONIC MAPPING \(\varphi_S: A \longrightarrow S^{-1}A\).
\end{proof}

\begin{rembox}
    \begin{remark}
        In the lemma above, the condition \(0 \not\in S\) is required because if \(S\) contains \(0\), then \(S^{-1}A = 0\) and by definition, an integral domain is a nonzero ring.
    \end{remark}
\end{rembox}

\begin{thmbox}
    \begin{proposition}
        Let \(A\) be a ring, \(S\) a multiplicative subset, and \(\mathfrak{a}_1, \ldots, \mathfrak{a}_n\) for \(n \in \mathbb{N}^+\) ideals in \(A\). It is
        \begin{align*}
            \left(\bigcap_{i=1}^n \mathfrak{a}_i\right) A_S &= \left(\bigcap_{i=1}^n \mathfrak{a}_i A_S\right)
            \intertext{or written differently}
            \left(\mathfrak{a}_1 \cap \cdots \cap \mathfrak{a}_n\right) A_S &= \mathfrak{a}_1 A_S \cap \cdots \cap \mathfrak{a}_n A_S \text{.}
        \end{align*}
    \end{proposition}
\end{thmbox}
\begin{proof}
    By induction, we reduce the case to \(n = 2\), that is, we want to show \((\mathfrak{a}_1 \cap \mathfrak{a}_2) A_S = \mathfrak{a}_1 A_S \cap \mathfrak{a}_2 A_S\). The inclusion \((\mathfrak{a}_1 \cap \mathfrak{a}_2) \xhookrightarrow{} \mathfrak{a}_1\) induces a natural inclusion \((\mathfrak{a}_1 \cap \mathfrak{a}_2) A_S \xhookrightarrow{} \mathfrak{a}_1 A_S\) which can be extended to a injective map \(f: (\mathfrak{a}_1 \cap \mathfrak{a}_2)A_S \rightarrow \mathfrak{a}_1 A_S \cap \mathfrak{a}_2 A_S\). It suffies to show \(f\) is surjective. Let \(y \in \mathfrak{a}_1 A_S \cap \mathfrak{a}_2 A_S\). We have
    \begin{align*}
        y = \frac{a_1}{s} = \frac{a_2}{t}
    \end{align*}
    with \(a_1 \in \mathfrak{a}_2\), \(a_2 \in \mathfrak{a}_2\), and \(s, t \in S\). Thus it is \(a_1 t u = a_2 s u\) for some \(u \in S\). Since \(a_1\) lies in \(\mathfrak{a}_1\), we have \(a_1 t u \in \mathfrak{a}_1\), and similary \(a_2 s u \in \mathfrak{a}_2\), hence \(a_1 t u \in \mathfrak{a}_1 \cap \mathfrak{a}_2\). But \(t\) and \(u\) are invertible in \(A_S\), therefore
    \begin{align*}
        \frac{a_1}{s} = \frac{a_1 t u}{s t u} \in (\mathfrak{a}_1 \cap \mathfrak{a}_2) A_S
    \end{align*}
    thus \(f\) is surjective.
\end{proof}

\begin{exmbox}
    \begin{example}
        Consider \(\mathbb{Q}[X]\)
    \end{example}
\end{exmbox}

\newpage
\subsection*{Exercises and Notes}

\begin{example}
    Let \(A_1\) and \(A_2\) be rings. Consider \(A = A_1 \times A_2\) and set \(S := \set{(1, 1), (1, 0)}\). Prove \(A_1 \simeq S^{-1}A\).
\end{example}

\begin{proof}[Solution]
    I don't understand the solution?
\end{proof}


\begin{example}
    Find all intermediate rings \(\mathbb{Z} \subset A \subset \mathbb{Q}\), and describe each \(A\) as a localization of \(\mathbb{Z}\). As a starter, prove \(\mathbb{Z}\left[\frac{2}{3}\right] = S_3^{-1} \mathbb{Z}\) where \(S_3 := \makeset{3^i}{i \in \mathbb{N}^+}\).
\end{example}