\chapter{Rings and Homomorphisms}
\subsection*{Definition and Theorems}
\subsubsection*{Rings}
\begin{defbox}
    \begin{definition}[Ring]
        A {\color{maththen}ring} is a {\color{mathobj}set} \(A\) equipped with two {\color{mathobj}binary operations} \(+\) ({\color{mathrem}addition}) and \(\cdot\) ({\color{mathrem}multiplication}) satisfying the following three sets of {\color{mathrem}axioms}, called the {\color{mathrem}ring axioms}.
    \begin{enumerate}
      \item \((A, +)\) is an {\color{mathif}abelian group}, i.e.
      \begin{enumerate}
        \item The operation \(+\) is well-defined meaning for all pairs \(a\) and \(b\) of \(A\), \(a + b\) is defined and belongs to \(A\).
        \item (Associativity) For all \(a\), \(b\), and \(c\) in \(A\), it is \((a + b) + c = a + (b + c)\).
        \item (Identity Element) There exists an element \(0\) in \(A\) such that for all elements \(a\) in \(A\), it is \(0 + a = a + 0 = 0\).
        \item (Inverse Element) For each \(a\) in \(A\) there exists an element \(b \in A\) such that \(a + b = b + a = 0\).
        \item (Commutativity) For all \(a\) and \(b\) in \(A\), it is \(a + b = b + a\).
      \end{enumerate}
      \item \((A, \cdot)\) is a {\color{mathif}semigroup}, i.e.
      \begin{enumerate}
        \item The operation \(\cdot\) is well-defined meaning for all pairs \(a\) and \(b\) of \(A\), \(a \cdot b\) is defined and belongs to \(A\).
        \item (Associativity) For all \(a\), \(b\), and \(c\) in \(A\), it is \((a \cdot b) \cdot c = a \cdot (b \cdot c)\).
      \end{enumerate}
      \item {\color{mathobj}Multiplication} is {\color{mathif}distributive} with respect to {\color{mathobj}addition}, meaning that
      \begin{itemize}
        \item \(a \cdot (b + c) = (a \cdot b) + (a \cdot c)\) for all \(a, b, c \in A\) ({\color{mathrem}left distributivity}).
        \item \((b + c) \cdot a = (b \cdot a) + (c \cdot a)\) for all \(a, b, c \in A\) ({\color{mathrem}right distributivity}).
      \end{itemize}
    \end{enumerate}
    A {\color{mathobj}ring} is called {\color{maththen}unitary} if it {\color{mathif}contains} the {\color{mathobj}multiplicative identity} and {\color{maththen}commutative} if {\color{mathobj}multiplication} is {\color{mathif}commutative}.
    \end{definition}
\end{defbox}
%
%
%
\begin{intbox}
    \begin{intuition}
        A ring may be understood as the generalization of the integers.

        Another way to see rings is a less well behaved field where the theory of dividing is due to rings missing the multiplicative identity richer.
    \end{intuition}
\end{intbox}
%
%
%
\begin{rembox}
    \begin{remark}
        In this text, we will primarily be concerned with commutative unitary rings, and thus, for brevity sake, we simply write ``ring'' and mean a commutative unitary ring.
    \end{remark}
\end{rembox}
%
%
%
\begin{exmbox}
    \begin{example}
        Some important examples of rings include the following.
        \begin{enumerate}
            \item The prototypical example is the ring of integers \(\mathbb{Z}\) with the two operations being of addition and multiplication.
            \item Any field is a ring. In particular, the rational numbers \(\mathbb{Q}\), the real numbers \(\mathbb{R}\), and the complex numbers \(\mathbb{C}\) are rings.
            \item The zero ring or trivial ring is the unique ring consisting of one element \(0\) with the operations \(+\) and \(\cdot\) defined such that \(0 + 0 = 0\) and \(0 \cdot 0 = 0\). It is the unique ring in which the additive and the multiplicative identity coincide.
            \item the set of polynomials
            \item an example of a finite ring
            \item If \(S\) is a set, then the power set \(\mathcal{P}(S)\) of \(S\) becomes a ring if we define addition to be the symmetric difference of sets and multiplication to be intersection.
        \end{enumerate}
    \end{example}
\end{exmbox}
%
%
%
\begin{example}
    Moreover, we have some examples of rings that are non-commutative or non-unitary.
    \begin{enumerate}
        \item Matrix ring is non-commutative
    \end{enumerate}
\end{example}
%
%
%
\begin{example}
    Counterexamples of rings include the following.
    \begin{enumerate}
        \item The set of natural numbers \(\mathbb{N}\) with the usual operations is not a ring, since \((\mathbb{N}, +)\) is not even a group.
        \item Trivially, the emptyset regardless of the operations is not a ring.
    \end{enumerate}
\end{example}
%
%
%
\begin{defbox}
    \begin{definition}[Subring]
        A subset \(S\) of \(A\) is called a subring if any of the following equivalent conditions holds.
    \end{definition}
\end{defbox}
%
%
%
\begin{thmbox}
    \begin{proposition}
        Let \(A\) be a ring and \(R\) and \(S\) subrings of \(A\).
        \begin{enumerate}
            \item (ANY?) intersection stable
            \item cartesian product is again a ring
        \end{enumerate}
    \end{proposition}
\end{thmbox}
%
%
%
\begin{exmbox}
    \begin{example}
        \begin{enumerate}
            \item Complement, of course not.
            \item union, of course not.
            \item difference, of course not
            \item symmetric difference, of course not
        \end{enumerate}
    \end{example}
\end{exmbox}

\subsubsection*{Ring Homomorphisms}
\begin{defbox}
    \begin{definition}[Ring Homomorphism]
        A homomorphism from ring \((A, +, \cdot)\) to a ring \((B, \boxplus, \boxdot)\) is a map \(\varphi\) from \(A\) to \(B\) that preserves the ring operations.
    \end{definition}
\end{defbox}
%
%
%
\begin{example}
    examples of ring homomorphism.
\end{example}
%
%
%
\begin{thmbox}
    \begin{proposition}
        Let \(f: A \rightarrow B\) be a ring homomorphism.
        \begin{enumerate}
            \item A ring homomorphism preserves the additive identity, i.e. \(f(0_A) = 0_B\).
        \end{enumerate}
    \end{proposition}
\end{thmbox}
\newpage
\subsection*{Notes}

