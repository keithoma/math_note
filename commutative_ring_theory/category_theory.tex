\documentclass[a4paper]{scrartcl}
\title{Category Theory}


% ---------------------------------------------------------------------
% P A C K A G E S
% ---------------------------------------------------------------------

% typography and formatting
\usepackage[english]{babel}
\usepackage[utf8]{inputenc}
\usepackage{geometry}
\usepackage{exsheets}
\usepackage{environ}
\usepackage{graphicx}
\usepackage{cutwin}
\usepackage{pifont}

% mathematics
\usepackage{xfrac}  
\usepackage{amsthm} % for theorems, and definitions
\usepackage{amssymb}
\usepackage{amsmath}
\usepackage{textcomp}
\usepackage{mathtools}
% \usepackage{MnSymbol} % for \cupdot

% extra
\usepackage{xcolor}
\usepackage{tikz}

% ---------------------------------------------------------------------
% S E T T I N G
% ---------------------------------------------------------------------

%maybe delete later, for colorbox
\usepackage{tcolorbox}
\newtcolorbox{defbox}{colback=blue!5!white,colframe=blue!75!black}
\newtcolorbox{defboxlight}{colback=cyan!5!white,colframe=cyan!75!black}
\newtcolorbox{thmbox}{colback=orange!5!white,colframe=orange!75!black}
\newtcolorbox{rembox}{colback=purple!5!white,colframe=purple!75!black}
\newtcolorbox{exmbox}{colback=gray!5!white,colframe=gray!75!black}
\newtcolorbox{intbox}{colback=violet!5!white,colframe=violet!75!black}

% typography and formatting
\geometry{margin=3cm}

\SetupExSheets{
  counter-format = ch.qu,
  counter-within = chapter,
  question/print = true,
  solution/print = true,
}

% mathematics
\newcounter{global}

\theoremstyle{definition}
\newtheorem{definition}{Definition}[]
\newtheorem{example}{Example}[definition]

\newtheorem{theorem}[definition]{Theorem}
\newtheorem{corollary}{Corollary}
\newtheorem{lemma}[definition]{Lemma}
\newtheorem{proposition}[definition]{Proposition}

\newtheorem*{remark}{Remark}
\newtheorem*{intuition}{Intuition}

% extra
\definecolor{mathif}{HTML}{0000A0} % for conditions
\definecolor{maththen}{HTML}{CC5500} % for consequences
\definecolor{mathrem}{HTML}{8b008b} % for notes
\definecolor{mathobj}{HTML}{008800}

\usetikzlibrary{positioning}
\usetikzlibrary{shapes.geometric, arrows}

% ---------------------------------------------------------------------
% C O M M A N D S
% ---------------------------------------------------------------------

\newcommand{\norm}[1]{\left\lVert#1\right\rVert}
\newcommand{\rank}{\text{rank}}
\newcommand{\Vol}{\text{Vol}}

\newcommand{\set}[1]{\left\{\, #1 \,\right\}}
\newcommand{\makeset}[2]{\left\{\, #1 \mid #2 \,\right\}}

\newcommand*\diff{\mathop{}\!\mathrm{d}}
\newcommand*\Diff{\mathop{}\!\mathrm{D}}

\newcommand\restr[2]{{% we make the whole thing an ordinary symbol
  \left.\kern-\nulldelimiterspace % automatically resize the bar with \right
  #1 % the function
  \vphantom{\big|} % pretend it's a little taller at normal size
  \right|_{#2} % this is the delimiter
  }}

% ---------------------------------------------------------------------
% R E N D E R
% ---------------------------------------------------------------------

\newif\ifshowproof
\showprooftrue

\NewEnviron{Proof}{%
    \ifshowproof%
        \begin{proof}%
            \BODY
        \end{proof}%
    \fi%
}%

\begin{document}
\maketitle

\section{Category}
\begin{defbox}
  \begin{definition}
    A category \(\mathcal{C}\) consists of
    \begin{enumerate}
      \item a class of objects, denoted by \(\mathrm{Ob}(\mathcal{C})\), and
      \item a collection of of morphisms, denoted by \(\mathrm{Hom}(\mathcal{C})\)
    \end{enumerate}
  \end{definition}
\end{defbox}

\begin{exmbox}
  \begin{example}
    \begin{enumerate}
      \item The category of sets
      \item The category of groups
      \item the category of rings
      \item the category of fields
      \item For a ring \(A\), the category of left modules, right modules
      \item the single element category
    \end{enumerate}
  \end{example}
\end{exmbox}

\begin{defbox}
  \begin{definition}[Subcategory]
    
  \end{definition}
\end{defbox}

\begin{defbox}
  \begin{definition}[Opposite Category]
    
  \end{definition}
\end{defbox}

\newpage
\section{Functors}

\begin{defbox}
  \begin{definition}
    Let \(\mathcal{C}\) and \(\mathcal{D}\) be two categories. A functor \(F\) from \(\mathcal{C}\) to \(\mathcal{D}\) is a mapping that
    \begin{enumerate}
      \item associates each object \(X\) in \(\mathcal{C}\) to an object \(F(X)\) in \(\mathcal{D}\), and
      \item associates each morphism \(f: X \rightarrow Y\) in \(\mathcal{C}\) to a morphism \(F(f): F(X) \rightarrow F(Y)\) in \(\mathcal{D}\) such that
      \begin{enumerate}
        \item \(F(\mathrm{id}_X) = \mathrm{id}_{F(X)}\) for every object \(X\) in \(\mathcal{C}\), and
        \item \(F(g \circ f) = F(g) \circ F(f)\) for all morphisms \(f: X \rightarrow Y\) and \(g: Y \rightarrow Z\) in \(\mathcal{C}\).
      \end{enumerate}
    \end{enumerate}
  \end{definition}
\end{defbox}

\begin{defbox}
  \begin{definition}[Covariant and Contravariant]
    
  \end{definition}
\end{defbox}

\begin{defbox}
  \begin{definition}[Faithful and Full]
    
  \end{definition}
\end{defbox}

\begin{intbox}
  \begin{intuition}
    Faithful is basically injective, full is basically surjective.
  \end{intuition}
\end{intbox}

\end{document}