\documentclass{book}
\usepackage[utf8]{inputenc}
\usepackage[english]{babel}

\usepackage{mathtools}

\usepackage{amsmath}

\usepackage{amssymb}

% page layout
\usepackage{geometry}
 \geometry{
 a4paper,
 total={170mm,257mm},
 left=20mm,
 top=20mm,
 }

% theorems
\usepackage{amsthm}

\newtheoremstyle{custom_definition}% name of the style to be used
  {\topsep} % measure of space to leave above the theorem. E.g.: 3pt
  {\topsep} % measure of space to leave below the theorem. E.g.: 3pt
  {\normalfont} % name of font to use in the body of the theorem
  {} % measure of space to indent
  {\bfseries} % name of head font
  {.\newline} % punctuation between head and body
  {\topsep}% space after theorem head; " " = normal interword space
  {\thmname{#1}\thmnumber{ #2} --- \thmnote{#3}} % Manually specify head

  \newtheoremstyle{custom_theorem}% name of the style to be used
  {\topsep} % measure of space to leave above the theorem. E.g.: 3pt
  {\topsep} % measure of space to leave below the theorem. E.g.: 3pt
  {\itshape} % name of font to use in the body of the theorem
  {} % measure of space to indent
  {\bfseries} % name of head font
  {.\newline} % punctuation between head and body
  {\topsep}% space after theorem head; " " = normal interword space
  {\thmname{#1}\thmnumber{ #2} --- \thmnote{#3}} % Manually specify head

\theoremstyle{custom_definition}
\newtheorem{definition}{Definition}

\theoremstyle{custom_theorem}
\newtheorem{lemma}{Lemma}
\newtheorem{theorem}{Theorem}

\newcommand{\bigslant}[2]{{\raisebox{.2em}{$#1$}\left/\raisebox{-.2em}{$#2$}\right.}}

\usepackage[lastexercise]{exercise}

\begin{document}
    \begin{definition}[Chart]
        An \(n\)-dimensional chart \((\mathcal{U}, x)\) on a set \(M\) consists of a subset \(\mathcal{U \subseteq M}\) and an injective map \(x: \mathcal{U} \longrightarrow \mathbb{R}^n\) whose image \(x(\mathcal{U}) \subseteq \mathbb{R}^n\) is an open set.
    \end{definition}
    \begin{definition}[Transition Maps]
        Any two charts \((\mathcal{U}, x)\) and \((\mathcal{V}, y)\) determine a pair of transition maps
        \begin{align}
            x(\mathcal{U} \cap \mathcal{V}) \xrightarrow{y \circ x^{-1}} y(\mathcal{U} \cap \mathcal{V}) \\
            y(\mathcal{U} \cap \mathcal{V}) \xrightarrow{x \circ y^{-1}} x(\mathcal{U} \cap \mathcal{V})
        \end{align}
        which are inverse to each other, and are thus bijections between subset of \(\mathbb{R}^n\).
    \end{definition}
    \begin{definition}[\(C^k\)-compatible]
        We say that the two charts are \(C^k\)-compatible for some \(k \in \mathbb{N} \cup \{0, \infty\}\) if the sets \(x(\mathcal{U} \cap \mathcal{V})\) and \(y(\mathcal{U} \cap \mathcal{V})\) are both open and the transition maps \(y \circ x^{-1}\) and \(x \circ y^{-1}\) are both of class \(C^k\). If \(k = \infty\), we say the charts are smoothly compatible.
    \end{definition}
    \begin{definition}[Atlas]
        An atlas of class \(C^k\) for the set \(M\) (or smooth atlas in the case \(k = \infty\)) is a collection of charts \(\mathcal{A} = \{(\mathcal{U}_\alpha, x_\alpha)\}_{\alpha \in I}\) that are all \(C^k\)-compatible with each other, such that \(\bigcup_{\alpha \in I} \mathcal{U} = M\).
    \end{definition}
    \begin{definition}
        For a set \(M\) with an atlas \(\mathcal{A}\) of class \(C^k\) and \(r \in \mathbb{N} \cup \{0, \infty\}\) with \(r \leq k\), a function \(f: M \longrightarrow \mathbb{R}\) is said to be of class \(C^k\) if and only if the function
        \begin{align}
            f \circ x^{-1}: x(\mathcal{U}) \longrightarrow \mathbb{R}
        \end{align}
        is of class \(C^r\) for every chart \((\mathcal{U}, x) \in \mathcal{A}\).
    \end{definition}
    \begin{definition}
        For \(k \in \mathbb{N} \cap \{\infty\}\), a \(C^k\)-structure or differentiable structure of class \(C^k\) on a set \(M\) is a maximal atlas \(\mathcal{A}\) of class \(C^k\) on \(M\). In the case \(k = \infty\), we also call this a smooth structure on \(M\). If \(M\) has been endowed with a \(C^k\)-structure \(\mathcal{A}\), then a chart \((\mathcal{U}, x)\) on \(M\) will be referred to as a \(C^k\)-chart if it belongs to the maximal atlas \(\mathcal{A}\).
    \end{definition}
    \begin{definition}[Topology]
        A topology on a set \(X\) is a collection \(\mathcal{T}\) of subsets of \(X\) satisfying the following axioms:
        \begin{enumerate}
            \item \(\emptyset \in \mathcal{T}\) and \(X \in \mathcal{T}\).
            \item For every subcollection \(I \subseteq \mathcal{T}\) it is \(\bigcup_{\mathcal{U} \in I} \mathcal{U} \in \mathcal{T}\).
            \item For every pair \(\mathcal{U}_1, \mathcal{U}_2 \in \mathcal{T}\) it is \(\mathcal{U}_1 \cap \mathcal{U}_2 \in \mathcal{T}\).
        \end{enumerate}
        The pair \((X, \mathcal{T})\) is then called a topological space, and we call the sets \(\mathcal{U} \in \mathcal{T}\) the open subsets in \((X, \mathcal{T})\).
    \end{definition}
    \begin{definition}[Metrizable]
        
    \end{definition}
    \begin{definition}[]
        Assume \(k \in \mathbb{N} \cup \{\infty\}\). A differentiable manifold of class \(C^k\) or \(C^k\)-manifold is a set \(M\) endowed with a \(C^k\)-structure such that the induced topology on \(M\) is metrizable and separable. In the case \(k = \infty\), we also call \(M\) are smooth manifold.

        We say that \(M\) is \(n\)-dimensional and refer to \(M\) as an \(n\)-manifold, written \(\text{dim}\, M = n\), if every chart in its differentiable structure is \(n\)-dimensional.
    \end{definition}
\end{document}


