\textbf{CHEET SHEET}
\begin{enumerate}
    \item 
\end{enumerate}

\begin{definition}
    Let \(\mathcal{R} \subset \mathcal{P}(X)\) a set of rings. Set
    \begin{align}
        \mathcal{R}^{\uparrow} := \left\{ A \in \mathcal{P}(X) \mid \exists (A_k)_{k \in \mathbb{N}} \text{ in } \mathcal{R} \text{ with } A_k \uparrow A \right\} \subset \mathcal{R} \text{.}
    \end{align}
\end{definition}
\begin{remark}
    \(\mathcal{R}^\uparrow\) is the set of all \(A \in \mathcal{P}(X)\) that can be expressed as a countable many unions of sets in \(\mathcal{R}\).

    In general, \(\mathcal{R}^\uparrow\) is not a set of rings.
\end{remark}
\begin{definition}
    Let \(\mathcal{R} \subset \mathcal{P}(X)\) be a ring of sets and \(\mu: \mathcal{R} \rightarrow [0, \infty]\) a premeasure. For \(A_k \uparrow A\) with \(A_k \in \mathcal{R}\) for \(k \in \mathbb{N}\) define
    \begin{align}
        \tilde{\mu}: \mathcal{R}^\uparrow \rightarrow [0, \infty], \, A \mapsto \tilde{\mu}(A) := \lim_{k \rightarrow \infty} \mu(A_k)\text{.}
    \end{align}
    \(\tilde{\mu}\) is called the first extension of the premeasure \(\mu\).
\end{definition}
\begin{remark}
    In general, \(\tilde{\mu}\) is not a premeasure as \(\mathcal{R}^\uparrow\) need not be a ring of sets.

    \(\tilde{\mu}\) restricted on \(\mathcal{R}\) is identical with \(\mu\), i.e. \(\restr{\tilde{\mu}}{\mathcal{R}} \equiv \mu \).
\end{remark}
\begin{lemma}
    The first extension \(\tilde{\mu}\) is well-defined.
\end{lemma}
\begin{proposition}[Properties of \(\mathcal{R}^\uparrow\)]
    
\end{proposition}
\begin{proposition}[Properties of the First Extension]
    
\end{proposition}
\begin{definition}[Second Extension or the Outer Measure]
    Let \(\mathcal{R} \subset \mathcal{P}(X)\) be a ring of sets, \(\mu: \mathcal{R} \rightarrow [0, \infty]\) a \(\sigma\)-finite premeasure on \(\mathcal{R}\), and \(\tilde{\mu}: \mathcal{R}^\uparrow \rightarrow [0, \infty]\) the first extension of \(\mu\) on \(\mathcal{R}^\uparrow\). Moreover, let \(B \subset X\) be a subset of \(X\). Then, the map
    \begin{align}
        \mu^*: \mathcal{P}(X) \rightarrow [0, \infty], \, B \mapsto \mu^* := \inf \left\{ \tilde\mu(A) \mid A \in \mathcal{R}^\uparrow, \, A \supset B \right\}
    \end{align}
    is called the outer measure induced by \(\tilde{\mu}\) on \(\mathcal{P}(X)\).
\end{definition}

\begin{proposition}[Properties of the Second Extension]

\end{proposition}

\begin{proposition}[Properties of the Outer Measure]

\end{proposition}

\begin{definition}[Lebesgue Outer Measure]
    Let \(\lambda^n: \mathcal{Q}(\mathbb{R}^n) \rightarrow \mathbb{R}_0^+\) the Lebesgue premeasure. The map
    \begin{align}
        \lambda^*: \mathcal{P}(\mathbb{R}^n) \rightarrow [0, \infty], \, B \mapsto \lambda^*(B):= \inf \left\{ \tilde{\lambda}^n (B) \mid A \in \mathcal{Q}(\mathbb{R}^n)^\uparrow, \, A \supset B \right\}
    \end{align}
    is called the Lebesgue outer measure induced by \(\tilde{\lambda^n}\).
\end{definition}

\begin{definition}[Pseudo Metric]
    
\end{definition}

\begin{proposition}
    The outer measure induces a pseudo metric.
\end{proposition}

\begin{proposition}
    The outer measure is continuous.
\end{proposition}

\begin{definition}[Approximation through elements of Rings]
    
\end{definition}

\begin{theorem}
    \begin{align}
        \hat{\mathcal{A}} := \{ A \in \mathcal{P}(X) \mid \text{\(A\) is \(\mathcal{A}\)-approximatable with \(\mu^*\)} \}
    \end{align}
    is a \(\sigma\)-algebra on \(X\).
\end{theorem}