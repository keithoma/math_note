\begin{definition}
    Let \(\mathcal{R} \subset \mathcal{P}(X)\) be a ring of sets. A set function \(\mu \rightarrow [0, \infty]\) is called
    \begin{itemize}
        \item finitely additive if for all disjoint \(A, B \in \mathcal{R}\) it is \(\mu (A \sqcup B) = \mu(A) + \mu(B)\).
        \item \(\sigma\)-additive if for all disjoint \(A_k \in \mathcal{R}\) with \(k \in \mathbb{N}\) and  \(\bigsqcup_{k=1}^\infty A_k \in \mathcal{R}\) it is
        \begin{align}
            \mu \left( \bigsqcup_{k=1}^\infty A_k \right) = \sum_{k=1}^\infty \mu(A_k) \text{.}
        \end{align}
        \item subadditive if for all \(A, B \in \mathcal{R}\) it is \(\mu(A \cup B) \leq \mu(A) + \mu(B)\)
        \item \(\sigma\)-subadditive if for all \(A_k \in \mathcal{R}\) with \(k \in \mathbb{N}\) and  \(\bigcup_{k=1}^\infty A_k \in \mathcal{R}\) it is
        \begin{align}
            \mu \left( \bigcup_{k=1}^\infty A_k \right) \leq \sum_{k=1}^\infty \mu(A_k) \text{.}
        \end{align}
        \item finite if for all \(A \in \mathcal{R}\) it is \(\mu(A) < \infty\).
        \item \(\sigma\)-finite if there exists a collection of subsets \(\{A_k\}_{k \in \mathbb{N}}\) in \(\mathcal{R}\) with \(\mu(A_k) < \infty\) for all \(k \in \mathbb{N}\) such that
        \begin{align}
            \bigcup_{k \in \mathbb{N}} A_k = X \text{.}
        \end{align}
        \item monotonous if for all \(A, B \in \mathcal{R}\) with \(A \subset B\) it is \(\mu(A) \leq \mu(B)\).
    \end{itemize}
\end{definition}
%
\begin{remark}
    In the definition of \(\sigma\)-additivity, checking whether \( \bigsqcup_{k=1}^\infty A_k\) is included in \(\mathcal{R}\) is required. For \(\sigma\)-rings and therefore \(\sigma\)-algebras, it is guranteed that a countable union of disjoint sets are included.

    In general, not all finite set functions \(\mu \rightarrow [0, \infty]\) are \(\sigma\)-finite as \(X\) need not be included in a ring of sets. 
\end{remark}
\begin{definition}[Content]
    Let \(\mathcal{R} \subset \mathcal{P}(X)\) be a ring of sets. A set function \(\mu \rightarrow [0, \infty]\) is called a content if
    \begin{enumerate}
        \item \(\mu(\varnothing) = 0\).
        \item \(\mu\) is finitely additive.
    \end{enumerate}
\end{definition}
%
\begin{definition}[Premeasure]
    Let \(\mathcal{R} \subset \mathcal{P}(X)\) be a ring of sets. A \(\sigma\)-additive content \(\mu \rightarrow [0, \infty]\) is called a premeasure.
\end{definition}
%
\begin{definition}[Measure]
    Let \(\mathcal{A} \subset \mathcal{P}(X)\) a \(\sigma\)-algebra. A \(\sigma\)-additive content \(\mu: \mathcal{A} \rightarrow [0, \infty]\) is called a measure.
\end{definition}
%
\begin{definition}[Lebesgue Content]
    Let \(\mathcal{Q}(\mathbb{R}^n)\) be the ring of sets over \(\mathbb{R}^n\). % how is endlichen Quadersummen called in English?
    \begin{align}
        \mathcal{Q}(\mathbb{R}^n) = \left\{ \bigsqcup_{k=1}^m \, [a_{1, k},\, b_{1, k}) \times \dots \times [a_{n, k},\, b_{n, k}) \, \middle| \, m \in \mathbb{N}; \, a_{i, k}, b_{i, k} \in \mathbb{R}; \, 1 \leq k \leq n \right\}
    \end{align}
    Set \(\lambda^n: \mathcal{Q}(\mathbb{R}^n) \rightarrow \mathbb{R}_0^+\) as
    \begin{align}
        \lambda^n(A) := \sum_{k=1}^m \prod_{i=1}^n (b_{i,k} - a_{i, k})
    \end{align}
    \(\lambda^n\) is the Lebesgue content.
\end{definition}
\begin{theorem}
    \(\lambda^n\) is a well-defined finite content.
\end{theorem}
\begin{theorem}
    \(\lambda^n\) is a premeasure.
\end{theorem}