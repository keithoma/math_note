\documentclass[a4paper]{book}
\title{Integration and Integration}
\author{K}


% ---------------------------------------------------------------------
% P A C K A G E S
% ---------------------------------------------------------------------

% typography and formatting
\usepackage[english]{babel}
\usepackage[utf8]{inputenc}
\usepackage{geometry}
\usepackage{exsheets}
\usepackage{environ}

% mathematics
\usepackage{amsthm} % for theorems, and definitions
\usepackage{amssymb}
\usepackage{amsmath}
\usepackage{textcomp}
%\usepackage{MnSymbol} % for \cupdot

% extra
\usepackage{xcolor}
\usepackage{tikz}

% ---------------------------------------------------------------------
% S E T T I N G
% ---------------------------------------------------------------------

% typography and formatting
\geometry{margin=3cm}

\SetupExSheets{
  counter-format = ch.qu,
  counter-within = chapter,
  question/print = true,
  solution/print = true,
}

% mathematics
\theoremstyle{definition}
\newtheorem{definition}{Definition}[chapter]
\newtheorem{example}{Example}[definition]

\newtheorem{theorem}{Theorem}[definition]
\newtheorem{corollary}{Corollary}
\newtheorem{lemma}{Lemma}[definition]
\newtheorem{proposition}{Proposition}[definition]

\newtheorem*{remark}{Remark}

% extra
\definecolor{mathif}{HTML}{0000A0} % for conditions
\definecolor{maththen}{HTML}{CC5500} % for consequences
\definecolor{mathrem}{HTML}{8b008b} % for notes

\usetikzlibrary{positioning}
\usetikzlibrary{shapes.geometric, arrows}

% ---------------------------------------------------------------------
% C O M M A N D S
% ---------------------------------------------------------------------

\newcommand{\norm}[1]{\left\lVert#1\right\rVert}
\newcommand{\rank}{\text{rank}}
\newcommand{\Vol}{\text{Vol}}
\newcommand*\diff{\mathop{}\!\mathrm{d}}
\newcommand*\Diff{\mathop{}\!\mathrm{D}}

\newcommand\restr[2]{{% we make the whole thing an ordinary symbol
  \left.\kern-\nulldelimiterspace % automatically resize the bar with \right
  #1 % the function
  \vphantom{\big|} % pretend it's a little taller at normal size
  \right|_{#2} % this is the delimiter
  }}

% ---------------------------------------------------------------------
% R E N D E R
% ---------------------------------------------------------------------

\newif\ifshowproof
\showprooftrue

\NewEnviron{Proof}{%
    \ifshowproof%
        \begin{proof}%
            \BODY
        \end{proof}%
    \fi%
}%

\begin{document}
\maketitle
\tableofcontents
%%%%%%%%%%%%%%%%%%%%%%%%%%%%%%%%%%%%%%%%%%%%%%%%%%%%%%%%%%%%%%%%%%%%%%%%%%%%%%%
\chapter*{Introduction}
\addcontentsline{toc}{chapter}{Introduction}
%
\part{\(\sigma\)-algebra and measures}
%
\chapter{Family of Sets}
%We have the following tree of inclusion.
\begin{figure}[h]
    \center
    \begin{tikzpicture}[node distance=2cm and 1cm]
        \tikzstyle{box} = [draw=none, minimum width=4cm]
        \tikzstyle{arrow} = [thick,-,>=stealth]
        \node [draw=none]                 (ring of sets) {ring of sets};
        \coordinate[below=of ring of sets] (c);
        \node [box, left=of c]      (algebra of sets) {algebra of sets};
        \node [box] at (c)     (s-ring)     {\(\sigma\)-ring};
        \node [box, right=of c] (monotone class)     {monotone class};
        \node [box, below=of c]     (s-algebra)    {\(\sigma\)-algebra};
        \node [box, below=of s-algebra] (b-algebra) {Borel \(\sigma\)-algebra};
        \draw [arrow] (ring of sets) -- (algebra of sets);
        \draw [arrow] (ring of sets) -- (s-ring);
        \draw [arrow] (algebra of sets) -- (s-algebra);
        \draw [arrow] (s-ring) -- (s-algebra);
        \draw [arrow] (monotone class) -- (s-algebra);
        \draw [arrow] (s-algebra) -- (b-algebra);
    \end{tikzpicture}
\end{figure}
NOTATION GUIDE:
\begin{enumerate}
    \item \(X\) as the superset
    \item \(\mathcal{P}(X)\) is the power set of \(X\).
    \item \(A, B \in \mathcal{P}(X)\) as subsets
    \item \(\mathcal{R}, \mathcal{A} \subset \mathcal{P}(X)\) system of subsets
\end{enumerate}
%
\section{Symmetric Difference}
\begin{definition}[Symmetric difference]
    Let {\color{mathif}\(A, B\)} be {\color{mathif}sets}. The binary set operation {\color{maththen}symmetric difference} is defined as
    \begin{align}
        A \triangle B := (A \setminus B) \cup (B \setminus A) \text{.}
    \end{align}
    In other words, \(x \in A \triangle B\) implies \(x\) is either in \(A\) or \(B\), but not in both.
\end{definition}
\begin{proposition}[Properties of Symmetric Difference]
    Let \(A, B, C, X\) and \(Y\) be {\color{mathif}sets}. Moreover, let \(A_i\) and \(X_i\) be {\color{mathif}sets} with an {\color{mathif}arbitary non-empty index set} \(i \in I\). Then, the following {\color{maththen}identities} hold.
    \begin{enumerate}
        \item \( A \triangle B = (A \cup B) \setminus (A \cap B) \).
        \item \( ( A \triangle B) \triangle C = A \triangle (B \triangle C) \). (Symmetric difference is {\color{mathrem}associative}.) 
        \item \( A \triangle B = B \triangle A \). (Symmetric difference is {\color{mathrem}commutative}.)
        \item \( A \triangle \emptyset = A\) and \( A \triangle A = \emptyset\)
        \item \( (A \triangle B) \cup C = (A \cup C) \triangle (B \cup C)\).
        \item \( A \cap B = \emptyset \Rightarrow A \triangle B = A \cup B \).
        \item \( B \subset A \Rightarrow A \triangle B = A \setminus B \).
        \item \(X \cap Y = \emptyset \Rightarrow A \cap B \subset (X \triangle A) \cup (Y \triangle B) \).
        \item \( (\bigcup_{i \in I} X_i) \triangle (\bigcup_{i \in I} A_i) \subset \bigcup_{i \in I} (X_i \triangle A_i) \)
    \end{enumerate}
\end{proposition}
\begin{Proof}
    Elementary.
\end{Proof}
%
%
%
%
%
\section{Ring of Sets}
\begin{definition}[Ring of sets]
    There are two equivalent definitions. Let {\color{mathif}\(X\)} be a {\color{mathif}set} and {\color{mathif}\(\mathcal{R} \subset \mathcal{P}(X)\)} a {\color{mathif}system of subsets}. Then {\color{maththen} \(\mathcal{R}\)}  is a {\color{maththen}ring of sets over \(X\)}, if
    \begin{enumerate}
        \item the following axioms are met.
        \begin{enumerate}
            \item \(\mathcal{R} \neq \emptyset\) (\(\mathcal{R}\) is {\color{mathrem}nonempty.})
            \item \(A, B \in \mathcal{R} \Rightarrow A \setminus B \in \mathcal{R}\) (\(\mathcal{R}\) is {\color{mathrem}closed under relative complement}.)
            \item \(A, B \in \mathcal{R} \Rightarrow A \cup B \in \mathcal{R}\) (\(\mathcal{R}\) is {\color{mathrem}closed under finite unions}.)
        \end{enumerate}
        \item \((\mathcal{R}, \triangle, \cap)\) is a ring in the algebraic sense, with \(\triangle\) as addition and \(\cap\) as multiplication.
    \end{enumerate}
\end{definition}
\begin{Proof}
    We show that the two definitions above are indeed equivalent.

    \((1 \Rightarrow 2)\) Let \(\mathcal{R}\) be nonempty, closed under the relative complement, and closed under finite unions. First, consider \((\mathcal{R}, \triangle)\). Let \(A, B \in \mathcal{R}\). It is
    \begin{enumerate}
        \item (Closure under addition) \(A \cup B \in \mathcal{R}\) because \(\mathcal{R}\) is closed under finite unions. We also have \(A \cap B = A \setminus (A \setminus B) \in \mathcal{R}\) as \(\mathcal{R}\) is closed under the relative complement. From these it follows that \( A \triangle B = (A \cup B) \setminus (A \cap B) \in \mathcal{R}\) by using the closure under the relative complement again.
        \item (Associativity)
        \item (Commutativity)
        \item (Neutral element) \(\emptyset\)
        \item (Inverse element) \(A\)
    \end{enumerate}
    Therefore, \((\mathcal{R}, \triangle)\) is an abelian group. Secondly, consider \((\mathcal{R}, \cap)\). \(\cap\) is associative and commutative. The identity element is the union of all sets (does this exist??).
\end{Proof}
%
\begin{remark}
    Since we have the identity \(A \cap B = A \setminus (A \setminus B)\), the condition that \(\mathcal{R}\) is closed under the relative complement, i.e.
    \begin{align}
        A, B \in \mathcal{R} \Rightarrow A \setminus B \in \mathcal{R}
    \end{align}
    can be replaced with closure under finite intersection, therefore
    \begin{align}
        A, B \in \mathcal{R} \Rightarrow A \cap B \in \mathcal{R} \text{.}
    \end{align}
\end{remark}
%%%%%%%%%% This is already implied in the definition
% \begin{proposition}[Properties of ring of sets]
%     Let \(\mathcal{R}\) be a ring of sets. It is
%     \begin{enumerate}
%         \item \(\emptyset \in \mathcal{R}\).
%         \item \(A, B \in \mathcal{R} \Rightarrow A \triangle B \in \mathcal{R}\)
%     \end{enumerate}
% \end{proposition}
%%%%%%%%%%
\begin{example}
    Let \(X\) be a set.
    \begin{enumerate}
        \item \(\mathcal{P}(X)\) and \(\{\emptyset, X\}\) are ring of sets.
        \item \(\{\emptyset\}\) is a ring of sets.
    \end{enumerate}
\end{example}
\section{Algebra of Sets}
\begin{definition}[Algebra of sets]
    There are two equivalent definitions. Let {\color{mathif}\(X\)} be a {\color{mathif}set} and {\color{mathif}\(\mathcal{R} \subset \mathcal{P}(X)\)} a {\color{mathif}system of subsets}. Then {\color{maththen} \(\mathcal{A}\)}  is a {\color{maththen}algebra of sets over \(X\)},
    \begin{enumerate}
        \item if \(\mathcal{A}\) is a {\color{mathif}ring of sets} that contains {\color{mathif}\(X\)}, or
        \item if the following axioms are met
        \begin{enumerate}
            \item \(\mathcal{A} \neq \emptyset\) (\(\mathcal{A}\) is {\color{mathrem}nonempty.})
            \item \(A \in \mathcal{A} \Rightarrow A^c \in \mathcal{A}\) (\(\mathcal{R}\) is {\color{mathrem}closed under the absolute complement}.)
            \item \(A, B \in \mathcal{A} \Rightarrow A \cup B \in \mathcal{A}\) (\(\mathcal{R}\) is {\color{mathrem}closed under finite unions}.)
        \end{enumerate}
    \end{enumerate}
\end{definition}
%
%
%
%
%
\section{\(\sigma\)-Ring}
\begin{definition}[\(\sigma\)-Ring]
    Let \(X\) be set and \(\mathcal{R} \subset \mathcal{P}(X)\) a system of subsets. \(\mathcal{R}\) is a \(\sigma\)-ring over \(X\), if
        \begin{enumerate}
            \item \(\mathcal{R} \neq \emptyset\). (\(\mathcal{A}\) is {\color{mathrem}nonempty.})
            \item \(A, B \in \mathcal{R} \Rightarrow A \setminus B \in \mathcal{R}\) ({\color{mathrem}closed under the relative complement}.)
            \item \(A_1, A_2, A_3, ... \in \mathcal{R} \Rightarrow \bigcup_{k=1}^\infty A_k \in \mathcal{R}\) ({\color{mathrem} Closed under countable unions.})
        \end{enumerate}
\end{definition}
%
%
%
%
%
\section{Monotone Class}
\begin{definition}[Notation for Monotonous Sequence of Sets]
    % Let \((X_k)_{k \in \mathbb{N}}\) and \((Y_k)_{k \in \mathbb{N}}\) be two sequence of sets. We write
    % \begin{align}
    %     X_k \uparrow X && Y_k \downarrow Y
    % \end{align}
    % if \(X_k\) and \(Y_k\) are monotonously increasing or decreasing, i.e. \(X_k \subset X_{k + 1}\) or \(Y_k \supset Y_{k + 1}\), and
\end{definition}
\begin{definition}[Monotone class]
    Let \(\mathcal{M} \subset \mathcal{P}(\Omega)\) a system of sets and \(k \in \mathbb{N}^*\). Then, \(\mathcal{M}\) is a monotone class, if
    \begin{enumerate}
        \item Let \(X_k \in \mathcal{M}\) with \(X_k \uparrow X\), then \(X \in \mathcal{M}\).
        \item Let \(Y_k \in \mathcal{M}\) with \(Y_k \downarrow X\), then \(Y \in \mathcal{M}\).
    \end{enumerate}
    Intersection of arbitary many monotonous class is again a monotonous class. Therefore, for all \(\mathcal{E} \subset \mathcal{P}(\Omega)\) with \(\mathcal{E} \neq \emptyset\) there exists the smallest monotonous class around \(\mathcal{E}\)
    \begin{align}
        \mathcal{M}_{\mathcal{E}} := \bigcap_{\mathcal{M} \text{ is monotonous class}, \mathcal{E} \subset \mathcal{M}} \mathcal{M}
    \end{align}
\end{definition}
\begin{remark}
    All \(\sigma\)-algebras are monotone class.
\end{remark}
\begin{theorem}
    Let \(\mathcal{A} \subset \mathcal{P}(\Omega)\) an algebra of sets. Then, the following are equivalent
    \begin{itemize}
        \item \(\mathcal{A}\) is a \(\sigma\)-algebra.
        \item For \(A_k \uparrow A\), \(A \in \mathcal{A}\).
    \end{itemize}
\end{theorem}
%
%
%
%
%
\section{\(\sigma\)-Algebra}
\textbf{CHEAT SHEET}
\begin{enumerate}
    \item closed under complementation (absolute and relative)
    \item closed under countable unions
    \item closed under countable intersections
    \item closed under symmetric differences
\end{enumerate}
\begin{definition}[\(\sigma\)-algebra]
    Let \(X\) be a set and \(\mathcal{A} \subset \mathcal{P}(X)\) a system of subsets. \(\mathcal{A}\) is a \(\sigma\)-algebra over \(X\), if
        \begin{enumerate}
            \item \(\mathcal{A} \neq \emptyset\).
            \item \(A \in \mathcal{A} \Rightarrow A^c \in \mathcal{A}\)
            \item \(A_1, A_2, A_3, ... \in \mathcal{A} \Rightarrow \bigcup_{k=1}^\infty A_k \in \mathcal{A}\)
        \end{enumerate}
\end{definition}
\begin{example}
    Trivial examples for the above structures.
\end{example}
\begin{definition}
    Let \(\mathcal{E} \subset \mathcal{P}(\Omega)\) be a system of sets. Define
    \begin{align}
        \mathcal{F}(\mathcal{E}) &:= \left\{ \mathcal{A} \subset \mathcal{P}(\Omega) \middle| \mathcal{E} \subset \mathcal{A}, \mathcal{A} \sigma\text{-Algebra} \right\} \\
        \left< \mathcal{E}  \right>^{\sigma} &:= \sigma(\mathcal{E}) := \bigcap_{\mathcal{A} \in \mathcal{F}(\mathcal{E})} \mathcal{A}
    \end{align}
    The first is the family of all \(\sigma\)-algebras that contain \(\mathcal{E}\).
    The second is the smallest \(\sigma\)-algebra that contains \(\mathcal{E}\).
\end{definition}
%
%
%
%
%
\section{Product Algebra??}
\begin{definition}
    Let \(\Omega_1\) and \(\Omega_1\) be sets; let \(\mathcal{R}_1 \subset \mathcal{P}(\Omega_1)\) and \(\mathcal{R}_2 \subset \mathcal{P}(\Omega_2)\) be ring of sets, and \(\Omega := \Omega_1 \times \Omega_2\). Define
    \begin{align}
        \mathcal{R} := \mathcal{R}_1 \boxtimes \mathcal{R}_2 := \left\{ \bigcup_{i=1}^m A_i \times B_i \middle| A_i \in \mathcal{R}_1, B_i \in \mathcal{R}_2, m \in \mathbb{N} \right\}
    \end{align}
    \(\mathcal{R}\) is a ring of sets over \(\Omega\).
\end{definition}
\begin{theorem}
    In above definition, if \(\mathcal{R}_1\) and \(\mathcal{R}_2\) are algebra of sets, then \(\mathcal{R}\) is a algebra of set.
\end{theorem}
\begin{theorem}
    \begin{align}
        \mathfrak{Q}(\mathbb{R}^n)
    \end{align}
    is a ring of sets.
\end{theorem}
\begin{remark}
    From \(\mathfrak{Q}(\mathbb{R}^n)\) we can construct one very important \(\sigma\)-algebra, the Borel-Algebra of \(\mathbb{R}^n\).
\end{remark}
\begin{definition}[Products of \(\sigma\)-algebras]
    Let \(\mathcal{A}_1\) and \(\mathcal{A}_2\) be \(\sigma\)-algebras on \(\Omega_1, \Omega_2\). Then, let
    \begin{align}
        \mathcal{A}_1 \otimes \mathcal{A}_2 = \sigma( \mathcal{A}_1 \boxtimes \mathcal{A}_2 )
    \end{align}
\end{definition}
\begin{example}
    \begin{align}
        \mathcal{B}(\mathbb{R}^{n+m}) = \mathcal{B}(\mathbb{R}^n) \otimes \mathcal{B}(\mathbb{R}^m)
    \end{align}
\end{example}
\begin{definition}
    Let \((X_k)_{k \in \mathbb{N}^*}\) be a sequence of sets with \(X_1 \subset X_2 \subset X_3 \subset \dots \) and \(X := \lim_{k \rightarrow \infty} := \bigcup_{k \in \mathbb{N}*} X_k\).
    Similar for monotonously decreasing.
\end{definition}

%
%
%
%
%
\section{Rectangles}
\begin{example}
    Let
    \begin{align}
        \mathfrak{Q}(\mathbb{R}) := \left\{ \bigcup_{i=1}^m [a_i, b_i)  \middle| m \in \mathbb{N}; a_i, b_i \in \mathbb{R} \right\}
    \end{align}
    be the set of all unions of finitely many right half open intervals on \(\mathbb{R}\). Then, \(\mathfrak{Q}(\mathbb{R})\) is a set of rings. Similary for the left half open sets, but not for open or closed intervals!
    \(\mathfrak{Q}(\mathbb{R})\) is neither \(\sigma\)-ring, \(\sigma\)-algebra nor an algebra of sets.
    One can generalize this to higher dimensions.
\end{example}
%
%
%
%
%
%
%
%
%
%
\section{Borel \(\sigma\)-algebra}
\begin{definition}
    Let \(\Omega\) be a set. A collection \(\mathcal{U} \subset \mathcal{P}(\Omega)\) of subsets of X is called a topology on \(X\) if it satisfies the following axioms.
    \begin{enumerate}
        \item \(\emptyset, X \in \mathcal{U}\).
        \item If \(n \in \mathbb{N}\) and \(U_1, \dots U_n \in \mathcal{U}\) then \(\bigcap_{i=1}^n U_i \in \mathcal{U}\).
        \item If \(I\) is any index set and \(U_i \in \mathcal{U}\) for \(i \in I\) then \(\bigcup_{i \in I} U_i \in \mathcal{U}\).
    \end{enumerate}
    A topological space is a pair \((\Omega, \mathcal{U})\) consisting of a set \(\Omega\) and a topology \(\mathcal{U} \in \mathcal{P}(\Omega)\).
\end{definition}
\begin{example}[Standard Topology on \(\overline{\mathbb{R}}\)]
    The set of open subsets \(\mathcal{T}\) of \(\overline{\mathbb{R}}\) is the standard topology on \(\overline{\mathbb{R}}\). Concretely, \(\mathcal{T}\) contains countable union of open intervals in \(\mathbb{R}\) and sets of the form \((a, \infty]\) or \([-\infty, b)\) for \(a, b \in \mathbb{R}\).
\end{example}
\begin{definition}[Borel algebra]
    Let \((\Omega, \mathcal{T})\) be a topological space, then \(\mathcal{B}(\Omega) := \sigma(\mathcal{T})\) is the Borel \(\sigma\)-algebra of \(\Omega\). The elments of \(\mathcal{B}\) are called Borel (measurable) sets.
    There are many ways to generate this algebra.
\end{definition}
\begin{theorem}
    Let \((\Omega, \mathcal{T})\) be a topological space. Then the following holds.
    \begin{enumerate}
        \item Every closed subset \(F \subset \Omega\) is a Borel set.
        \item Every countable union \(\bigcup_{i=1}^\infty F_i\) of closed subsets \(F_i \subset \Omega\) is a Borel set.
        \item Every countable intersection \(\bigcap_{i=1}^\infty F_i\) of open subsets \(F_i \subset \Omega\) is a Borel set.
    \end{enumerate}
\end{theorem}
\begin{theorem}
    It is
    \begin{align}
        \mathcal{B}(\mathbb{R}^n) = \sigma(\mathfrak{Q}(\mathbb{R}^n))
    \end{align}
    Moreover, define
    \begin{align}
        \mathfrak{Q}_{\mathbb{Q}}(\mathbb{R}^n) := \left\{ \bigcup_{i=1}^m [a_{1, i}, b_{1, i}) \times \dots [a_{n, i} \times b_{n, i}) \middle| m \in \mathbb{N}; a_{\nu, i}, b_{\nu, i} \in \mathbb{Q}; \nu = 1, \dots, n \right\}
    \end{align}
    the ring of sets of finite unions of quadern with rational edge points. Then, we even have
    \begin{align}
        \mathcal{R}(\mathbb{R}^n) = \sigma( \mathfrak{Q}_{\mathbb{Q}} (\mathbb{R}^n))
    \end{align}
\end{theorem}
\begin{lemma}
    Open subsets \(U \subset \mathbb{R}^n\) are disjoint union of countably many right half open dices with edge points in \(\mathbb{Q}^n\)
\end{lemma}
%
%
%
\section{Exercises}
\begin{question}
    Let \(X\) be a nonempty set and for all \(1 \leq i \leq m\) with \(m \in \mathbb{N}\) let \(A_i \subset X\) be a finite amount of subsets. Set
    \begin{align}
        S := A_1 \triangle A_2 \triangle \dots \triangle A_m \text{.}
    \end{align}

    Because of the associative property of the symmetric difference, \(S\) is uniquely defined regardless of the order of the operations.

    Show that \(x \in X\) belongs to \(S\) if and only if \(x\) belongs to an odd number of sets \(A_k\), i.e. when the number of indices \(k \in \{1, 2, \dots, m\}\) with \(x \in A_k\) is odd.
\end{question}
\begin{solution}
    % trivial 
\end{solution}
%
%
%
\begin{question}
    Let \(X\) be a nonempty set and \(R := \{f: X \rightarrow \mathbb{F}_2\}\) where \(\mathbb{F}_2 = \{0, 1\}\) is a field of two elements equipped with the common addition and the common multiplication. Moreover, define the operations
    \begin{align}
        (f \oplus g)(x) &:= f(x) + g(x) \\
        (f \otimes g)(x) &:= f(x) \cdot g(x) \text{.}
    \end{align}
    Show the following statements.
    \begin{enumerate}
        \item \((R, \oplus, \otimes)\) is a commutative ring with the identity element.
        \item The map \(\mathcal{P}(X) \rightarrow R, A \mapsto \chi_A\) that maps a subset \(A \subset X\) to its characteristic function is bijective.
        \item For all \(A, B \in \mathcal{P}(X)\) we have
        \begin{align}
            \chi_{A \triangle B} = \chi_A \oplus \chi_B && \chi_{A \cap B} = \chi_A \otimes \chi_B \text{.}
        \end{align}
        \item Conclude from the statements above that \(\mathcal{P}(X)\) is isomorphic to \(R\) as a ring with \(\triangle\) as its addition and with \(\cap\) as its multiplication.
        \item A subset \(\mathcal{R} \subset \mathcal{P}(X)\) is a ring of sets if and only if \(\mathcal{R}\) is a subring of \(\mathcal{P}(X)\) with respects to the ring structure defined above.
    \end{enumerate}
\end{question}
\begin{solution}
    
\end{solution}
%
%
%
\begin{question}
    %PRAESENZUEBUNG 2.1
\end{question}
%
%
%
\begin{question}
    %PREASENZUEBUNG 2.2
\end{question}
% UEBUNGSBLATT 2
\begin{question}
    Show explicitly that the following subsets generate the same \(\sigma\)-algebra on \(\mathbb{R}\).
    \begin{align}
        \mathcal{E}_1 &:= \{(a, b) \mid a, b \in \mathbb{R}, \, a \leq b\} &         \mathcal{E}_2 &:= \{[a, b) \mid a, b \in \mathbb{Q}, \, a \leq b\} \\
        \mathcal{E}_3 &:= \{[a, b) \mid a, b \in \mathbb{R}, \, a \leq b\} &
        \mathcal{E}_4 &:= \{(-\infty, b) \mid a, b \in \mathbb{Q}, \, a \leq b\}
    \end{align}
\end{question}
\begin{solution}
    We want to proof
    \begin{align}
        \sigma (\mathcal{E}_1) = \sigma (\mathcal{E}_2) = \sigma (\mathcal{E}_3) = \sigma (\mathcal{E}_4) \text{.}
    \end{align}
    We will do this by showing four inclusions. In each step, our goal is to show that an arbitary interval from the generator of the superset is included in the \(\sigma\)-algebra of the subset.
    \begin{enumerate}
    \item First we show \(\sigma (\mathcal{E}_1) \subset \sigma (\mathcal{E}_2)\). Fix \(a, b \in \mathbb{Q}\) with \(a \leq b\) and consider the interval \([a, b)\). If \(a = b\), then the interval is empty and \([a, b) = \varnothing \in \sigma (\mathcal{E}_1)\) immediately. Now let \(x, y \in \mathbb{R}\) with \(x < y < a\). The set \((x, a)^c \cap (y, b)\) is included in \(\sigma (\mathcal{E}_1)\) as \(\sigma\)-algebras are closed under absolute complements and intersections. We also have
    \begin{align}
        (x, a)^c \cap (y, b) = \left( (-\infty, x] \cup [a, \infty] \right) \cap (y, b) = [a, b) \text{.}
    \end{align}
    Therefore, it follows that \([a, b) \in \sigma (\mathcal{E}_1) \) and hence \(\sigma (\mathcal{E}_1) \subset \sigma (\mathcal{E}_2)\).
    \item Next, we show \( \sigma (\mathcal{E}_2) \subset \sigma (\mathcal{E}_3) \). As before, fix \(a, b \in \mathbb{R}\) with \(a \leq b\) and consider the interval \([a, b)\). If this interval is empty, it is included in \(\sigma (\mathcal{E}_2)\), so assume \(a < b\). Let \((a_k)_{k \in \mathbb{N}}\) and \((b_k)_{k \in \mathbb{N}}\) sequences in \(\mathbb{Q}\) with \(a < a_k\) and \(b_k < b\) and with \(a\) and \(b\) as their limits respectively. Since a \(\sigma\)-algebra is closed under countable unions, \(\bigcup_{k=1}^\infty [a_k, b_k)\) is included in \(\sigma (\mathcal{E}_2)\), but we also have
    \begin{align}
        \bigcup_{k=1}^\infty [a_k, b_k) = \lim_{k \rightarrow \infty} [a_k, b_k) = [a, b)
    \end{align}
    We conclude that \([a, b) \in \sigma (\mathcal{E}_2)\) and therefore, \(\sigma (\mathcal{E}_2) \subset \sigma (\mathcal{E}_3)\).
    \item Now we will show \( \sigma (\mathcal{E}_3) \subset \sigma (\mathcal{E}_4)\). Again, fix \(b \in \mathbb{Q}\) and consider \((-\infty, b)\). Let \((x_k)_{k \in \mathbb{N}}\) be a sequence in \(\mathbb{Q}\) with \(x_k < b\) for each \(k \in \mathbb{N}\) and diverging to negative infinity. As \(\sigma\)-algebras are closed under countable unions, we have \(\bigcup_{k=1}^\infty (x_k, b) \in \sigma( \mathcal{E}_3 )\). On the other hand, it is
    \begin{align}
        \bigcup_{k=1}^\infty (x_k, b) = \lim_{k \rightarrow \infty} (x_k, b) = (-\infty, b) \text{.}
    \end{align}
    This means that \((-\infty, b) \in \sigma (\mathcal{E}_3)\) and from this we have \(\sigma (\mathcal{E}_3) \subset \sigma (\mathcal{E}_4)\).
    \item Lastly, we want to show \(\mathcal{E}_4 \subset \mathcal{E}_1\). Fix \(a, b \in \mathbb{R}\) with \(a \leq b\) and consider \((a, b)\). Again, if \(a = b\), then the interval is empty and included in \(\sigma (\mathcal{E}_4) \). So assume \(a < b\). \\
    Let \( [x, y)\) with \(x, y \in \mathbb{Q}\) and \(x < y\) be a half left open interval. This is included in \(\sigma( \mathcal{E}_1)\) because \((-\infty, x)^c \cap (-\infty, y) = [x, y)\) and \(\sigma\)-algebras are closed under absolute complements and intersections. \\
    Now let \((x_k)_{k \in \mathbb{N}}\) be a sequence in \(\mathbb{Q}\) with \(x_k < y\) for all \(k \in \mathbb{N}\) and converging to \(x\), then \(\bigcap_{k = 1}^\infty [x, x_k) = \lim_{k \rightarrow \infty} [x, x_k]= \{x\}\) is also in \(\sigma(\mathcal{E}_1)\).
    \end{enumerate}
\end{solution}
%
\chapter{Measure}
\section{Content, Premeasure, and Measure}
\begin{definition}
    Let \(\mathcal{R} \subset \mathcal{P}(X)\) be a ring of sets. A set function \(\mu \rightarrow [0, \infty]\) is called
    \begin{itemize}
        \item finitely additive if for all disjoint \(A, B \in \mathcal{R}\) it is \(\mu (A \sqcup B) = \mu(A) + \mu(B)\).
        \item \(\sigma\)-additive if for all disjoint \(A_k \in \mathcal{R}\) with \(k \in \mathbb{N}\) and  \(\bigsqcup_{k=1}^\infty A_k \in \mathcal{R}\) it is
        \begin{align}
            \mu \left( \bigsqcup_{k=1}^\infty A_k \right) = \sum_{k=1}^\infty \mu(A_k) \text{.}
        \end{align}
        \item subadditive if for all \(A, B \in \mathcal{R}\) it is \(\mu(A \cup B) \leq \mu(A) + \mu(B)\)
        \item \(\sigma\)-subadditive if for all \(A_k \in \mathcal{R}\) with \(k \in \mathbb{N}\) and  \(\bigcup_{k=1}^\infty A_k \in \mathcal{R}\) it is
        \begin{align}
            \mu \left( \bigcup_{k=1}^\infty A_k \right) \leq \sum_{k=1}^\infty \mu(A_k) \text{.}
        \end{align}
        \item finite if for all \(A \in \mathcal{R}\) it is \(\mu(A) < \infty\).
        \item \(\sigma\)-finite if there exists a collection of subsets \(\{A_k\}_{k \in \mathbb{N}}\) in \(\mathcal{R}\) with \(\mu(A_k) < \infty\) for all \(k \in \mathbb{N}\) such that
        \begin{align}
            \bigcup_{k \in \mathbb{N}} A_k = X \text{.}
        \end{align}
        \item monotonous if for all \(A, B \in \mathcal{R}\) with \(A \subset B\) it is \(\mu(A) \leq \mu(B)\).
    \end{itemize}
\end{definition}
%
\begin{remark}
    In the definition of \(\sigma\)-additivity, checking whether \( \bigsqcup_{k=1}^\infty A_k\) is included in \(\mathcal{R}\) is required. For \(\sigma\)-rings and therefore \(\sigma\)-algebras, it is guranteed that a countable union of disjoint sets are included.

    In general, not all finite set functions \(\mu \rightarrow [0, \infty]\) are \(\sigma\)-finite as \(X\) need not be included in a ring of sets. 
\end{remark}
\begin{definition}[Content]
    Let \(\mathcal{R} \subset \mathcal{P}(X)\) be a ring of sets. A set function \(\mu \rightarrow [0, \infty]\) is called a content if
    \begin{enumerate}
        \item \(\mu(\varnothing) = 0\).
        \item \(\mu\) is finitely additive.
    \end{enumerate}
\end{definition}
%
\begin{definition}[Premeasure]
    Let \(\mathcal{R} \subset \mathcal{P}(X)\) be a ring of sets. A \(\sigma\)-additive content \(\mu \rightarrow [0, \infty]\) is called a premeasure.
\end{definition}
%
\begin{definition}[Measure]
    Let \(\mathcal{A} \subset \mathcal{P}(X)\) a \(\sigma\)-algebra. A \(\sigma\)-additive content \(\mu: \mathcal{A} \rightarrow [0, \infty]\) is called a measure.
\end{definition}
\section{Lebesgue Content}
\begin{definition}[Lebesgue Content]
    Let \(\mathcal{Q}(\mathbb{R}^n)\) be the ring of sets over \(\mathbb{R}^n\). % how is endlichen Quadersummen called in English?
    \begin{align}
        \mathcal{Q}(\mathbb{R}^n) = \left\{ \bigsqcup_{k=1}^m \, [a_{1, k},\, b_{1, k}) \times \dots \times [a_{n, k},\, b_{n, k}) \, \middle| \, m \in \mathbb{N}; \, a_{i, k}, b_{i, k} \in \mathbb{R}; \, 1 \leq k \leq n \right\}
    \end{align}
    Set \(\lambda^n: \mathcal{Q}(\mathbb{R}^n) \rightarrow \mathbb{R}_0^+\) as
    \begin{align}
        \lambda^n(A) := \sum_{k=1}^m \prod_{i=1}^n (b_{i,k} - a_{i, k})
    \end{align}
    \(\lambda^n\) is the Lebesgue content.
\end{definition}
\begin{theorem}
    \(\lambda^n\) is a well-defined finite content.
\end{theorem}
\begin{theorem}
    \(\lambda^n\) is a premeasure.
\end{theorem}
\section{Lebesgue Measure}
\textbf{CHEET SHEET}
\begin{enumerate}
    \item 
\end{enumerate}

\begin{definition}
    Let \(\mathcal{R} \subset \mathcal{P}(X)\) a set of rings. Set
    \begin{align}
        \mathcal{R}^{\uparrow} := \left\{ A \in \mathcal{P}(X) \mid \exists (A_k)_{k \in \mathbb{N}} \text{ in } \mathcal{R} \text{ with } A_k \uparrow A \right\} \subset \mathcal{R} \text{.}
    \end{align}
\end{definition}
\begin{remark}
    \(\mathcal{R}^\uparrow\) is the set of all \(A \in \mathcal{P}(X)\) that can be expressed as a countable many unions of sets in \(\mathcal{R}\).

    In general, \(\mathcal{R}^\uparrow\) is not a set of rings.
\end{remark}
\begin{definition}
    Let \(\mathcal{R} \subset \mathcal{P}(X)\) be a ring of sets and \(\mu: \mathcal{R} \rightarrow [0, \infty]\) a premeasure. For \(A_k \uparrow A\) with \(A_k \in \mathcal{R}\) for \(k \in \mathbb{N}\) define
    \begin{align}
        \tilde{\mu}: \mathcal{R}^\uparrow \rightarrow [0, \infty], \, A \mapsto \tilde{\mu}(A) := \lim_{k \rightarrow \infty} \mu(A_k)\text{.}
    \end{align}
    \(\tilde{\mu}\) is called the first extension of the premeasure \(\mu\).
\end{definition}
\begin{remark}
    In general, \(\tilde{\mu}\) is not a premeasure as \(\mathcal{R}^\uparrow\) need not be a ring of sets.

    \(\tilde{\mu}\) restricted on \(\mathcal{R}\) is identical with \(\mu\), i.e. \(\restr{\tilde{\mu}}{\mathcal{R}} \equiv \mu \).
\end{remark}
\begin{lemma}
    The first extension \(\tilde{\mu}\) is well-defined.
\end{lemma}
\begin{proposition}[Properties of \(\mathcal{R}^\uparrow\)]
    
\end{proposition}
\begin{proposition}[Properties of the First Extension]
    
\end{proposition}
\begin{definition}[Second Extension or the Outer Measure]
    Let \(\mathcal{R} \subset \mathcal{P}(X)\) be a ring of sets, \(\mu: \mathcal{R} \rightarrow [0, \infty]\) a \(\sigma\)-finite premeasure on \(\mathcal{R}\), and \(\tilde{\mu}: \mathcal{R}^\uparrow \rightarrow [0, \infty]\) the first extension of \(\mu\) on \(\mathcal{R}^\uparrow\). Moreover, let \(B \subset X\) be a subset of \(X\). Then, the map
    \begin{align}
        \mu^*: \mathcal{P}(X) \rightarrow [0, \infty], \, B \mapsto \mu^* := \inf \left\{ \tilde\mu(A) \mid A \in \mathcal{R}^\uparrow, \, A \supset B \right\}
    \end{align}
    is called the outer measure induced by \(\tilde{\mu}\) on \(\mathcal{P}(X)\).
\end{definition}

\begin{proposition}[Properties of the Second Extension]

\end{proposition}

\begin{proposition}[Properties of the Outer Measure]

\end{proposition}

\begin{definition}[Lebesgue Outer Measure]
    Let \(\lambda^n: \mathcal{Q}(\mathbb{R}^n) \rightarrow \mathbb{R}_0^+\) the Lebesgue premeasure. The map
    \begin{align}
        \lambda^*: \mathcal{P}(\mathbb{R}^n) \rightarrow [0, \infty], \, B \mapsto \lambda^*(B):= \inf \left\{ \tilde{\lambda}^n (B) \mid A \in \mathcal{Q}(\mathbb{R}^n)^\uparrow, \, A \supset B \right\}
    \end{align}
    is called the Lebesgue outer measure induced by \(\tilde{\lambda^n}\).
\end{definition}

\begin{definition}[Pseudo Metric]
    
\end{definition}

\begin{proposition}
    The outer measure induces a pseudo metric.
\end{proposition}

\begin{proposition}
    The outer measure is continuous.
\end{proposition}

\begin{definition}[Approximation through elements of Rings]
    
\end{definition}

\begin{theorem}
    \begin{align}
        \hat{\mathcal{A}} := \{ A \in \mathcal{P}(X) \mid \text{\(A\) is \(\mathcal{A}\)-approximatable with \(\mu^*\)} \}
    \end{align}
    is a \(\sigma\)-algebra on \(X\).
\end{theorem}
\section{Measure Space}
\begin{definition}
    Let \(\mathcal{A} \subset \mathcal{P}(X)\) a \(\sigma\)-algebra. The tupel \(X, \mathcal{A}\) is called measurable space and the sets in the \(\sigma\)-algebra \(A \in \mathcal{A}\) are called measurable sets.

    Morover, let \(\mu: \mathcal{A} \rightarrow [0, \infty]\) be a measure on \(\mathcal{P}(X)\). Then, \((X, \mathcal{A}, \mu)\) a measure space.
\end{definition}

\begin{definition}[Null Sets]
    Let \((X, \mathcal{A}, \mu)\) be a measure space and \(\mu^*: \mathcal{P}(X) \rightarrow [0, \infty]\) the induced outer measure. Then \(N \subset X\) with \(\mu^*(N) = 0\) is called null set.

    For \(X = \mathbb{R}^n\) with \(\lambda^n(N) = 0\) called Lebesgue null set.

    \(S = \varnothing\) is called the trivial null set.
\end{definition}

\begin{definition}[Completion of a Measure Space]
    Let \((X, \mathcal{A}, \mu)\) be a measure space. This measure space is called complete if all null sets are included in \(\mathcal{A}\), i.e. for all \(N \in \mathcal{A}\)
    \begin{align}
        \mu^*{N} = 0 \Rightarrow N \in \mathcal{A} \text{.}
    \end{align}
\end{definition}

\begin{definition}
    Let
    \begin{align}
        \overline{\mathcal{A}}^\mu := \{ A \cup N \mid A \in \mathcal{A}, \, N \subset X \text{ with } \mu^*(N) = 0 \}
    \end{align}
    then \(\overline{\mathcal{A}}^\mu\) is called the completion of \((X, \mathcal{A}, \mu)\).
\end{definition}

\begin{definition}
    The completion of the Lebesgue-Borel measure space \( (\mathbb{R}^n, \mathcal{B}(\mathbb{R}^n), \hat{\lambda}^n) \) to \( (\mathbb{R}^n, \mathcal{B}^{\hat{\lambda}^n}(\mathbb{R}^n), \hat{\lambda}^n) \) or shorter \( (\mathbb{R}^n, \overline{\mathcal{B}}^\lambda (\mathbb{R}^n), \lambda^n)\) is called the (completed) Lebesgue measure space.

    \(B \in \overline{\mathbb{B}}^\lambda(\mathbb{R}^n)\) is called Lebesgue measurable to differentiate from \(B \in \mathcal{B}(\mathbb{R}^n)\) Borel measurable.
\end{definition}

% https://math.stackexchange.com/questions/141017/lebesgue-measurable-set-that-is-not-a-borel-measurable-set
%
\part{Lebesgue Integral}
\section{Measurable Maps}
There is measurable, Borel measurable and Lebesgue measurable.
\begin{definition}[Measurable Function]
    Let \((X, \mathcal{A}_X)\) and \((Y, \mathcal{A}_Y)\) be measurable spaces. A map \(f: X \rightarrow Y\) is called measurable if the pre-image of every measurable subset of \(Y\) under \(f\) is measurable subset of \(X\), i.e.
        \begin{align}
            B \in \mathcal{A}_Y \Rightarrow f^{-1}(B) \in \mathcal{A}_X \text{.}
        \end{align}
\end{definition}

\begin{definition}
    Let \((X, \mathcal{A}_X)\) be a measurable space. A function \(f: X \rightarrow \overline{\mathbb{R}}\) is called measurable if it is measurable with respect to the Borel \(\sigma\)-algebra on \(\overline{\mathbb{R}}\) associated to the standard topology.
\end{definition}
%
\begin{definition}[Borel Measurable Maps]
    Let \(X, \mathcal{U}_X\) and \(Y, \mathcal{U}_Y\) be topological spaces. A map \(f: X \rightarrow Y\) is called Borel measurable if the pre-image of every Borel measurable subset of \(Y\) under \(f\) is a Borel measurable subset of \(X\).
\end{definition}
%
\begin{definition}[Pushforward]
    Let \(f: X \rightarrow Y\) be any map. Then the set
    \begin{align}
        f_{*} \mathcal{A}_X := \{B \subset Y \mid f^{-1}(B) \in \mathcal{A}_X\}
    \end{align}
    is a \(\sigma\)-algebra on \(Y\), called the pushforward of \(\mathcal{A}_X\) under \(f\).
\end{definition}
%
\begin{theorem}
    Let \((X, \mathcal{A}_X)\), \((Y, \mathcal{A}_Y)\), and \((Z, \mathcal{A}_Z)\) be measurable spaces.
    \begin{enumerate}
        \item The identity map \(\text{id}_X: X \rightarrow X\) is measurable.
        \item If \(f: X \rightarrow Y\) and \(g: Y \rightarrow Z\) are measurable maps then so is the composition \(g \circ f: X \rightarrow Z\).
        \item A map \(f: X \rightarrow Y\) is measurable if and only if \(\mathcal{A}_Y \subset f_* \mathcal{A}_X\).
        \item A map \(f: X \rightarrow Y\) is measurable if and only if the pre-image of every oben subset \(V \subset Y\) under \(f\) is measurable, i.e.
        \begin{align}
            V \in \mathcal{U}_Y \Rightarrow f^{-1}(V) \in \mathcal{A}_X \text{.}
        \end{align}
        \item Assume \(\mathcal{U}_X \subset \mathcal{P}(X)\) is a topology on \(X\) such that \(\mathcal{A}_X\) is a Borel \(\sigma\)-algebra of \((X, \mathcal{U}_X)\). Then every continuous map \(f: X \rightarrow Y\) is (Borel) measurable.
        \item Let \(f = (f_1, \dots, f_n): X \rightarrow \mathbb{R}^n\) be a function. Then \(f\) is measurable if and only if \(f_i: X \rightarrow \mathbb{R}\) is measurable for each \(i\).
    \end{enumerate}
\end{theorem}
%
\begin{theorem}
    Let \((X, \mathcal{A})\) be a measurable space and let \(f: X \rightarrow \overline{\mathbb{R}}\) be any function. Then the following are equivalent.
    \begin{itemize}
        \item \(f\) is measurable.
        \item \(f^{-1}( (a, \infty])\) is a measurable subset of \(X\) for every \(a \in \mathbb{R}\).
        \item \(f^{-1}( [a, \infty])\) is a measurable subset of \(X\) for every \(a \in \mathbb{R}\).
        \item \(f^{-1}( [-\infty, b))\) is a measurable subset of \(X\) for every \(b \in \mathbb{R}\).
        \item \(f^{-1}( [-\infty, b])\) is a measurable subset of \(X\) for every \(b \in \mathbb{R}\).
    \end{itemize}
\end{theorem}
%
\begin{lemma}
    Let \((X, \mathcal{A})\) be a measurable space and let \(u, v: X \rightarrow \mathbb{R}\) be measurable functions. If \(\phi: \mathbb{R}^2 \rightarrow \mathbb{R}\) is continuous then the function \(h: X \rightarrow \mathbb{R}\), defined by \(h(x) := \phi(u(x), v(x))\) for \(x \in X\), is measurable.
\end{lemma}
%
\begin{theorem}
    Let \(X, \mathcal{A}\) be a measurable space.
    \begin{enumerate}
        \item If \(f, g: X \rightarrow \mathbb{R}\) are measurable functions then so are the functions
        \begin{align}
            f+g, && fg, && \max\{f, g\}, && |f| \text{.}
        \end{align}
        \item Let \(f_k: X \rightarrow \overline{\mathbb{R}}\), \(k \in \mathbb{B}\) be a sequence of measurable functions. Then the following functions from \(X\) to \(\overline{\mathbb{R}}\) are measurable
        \begin{align}
            \inf_k f_k, && \sup_k f_k, && \limsup_{k \rightarrow \infty} f_k, && \liminf_{k \rightarrow \infty} f_k \text{.}
        \end{align}
    \end{enumerate}
\end{theorem}
%
\begin{theorem}
    Let \((\Omega, \mathcal{A})\) be a measurable space, and \(\mathcal{B} = \sigma(\mathcal{E})\) for a generator \(\mathcal{E} \subset \mathcal{P}(\Omega)\). If for all \(E \in \mathcal{E}\) it is \(f^{-1}(E) \in \mathcal{A}\), then \(f\) is measurable.
\end{theorem}
\begin{example}
    Let \(f:(\mathbb{R}, \mathcal{B}) \rightarrow (\mathbb{R}, \mathcal{B})\) defined as
    \begin{align}
        f(x) := \begin{cases}
            1 x \in Q \\
            -1 x \notin Q
        \end{cases}
    \end{align}
    for a \(Q \notin \mathcal{B}(\mathbb{R})\). Then, \(f^{-1}({1})=Q \notin \mathcal{B}\) and therefore, \(f\) is not measurable even though \(|f| = 1\) is measurable.
\end{example}
\section{Lebesgue Integral}
%
\part{Applications}
%
\part{More Theory}
\chapter{Lebesgue Space}
\section{Lebesgue Space}
\begin{definition}[\(L^p\)-Norm]
    Let \(X, \mathcal{A}, \mu\) a measure space, and \(f: X \rightarrow \overline{\mathbb{R}}\) measurable. Then for \(p \in [1, \infty)\) the \(L^p\)-norm is defined as
    \begin{align}
        \norm{f}_p := \left( \int_X |f|^p d\mu \right)^{\frac{1}{p}} \text{.}
    \end{align}
\end{definition}
%
\begin{theorem}[Holder Inequality]
    Let \(p, q \in (1, \infty)\) such that \(p^{-1} + q^{-1} =1\). Let \(f, g: X \rightarrow \overline{\mathbb{R}}\) measurable. Then we have
    \begin{align}
        \norm{fg}_1 \leq \norm{f}_p \cdot \norm{g}_q
    \end{align}
\end{theorem}
%
\begin{theorem}[Minkowski Inequality]
    Let \(f, g: X \rightarrow \overline{\mathbb{R}}\) measurable and \(f + g\) well defined on \(X\). Then
    \begin{align}
        \forall p \in [1, \infty) : \norm{f + g}_p \leq \norm{f}_p + \norm{g}_p
    \end{align}
\end{theorem}
\begin{definition}
    Let \(X, \mathcal{A}, \mu\) be a measure space and \(p \in [1, \infty)\). Define
    \begin{align}
        \mathcal{L}^p(X, \mathcal{A}, \mu) = \left\{f: X \rightarrow \mathbb{R} \middle| f \text{ is \(\mathcal{A}\)-measurable and } \norm{f}_p < \infty \right\}
    \end{align}
    aaa

\end{definition}
\section{Convergence Theorems}
\begin{theorem}[Lebesgue Monotone Convergence Theorem]
    \textit{Also called the theorem of Beppo Levi.}
    Let \( (X, \mathcal{A}, \mu) \) be a measure space and let \(f_n: X \rightarrow [0, \infty]\) be a sequence of measurable functions such that
    \begin{align}
        f_n(x) \leq f_{n+1}(x)
    \end{align}
    for all \(x \in X\) and all \(n \in \mathbb{N}\). Define \(f: X \rightarrow [0, \infty]\) by
    \begin{align}
        f(x) := \lim_{n \rightarrow \infty} f_n(x) \text{.}
    \end{align}
    Then \(f\) is measurable and
    \begin{align}
        \lim_{n \rightarrow \infty} \int_X f_n \diff \mu = \int_X f \diff \mu \text{.}
    \end{align}
\end{theorem}
%
\begin{theorem}[Lebesgue Dominated Convergence Theorem]
    Let \( (X, \mathcal{A}, \mu) \) be a measure space, let \(g: X \rightarrow \mathbb{R}_0^+\) be an integrable function, and let \(f_n: X \rightarrow \mathbb{R}\) be a sequence of integrable functions satisfying
    \begin{align}
        |f_n(x)| \leq g(x)
    \end{align}
    for all \(x \in X\) and \(n \in \mathbb{N}\) and converging pointwise to \(f: X \rightarrow \mathbb{R}\), i.e.
    \begin{align}
        f(x) = \lim_{n \rightarrow \infty} f_n(x) \qquad \text{for all \(x \in X\).}
    \end{align}
    Then \(f\) is integrable and, for every \(E \in \mathcal{A}\),
    \begin{align}
        \int_E f \diff \mu = \lim_{n \rightarrow \infty} \int_E f_n \diff \mu \text{.}
    \end{align}
\end{theorem}
\section{Convergence}
\
\chapter{Fourier}
\section{Fourier Series}
\begin{definition}
    Let \(Y\) be a set and \(f: \mathbb{R} \rightarrow Y\) be a function. \(f\) is called periodic with periodicity \(L \in \mathbb{R}^+\) if for all \(x \in \mathbb{R}\) it is \(f(x + L) = f(x)\).
\end{definition}
\begin{remark}
    In the following, if the periodicity of the function is not given, let it be \(2\pi\).
\end{remark}
\begin{definition}
    For all \(k \in \mathbb{N}\) let \(a_k, b_k \in \mathbb{R}\). Then \(f: \mathbb{R} \rightarrow \mathbb{R}\) with
    \begin{align}
        f(x) := \frac{a_0}{2} + \sum_{k = 1}^n (a_k \cos(kx) + b_k \sin(kx))
    \end{align}
    is called the trigonometric polynomial of the order \(n\).
\end{definition}
\begin{remark}
    \begin{itemize}
        \item \(f\) sets the constants \(a_k\) and \(b_k\) uniquely.
        \item \(f\) is indeed a polynomial with the degree \(2n\).
    \end{itemize}
\end{remark}
\begin{definition}
    Let \(u, v: [a, b] \rightarrow \mathbb{R}\) integratable. Then \(\varphi: [a, b] \rightarrow \mathbb{C}, \, x \mapsto \varphi(x) := u(x) + iv(x)\) integratable with
    \begin{align}
        \int_a^b \varphi(x) := \int_a^b u(x) \diff x + i \int_a^b v(x) \diff x \text{.}
    \end{align}
\end{definition}
\begin{theorem}
    something
\end{theorem}
%
\begin{definition}[Fourier Series]
    Let \(f: \mathbb{R} \rightarrow \mathbb{C}\) periodic and integratable on \([0, 2\pi]\). Then the constants
    \begin{align}
        c_k = \frac{1}{2\pi} \int_0^{2\pi} f(x) e^{ikx} \diff x 
    \end{align}
    are called the Fourier-coefficients of \(f\). The series
    \begin{align}
        \mathcal{F}[f](x) := \sum_{k = -\infty}^\infty c_k e^{ikx}
    \end{align}
    is called the Fourier-series of \(f\).
\end{definition}

\end{document}