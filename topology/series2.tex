\documentclass[a4paper]{article}
\title{Topology}
\author{K}


% ---------------------------------------------------------------------
% P A C K A G E S
% ---------------------------------------------------------------------

% typography and formatting
\usepackage[english]{babel}
\usepackage[utf8]{inputenc}
\usepackage{geometry}
\usepackage{exsheets}
\usepackage{environ}

% mathematics
\usepackage{amsthm} % for theorems, and definitions
\usepackage{amssymb}
\usepackage{amsmath}
\usepackage{textcomp}
%\usepackage{MnSymbol} % for \cupdot

% extra
\usepackage{xcolor}
\usepackage{tikz}

% ---------------------------------------------------------------------
% S E T T I N G
% ---------------------------------------------------------------------

% typography and formatting
\geometry{margin=3cm}

\SetupExSheets{
  counter-format = ch.qu,
  counter-within = chapter,
  question/print = true,
  solution/print = true,
}

% mathematics

% extra
\definecolor{mathif}{HTML}{0000A0} % for conditions
\definecolor{maththen}{HTML}{CC5500} % for consequences
\definecolor{mathrem}{HTML}{8b008b} % for notes

\usetikzlibrary{positioning}
\usetikzlibrary{shapes.geometric, arrows}

\theoremstyle{definition}
\newtheorem*{lemma}{Lemma}

% ---------------------------------------------------------------------
% C O M M A N D S
% ---------------------------------------------------------------------

\newcommand{\norm}[1]{\left\lVert#1\right\rVert}
\newcommand{\rank}{\text{rank}}
\newcommand{\Vol}{\text{Vol}}

\newcommand{\set}[1]{\left\{\, #1 \,\right\}}
\newcommand{\makeset}[2]{\left\{\, #1 \mid #2 \,\right\}}


\newcommand*\diff{\mathop{}\!\mathrm{d}}
\newcommand*\Diff{\mathop{}\!\mathrm{D}}

\newcommand\restr[2]{{% we make the whole thing an ordinary symbol
  \left.\kern-\nulldelimiterspace % automatically resize the bar with \right
  #1 % the function
  \vphantom{\big|} % pretend it's a little taller at normal size
  \right|_{#2} % this is the delimiter
  }}

% ---------------------------------------------------------------------
% R E N D E R
% ---------------------------------------------------------------------

\newif\ifshowproof
\showprooftrue

\NewEnviron{Proof}{%
    \ifshowproof%
        \begin{proof}%
            \BODY
        \end{proof}%
    \fi%
}%
\title{Series 2}

\begin{document}
\maketitle

\section*{Exercise 3}
Suppose \(\mathcal{B}\) is a subbase for a topology \(\mathcal{T}\) on a set \(X\).

\subsection*{(a)}
Show that a sequence \(x_n \in X\) converges to \(x \in X\) if and only if for every \(\mathcal{U} \in \mathcal{B}\) containing \(x\) it is \(x_n \in \mathcal{U}\) for all \(n\) sufficiently large.
\begin{proof}[Solution]
    Let \(\mathcal{B}\) be a subbase for a topology \(\mathcal{T}\) on a set \(X\).

    ``\(\Rightarrow\)'': Let \(x_n \in X\) be a sequence that converges to a point \(x \in X\). By the definition of convergence, we have that for every neighbourhood \(\mathcal{N} \subset X\) of \(x\) it is \(x_n \in \mathcal{N}\) for all \(n \in \mathbb{N}\) sufficiently large.

    Fix a subset \(\mathcal{U} \in \mathcal{B}\) that contains \(x\). Since any subbase consists of only open sets, \(\mathcal{U}\) is open as well. Open subsets that contain \(x\) are themselves a neighbourhood of \(x\), thus \(x_n \in \mathcal{U}\) for all \(n \in \mathbb{N}\) sufficiently large as desired.

    ``\(\Leftarrow\)'': Let \(x_n \in X\) be a sequence and \(x \in X\) be a point such that for every \(\mathcal{U} \in \mathcal{B}\) containing \(x\), it is \(x_n \in \mathcal{U}\) for all \(n\) sufficiently large. Fix a neighbourhood \(\mathcal{N} \subset X\) of \(x\). By the definition of a neighbourhood, there is an open set \(\mathcal{V} \in \mathcal{T}\) such that \(x \in \mathcal{V} \subset \mathcal{N}\). Now by the definition of a subbase, we may write
    \begin{align*}
        \mathcal{V} = \bigcup_{\alpha \in I} \left( \mathcal{U}_\alpha^1 \cap \cdots \cap \mathcal{U}_\alpha^{N_\alpha} \right)
    \end{align*}
    for some collection of subsets \(\mathcal{U}_\alpha \subset X\) indexed by a set \(I\), for each \(\alpha \in I\) it is \(N_\alpha \in \mathbb{N}\), and it is \(\mathcal{U}_\alpha^1, \ldots, \mathcal{U}_\alpha^{N_\alpha} \in \mathcal{B}\). Since \(x \in \mathcal{V}\), there is an \(\beta \in I\) such that \(x \in \mathcal{U}_\beta^1 \cap \cdots \cap \mathcal{U}_\beta^{N_\beta}\) and hence \(x \in \mathcal{U}_\beta^{i}\) for all \(1 \leq i \leq N_\beta\). By the given condition, this means that \(x_n \in \mathcal{U}_\beta^i\) for all \(1 \leq i \leq N_\beta\) and for all \(n\) sufficiently large. In turn, this gives \(x_n \in \mathcal{U}_\beta^1 \cap \cdots \cap \mathcal{U}_\beta^{N_\beta}\) for all \(n\) sufficiently large. Thus, \(x_n \in \mathcal{V}\) for all \(n\) sufficiently larg. Conclude \(x_n \in \mathcal{N}\) for all \(n\) sufficiently large which means \(x_n\) converges to \(x\).
\end{proof}

\end{document}