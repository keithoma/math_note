\chapter{Quotient Space}
% $$$$$$\  $$\                            $$\     
% $$  __$$\ $$ |                           $$ |    
% $$ /  \__|$$$$$$$\   $$$$$$\   $$$$$$\ $$$$$$\   
% $$ |      $$  __$$\ $$  __$$\ $$  __$$\\_$$  _|  
% $$ |      $$ |  $$ |$$$$$$$$ |$$$$$$$$ | $$ |    
% $$ |  $$\ $$ |  $$ |$$   ____|$$   ____| $$ |$$\ 
% \$$$$$$  |$$ |  $$ |\$$$$$$$\ \$$$$$$$\  \$$$$  |
%  \______/ \__|  \__| \_______| \_______|  \____/
\section{Definitions and Theorems}
% $$$$$$$\                                 $$$$$$\  
% $$  __$$\                               $$  __$$\ 
% $$ |  $$ | $$$$$$\   $$$$$$\   $$$$$$\  $$ /  \__|
% $$$$$$$  |$$  __$$\ $$  __$$\ $$  __$$\ $$$$\     
% $$  ____/ $$ |  \__|$$ /  $$ |$$ /  $$ |$$  _|    
% $$ |      $$ |      $$ |  $$ |$$ |  $$ |$$ |      
% $$ |      $$ |      \$$$$$$  |\$$$$$$  |$$ |      
% \__|      \__|       \______/  \______/ \__|
\section{Proofs, Remarks, and Examples}
\begin{defbox}
    \begin{definition}
        Let \((X, \mathcal{O})\) be a {\color{mathif}topological space}, and let \(\sim\) be an {\color{mathif}equivalence relation} on \(X\). The {\color{maththen}quotient set}, \(X / \sim\) is the {\color{mathobj}set} of {\color{mathobj}equivalence classes} of elements of \(X\). The equivalence class of \(x \in X\) is {\color{mathrem}denoted} \([x]\). The {\color{maththen}projection map} (also {\color{mathrem}quotient} or {\color{mathrem}canonical map}) associated with \(\sim\) refers to the following {\color{mathif}surjective map}:
        \begin{align*}
            \pi: X \longrightarrow X / \sim, \qquad x \mapsto [x]
        \end{align*}
        For any subset \(S \subset X / \sim\) (so in particular, \(s \subset X\) for every \(s \in S\)) the following holds.
        \begin{align*}
            q^{-1}(S) = \makeset{x \in X}{[x] \in S} = \bigcup_{s \in S} s \text{.}
        \end{align*}

        The quotient space under \(\sim\) is the quotient set \(X / \sim\) equipped with the quotient topology, which is the topology whose open sets are subsets \(U \subset X / \sim\) such that 
        \begin{align*}
            \makeset{x \in X}{[x] \in U} = \bigcup_{u \in U}u
        \end{align*}
        is an open subset of \((X, \mathcal{O}_X)\); that is, \(U \subset X / \sim\)
    \end{definition}
\end{defbox}
%
\begin{proof}
    We will show that \(\mathcal{O}_\sim\) is a topology.
    \begin{enumerate}
        \item Clearly, \(\pi^{-1}(\varnothing) = \varnothing \in \mathcal{O}\), thus \(\varnothing \in \mathcal{O}_\sim\). Moreover, \(\pi(X) = X / \sim\), hence \(\pi^{-1}(X / \sim) = X \in \mathcal{O}\), and it is \(X / \sim \in \mathcal{O}_\sim\).
        \item Let \(I\) be an arbitary index set and \(\{U_i\}_{i \in I}\) a family of open sets in \(X / \sim\). It is
        \begin{align}
            \pi^{-1}\left( \bigcup_{i \in I} U_i \right) = \bigcup_{i \in I} \pi^{-1}(U_i) \text{.}
        \end{align}
        Since \(U_i\) is open and \(\pi\) is continuous, \(\pi^{-1}(U_i)\) is open for all \(i \in I\). Thus \(\bigcup_{i \in I} U_i \in \mathcal{O}_\sim\).
        \item Similar as above as preimages preserve unions and intersections.
    \end{enumerate}
    Let \(\mathcal{O}_\sim \subsetneq \mathcal{O}^\prime\). Then there is a open set \(U \in \mathcal{O}^\prime\) but \(U \not\in \mathcal{O}_\sim\).
\end{proof}
%
\begin{exmbox}
    \begin{example}
        \begin{enumerate}
            \item \(\mathbb{R} / \mathbb{Z}\)

            This space is homeomorph to \(S^{-1}\) and is compact.
            \item \((\mathbb{R} / \mathbb{Q}, \mathcal{O}_{\mathbb{R} / \mathbb{Q}})\) is the trivial topology.
        \end{enumerate}
    \end{example}
\end{exmbox}

\begin{thmbox}
    \begin{proposition}
        \(\mathcal{O}_{X / \sim}\) is the {\color{maththen}finest} {\color{mathobj}topology} in which the {\color{mathif}projection map} \(\pi: X \longrightarrow X / \sim\) is {\color{mathif}continuous}.
    \end{proposition}
\end{thmbox}
\begin{defbox}
    Let \(X\) and \(Y\) be topological spaces and let \(p: X \longrightarrow Y\) be a surjective map. The map is a quotient map (also said strong continuity) if one of the equivalent condition hold.
    \begin{enumerate}
        \item A subset \(U \subset Y\) is open in \(Y\) if and only if the preimage \(p^{-1}(U)\) is open in \(X\).
        \item A subset \(U \subset Y\) is closed in \(Y\) if and only if the preimage \(p^{-1}(U)\) is closed in \(X\).
    \end{enumerate}
\end{defbox}
\begin{rembox}
    \begin{remark}
        Quotient maps are continuous. There are quotient maps that are neither open nor closed maps.
    \end{remark}
\end{rembox}

\begin{thmbox}
    \begin{theorem}
        Let \(Y\) be a topological space. Then the following are equivalent.
        \begin{enumerate}
            \item \(f: X / \sim \longrightarrow Y\) continuous
            \item \(f \circ \pi : X \longrightarrow Y\) is continuous.
        \end{enumerate}

        Moreover, if \(X\) is connected, then \(X / \sim\) is connected. Same is true for path-connectedness and compactness.
    \end{theorem}
\end{thmbox}

\begin{defbox}
    \begin{definition}
        A topological group \(G\) is a topological space that is also a group such that the group operation
        \begin{align*}
            \circ: G \times G \longrightarrow G, (x, y) \mapsto x \circ y
        \end{align*}
        and the inversion map
        \begin{align*}
            ^{-1}: G \longrightarrow G, x \mapsto x^{-1}
        \end{align*}
        are continuous. Here \(G \times G\) is viewed as a topological space with the product topology. Such a topoloy is said to be compatible with the group operations and is called a group topology.
    \end{definition}
\end{defbox}

\begin{rembox}
    \begin{remark}
        About homoemorphism of G-space.
    \end{remark}
\end{rembox}

\begin{defbox}
    \begin{definition}
        Consider a group acting on a set \(X\). The orbit of an element \(x\) in \(X\) is the set of elements in \(X\) to which \(x\) can be moved by the elements of \(G\). The orbit of \(x\) is denoted by \(G \cdot x\).
        \begin{align*}
            G \cdot x := \makeset{g \cdot x}{g \in G} \text{.}
        \end{align*}
    \end{definition}
\end{defbox}

\begin{defbox}
    \begin{definition}
        \(X / G := X / \sim\) such that \(x \sim y\) if and only if there is a \(g \in G\) such that \(x = gy\).
    \end{definition}
\end{defbox}

\begin{defbox}
    \begin{definition}[Hilbert]
        \begin{itemize}
            \item Matrix \(A\) is semi-stable if \(A\) is diagonizable.
            \item Matrix \(A\) is stable, if it is semi-stable and all Eigenvalues are distinct.
        \end{itemize}
    \end{definition}
\end{defbox}

\begin{defbox}
    \begin{definition}[Stabilisator]
        \(x \in X\)

        \(G \supset G_x := \makeset{h}{h \cdot x = x}\)
    \end{definition}
\end{defbox}

\begin{thmbox}
    \begin{lemma}
        \begin{enumerate}
            \item \(G_x \subset G\) is a subgroup.
            \item \(G/G_x \longrightarrow G_x\) is well-defined and \([x] \mapsto gx\) is a continuous bijection (repsective of the quotient topology on \(G / G_x\)).
        \end{enumerate}
    \end{lemma}
\end{thmbox}

\begin{thmbox}
    \begin{corollary}
        If \(G\) is compact and \(X\) is \(T_2\), then \(f: G/G_x \longrightarrow G\) is a homeomorphism.
    \end{corollary}
\end{thmbox}

\begin{defbox}
    \begin{definition}
        
    \end{definition}
\end{defbox}

\newpage
% $$\   $$\            $$\                         
% $$$\  $$ |           $$ |                        
% $$$$\ $$ | $$$$$$\ $$$$$$\    $$$$$$\   $$$$$$$\ 
% $$ $$\$$ |$$  __$$\\_$$  _|  $$  __$$\ $$  _____|
% $$ \$$$$ |$$ /  $$ | $$ |    $$$$$$$$ |\$$$$$$\  
% $$ |\$$$ |$$ |  $$ | $$ |$$\ $$   ____| \____$$\ 
% $$ | \$$ |\$$$$$$  | \$$$$  |\$$$$$$$\ $$$$$$$  |
% \__|  \__| \______/   \____/  \_______|\_______/
\section{Exercises and Notes}