\chapter{Pasting}

\begin{defbox}
    \begin{definition}
        Let \(X_0\) and \(X\) be topological spaces and \(\varphi: X_0 \longrightarrow X\) a continuous function. Set \(X_\varphi := X / \sim\) where \(\sim\) is generated by
        \begin{align*}
            \makeset{x \sim \varphi(x)}{x \in X_0} \text{.}
        \end{align*}
    \end{definition}
\end{defbox}

\begin{exmbox}
    \begin{example}
        idk
    \end{example}
\end{exmbox}

\begin{defbox}
    \begin{definition}
        Abbildungstorus
    \end{definition}
\end{defbox}

\begin{defbox}
    \begin{definition}
        Let \(X\) be a topological space.
        \begin{enumerate}
            \item \(X\) is said to be a first-countable space or to satisfy the first axiom of countability if each point has a countable neighbourhood basis (local base). That is, for each point \(x \in X\) there exists a sequence \(N_1, N_2, \ldots\) of neighbourhoods of \(x\) such that for any neighbourhood \(N\) of \(x\) there exists an integer \(i\) with \(N_i\) contained in \(N\). Since every neighbourhood of any point contains an open neighbourhood of that point, the neighbourhood basis can be chosen without loss of generality to consist of open neighbourhoods.
            \item \(X\) is said to be a second-countable space, also called completely separable space, or to satisfy the second axiom of countability if it has a countable base.
        \end{enumerate}
    \end{definition}
\end{defbox}

\begin{rembox}
    \begin{remark}
        A countable subbase induces a countable base.
    \end{remark}
\end{rembox}

\begin{thmbox}
    \begin{lemma}
        The second-countable axiom implies the first.
    \end{lemma}
\end{thmbox}

\begin{thmbox}
    \begin{lemma}
        Is \(X\) a first-countable space, then
        \begin{enumerate}
            \item all sequentially-continuous function is continuous.
            \item all compact spaces are also sequentially compact.
        \end{enumerate}
    \end{lemma}
\end{thmbox}

\begin{defbox}
    \begin{definition}
        A topological mannifold \(\mathcal{M}^n\) of dimension \(n\) is a topological space that is \(T_2\), is a second-countable space, and for each point has a neighbourhood that is homeomorph to a open subset of
        \begin{align*}
            \mathbb{H}^n := \makeset{x = (x_1, \ldots, x_n) \in \mathbb{R}^n}{x_n \geq 0}
        \end{align*}
    \end{definition}
\end{defbox}

\begin{defbox}
    \begin{definition}[Pasting]
        Let \(X\) and \(Y\) be two disjoint topological spaces, \(A \subset X\) a closed subset, and \(f: A \longrightarrow Y\) continuous. Define a equivalence relation on \(X \cup Y\) by
        \begin{align*}
            v \sim w :
            \begin{cases}
                v = w \text{ if } v, w \in X \cup Y \\
                f(v) = f(w) \text{ if } v, w \in A \\
                v = f(w) \text{ if } v \in Y, w \in A \\
                w = f(v) \text{ if } v \in A, w \in Y
            \end{cases}
        \end{align*}
        We write
        \begin{align*}
            Y \cup_f X := X \cup Y / \sim
        \end{align*}
    \end{definition}
\end{defbox}
\begin{example}
    Consider \(X = [0, 1]\) on the \(x\)-axis and \(Y = [0, 1]\).
\end{example}