\chapter{Compact Spaces}
\begin{defbox}
    \begin{definition}
        \begin{enumerate}
            \item A {\color{mathobj}topological space} \(X\) is called {\color{maththen}compact} if each of its {\color{mathif}open cover} has a \textbf{finite} {\color{mathif}subcover}.
            \item A {\color{mathobj}topological space} \(X\) is called {\color{maththen}sequentially compact} if every {\color{mathif}sequence} in \(X\) has a {\color{mathif}convergent subsequence} whose limit is in \(X\).
        \end{enumerate}
    \end{definition}
\end{defbox}
%
\begin{thmbox}
    \begin{theorem}
        Satz 17
    \end{theorem}
\end{thmbox}
%
\begin{thmbox}
    \begin{theorem}
        Let \(A \subset \mathbb{R}^n\) be a subset. \(A\) is compact if and only if it is closed and bounded.
    \end{theorem}
\end{thmbox}
%
\begin{thmbox}
    \begin{theorem}
        Let \(X\) be a \(T_2\) space. If a subset \(K \subset X\) is compact, then it is closed.
    \end{theorem}
\end{thmbox}
%
\begin{thmbox}
    \begin{theorem}
        Let \(X\) and \(Y\) be topological spaces, \(X\) compact, and \(Y\) be a \(T_2\) space. If \(f: X \longrightarrow Y\) is bijective and continuous, then the inverse function \(f^{-1}\) is continuous.
    \end{theorem}
\end{thmbox}
%
\newpage
\section{Proofs, Remarks, and Examples}

\begin{thmbox}
    \begin{lemma}
        \([0, 1] \subset \mathbb{R}\) is {\color{maththen}compact}.
    \end{lemma}
\end{thmbox}