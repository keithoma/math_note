\chapter{Compact Spaces}
% $$$$$$\  $$\                            $$\     
% $$  __$$\ $$ |                           $$ |    
% $$ /  \__|$$$$$$$\   $$$$$$\   $$$$$$\ $$$$$$\   
% $$ |      $$  __$$\ $$  __$$\ $$  __$$\\_$$  _|  
% $$ |      $$ |  $$ |$$$$$$$$ |$$$$$$$$ | $$ |    
% $$ |  $$\ $$ |  $$ |$$   ____|$$   ____| $$ |$$\ 
% \$$$$$$  |$$ |  $$ |\$$$$$$$\ \$$$$$$$\  \$$$$  |
%  \______/ \__|  \__| \_______| \_______|  \____/
\begin{defbox}
    \begin{definition}
        \begin{enumerate}
            \item A {\color{mathobj}topological space} \(X\) is called {\color{maththen}compact} if each of its {\color{mathif}open cover} has a \textbf{finite} {\color{mathif}subcover}.
            \item A {\color{mathobj}topological space} \(X\) is called {\color{maththen}sequentially compact} if every {\color{mathif}sequence} in \(X\) has a {\color{mathif}convergent subsequence} whose limit is in \(X\).
        \end{enumerate}
    \end{definition}
\end{defbox}
%
\begin{thmbox}
    \begin{theorem}
        Satz 17
    \end{theorem}
\end{thmbox}
%
\begin{thmbox}
    \begin{theorem}
        Let \(A \subset \mathbb{R}^n\) be a subset. \(A\) is compact if and only if it is closed and bounded.
    \end{theorem}
\end{thmbox}
%
\begin{thmbox}
    \begin{theorem}
        Let \(X\) be a \(T_2\) space. If a subset \(K \subset X\) is compact, then it is closed.
    \end{theorem}
\end{thmbox}
%
\begin{thmbox}
    \begin{theorem}
        Let \(X\) and \(Y\) be topological spaces, \(X\) compact, and \(Y\) be a \(T_2\) space. If \(f: X \longrightarrow Y\) is bijective and continuous, then the inverse function \(f^{-1}\) is continuous.
    \end{theorem}
\end{thmbox}
% $$$$$$$\                                 $$$$$$\  
% $$  __$$\                               $$  __$$\ 
% $$ |  $$ | $$$$$$\   $$$$$$\   $$$$$$\  $$ /  \__|
% $$$$$$$  |$$  __$$\ $$  __$$\ $$  __$$\ $$$$\     
% $$  ____/ $$ |  \__|$$ /  $$ |$$ /  $$ |$$  _|    
% $$ |      $$ |      $$ |  $$ |$$ |  $$ |$$ |      
% $$ |      $$ |      \$$$$$$  |\$$$$$$  |$$ |      
% \__|      \__|       \______/  \______/ \__|
\newpage
\section{Proofs, Remarks, and Examples}

\begin{defbox}
    \begin{definition}
        \begin{enumerate}
            \item A topological space \(X\) is called {\color{maththen}compact} if each of its {\color{mathif}open cover} has a \textbf{finite} {\color{mathif}subcover}. That is, \(X\) is compact if for every arbitary collection \(\mathcal{C}\) of open subsets of \(X\) such that
            \begin{align*}
                X = \bigcup_{U \in \mathcal{C}} U
            \end{align*}
            there is a finite subcollection \(\mathcal{F} \subset \mathcal{C}\) such that
            \begin{align*}
                X = \bigcup_{U \in \mathcal{F}} U \text{.}
            \end{align*}
            \item A subset is said to be compact if it is compact as a subspace.
        \end{enumerate}
    \end{definition}
\end{defbox}

\begin{defbox}
    \begin{definition}
        \(X\) is called {\color{maththen}sequentially compact} if every {\color{mathif}sequence} in \(X\) has a {\color{mathif}convergent subsequence} whose limit is in \(X\).
    \end{definition}
\end{defbox}

\begin{rembox}
    \begin{remark}
        The notion of compact and sequentially compact are not equivalent.
    \end{remark}
\end{rembox}

\begin{exmbox}
    \begin{example}
        \begin{enumerate}
            \item Example of a space that is compact but not sequentially compact.
            \item Example of a space that is sequentially compact but not compact.
        \end{enumerate}
    \end{example}
\end{exmbox}

\begin{thmbox}
    \begin{proposition}
        Let \(X\) and \(Y\) be two topological spaces.
        \begin{enumerate}
            \item Continuous functions preserve compactness, i.e. if \(f: X \longrightarrow Y\) is continuous and \(X\) is compact, then \(f(X) \subset Y\) is compact.
            \item In a compact space, every closed subset is compact, i.e. if \(X\) is compact and \(A \subset X\) is a closed subset, then \(A\) is compact.
            \item The product of compact spaces is again compact. If \(X\) and \(Y\) are both compact, so is \(X \times Y\).
        \end{enumerate}
    \end{proposition}
\end{thmbox}

\begin{proof}
    \begin{enumerate}
        \item Let \(f: X \longrightarrow Y\) be continuous and \(X\) compact. Denote the open cover of the continuous image of \(X\) by \(\mathcal{C}\), i.e.
        \begin{align*}
            f(X) \subset \bigcup_{U \in \mathcal{C}} U \text{.}
        \end{align*}
        Because \(f\) is continuous, each of the preimages \(f^{-1}(U)\) with \(U \in \mathcal{C}\) is open. Now, \(X\) is compact, there are finitely many \(f^{-1}(U)\) such that
        \begin{align*}
            X \subset \bigcup_{U \in \mathcal{F}} f^{-1}(U)
        \end{align*}
        Conclude that \(f(X)\) is compact.
        \item Let \(\mathcal{U}\) be an open cover of \(A\). Every open set in \(\mathcal{U}\) is in the form \(U \cap A\) for some open set \(U \subset X\). Define
        \begin{align*}
            \mathcal{V} := \makeset{U \in \mathcal{O}}{U \cap A \in \mathcal{U}}
        \end{align*}
        then \(\mathcal{V}\) is an open cover of \(A\) as well. Since \(A\) is closed, \(X \setminus A\) is open, so \(\mathcal{V} \cup (X \setminus A)\) is an open cover of \(X\). By compactness of \(X\), there is a finite subcover that XXXXX.
        \item I think this is clear.
    \end{enumerate}
\end{proof}

\begin{thmbox}
    \begin{lemma}
        \([0, 1] \subset \mathbb{R}\) is {\color{maththen}compact}.
    \end{lemma}
\end{thmbox}

\begin{proof}
    skipped
\end{proof}

\begin{thmbox}
    \begin{theorem}[Heine-Borel]
        A compact subset of a Euclidean space \(A \subset \mathbb{R}^n\) is compact if and only if it is closed and bounded.
    \end{theorem}
\end{thmbox}

\begin{thmbox}
    \begin{proposition}
        Let \(X\) be a \(T_2\)-space. If \(K \subset X\), then \(K\) is closed.
    \end{proposition}
\end{thmbox}

\begin{proof}
    for the two things above, skipped.
\end{proof}

\begin{thmbox}
    \begin{lemma}
        Let \(f: X \longrightarrow Y\) be continuous and bijective. If \(X\) is compact and \(Y\) is a \(T_2\)-space, then \(f^{-1}\) is continuous.
    \end{lemma}
\end{thmbox}