\chapter{Connected Spaces and Sets}
\section{Definition and Theorems}
\begin{defbox}
    \begin{definition}
        A {\color{mathobj}topological space} \(X\) is said to be {\color{maththen}connected}, if one of the following {\color{mathrem}equivalent} conditions is met.
        \begin{enumerate}
            \item \(X\) is \textbf{not} a {\color{mathif}union} of two {\color{mathif}disjoint} sets.
            \item The \textbf{only} {\color{mathif}subsets} of \(X\) that are \textbf{both} {\color{mathif}open} and {\color{mathif}closed} ({\color{mathrem}clopen}) are the emptyset \(\varnothing\) and the entire set \(X\).
            \item The \textbf{only} {\color{mathif}subsets} of \(X\) with empty {\color{mathif}boundary} are the emptyset \(\varnothing\) and the entire set \(X\).
            \item All {\color{mathif}continuous} maps from \(X\) to the two point space \(\{0, 1\}\) endowed with the {\color{mathif}discrete} topology is {\color{mathif}constant}. 
        \end{enumerate}
    \end{definition}
\end{defbox}

\begin{thmbox}
    \begin{lemma}
        Any {\color{mathif}interval} \(I \subset \mathbb{R}\) is {\color{maththen}connected}.
    \end{lemma}
\end{thmbox}
%
\begin{defbox}
    \begin{definition}
        A connected component of a topological space is a maximally connected subset \(X_0 \subseteq X\), i.e. \(X_0\) connected and for all \(X_0 \subsetneq X_1\) then \(X_1\) is not connected.
    \end{definition}
\end{defbox}
%
\begin{thmbox}
    \begin{proposition}
        Connected components are closed subsets.
    \end{proposition}
\end{thmbox}
%
\begin{thmbox}
    \begin{lemma}[Lemma 11]
        Let \(X\) be connected and \(f: X \longrightarrow Y\) and locally constant, i.e. for all \(x \in X\) there exists a \(U_x \in \mathcal{O}_X\), \(x \in U_x\) such that \(f\) restricted on \(U_x\) is identical to \(f(x)\)., then \(f\) is constant.
    \end{lemma}
\end{thmbox}
%
\begin{defbox}
    \begin{definition}
        \(X\) is said to be {\color{maththen}path connected}, if for every pair of points \(x\) and \(x_0\) in \(X\) there is a continuous map (called path) \(\gamma: [0, 1] \longrightarrow X\) with \(\gamma(0) = x_0\) and \(\gamma(1) = x\).
    \end{definition}
\end{defbox}
%
\begin{thmbox}
    \begin{lemma}
        If \(X\) is path connected, then it is also connected.
    \end{lemma}
\end{thmbox}

\section{Proofs, Remarks, and Examples}

\begin{proof}
    Fix an interval \(I \subset \mathbb{R}\), and let \(A, B \subset \mathbb{R}\) be two nonempty, open and disjoint subsets such that \(A \sqcup B = I\). Moreover, let \(a \in A\) and \(b \in B\) and assume without loss of generality that \(a < b\). If we set
    \begin{align}
        s := \inf \makeset{x \in B}{a < x} \text{,}
    \end{align}
    then \(s \in [a, b] \subset I\) because \(I\) is an interval.   
\end{proof}

\begin{example}
    The general linear group \(\mathrm{GL}_n(K)\) for a field \(K\) and \(n \in \mathbb{N}\) is not connected for \(K = \mathbb{R}\) and \(K = \mathbb{C}\).
\end{example}

\begin{remark}
    Let \(f: X \longrightarrow Y\) be continuous and \(X\) be connected, then \(f(X) \subset Y\) is connected.
\end{remark}
\begin{proof}
    Let \(f(X) = A \sqcup B\) with \(A\) and \(B\) being two open disjoint sets. \(f^{-1}(A)\) and \(f^{-1}(B)\) are open since \(f\) is continuous. We also have \(f^{-1}(A) \cap f^{-1}B = f^{-1}(A \cap B) = \varnothing\) so \(f^{-1}(A) = \varnothing\) or \(f^{-1}(B) = \varnothing\), so \(A = \varnothing\) or \(B = \varnothing\) and we are done.
\end{proof}
\begin{proof}
    % proof missing
\end{proof}
\begin{example}
    For \(\mathbb{Q} \subset \mathbb{R}\) the connected components are points and those are not open.
\end{example}

\begin{proof}
    Locally constant implies continuous with regards to the discrete topology on \(Y\). Let \(x \in X\), \(X = f^{-1}(f(x)) \cup f^{-1}(Y \setminus \{f(x)\})\) is a disjoint union and since \(X\) is connected \(f^{-1}(Y \setminus \{f(x)\}) = \varnothing\). Conclude \(f\) is identical to \(f(x)\).
\end{proof}

\textbf{Application:} \(f: X \longrightarrow \{0, 1\}\), \(X\) is connected, \(f\) locally constant, there is a \(x \in X\) such that \(f(x) = 1\), then \(f\) is identical to \(1\).

\begin{proof}
    Let \(A\) and \(B\) two disjoint open sets such that \(A \sqcup B = X\), and let \(a \in A\) and \(b \in B\). Let \(\gamma: [0, 1] \longrightarrow X\) be continuous path with \(\gamma(0) = x_0\) and \(\gamma(1) = x_1\). We have that \(\gamma^{-1}\)
\end{proof}

\begin{remark}
    The converse statement is not true in general.
\end{remark}

\begin{example}
    \(X = \makeset{(x, \sin(\frac{1}{x}))}{x > 0} \cup \{0\} \times [-1, 1] \subset \mathbb{R}^2\) is connected but not path connected.
\end{example}
\begin{proof}
    Homework
\end{proof}
\begin{remark}
    missing
\end{remark}
