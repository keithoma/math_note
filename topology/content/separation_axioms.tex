\chapter{Separation Axioms}
Literature: Groessere Liste in Sten, Seibeck

\begin{defbox}
    \begin{definition}[\(T_1\) Space]
        Let \(X\) be a {\color{mathif}topological space}.
        \begin{enumerate}
            \item We say that two {\color{mathobj}points} \(x\) and \(y\) can be {\color{maththen}separated} if each lies in a {\color{mathif}neighborhood} that does \textbf{not} contain the other point.

            \item A {\color{mathobj}topological space} \(X\) is a {\color{maththen}\(T_1\) space} if any two distinct points in \(X\) are {\color{mathif}separated}.
        \end{enumerate}
    \end{definition}
\end{defbox}
%
\begin{thmbox}
    \begin{proposition}
        Let \(X\) be a {\color{mathif}topological space}. Then, the following are {\color{mathrem}equivalent}.
        \begin{enumerate}
            \item \(X\) is a {\color{maththen}\(T_1\) space}.
            \item {\color{mathif}Points} are {\color{maththen}closed} in \(X\), i.e. given any \(x \in X\), the {\color{mathif}singleton} set \(\{x\}\) is a {\color{maththen}closed} set.
        \end{enumerate}
    \end{proposition}
\end{thmbox}
%
\begin{defbox}
    \begin{definition}[\(T_2\) Space]
        Let \(X\) be a {\color{mathif}topological space}.
        \begin{enumerate}
            \item {\color{mathobj}Points} \(x\) and \(y\) in \(X\) can be {\color{maththen}separated by neighborhood} if there exists a {\color{mathif}neighborhood} \(U\) of \(x\) and a {\color{mathif}neighborhood} \(V\) of \(y\) such that \(U\) and \(V\) are {\color{mathif}disjoint}, i.e. \(U \cap V = \varnothing\).
            \item A {\color{mathobj}topological space} \(X\) is a {\color{maththen}\(T_2\) space} if any two distinct points in \(X\) are {\color{mathif}separated by neighborhood}.
        \end{enumerate}
    \end{definition}
\end{defbox}
%
\begin{thmbox}
    \begin{proposition}
        Let \(X\) be a {\color{mathif}topological space}. Then, the following are {\color{mathrem}equivalent}.
        \begin{enumerate}
            \item \(X\) is a {\color{maththen}\(T_2\) space}.
            \item Any singleton set \(\{x\}\) is the intersection of all closed neighborhoods of \(x\).
            \item The diagonal \(\Delta = \makeset{(x, x)}{x \in X}\) is closed as a subset of the product space \(X \times X\).
        \end{enumerate}
    \end{proposition}
\end{thmbox}
%
\begin{thmbox}
    \begin{proposition}
        \(T_2\) spaces are also \(T_1\) spaces.
    \end{proposition}
\end{thmbox}