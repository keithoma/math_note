\chapter{Topological Space}
\section{Definitions and Theorems}
\begin{defbox}
    \begin{definition}[Topological Space]
        A {\color{maththen}topological space} is an {\color{mathobj}ordered pair} \((X, \tau)\), where \(X\) is a {\color{mathif}set} and \(\tau\) is a {\color{mathif}collection of subsets} that satisfies the following {\color{mathrem}axioms}.
        \begin{enumerate}
            \item The {\color{mathif}empty set} \(\varnothing\) and the {\color{mathif}entire set} \(X\) belongs to \(\tau\).
            \item Any \textbf{arbitary} {\color{mathif}union} of members of \(\tau\) belongs to \(\tau\).
            \item The {\color{mathif}intersection} of \textbf{finite number} of members of \(\tau\) belongs to \(\tau\).
        \end{enumerate}
        The {\color{mathobj}collection} \(\tau\) is called a {\color{maththen}topology} on \(X\) and the {\color{mathobj}elements} of \(\tau\) are called {\color{maththen}open sets}. A {\color{mathobj}subset} \(A \subset X\) is said to be {\color{maththen}closed} if its {\color{mathif}complement} \(X \setminus A\) is {\color{mathif}open}.
    \end{definition}
\end{defbox}
%TODO: There are other equivalent definitions
%TODO: write something about the notation

\begin{defbox}
    \begin{definition}[Continuous Maps]
        Let \((X, \tau_X)\) and \((Y, \tau_Y)\) be {\color{mathif}topological spaces}. A {\color{mathif}map} \(f: X \longrightarrow Y\) is said to be {\color{maththen}continuous} if the preimage of an open subset is again open, i.e.
        \begin{equation}
            \text{for all } U \in \tau_Y \text{ it is } f^{-1}(U) \in \tau_X \text{.}
        \end{equation}
    \end{definition}
\end{defbox}
\begin{thmbox}
    \begin{lemma}
        The different definitions of continuity in a topological space and a metric space are equivalent, i.e. if \(X\) and \(Y\) are metric spaces, then \(f: X \longrightarrow Y\) is \(\epsilon\)-\(\delta\)-continuous if and only if \(f\) is continuous.
    \end{lemma}
\end{thmbox}

\begin{defbox}
    \begin{definition}[Homeomorphism]
        Let \(X\) and \(Y\) be {\color{mathif}topological spaces}.
        \begin{enumerate}
            \item A {\color{mathobj}map} \(f: X \longrightarrow Y\) is a {\color{maththen}homeomorphism} if it has the following properties.

            \begin{enumerate}
                \item \(f\) is {\color{mathif}bijective}.
                \item \(f\) and the {\color{mathif}inverse map} \(f^{-1}\) is {\color{mathif}continuous}.
            \end{enumerate}

            \item Two topological spaces \(X\) and \(Y\) are said to be {\color{maththen}homeomorphic} if a homeomorphism exists.

            \item We denote the set of all homeomorphisms from \(X\) to \(Y\) by \(\mathrm{Homeo}(X, Y)\). If \(Y = X\) we also write \(\mathrm{Homeo}(X)\).
        \end{enumerate}
    \end{definition}
\end{defbox}

\begin{defbox}
    \begin{definition}[Homeomorphism]
        Let \((X, \tau)\) a {\color{mathif}topological space}.
        \begin{enumerate}
            \item \(\mathcal{B} \subset \mathcal{O}\) is a {\color{maththen}basis} of the topology, if any member of \(\mathcal{O}\) is the {\color{mathif}union of subsets} from \(\mathcal{B}\).
            \item \(\mathcal{S} \subset \mathcal{O}\) is a {\color{maththen}subbasis} of the topology, if any member of \(\mathcal{O}\) is the {\color{mathif}union of finite intersections of subsets} from \(\mathcal{S}\).
        \end{enumerate}
        We say that \(\mathcal{B}\) and \(\mathcal{S}\) {\color{maththen}generates} \(\mathcal{O}\) and write \(\overline{\mathcal{S}} = \overline{\mathcal{B}} = \mathcal{O}\).
    \end{definition}
\end{defbox}

\begin{thmbox}
    \begin{lemma}
    Let \(\mathcal{S} \subset \mathcal{P}(X)\) be a {\color{mathobj}collection of subsets}, then there {\color{maththen}exists} \textbf{exactly one} topology \(\tau \subset \mathcal{P}(X)\) of \(X\) such that
    \begin{enumerate}
        \item \(\mathcal{S} \subset \tau\)
        \item If \(\tau' \subset \mathcal{P}(X)\) a topology with \(S \subset \tau'\), then \(\tau \subset \tau'\).
    \end{enumerate}
    \end{lemma}
\end{thmbox}

\begin{defbox}
    \begin{definition}
        \begin{enumerate}
            \item Given \((X, \tau)\) be a {\color{mathif}topological space}, \(S \subset X\) a subset, the {\color{maththen}subspace topology} (also the induced topology or the relative topology) on \(S\) is defined by
            \begin{equation*}
                \tau_S = \makeset{S \cap U}{U \in \tau} \text{.}
            \end{equation*}
            \item Let \((X, \tau_X)\) and \((Y, \tau_Y)\) be two {\color{mathif}topological spaces}. The product topology of \(X\) and \(Y\) is defined by
            \begin{equation*}
                \tau_{X \times Y} := \makeset{U \times V}{U \in \tau_X \text{ and } V \in \tau_Y} \text{.}
            \end{equation*}
            \item Let \((X, \tau_X)\) and \((Y, \tau_Y)\) be two {\color{mathif}topological spaces}. The topological sum of \(X\) and \(Y\) is defined by
            \begin{align*}
                \tau_{X \sqcup Y} := \makeset{U \sqcup V}{U \in \tau_X \text{ and } V \in \tau_Y} \text{.}
            \end{align*}
        \end{enumerate}
    \end{definition}
\end{defbox}

\begin{defbox}
    \begin{definition}
        Let \((X, \tau)\) be a topological space.
        \begin{enumerate}
            \item Given a {\color{mathobj}point} \(p \in X\), a subset \(U \subset X\) is a neighborhood of \(p\) if there is an open subset \(V \in U\) such that \(p \in V\). If such a neighborhood exists, \(p\) is called a interior point of \(U\).
            \item Let \(S \subset X\) be a subset. The interior of \(S\), denoted by \(\mathring{S}\) or \(\mathrm{int}(S)\), is the {\color{mathobj}set} of all interior points of \(S\).
            \item Let \(S \subset X\) be a subset. The closure of \(S\), denoted by \(\overline{S}\) or \(\mathrm{cl}(S)\), is defined by
            \begin{equation*}
                \mathrm{cl}(S) := X \setminus \mathrm{int}(X \setminus S) \text{.}
            \end{equation*}
        \end{enumerate}
    \end{definition}
\end{defbox}
\newpage
%%%%%%%%%%%%%%%%%%%%%%%%%%%%%%%%%%%%%%%%
%%%%%%%%%%%%%%%%%%%%%%%%%%%%%%%%%%%%%%%%
%%%%%%%%%%%%%%%%%%%%%%%%%%%%%%%%%%%%%%%%
\section{Proofs, Remarks, and Examples}
%%%%%%%%%%%%%%%%%%%%%%%%%%%%%%%%%%%%%%%%
%%%%%%%%%%%%%%%%%%%%%%%%%%%%%%%%%%%%%%%%
%%%%%%%%%%%%%%%%%%%%%%%%%%%%%%%%%%%%%%%%

\begin{example}
    Let \(X\) be a {\color{mathif}set}.
    \begin{enumerate}
        \item \(\tau = \mathcal{P}(X)\) is called the {\color{maththen}discrete topology}. In this case, \((X, \tau)\) is called the {\color{maththen}discrete space}. It is the {\color{mathrem}finest topology} that can be defined on a set. (The set of all possible topologies on a given set forms a partially ordered set.)
        \item \(\tau = \{\varnothing, \mathcal{P}(X)\}\) is called the {\color{maththen}trivial topology}.
        \item Let \((X, d)\) be a {\color{mathif}metric space}. Set
        \begin{equation}
            \tau_d := \makeset{U \in X}{U \text{ is a open subset in the metric space } (X, d)} \text{.}
        \end{equation}
        Recall that \(U\) being an open subset in the metric space \((X, d)\) means that for all \(x \in U\) there is an \(r > 0\) such that \(B_d(x, r)\) is contained in \(U\).

        Here, \(\tau\) is a topology. In other words, a metric induces a topology.

        (Proof as homework.)
        \item The Zariski-topology.
    \end{enumerate}
\end{example}


\begin{remark}
    The set of all homeomorphisms of \(X\) to itself \(\mathrm{Homeo}(X)\) is a group with composition as its operation.
\end{remark}

\begin{remark}
    This lemma does not hold for basis.
\end{remark}


\begin{remark}
    \begin{enumerate}
        \item \(\tau_{X \times Y}\) is the most coarse topology for which both of the projections are continuous.
        \item \(\tau_{X \sqcup Y}\) is the finest topology for which both the inclusions are continuous.
    \end{enumerate}
\end{remark}

    %%% lecture 2 missing
    Note about product topology: \(\makeset{U \times V}{U \in \mathcal{O}_X, V \in \mathcal{O}_Y}\); often \(W \subset X \times Y \iff \forall (x, y) \in W \exists U_X \in \mathcal{O}_X, V_Y \in \mathcal{O}_Y, x \in U_X, y \in V_Y\)