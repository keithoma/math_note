% TO-DO
% 1. proof for lemma 17
% 2. where does "this lemma does not hold for basis belong to?
% 3. general clean up
\chapter{Topological Space}
% $$$$$$\  $$\                            $$\     
% $$  __$$\ $$ |                           $$ |    
% $$ /  \__|$$$$$$$\   $$$$$$\   $$$$$$\ $$$$$$\   
% $$ |      $$  __$$\ $$  __$$\ $$  __$$\\_$$  _|  
% $$ |      $$ |  $$ |$$$$$$$$ |$$$$$$$$ | $$ |    
% $$ |  $$\ $$ |  $$ |$$   ____|$$   ____| $$ |$$\ 
% \$$$$$$  |$$ |  $$ |\$$$$$$$\ \$$$$$$$\  \$$$$  |
%  \______/ \__|  \__| \_______| \_______|  \____/

\section{Cheat Sheet}

% $$$$$$$\                                 $$$$$$\  
% $$  __$$\                               $$  __$$\ 
% $$ |  $$ | $$$$$$\   $$$$$$\   $$$$$$\  $$ /  \__|
% $$$$$$$  |$$  __$$\ $$  __$$\ $$  __$$\ $$$$\     
% $$  ____/ $$ |  \__|$$ /  $$ |$$ /  $$ |$$  _|    
% $$ |      $$ |      $$ |  $$ |$$ |  $$ |$$ |      
% $$ |      $$ |      \$$$$$$  |\$$$$$$  |$$ |      
% \__|      \__|       \______/  \______/ \__|
\section{Proofs, Remarks, and Examples}
%
\subsection{Think of a Title}
\begin{defbox}
    \begin{definition}[Topological Space]
        A {\color{maththen}topological space} is an {\color{mathobj}ordered pair} \((X, \mathcal{O})\), where \(X\) is a {\color{mathif}set} and \(\mathcal{O}\) is a {\color{mathif}collection of subsets} that satisfies the following {\color{mathrem}axioms}.
        \begin{enumerate}
            \item The {\color{mathif}empty set} \(\varnothing\) and the {\color{mathif}entire set} \(X\) belongs to \(\mathcal{O}\).
            \item Any \textbf{arbitary} {\color{mathif}union} of members of \(\mathcal{O}\) belongs to \(\mathcal{O}\).
            \item The {\color{mathif}intersection} of \textbf{finite number} of members of \(\mathcal{O}\) belongs to \(\mathcal{O}\).
        \end{enumerate}
        
        The {\color{mathobj}collection} \(\mathcal{O}\) is called a {\color{maththen}topology} on \(X\) and the {\color{mathobj}elements} of \(\mathcal{O}\) are called {\color{maththen}open sets}. A {\color{mathobj}subset} \(A \subset X\) is said to be {\color{maththen}closed} if its {\color{mathif}complement} \(X \setminus A\) is {\color{mathif}open}. Moreover, a {\color{mathobj}subset} that is \textbf{both} {\color{mathif}open} and {\color{mathif}closed} is called {\color{maththen}clopen}.
    \end{definition}
\end{defbox}
%
\begin{rembox}
    \begin{remark}
        We often just write \(X\) instead of \((X, \mathcal{O})\) if the given topology is clear.
    \end{remark}
\end{rembox}
%
\begin{rembox}
    \begin{remark}
        \begin{enumerate}
            \item One might believe that openness and closedness are mutually exclusive, but this is not true, i.e. there are indeed subsets that are both open and closed. Trivial examples include the emptyset \(\varnothing\) and \(X\) itself.
            \item Furthermore, a subset may neither be open nor closed. If \(X = \{1, 2, 3\}\) is a set, then \((X, \set{\varnothing, X})\) is a well-defined topological space and the subsets \(\{1\}\) and \(\{2, 3\}\) are neither open nor closed.
        \end{enumerate}
    \end{remark}
\end{rembox}
%
\begin{rembox}
    \begin{remark}
        The definition of a topological space does not require \(X\) to be nonempty. If \(X\) is the emptyset, then the only topology that can be defined on \(X\) is \(\mathcal{O} = \{\varnothing\}\).
    \end{remark}
\end{rembox}
%
\begin{example}
    Let \(X = \set{1, 2, 3}\) be a set with three elements. The power set of \(X\) is
    \begin{align*}
        \mathcal{P}(X) = \set{\varnothing, \{1\}, \{2\}, \{3\}, \{1, 2\}, \{1, 3\}, \{2, 3\}, X}
    \end{align*}
    and has eight elements. There are \(2^8 = 256\) subsets of \(\mathcal{P}(X)\)\footnote{The number of subsets of any finite set is \(2^n\) where \(n\) is the number of elements because for each element one can choose this element to be inside the subset or not.}.
    \begin{enumerate}
        \item There are no singleton topologies because the first axiom of a topology requires at least two subsets. Thus, the only topology on \(X\) with two elements is \(\set{\varnothing, X}\).
        %
        \item Any collection of three subsets of \(X\) is a well-defined topology as long as it contains the emptyset and the entire set \(X\). Therefore, there are six topologies on \(X\) containing exactly three elements.
        \begin{align*}
            \set{\varnothing, \{1\}, X} \text{,} & & \set{\varnothing, \{2, 3\}, X} \text{,}\\
            \set{\varnothing, \{2\}, X} \text{,} & & \set{\varnothing, \{1, 3\}, X} \text{,}\\
            \set{\varnothing, \{3\}, X} \text{,} & & \set{\varnothing, \{1, 2\}, X} \text{.}
        \end{align*}
        The collection of closed subsets for each of the topologies above corresponds to the other topology in the same row, i.e. \(\set{\varnothing, \{2, 3\}, X}\) is the collection of closed subsets for the topology \(\set{\varnothing, \{1\}, X}\).
        
        Nonexamples of a topology include \(\set{\{1\}, \{2\}, X}\), \(\set{\varnothing, \{1\}, \{2, 3\}}\), \(\set{\{1\}, \{2\}, \{3\}}\) because these already fail to fulfill the first axiom.
        %
        \item Since any topology must include the union of members, the topologies on \(X\) with four elements contain a subset that is included in the other one that is not \(X\).
        \begin{align*}
            \set{\varnothing, \{1\}, \{1, 2\}, X} \text{,} & & \set{\varnothing, \{3\}, \{2, 3\}, X} \text{,} \\
            \set{\varnothing, \{1\}, \{1, 3\}, X} \text{,} & & \set{\varnothing, \{2\}, \{2, 3\}, X} \text{,} \\
            \set{\varnothing, \{2\}, \{1, 2\}, X} \text{,} & & \set{\varnothing, \{3\}, \{1, 3\}, X} \text{,}\\
            & \set{\varnothing, \{1\}, \{2, 3\}, X} \text{,} & \\
            & \set{\varnothing, \{2\}, \{1, 3\}, X} \text{,} & \\
            & \set{\varnothing, \{3\}, \{1, 2\}, X} &
        \end{align*}
        Again, the collection of closed subsets for each of the topologies listed are in the same row but on the other side, i.e. \(\set{\varnothing, \{3\}, \{2, 3\}, X}\) is the collection of closed subset for \(\set{\varnothing, \{1\}, \{1, 2\}, X}\).

        However, \(\set{\varnothing, \{1\}, \{2\}, X}\) and \(\set{\varnothing, \{1\}, \{2, 3\}, X}\) are not topologies because they do not contain the union of its members.
        %
        \item Similary, there are exactly six topologies on \(X\) that contain five elements. Those are
        \begin{align*}
            \set{\varnothing, \{1\}, \{2\}, \{1, 2\}, X}, & & \set{\varnothing, \{3\}, \{1, 3\}, \{2, 3\}, X}, \\
            \set{\varnothing, \{1\}, \{3\}, \{1, 3\}, X}, & & \set{\varnothing, \{2\}, \{1, 2\}, \{2, 3\}, X}, \\
            \set{\varnothing, \{2\}, \{3\}, \{2, 3\}, X}, & & \set{\varnothing, \{1\}, \{1, 2\}, \{1, 3\}, X} \text{.}
        \end{align*}
        Note that the collection of closed subsets corresponds to the one in the same row, i.e. for \(\set{\varnothing, \{1\}, \{2\}, \{1, 2\}, X}\) the collection of closed subsets are \(\set{\varnothing, \{3\}, \{1, 3\}, \{2, 3\}, X}\) and vice versa.
        \item a
        \begin{align*}
            \set{\varnothing, \{1\}, \{2\}, \{1, 2\}, \{1, 3\}, X} & & \set{\varnothing, \{2\}, \{3\}, \{1,3\}, \{2, 3\}, X} \\
            \set{\varnothing, \{1\}, \{2\}, \{1, 2\}, \{2, 3\}, X} & & \set{\varnothing, \{1\}, \{3\}, \{1, 3\}, \{2, 3\}} \\
            \set{\varnothing, \{2\}, \{3\}, \{1, 2\}, \{2, 3\}, X} & & \set{\varnothing, \{1\}, \{3\}, \{1, 2\}, \{1, 3\}}
        \end{align*}
        \item There are no topolgoies on \(X\) that contain exactly seven subsets. Take
        \begin{align*}
            \set{\varnothing, \{1\}, \{2\}, \{3\}, \{1, 2\}, X}
        \end{align*}
        for example. The missing elements \(\{2, 3\}\) and \(\{1, 3\}\) can be generated by \(\{2\} \cup \{3\}\) and \(\{1\} \cup \{3\}\).
        \item Lastly, \(\mathcal{P}(X)\) itself is a well-defined topology. Each subset of this topology is clopen.
    \end{enumerate}
    We have shown that on \(X = \{1, 2, 3\}\) there are \(20\) possible topologies. While the total amount of all possible subsets were \(256\), if we consider that any topology must contain the emptyset \(\varnothing\) and \(X\) itself, then the number of valid collection of subsets shrinks to \(2^6 = 64\). Thus, around \(1/3\) of sensible collection of subsets were topologies in this example.
\end{example}
There are two key takeaways from this example. Firstly, the collection of closed subsets for each topology were topologies themselves. Another is that the collection of all possible topologies can be partially ordered using the inclusion relation.
%
\begin{thmbox}
    \begin{proposition}
        Let \((X, \mathcal{O})\) be a topological space. The collection of all closed subsets of \(X\) with regards to \(\mathcal{O}\) is a topology.
    \end{proposition}
\end{thmbox}
%
\begin{proof}
    Denote \(\mathcal{C}\) to be the collection of closed subsets of \(X\).
    \begin{enumerate}
        \item Because \(X \setminus \varnothing = X \in \mathcal{O}\) and \(X \setminus X = \varnothing \in \mathcal{O}\), the emptyset \(\varnothing\) and \(X\) are closed subsets.
        \item Let \(I\) be an arbitary index set and \(\{C_i\}_{i \in I}\) be a familiy of closed subsets in \(\mathcal{C}\).
    \end{enumerate}
\end{proof}
%
\begin{defboxlight}
    \begin{definition}[Definition of a Topology via Closed Sets]
        A topology on a set \(X\) is a collection \(\mathcal{O}\) of subsets of \(X\) that satisfies the following axioms.
        \begin{enumerate}
            \item The emptyset \(\varnothing\) and the entire set \(X\) belong to \(\mathcal{O}\).
            \item Any arbitary intersection of members of \(\mathcal{O}\) belongs to \(\mathcal{O}\).
            \item The union of finite number of members of \(\mathcal{O}\) belongs to \(\mathcal{O}\).
        \end{enumerate}
        In this case, the elements of \(\mathcal{O}\) are called closed sets. A subset \(A \subset X\) is said to be open if its complement \(X \setminus A\) is open.
    \end{definition}
\end{defboxlight}
%
\begin{defboxlight}
    \begin{definition}[Definition of a Topology via Neighborhoods]
        
    \end{definition}
\end{defboxlight}
%
\begin{thmbox}
    \begin{proposition}
        The three definitions of a topology are equivalent.
    \end{proposition}
\end{thmbox}
%
\begin{defbox}
    \begin{definition}[Comparison of Topologies]
        Let \(\mathcal{O}_1\) and \(\mathcal{O}_2\) be two topologies on a set \(X\) such that \(\mathcal{O}_1 \subseteq \mathcal{O}_2\). Then the topology \(\mathcal{O}_1\) is said to be coarser (also weaker or smaller) than \(\mathcal{O}_2\), and \(\mathcal{O}_2\) is said to be finer (also stronger or larger) than \(\mathcal{O}_1\). The binary relation \(\subseteq\) defines a partial ordering relation on the set of all possible topologies on \(X\).
    \end{definition}
\end{defbox}
%
\begin{exmbox}
    \begin{example}
        Let \(X\) be a {\color{mathif}set}.
        \begin{enumerate}
            \item \(\mathcal{O} = \{\varnothing, \mathcal{P}(X)\}\) is called the {\color{maththen}trivial topology}. It is the coarsest topology that can be defined on a set.
            \item \(\mathcal{O} = \mathcal{P}(X)\) is called the {\color{maththen}discrete topology}. In this case, \((X, \mathcal{O})\) is called the {\color{maththen}discrete space}. It is the {\color{mathrem}finest topology} that can be defined on a set.
        \end{enumerate}
    \end{example}
\end{exmbox}

\begin{rembox}
    \begin{remark}
        The only sets in which the trivial topology and the discrete topology coincide are the emptyset and a singleton set.
    \end{remark}
\end{rembox}

\newpage
\subsection*{Metric Space}

\begin{thmbox}
    \begin{proposition}
        Let \((X, d)\) be a metric space. The collection of subsets
        \begin{align*}
            \mathcal{O}_d := \makeset{U \subset X}{U \text{ is a open subset in the metric space } (X, d)}
        \end{align*}
        defines a topology on \(X\). In other words, a metric induces a topology.
    \end{proposition}
\end{thmbox}

\begin{proof}
    We will show that \(\mathcal{O}_d\) fullfills the axioms of a topology.
    \begin{enumerate}
        \item The emptyset \(\varnothing\) is open in the metric space vacuously, hence \(\varnothing \in \mathcal{O}_d\). For the entire set \(X\), if \(x \in X\), then clearly \(B_\epsilon(x) \subset X\) for any \(\epsilon \in \mathbb{R}^+\), therefore \(X \in \mathcal{O}_d\).
        \item Let \(S \subset \mathcal{O}_d\) be a collection of subsets. Consider
        \begin{align*}
            x \in \bigcup_{U \in S} U \text{,}
        \end{align*}
        then \(x \in U_0\) for some set in \(\mathcal{O}_d\). \(U_0\) is open in the metric space, therefore, there is an \(\epsilon \in \mathbb{R}^+\) such that \(B_\epsilon(x) \in U_0\). The \(\epsilon\)-ball \(B_\epsilon(x)\) is also contained in the union of the subsets in \(S\). In other words, any arbitary union of members of \(\mathcal{O}_d\) are again in \(\mathcal{O}_d\).
        \item Let \(U, V \in \mathcal{O}_d\) and consider \(x \in U \cap V\). We have that \(x \in U\) and \(x \in V\). Since \(U, V \in \mathcal{O}_d\), they are open subsets in the metric space, hence there are \(\epsilon_1, \epsilon_2 \in \mathbb{R}^+\) such that \(B_{\epsilon_1}(x) \subset U\) and \(B_{\epsilon_2}(x) \subset V\). Without loss of generality assume \(\epsilon_1 \leq \epsilon_2\). Then, \(B_{\epsilon_1}(x) \subset B_{\epsilon_2}(x)\), so \(B_{\epsilon_1}(x) \subset V\) also. This implies \(B_{\epsilon_1}(x) \subset U \cap V\), so \(U \cap V \in \mathcal{O}_d\). By simple induction, we may conclude that the intersection of finite number of members of \(\mathcal{O}_d\) is again in \(\mathcal{O}_d\).
    \end{enumerate}
\end{proof}

\begin{rembox}
    \begin{remark}
        The proof above coincides with the fact that in a metric space arbitary union of open subsets and finite intersection of open subsets are open.
    \end{remark}
\end{rembox}

\begin{exmbox}
    \begin{example}
        The Zariski-topology.
    \end{example}
\end{exmbox}
%
\begin{exmbox}
    \begin{example}
        List of natural topologies.
        \begin{enumerate}
            \item On \(\mathbb{R}^n\) the canonical topology, called the Euclidean topology, is generated by the basis that is formed by open balls, i.e. open subsets of \(\mathbb{R}^n\) are arbitary unions of open balls. In other words, if \(A \in \mathcal{O}_{\mathbb{R}^n}\) and \(I\) is an index set, then
            \begin{equation*}
                A = \bigcup_{i \in I} B_r(p) = \bigcup_{i \in I} \makeset{x \in \mathbb{R}^n}{d(p, x) < r} \text{.}
            \end{equation*}
            This definition agrees with the topology endowed on arbitary metric spaces.
            \item The matrix space \(\mathrm{Mat}_{n \times m}(\mathbb{K})\) for a field \(\mathbb{K}\) does not have one canonical topology. Depending on the context and literature different ones are used.
            \begin{itemize}
                \item Since \(\mathrm{Mat}_{n \times m}(\mathbb{K})\) is isomorphic to \(\mathbb{R}^{n \cdot m}\), one could use the Euclidean topology as defined above.
                \item \(\mathrm{Mat}_{n \times m}(\mathbb{K})\) is a metric space via multitude of operator norms. The metric space induces the topology.
                \item Another metric on \(\mathrm{Mat}_{n \times m}(\mathbb{K})\) is the rank distance for \(A, B \in \mathrm{Mat}_{n \times m}\) defined as \(d(A, B) := \mathrm{rank}(B - A)\) which again would induce a topology.
            \end{itemize}
        \end{enumerate}
    \end{example}
\end{exmbox}
%
\begin{defbox}
    \begin{definition}
        Convergence in the topological sense
    \end{definition}
\end{defbox}
%
\begin{defbox}
    \begin{definition}[Continuous Maps]
        Let \((X, \tau_X)\) and \((Y, \tau_Y)\) be {\color{mathif}topological spaces}. A {\color{mathif}map} \(f: X \longrightarrow Y\) is said to be {\color{maththen}continuous} if the preimage of an open subset is again open, i.e.
        \begin{equation}
            \text{for all } U \in \tau_Y \text{ it is } f^{-1}(U) \in \tau_X \text{.}
        \end{equation}
    \end{definition}
\end{defbox}

\begin{thmbox}
    \begin{proposition}
        The different definitions of continuity in a topological space and a metric space are equivalent, i.e. if \(X\) and \(Y\) are metric spaces, then \(f: X \longrightarrow Y\) is \(\epsilon\)-\(\delta\)-continuous if and only if \(f\) is continuous.
    \end{proposition}
\end{thmbox}
\begin{proof}
    Let \(X\) and \(Y\) be metric spaces and \(f: X \longrightarrow Y\) a function.
    \begin{enumerate}
        \item ``\(\Rightarrow\)'': Let \(f\) be \(\epsilon\)-\(\delta\)-continuous and \(V \in \mathcal{O}_Y\) be an open subset. If \(f^{-1}(V)\) is empty, then we are finished, so consider \(x \in f^{-1}(V)\). We have that \(f(x) \in V\). Since \(V\) is an open subset, there is an \(\epsilon \in \mathbb{R}^+\) such that \(B_Y(f(x), \epsilon) \subset V\). Using the \(\epsilon\)-\(\delta\)-continuity of \(f\) yields
        \begin{align*}
            f(B_X(x, \delta)) \subset B_Y(f(x), \epsilon) \subset V \text{.}
        \end{align*}
        If we apply the definition of a preimage, we get \(B_X(x, \delta) \subset f^{-1}(V)\) which implies that \(f^{-1}(V)\) is open in the topological sense. Therefore, \(f\) is continuous.
        \item ``\(\Leftarrow\)'': Let \(f\) be continuous in the topological sense and consider \(x \in X\). The \(\epsilon\)-ball \(B_Y(f(x), \epsilon)\) is open in \(Y\), hence the preimage \(f^{-1}(B_Y(f(x), \epsilon))\) is also open and contains \(x\). Now, there exists a \(\delta \in \mathbb{R}^+\) such that
        \begin{align*}
            B_X(x, \delta) \subset f^{-1}(B_Y(f(x), \epsilon)) \text{.}
        \end{align*}
        Applying the definition of a preimage we get \(f(B_X(x, \delta)) \subset B_Y(f(x), \epsilon)\) which means \(f\) is \(\epsilon\)-\(\delta\)-continuous at \(x\). Since \(x\) was chosen arbitary, \(f\) is \(\epsilon\)-\(\delta\)-continuous.
    \end{enumerate}
\end{proof}

\begin{rembox}
    \begin{remark}
        Again, the proof above coincides with the fact that in a metric space, a function is \(\epsilon\)-\(\delta\)-continuous if and only if the preimage of any open subset is open.
    \end{remark}
\end{rembox}

\begin{thmbox}
    \begin{lemma}
        Let \(X\), \(Y\), and \(Z\) be topological spaces.
        \begin{enumerate}
            \item Any constant map \(f: X \longrightarrow Y\) is continuous.
            \item The identity map \(\mathrm{id}: X \longrightarrow Y\) is continuous.
            \item If \(f: X \longrightarrow Y\) is continuous, so is the restriction of \(f\) to any open subset of \(X\).
            \item If \(f: X \longrightarrow Y\) and \(g: Y \longrightarrow Z\) are continuous, so is their composition \(g \circ f: X \longrightarrow Z\).
        \end{enumerate}
    \end{lemma}
\end{thmbox}

\begin{thmbox}
    \begin{lemma}
        A map \(f: X \longrightarrow Y\) between topological spaces is continuous if and only if each point of \(X\) is contained in an open subset on which the restriction of \(f\) is continuous.
    \end{lemma}
\end{thmbox}

NOTES:
Depending on the choice of the topology, the only convergent sequences are the ones that are constant, and in some all sequence converges to any other point!!



\newpage
\subsection*{Homeomorphism}

\begin{defbox}
    \begin{definition}[Homeomorphism]
        Let \(X\) and \(Y\) be {\color{mathif}topological spaces}.
        \begin{enumerate}
            \item A {\color{mathobj}map} \(f: X \longrightarrow Y\) is a {\color{maththen}homeomorphism} if it has the following properties.

            \begin{enumerate}
                \item \(f\) is {\color{mathif}bijective}.
                \item \(f\) and the {\color{mathif}inverse map} \(f^{-1}\) is {\color{mathif}continuous}.
            \end{enumerate}

            \item Two topological spaces \(X\) and \(Y\) are said to be {\color{maththen}homeomorphic} if a homeomorphism exists.

            \item We denote the set of all homeomorphisms from \(X\) to \(Y\) by \(\mathrm{Homeo}(X, Y)\). If \(Y = X\) we also write \(\mathrm{Homeo}(X)\).
        \end{enumerate}
    \end{definition}
\end{defbox}

%
\begin{thmbox}
    \begin{proposition}
        The set of all homeomorphisms of \(X\) to itself \(\mathrm{Homeo}(X)\) is a group with composition as its operation.
    \end{proposition}
\end{thmbox}

\begin{proof}
    The identity function is contained in \(\mathrm{Homeo}(X)\) and is the identity element. Composition is associative and closed in \(\mathrm{Homeo}(X)\). By definition, \(\mathrm{Homeo}(X)\) contains the inverse of all its elements. Thus, \(\mathrm{Homeo}(X)\) is a group with composition as its operation.
\end{proof}

\begin{defbox}
    \begin{definition}[Base]
        Let \((X, \mathcal{O})\) a {\color{mathif}topological space}.
        \begin{enumerate}
            \item \(\mathcal{B} \subset \mathcal{O}\) is a {\color{maththen}basis} of the topology, if any member of \(\mathcal{O}\) is the {\color{mathif}union of subsets} from \(\mathcal{B}\).
            \item \(\mathcal{S} \subset \mathcal{O}\) is a {\color{maththen}subbasis} of the topology, if any member of \(\mathcal{O}\) is the {\color{mathif}union of finite intersections of subsets} from \(\mathcal{S}\).
        \end{enumerate}
        We say that \(\mathcal{B}\) and \(\mathcal{S}\) {\color{maththen}generates} \(\mathcal{O}\) and write \(\overline{\mathcal{S}} = \overline{\mathcal{B}} = \mathcal{O}\).
    \end{definition}
\end{defbox}

\begin{rembox}
    \begin{remark}
        We define the following as convention:
        \begin{enumerate}
            \item The empty union generates the emptyset, i.e.
            \begin{align*}
                \bigcup_{U \in \varnothing} U = \varnothing
            \end{align*}
            \item The empty intersection generates the entire set, i.e.
            \begin{align*}
                \bigcap_{U \in \varnothing} U = X
            \end{align*}
        \end{enumerate}
    \end{remark}
\end{rembox}

\begin{example}
    \begin{enumerate}
        \item \(\mathcal{O} = \set{\varnothing, \{a\}, X}\) then \(\mathcal{B} = \set{\{a\}, X}\)
        \item \(\mathcal{O} = \set{\varnothing, \{a\}, \{a, b\}}\) then \(\mathcal{B} = \set{\{a\}, \{a, b\}}\) and 
    \end{enumerate}
\end{example}

\begin{exmbox}
    \begin{example}
        \begin{enumerate}
            \item The set \(\Gamma\) of all open intervals in \(\mathbb{R}\) form a basis for the Euclidean topology on \(\mathbb{R}\). If we require \(\Gamma\) to be of all bounded open intervals, it will still generate the Euclidean topology.
        \end{enumerate}
    \end{example}
\end{exmbox}

\begin{thmbox}
    \begin{lemma}
    For any collection of subsets \(S \subset \mathcal{P}(X)\), there exists exactly one topology \(\mathcal{O} \subset \mathcal{P}(X)\) that contains \(S\) and is the coarsest topology to do so, i.e.
    \begin{enumerate}
        \item \(\mathcal{S} \subset \mathcal{O}\), and
        \item if \(\mathcal{O}^\prime \subset \mathcal{P}(X)\) is an another topology with \(S \subset \mathcal{O}'\), then \(\mathcal{O} \subset \mathcal{O}'\).
    \end{enumerate}
    \end{lemma}
\end{thmbox}

\begin{proof}
    Let \(S \subset \mathcal{P}(X)\) be a collection of subsets and \(\mathcal{O}(S)\) be the set of topologies that contain \(S\), i.e.
    \begin{align*}
        \mathcal{O}(S) = \makeset{\tau \subset \mathcal{P}(X)}{\tau \text{ is a topology and } S \subset \tau} \text{.}
    \end{align*}
    We know that \(\mathcal{O}(S)\) is not empty because \(\mathcal{P}(X) \in \mathcal{O}(S)\). Now define
    \begin{align*}
        \mathcal{O} := \bigcap_{\tau \in \mathcal{O}(S)} \tau \text{.}
    \end{align*}
    Our claim is that this \(\mathcal{O}\) is a topology.
    \begin{enumerate}
        \item The emptyset \(\varnothing\) and the entire set \(X\) is contained in each \(\tau \in \mathcal{O}(S)\) since these are topolgoies. Thus, the empty set and the entire set lie also in the intersection, i.e. \(\varnothing, X \in \mathcal{O}\).
        \item Let \(\{U_i\}_{i \in I}\) be a familiy of subsets in \(\mathcal{O}\) for an arbitary index set \(I\). This means for each \(i \in I\) it is \(U_i \subset \mathcal{O}\), therefore, again for each \(i \in I\) we have \(U_i \in \tau\). Since \(\tau\) was a topology, the arbitary union of \(\{U_i\}_{i \in I}\) will lie in \(\tau\).
        \item Similar for the finite intersection.
    \end{enumerate}
    In particular, \(\mathcal{O}\) lies in \(\mathcal{O}(S)\).

    MISSING THAT IT IS UNIQUE!
\end{proof}

\begin{defbox}
    \begin{definition}
        \begin{enumerate}
            \item Given \((X, \tau)\) be a {\color{mathif}topological space}, \(S \subset X\) a subset, the {\color{maththen}subspace topology} (also the induced topology or the relative topology) on \(S\) is defined by
            \begin{equation*}
                \tau_S = \makeset{S \cap U}{U \in \tau} \text{.}
            \end{equation*}
            \item Let \((X, \tau_X)\) and \((Y, \tau_Y)\) be two {\color{mathif}topological spaces}. The product topology of \(X\) and \(Y\) is defined by
            \begin{equation*}
                \tau_{X \times Y} := \makeset{U \times V}{U \in \tau_X \text{ and } V \in \tau_Y} \text{.}
            \end{equation*}
            \item Let \((X, \tau_X)\) and \((Y, \tau_Y)\) be two {\color{mathif}topological spaces}. The topological sum of \(X\) and \(Y\) is defined by
            \begin{align*}
                \tau_{X \sqcup Y} := \makeset{U \sqcup V}{U \in \tau_X \text{ and } V \in \tau_Y} \text{.}
            \end{align*}
        \end{enumerate}
    \end{definition}
\end{defbox}

\begin{defbox}
    \begin{definition}
        Let \((X, \tau)\) be a topological space.
        \begin{enumerate}
            \item Given a {\color{mathobj}point} \(p \in X\), a subset \(U \subset X\) is a neighborhood of \(p\) if there is an open subset \(V \in U\) such that \(p \in V\). If such a neighborhood exists, \(p\) is called a interior point of \(U\).
            \item Let \(S \subset X\) be a subset. The interior of \(S\), denoted by \(\mathring{S}\) or \(\mathrm{int}(S)\), is the {\color{mathobj}set} of all interior points of \(S\).
            \item Let \(S \subset X\) be a subset. The closure of \(S\), denoted by \(\overline{S}\) or \(\mathrm{cl}(S)\), is defined by
            \begin{equation*}
                \mathrm{cl}(S) := X \setminus \mathrm{int}(X \setminus S) \text{.}
            \end{equation*}
        \end{enumerate}
    \end{definition}
\end{defbox}

\begin{remark}
    This lemma does not hold for basis.
\end{remark}


\begin{remark}
    \begin{enumerate}
        \item \(\tau_{X \times Y}\) is the most coarse topology for which both of the projections are continuous.
        \item \(\tau_{X \sqcup Y}\) is the finest topology for which both the inclusions are continuous.
    \end{enumerate}
\end{remark}

    %%% lecture 2 missing
    Note about product topology: \(\makeset{U \times V}{U \in \mathcal{O}_X, V \in \mathcal{O}_Y}\); often \(W \subset X \times Y \iff \forall (x, y) \in W \exists U_X \in \mathcal{O}_X, V_Y \in \mathcal{O}_Y, x \in U_X, y \in V_Y\)

\begin{rembox}
    \begin{remark}
        Let \((X, \mathcal{O})\) be a {\color{mathobj}topological space}. A {\color{mathobj}subset} that is \textbf{both} {\color{mathif}open} and {\color{mathif}closed} is called {\color{maththen}clopen}. Moreover, a subset is {\color{maththen}clopen} if and only if its {\color{maththen}boundary} is {\color{maththen}empty}.
    \end{remark}
\end{rembox}
%
\begin{proof}
    Let \(A \subset X\) be clopen. Because \(A\) is closed, we have \(\mathrm{cl}(A) = A\), but on the other hand, \(A\) is open, so we also have \(\mathrm{int}(A) = A\). Then, the boundary of \(A\) is \(\partial A = \mathrm{cl}(A) \setminus \mathrm{int}(A) = A \setminus A = \varnothing\). All steps we have taken are not just implications, but equivalencies, therefore we have proven the statement.
\end{proof}
\newpage
% $$\   $$\            $$\                         
% $$$\  $$ |           $$ |                        
% $$$$\ $$ | $$$$$$\ $$$$$$\    $$$$$$\   $$$$$$$\ 
% $$ $$\$$ |$$  __$$\\_$$  _|  $$  __$$\ $$  _____|
% $$ \$$$$ |$$ /  $$ | $$ |    $$$$$$$$ |\$$$$$$\  
% $$ |\$$$ |$$ |  $$ | $$ |$$\ $$   ____| \____$$\ 
% $$ | \$$ |\$$$$$$  | \$$$$  |\$$$$$$$\ $$$$$$$  |
% \__|  \__| \______/   \____/  \_______|\_______/
\section{Exercises and Notes}

\begin{defbox}
    \begin{definition}[Metric Space]
        
    \end{definition}
\end{defbox}
\begin{defbox}
    \begin{definition}[Open and Closed Subsets]
        
    \end{definition}
\end{defbox}
\begin{thmbox}
    \begin{theorem}[Union and Intersection of Open Subsets]
        
    \end{theorem}
\end{thmbox}


\begin{rembox}
    \begin{definition}
        There are many equivalent ways to define continuity.
        \begin{itemize}
            \item \textit{\(\epsilon\)-\(\delta\)-continuity}:
            \item \textit{sequential continuity}:
        \end{itemize}
    \end{definition}
\end{rembox}