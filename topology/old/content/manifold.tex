\chapter{Manifold}

\begin{defbox}
    \begin{definition}
        A topological space \(M\) is said to be locally Euclidean of dimension \(n\) if one of the following equivalent conditions hold.
        \begin{enumerate}
            \item Every point \(p \in M\) has a neighborhood that is homeomorphic to an open subset of \(\mathbb{R}^n\).
            \item Every point \(p \in M\) has a neighborhood that is homeomorphic to an open ball in \(\mathbb{R}^n\).
            \item if every point \(p \in M\) has a neighborhood that is homeomorphic to \(\mathbb{R}^n\)
        \end{enumerate}
    \end{definition}
\end{defbox}

\begin{defbox}
    \begin{definition}
        An \(n\)-dimensional topological manifold is a second countable Hausdorff space that is locally Euclidean of dimension \(n\).
    \end{definition}
\end{defbox}

\begin{example}
    Some examples of topological manifolds.
    \begin{enumerate}
        \item The closed unit ball \(D^n \subset \mathbb{R}^n\) is a \(n\)-dimensional manifold.
        \begin{proof}
            Since \(D^n\) is a subset of \(\mathbb{R}^n\) which is second-countable Hausdorff space, the only thing to verify is that \(D^n\) is locally Euclidean.

            Let \(p \in D^n\) be a point. If \(p\) is an interior point, then by definition, there is an open ball centered at \(p\) contained in \(D^n\). Set this open ball as the neighborhood of \(p\), then this is simply homeomorphic to an open ball in \(\mathbb{R}^n\).

            Now let \(p \in \partial D^n\).
        \end{proof}
        \item The complex projective space \(\mathbb{C}\mathbb{P}^n\).
    \end{enumerate}
\end{example}