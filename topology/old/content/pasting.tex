\chapter{Pasting}

\begin{defbox}
    \begin{definition}
        Let \(X_0\) and \(X\) be topological spaces and \(\varphi: X_0 \longrightarrow X\) a continuous function. Set \(X_\varphi := X / \sim\) where \(\sim\) is generated by
        \begin{align*}
            \makeset{x \sim \varphi(x)}{x \in X_0} \text{.}
        \end{align*}
    \end{definition}
\end{defbox}

\begin{exmbox}
    \begin{example}
        idk
    \end{example}
\end{exmbox}

\begin{defbox}
    \begin{definition}
        Abbildungstorus
    \end{definition}
\end{defbox}

\begin{defbox}
    \begin{definition}
        Let \(X\) be a topological space.
        \begin{enumerate}
            \item \(X\) is said to be a first-countable space or to satisfy the first axiom of countability if each point has a countable neighbourhood basis (local base). That is, for each point \(x \in X\) there exists a sequence \(N_1, N_2, \ldots\) of neighbourhoods of \(x\) such that for any neighbourhood \(N\) of \(x\) there exists an integer \(i\) with \(N_i\) contained in \(N\). Since every neighbourhood of any point contains an open neighbourhood of that point, the neighbourhood basis can be chosen without loss of generality to consist of open neighbourhoods.
            \item \(X\) is said to be a second-countable space, also called completely separable space, or to satisfy the second axiom of countability if it has a countable base.
        \end{enumerate}
    \end{definition}
\end{defbox}

\begin{rembox}
    \begin{remark}
        The majority of 'everyday' spaces in mathematics are first-countable.
    \end{remark}
\end{rembox}

\begin{exmbox}
    \begin{example}
        Every metric space is first-countable.
    \end{example}
\end{exmbox}

\begin{proof}
    Let \((M, d)\) be a metric space and fix a point \(x \in M\). The open balls \(\{B_n(x)\}_{n \in \mathbb{N}}\) around \(x\) where
    \begin{align*}
        B_n(x) := \makeset{y \in M}{d(x, y) < \frac{1}{n}}
    \end{align*}
    defines a sequence of neighbourhoods of \(x\). Now, any neighbourhood \(N\) of \(x\) is contained in at least one of these balls. Thus, every metric space is first-countable.
\end{proof}

\begin{exmbox}
    \begin{example}
        Not all metric spaces are second-countable. For example, \(\mathbb{R}\) equipped with the discrete metric is not second-countable.
    \end{example}
\end{exmbox}

\begin{rembox}
    \begin{remark}
        A countable subbase induces a countable base.
    \end{remark}
\end{rembox}

\begin{thmbox}
    \begin{lemma}
        The second-countable axiom implies the first.
    \end{lemma}
\end{thmbox}

\begin{exmbox}
    \begin{example}
        The converse of proposition XXX is not true. For example, the Sorgenfrey line \((\mathbb{R}, \mathcal{O})\) where the topology \(\mathcal{O}\) is generated by all the right-open, left-closed intervals, i.e.
        \begin{align*}
            \mathcal{O} = \{\varnothing\} \cup \makeset{\bigcup_{i \in I} [a_i, b_i) \subset \mathbb{R}}{-\infty < a_i < b_i < \infty}
        \end{align*}
        satisfies the first-countable axiom but not the second.
    \end{example}
\end{exmbox}
%
\begin{proof}
    Let \((\mathbb{R}, \mathcal{O})\) be the Sorgenfrey line as defined abvoe.
    \begin{enumerate}
        \item We show that \((\mathbb{R}, \mathcal{O})\) satisfies the first-countable axiom. Fix a point \(x \in \mathbb{R}\) and set
        \begin{align*}
            I_n(x) := \bigg[x, x + \frac{1}{n}\bigg) \text{.}
        \end{align*}
        Then, \(\{I_n\}_{n \in \mathbb{N}}\) is a sequence of neighbourhoods of \(x\). For any neighbourhood \(N\) of \(x\) there is a \(n \in \mathbb{N}\) large enough such that \(I_n \subset N\), and thus, the Sorgenfrey line \((\mathbb{R}, \mathcal{O})\) is first-countable.
        \item We show that \((\mathbb{R}, \mathcal{O})\) is not second-countable. Consider a point \(x \in \mathbb{R}\). Because \([x, \infty)\) is open, there must be a base element \(B \subset [x, \infty]\). This is true for all points on \(\mathbb{R}\) and all the base elements are distinct. \(\mathbb{R}\) is not countable, thus \((\mathbb{R}, \mathcal{O})\) is not second-countable.
    \end{enumerate}
\end{proof}
%
\begin{thmbox}
    \begin{lemma}
        Is \(X\) a first-countable space, then
        \begin{enumerate}
            \item all sequentially-continuous function is continuous.
            \item all compact spaces are also sequentially compact.
        \end{enumerate}
    \end{lemma}
\end{thmbox}

\begin{defbox}
    \begin{definition}
        A topological mannifold \(\mathcal{M}^n\) of dimension \(n\) is a topological space that is \(T_2\), is a second-countable space, and for each point has a neighbourhood that is homeomorph to a open subset of
        \begin{align*}
            \mathbb{H}^n := \makeset{x = (x_1, \ldots, x_n) \in \mathbb{R}^n}{x_n \geq 0}
        \end{align*}
    \end{definition}
\end{defbox}

\begin{defbox}
    \begin{definition}[Pasting]
        Let \(X\) and \(Y\) be two disjoint topological spaces, \(A \subset X\) a closed subset, and \(f: A \longrightarrow Y\) continuous. Define a equivalence relation on \(X \cup Y\) by
        \begin{align*}
            v \sim w :
            \begin{cases}
                v = w \text{ if } v, w \in X \cup Y \\
                f(v) = f(w) \text{ if } v, w \in A \\
                v = f(w) \text{ if } v \in Y, w \in A \\
                w = f(v) \text{ if } v \in A, w \in Y
            \end{cases}
        \end{align*}
        We write
        \begin{align*}
            Y \cup_f X := X \cup Y / \sim
        \end{align*}
    \end{definition}
\end{defbox}
\begin{example}
    Consider \(X = [0, 1]\) on the \(x\)-axis and \(Y = [0, 1]\).
\end{example}
\begin{example}
    Portals can be thought of a Pasting
\end{example}

\begin{example}
    Let \(D^n := \makeset{x \in \mathbb{R}^n}{||x|| \leq 1}\in \mathbb{R}^n\) be the \(n\)-dimensional ball and define
    \begin{align*}
        id_{}: \partial D^n \longrightarrow \partial D^n, x \mapsto x
    \end{align*}
    It is
    \begin{align*}
        D^n \cup_{\varphi} D^n = D^n \cup D^n / \sim
    \end{align*}
\end{example}

\begin{thmbox}
    \begin{lemma}
        Any open subset of an \(n\)-manifold is an \(n\)-manifold.
    \end{lemma}
\end{thmbox}