\documentclass[a4paper]{book}
\title{Topology}
\author{K}


% ---------------------------------------------------------------------
% P A C K A G E S
% ---------------------------------------------------------------------

% typography and formatting
\usepackage[english]{babel}
\usepackage[utf8]{inputenc}
\usepackage{geometry}
\usepackage{exsheets}
\usepackage{environ}
\usepackage{graphicx}
\usepackage{cutwin}

% mathematics
\usepackage{amsthm} % for theorems, and definitions
\usepackage{amssymb}
\usepackage{amsmath}
\usepackage{textcomp}
% \usepackage{MnSymbol} % for \cupdot

% extra
\usepackage{xcolor}
\usepackage{tikz}

% ---------------------------------------------------------------------
% S E T T I N G
% ---------------------------------------------------------------------

%maybe delete later, for colorbox
\usepackage{tcolorbox}
\newtcolorbox{defbox}{colback=blue!5!white,colframe=blue!75!black}
\newtcolorbox{defboxlight}{colback=cyan!5!white,colframe=cyan!75!black}
\newtcolorbox{thmbox}{colback=orange!5!white,colframe=orange!75!black}
\newtcolorbox{rembox}{colback=purple!5!white,colframe=purple!75!black}
\newtcolorbox{exmbox}{colback=gray!5!white,colframe=gray!75!black}

% typography and formatting
\geometry{margin=3cm}

\SetupExSheets{
  counter-format = ch.qu,
  counter-within = chapter,
  question/print = true,
  solution/print = true,
}

% mathematics
\newcounter{global}

\theoremstyle{definition}
\newtheorem{definition}{Definition}[]
\newtheorem{example}{Example}[definition]

\newtheorem{theorem}[definition]{Theorem}
\newtheorem{corollary}{Corollary}
\newtheorem{lemma}[definition]{Lemma}
\newtheorem{proposition}[definition]{Proposition}

\newtheorem*{remark}{Remark}

% extra
\definecolor{mathif}{HTML}{0000A0} % for conditions
\definecolor{maththen}{HTML}{CC5500} % for consequences
\definecolor{mathrem}{HTML}{8b008b} % for notes
\definecolor{mathobj}{HTML}{008800}

\usetikzlibrary{positioning}
\usetikzlibrary{shapes.geometric, arrows}

% ---------------------------------------------------------------------
% C O M M A N D S
% ---------------------------------------------------------------------

\newcommand{\norm}[1]{\left\lVert#1\right\rVert}
\newcommand{\rank}{\text{rank}}
\newcommand{\Vol}{\text{Vol}}

\newcommand{\set}[1]{\left\{\, #1 \,\right\}}
\newcommand{\makeset}[2]{\left\{\, #1 \mid #2 \,\right\}}

\newcommand*\diff{\mathop{}\!\mathrm{d}}
\newcommand*\Diff{\mathop{}\!\mathrm{D}}

\newcommand\restr[2]{{% we make the whole thing an ordinary symbol
  \left.\kern-\nulldelimiterspace % automatically resize the bar with \right
  #1 % the function
  \vphantom{\big|} % pretend it's a little taller at normal size
  \right|_{#2} % this is the delimiter
  }}

% ---------------------------------------------------------------------
% R E N D E R
% ---------------------------------------------------------------------

\newif\ifshowproof
\showprooftrue

\NewEnviron{Proof}{%
    \ifshowproof%
        \begin{proof}%
            \BODY
        \end{proof}%
    \fi%
}%

\begin{document}
\maketitle
\tableofcontents
%%%%%%%%%%%%%%%%%%%%%%%%%%%%%%%%%%%%%%%%%%%%%%%%%%%%%%%%%%%%%%%%%%%%%%%%%%%%%%%
\chapter*{Conventions}
\(\mathbb{N}\) contains \(0\), that is \(\mathbb{N} = \set{0, 1, 2, \ldots}\).
\chapter{Topological Space}
\section{Definitions and Theorems}
\begin{defbox}
    \begin{definition}[Topological Space]
        A {\color{maththen}topological space} is an {\color{mathobj}ordered pair} \((X, \tau)\), where \(X\) is a {\color{mathif}set} and \(\tau\) is a {\color{mathif}collection of subsets} that satisfies the following {\color{mathrem}axioms}.
        \begin{enumerate}
            \item The {\color{mathif}empty set} \(\varnothing\) and the {\color{mathif}entire set} \(X\) belongs to \(\tau\).
            \item Any \textbf{arbitary} {\color{mathif}union} of members of \(\tau\) belongs to \(\tau\).
            \item The {\color{mathif}intersection} of \textbf{finite number} of members of \(\tau\) belongs to \(\tau\).
        \end{enumerate}
        The {\color{mathobj}collection} \(\tau\) is called a {\color{maththen}topology} on \(X\) and the {\color{mathobj}elements} of \(\tau\) are called {\color{maththen}open sets}. A {\color{mathobj}subset} \(A \subset X\) is said to be {\color{maththen}closed} if its {\color{mathif}complement} \(X \setminus A\) is {\color{mathif}open}.
    \end{definition}
\end{defbox}
%TODO: There are other equivalent definitions
%TODO: write something about the notation

\begin{defbox}
    \begin{definition}[Continuous Maps]
        Let \((X, \tau_X)\) and \((Y, \tau_Y)\) be {\color{mathif}topological spaces}. A {\color{mathif}map} \(f: X \longrightarrow Y\) is said to be {\color{maththen}continuous} if the preimage of an open subset is again open, i.e.
        \begin{equation}
            \text{for all } U \in \tau_Y \text{ it is } f^{-1}(U) \in \tau_X \text{.}
        \end{equation}
    \end{definition}
\end{defbox}
\begin{thmbox}
    \begin{lemma}
        The different definitions of continuity in a topological space and a metric space are equivalent, i.e. if \(X\) and \(Y\) are metric spaces, then \(f: X \longrightarrow Y\) is \(\epsilon\)-\(\delta\)-continuous if and only if \(f\) is continuous.
    \end{lemma}
\end{thmbox}

\begin{defbox}
    \begin{definition}[Homeomorphism]
        Let \(X\) and \(Y\) be {\color{mathif}topological spaces}.
        \begin{enumerate}
            \item A {\color{mathobj}map} \(f: X \longrightarrow Y\) is a {\color{maththen}homeomorphism} if it has the following properties.

            \begin{enumerate}
                \item \(f\) is {\color{mathif}bijective}.
                \item \(f\) and the {\color{mathif}inverse map} \(f^{-1}\) is {\color{mathif}continuous}.
            \end{enumerate}

            \item Two topological spaces \(X\) and \(Y\) are said to be {\color{maththen}homeomorphic} if a homeomorphism exists.

            \item We denote the set of all homeomorphisms from \(X\) to \(Y\) by \(\mathrm{Homeo}(X, Y)\). If \(Y = X\) we also write \(\mathrm{Homeo}(X)\).
        \end{enumerate}
    \end{definition}
\end{defbox}

\begin{defbox}
    \begin{definition}[Homeomorphism]
        Let \((X, \tau)\) a {\color{mathif}topological space}.
        \begin{enumerate}
            \item \(\mathcal{B} \subset \mathcal{O}\) is a {\color{maththen}basis} of the topology, if any member of \(\mathcal{O}\) is the {\color{mathif}union of subsets} from \(\mathcal{B}\).
            \item \(\mathcal{S} \subset \mathcal{O}\) is a {\color{maththen}subbasis} of the topology, if any member of \(\mathcal{O}\) is the {\color{mathif}union of finite intersections of subsets} from \(\mathcal{S}\).
        \end{enumerate}
        We say that \(\mathcal{B}\) and \(\mathcal{S}\) {\color{maththen}generates} \(\mathcal{O}\) and write \(\overline{\mathcal{S}} = \overline{\mathcal{B}} = \mathcal{O}\).
    \end{definition}
\end{defbox}

\begin{thmbox}
    \begin{lemma}
    Let \(\mathcal{S} \subset \mathcal{P}(X)\) be a {\color{mathobj}collection of subsets}, then there {\color{maththen}exists} \textbf{exactly one} topology \(\tau \subset \mathcal{P}(X)\) of \(X\) such that
    \begin{enumerate}
        \item \(\mathcal{S} \subset \tau\)
        \item If \(\tau' \subset \mathcal{P}(X)\) a topology with \(S \subset \tau'\), then \(\tau \subset \tau'\).
    \end{enumerate}
    \end{lemma}
\end{thmbox}

\begin{defbox}
    \begin{definition}
        \begin{enumerate}
            \item Given \((X, \tau)\) be a {\color{mathif}topological space}, \(S \subset X\) a subset, the {\color{maththen}subspace topology} (also the induced topology or the relative topology) on \(S\) is defined by
            \begin{equation*}
                \tau_S = \makeset{S \cap U}{U \in \tau} \text{.}
            \end{equation*}
            \item Let \((X, \tau_X)\) and \((Y, \tau_Y)\) be two {\color{mathif}topological spaces}. The product topology of \(X\) and \(Y\) is defined by
            \begin{equation*}
                \tau_{X \times Y} := \makeset{U \times V}{U \in \tau_X \text{ and } V \in \tau_Y} \text{.}
            \end{equation*}
            \item Let \((X, \tau_X)\) and \((Y, \tau_Y)\) be two {\color{mathif}topological spaces}. The topological sum of \(X\) and \(Y\) is defined by
            \begin{align*}
                \tau_{X \sqcup Y} := \makeset{U \sqcup V}{U \in \tau_X \text{ and } V \in \tau_Y} \text{.}
            \end{align*}
        \end{enumerate}
    \end{definition}
\end{defbox}

\begin{defbox}
    \begin{definition}
        Let \((X, \tau)\) be a topological space.
        \begin{enumerate}
            \item Given a {\color{mathobj}point} \(p \in X\), a subset \(U \subset X\) is a neighborhood of \(p\) if there is an open subset \(V \in U\) such that \(p \in V\). If such a neighborhood exists, \(p\) is called a interior point of \(U\).
            \item Let \(S \subset X\) be a subset. The interior of \(S\), denoted by \(\mathring{S}\) or \(\mathrm{int}(S)\), is the {\color{mathobj}set} of all interior points of \(S\).
            \item Let \(S \subset X\) be a subset. The closure of \(S\), denoted by \(\overline{S}\) or \(\mathrm{cl}(S)\), is defined by
            \begin{equation*}
                \mathrm{cl}(S) := X \setminus \mathrm{int}(X \setminus S) \text{.}
            \end{equation*}
        \end{enumerate}
    \end{definition}
\end{defbox}
\newpage
%%%%%%%%%%%%%%%%%%%%%%%%%%%%%%%%%%%%%%%%
%%%%%%%%%%%%%%%%%%%%%%%%%%%%%%%%%%%%%%%%
%%%%%%%%%%%%%%%%%%%%%%%%%%%%%%%%%%%%%%%%
\section{Proofs, Remarks, and Examples}
%%%%%%%%%%%%%%%%%%%%%%%%%%%%%%%%%%%%%%%%
%%%%%%%%%%%%%%%%%%%%%%%%%%%%%%%%%%%%%%%%
%%%%%%%%%%%%%%%%%%%%%%%%%%%%%%%%%%%%%%%%

\begin{example}
    Let \(X\) be a {\color{mathif}set}.
    \begin{enumerate}
        \item \(\tau = \mathcal{P}(X)\) is called the {\color{maththen}discrete topology}. In this case, \((X, \tau)\) is called the {\color{maththen}discrete space}. It is the {\color{mathrem}finest topology} that can be defined on a set. (The set of all possible topologies on a given set forms a partially ordered set.)
        \item \(\tau = \{\varnothing, \mathcal{P}(X)\}\) is called the {\color{maththen}trivial topology}.
        \item Let \((X, d)\) be a {\color{mathif}metric space}. Set
        \begin{equation}
            \tau_d := \makeset{U \in X}{U \text{ is a open subset in the metric space } (X, d)} \text{.}
        \end{equation}
        Recall that \(U\) being an open subset in the metric space \((X, d)\) means that for all \(x \in U\) there is an \(r > 0\) such that \(B_d(x, r)\) is contained in \(U\).

        Here, \(\tau\) is a topology. In other words, a metric induces a topology.

        (Proof as homework.)
        \item The Zariski-topology.
    \end{enumerate}
\end{example}

\begin{example}
    List of natural topologies.
    \begin{enumerate}
        \item On \(\mathbb{R}^n\) the canonical topology, called the Euclidean topology, is generated by the basis that is formed by open balls, i.e. open subsets of \(\mathbb{R}^n\) are arbitary unions of open balls. In other words, if \(A \in \mathcal{O}_{\mathbb{R}^n}\) and \(I\) is an index set, then
        \begin{equation*}
            A = \bigcup_{i \in I} B_r(p) = \bigcup_{i \in I} \makeset{x \in \mathbb{R}^n}{d(p, x) < r} \text{.}
        \end{equation*}
        This definition agrees with the topology endowed on arbitary metric spaces.
    \end{enumerate}
\end{example}

\begin{remark}
    The set of all homeomorphisms of \(X\) to itself \(\mathrm{Homeo}(X)\) is a group with composition as its operation.
\end{remark}

\begin{remark}
    This lemma does not hold for basis.
\end{remark}


\begin{remark}
    \begin{enumerate}
        \item \(\tau_{X \times Y}\) is the most coarse topology for which both of the projections are continuous.
        \item \(\tau_{X \sqcup Y}\) is the finest topology for which both the inclusions are continuous.
    \end{enumerate}
\end{remark}

    %%% lecture 2 missing
    Note about product topology: \(\makeset{U \times V}{U \in \mathcal{O}_X, V \in \mathcal{O}_Y}\); often \(W \subset X \times Y \iff \forall (x, y) \in W \exists U_X \in \mathcal{O}_X, V_Y \in \mathcal{O}_Y, x \in U_X, y \in V_Y\)
\chapter{Connected Spaces and Sets}
\section{Definition and Theorems}
\begin{defbox}
    \begin{definition}
        A {\color{mathobj}topological space} \(X\) is said to be {\color{maththen}connected}, if one of the following {\color{mathrem}equivalent} conditions is met.
        \begin{enumerate}
            \item \(X\) is \textbf{not} a {\color{mathif}union} of two {\color{mathif}disjoint} sets.
            \item The \textbf{only} {\color{mathif}subsets} of \(X\) that are \textbf{both} {\color{mathif}open} and {\color{mathif}closed} ({\color{mathrem}clopen}) are the emptyset \(\varnothing\) and the entire set \(X\).
            \item The \textbf{only} {\color{mathif}subsets} of \(X\) with empty {\color{mathif}boundary} are the emptyset \(\varnothing\) and the entire set \(X\).
            \item All {\color{mathif}continuous} maps from \(X\) to the two point space \(\{0, 1\}\) endowed with the {\color{mathif}discrete} topology is {\color{mathif}constant}. 
        \end{enumerate}
    \end{definition}
\end{defbox}

\begin{thmbox}
    \begin{lemma}
        Any {\color{mathif}interval} \(I \subset \mathbb{R}\) is {\color{maththen}connected}.
    \end{lemma}
\end{thmbox}
%
\begin{defbox}
    \begin{definition}
        A connected component of a topological space is a maximally connected subset \(X_0 \subseteq X\), i.e. \(X_0\) connected and for all \(X_0 \subsetneq X_1\) then \(X_1\) is not connected.
    \end{definition}
\end{defbox}
%
\begin{thmbox}
    \begin{proposition}
        Connected components are closed subsets.
    \end{proposition}
\end{thmbox}
%
\begin{thmbox}
    \begin{lemma}[Lemma 11]
        Let \(X\) be connected and \(f: X \longrightarrow Y\) and locally constant, i.e. for all \(x \in X\) there exists a \(U_x \in \mathcal{O}_X\), \(x \in U_x\) such that \(f\) restricted on \(U_x\) is identical to \(f(x)\)., then \(f\) is constant.
    \end{lemma}
\end{thmbox}
%
\begin{defbox}
    \begin{definition}
        \(X\) is said to be {\color{maththen}path connected}, if for every pair of points \(x\) and \(x_0\) in \(X\) there is a continuous map (called path) \(\gamma: [0, 1] \longrightarrow X\) with \(\gamma(0) = x_0\) and \(\gamma(1) = x\).
    \end{definition}
\end{defbox}
%
\begin{thmbox}
    \begin{lemma}
        If \(X\) is path connected, then it is also connected.
    \end{lemma}
\end{thmbox}

\section{Proofs, Remarks, and Examples}

\begin{proof}
    Fix an interval \(I \subset \mathbb{R}\), and let \(A, B \subset \mathbb{R}\) be two nonempty, open and disjoint subsets such that \(A \sqcup B = I\). Moreover, let \(a \in A\) and \(b \in B\) and assume without loss of generality that \(a < b\). If we set
    \begin{align}
        s := \inf \makeset{x \in B}{a < x} \text{,}
    \end{align}
    then \(s \in [a, b] \subset I\) because \(I\) is an interval.   
\end{proof}

\begin{example}
    The general linear group \(\mathrm{GL}_n(K)\) for a field \(K\) and \(n \in \mathbb{N}\) is not connected for \(K = \mathbb{R}\) and \(K = \mathbb{C}\).
\end{example}

\begin{remark}
    Let \(f: X \longrightarrow Y\) be continuous and \(X\) be connected, then \(f(X) \subset Y\) is connected.
\end{remark}
\begin{proof}
    Let \(f(X) = A \sqcup B\) with \(A\) and \(B\) being two open disjoint sets. \(f^{-1}(A)\) and \(f^{-1}(B)\) are open since \(f\) is continuous. We also have \(f^{-1}(A) \cap f^{-1}B = f^{-1}(A \cap B) = \varnothing\) so \(f^{-1}(A) = \varnothing\) or \(f^{-1}(B) = \varnothing\), so \(A = \varnothing\) or \(B = \varnothing\) and we are done.
\end{proof}
\begin{proof}
    % proof missing
\end{proof}
\begin{example}
    For \(\mathbb{Q} \subset \mathbb{R}\) the connected components are points and those are not open.
\end{example}

\begin{proof}
    Locally constant implies continuous with regards to the discrete topology on \(Y\). Let \(x \in X\), \(X = f^{-1}(f(x)) \cup f^{-1}(Y \setminus \{f(x)\})\) is a disjoint union and since \(X\) is connected \(f^{-1}(Y \setminus \{f(x)\}) = \varnothing\). Conclude \(f\) is identical to \(f(x)\).
\end{proof}

\textbf{Application:} \(f: X \longrightarrow \{0, 1\}\), \(X\) is connected, \(f\) locally constant, there is a \(x \in X\) such that \(f(x) = 1\), then \(f\) is identical to \(1\).

\begin{proof}
    Let \(A\) and \(B\) two disjoint open sets such that \(A \sqcup B = X\), and let \(a \in A\) and \(b \in B\). Let \(\gamma: [0, 1] \longrightarrow X\) be continuous path with \(\gamma(0) = x_0\) and \(\gamma(1) = x_1\). We have that \(\gamma^{-1}\)
\end{proof}

\begin{remark}
    The converse statement is not true in general.
\end{remark}

\begin{example}
    \(X = \makeset{(x, \sin(\frac{1}{x}))}{x > 0} \cup \{0\} \times [-1, 1] \subset \mathbb{R}^2\) is connected but not path connected.
\end{example}
\begin{proof}
    Homework
\end{proof}
\begin{remark}
    missing
\end{remark}

\chapter{Separation Axioms}
Literature: Groessere Liste in Sten, Seibeck

\begin{defbox}
    \begin{definition}[\(T_1\) Space]
        Let \(X\) be a {\color{mathif}topological space}.
        \begin{enumerate}
            \item We say that two {\color{mathobj}points} \(x\) and \(y\) can be {\color{maththen}separated} if each lies in a {\color{mathif}neighborhood} that does \textbf{not} contain the other point.

            \item A {\color{mathobj}topological space} \(X\) is a {\color{maththen}\(T_1\) space} if any two distinct points in \(X\) are {\color{mathif}separated}.
        \end{enumerate}
    \end{definition}
\end{defbox}
%
\begin{thmbox}
    \begin{proposition}
        Let \(X\) be a {\color{mathif}topological space}. Then, the following are {\color{mathrem}equivalent}.
        \begin{enumerate}
            \item \(X\) is a {\color{maththen}\(T_1\) space}.
            \item {\color{mathif}Points} are {\color{maththen}closed} in \(X\), i.e. given any \(x \in X\), the {\color{mathif}singleton} set \(\{x\}\) is a {\color{maththen}closed} set.
        \end{enumerate}
    \end{proposition}
\end{thmbox}
%
\begin{defbox}
    \begin{definition}[\(T_2\) Space]
        Let \(X\) be a {\color{mathif}topological space}.
        \begin{enumerate}
            \item {\color{mathobj}Points} \(x\) and \(y\) in \(X\) can be {\color{maththen}separated by neighborhood} if there exists a {\color{mathif}neighborhood} \(U\) of \(x\) and a {\color{mathif}neighborhood} \(V\) of \(y\) such that \(U\) and \(V\) are {\color{mathif}disjoint}, i.e. \(U \cap V = \varnothing\).
            \item A {\color{mathobj}topological space} \(X\) is a {\color{maththen}\(T_2\) space} if any two distinct points in \(X\) are {\color{mathif}separated by neighborhood}.
        \end{enumerate}
    \end{definition}
\end{defbox}
%
\begin{thmbox}
    \begin{proposition}
        Let \(X\) be a {\color{mathif}topological space}. Then, the following are {\color{mathrem}equivalent}.
        \begin{enumerate}
            \item \(X\) is a {\color{maththen}\(T_2\) space}.
            \item Any singleton set \(\{x\}\) is the intersection of all closed neighborhoods of \(x\).
            \item The diagonal \(\Delta = \makeset{(x, x)}{x \in X}\) is closed as a subset of the product space \(X \times X\).
        \end{enumerate}
    \end{proposition}
\end{thmbox}
%
\begin{thmbox}
    \begin{proposition}
        \(T_2\) spaces are also \(T_1\) spaces.
    \end{proposition}
\end{thmbox}
\chapter{Compact Spaces}
\begin{defbox}
    \begin{definition}
        \begin{enumerate}
            \item A {\color{mathobj}topological space} \(X\) is called {\color{maththen}compact} if each of its {\color{mathif}open cover} has a \textbf{finite} {\color{mathif}subcover}.
            \item A {\color{mathobj}topological space} \(X\) is called {\color{maththen}sequentially compact} if every {\color{mathif}sequence} in \(X\) has a {\color{mathif}convergent subsequence} whose limit is in \(X\).
        \end{enumerate}
    \end{definition}
\end{defbox}
%
\begin{thmbox}
    \begin{theorem}
        Satz 17
    \end{theorem}
\end{thmbox}
%
\begin{thmbox}
    \begin{theorem}
        Let \(A \subset \mathbb{R}^n\) be a subset. \(A\) is compact if and only if it is closed and bounded.
    \end{theorem}
\end{thmbox}
%
\begin{thmbox}
    \begin{theorem}
        Let \(X\) be a \(T_2\) space. If a subset \(K \subset X\) is compact, then it is closed.
    \end{theorem}
\end{thmbox}
%
\begin{thmbox}
    \begin{theorem}
        Let \(X\) and \(Y\) be topological spaces, \(X\) compact, and \(Y\) be a \(T_2\) space. If \(f: X \longrightarrow Y\) is bijective and continuous, then the inverse function \(f^{-1}\) is continuous.
    \end{theorem}
\end{thmbox}
%
\newpage
\section{Proofs, Remarks, and Examples}

\begin{thmbox}
    \begin{lemma}
        \([0, 1] \subset \mathbb{R}\) is {\color{maththen}compact}.
    \end{lemma}
\end{thmbox}
\chapter{Quotient Space}
% $$$$$$\  $$\                            $$\     
% $$  __$$\ $$ |                           $$ |    
% $$ /  \__|$$$$$$$\   $$$$$$\   $$$$$$\ $$$$$$\   
% $$ |      $$  __$$\ $$  __$$\ $$  __$$\\_$$  _|  
% $$ |      $$ |  $$ |$$$$$$$$ |$$$$$$$$ | $$ |    
% $$ |  $$\ $$ |  $$ |$$   ____|$$   ____| $$ |$$\ 
% \$$$$$$  |$$ |  $$ |\$$$$$$$\ \$$$$$$$\  \$$$$  |
%  \______/ \__|  \__| \_______| \_______|  \____/
\section{Definitions and Theorems}
% $$$$$$$\                                 $$$$$$\  
% $$  __$$\                               $$  __$$\ 
% $$ |  $$ | $$$$$$\   $$$$$$\   $$$$$$\  $$ /  \__|
% $$$$$$$  |$$  __$$\ $$  __$$\ $$  __$$\ $$$$\     
% $$  ____/ $$ |  \__|$$ /  $$ |$$ /  $$ |$$  _|    
% $$ |      $$ |      $$ |  $$ |$$ |  $$ |$$ |      
% $$ |      $$ |      \$$$$$$  |\$$$$$$  |$$ |      
% \__|      \__|       \______/  \______/ \__|
\section{Proofs, Remarks, and Examples}
\begin{defbox}
    \begin{definition}
        Let \((X, \mathcal{O})\) be a {\color{mathif}topological space}, and let \(\sim\) be an {\color{mathif}equivalence relation} on \(X\). The {\color{maththen}quotient set}, \(X / \sim\) is the {\color{mathobj}set} of {\color{mathobj}equivalence classes} of elements of \(X\). The equivalence class of \(x \in X\) is {\color{mathrem}denoted} \([x]\). The {\color{maththen}projection map} (also {\color{mathrem}quotient} or {\color{mathrem}canonical map}) associated with \(\sim\) refers to the following {\color{mathif}surjective map}:
        \begin{align*}
            \pi: X \longrightarrow X / \sim, \qquad x \mapsto [x]
        \end{align*}
        For any subset \(S \subset X / \sim\) (so in particular, \(s \subset X\) for every \(s \in S\)) the following holds.
        \begin{align*}
            q^{-1}(S) = \makeset{x \in X}{[x] \in S} = \bigcup_{s \in S} s \text{.}
        \end{align*}

        The quotient space under \(\sim\) is the quotient set \(X / \sim\) equipped with the quotient topology, which is the topology whose open sets are subsets \(U \subset X / \sim\) such that 
        \begin{align*}
            \makeset{x \in X}{[x] \in U} = \bigcup_{u \in U}u
        \end{align*}
        is an open subset of \((X, \mathcal{O}_X)\); that is, \(U \subset X / \sim\)
    \end{definition}
\end{defbox}
%
\begin{proof}
    We will show that \(\mathcal{O}_\sim\) is a topology.
    \begin{enumerate}
        \item Clearly, \(\pi^{-1}(\varnothing) = \varnothing \in \mathcal{O}\), thus \(\varnothing \in \mathcal{O}_\sim\). Moreover, \(\pi(X) = X / \sim\), hence \(\pi^{-1}(X / \sim) = X \in \mathcal{O}\), and it is \(X / \sim \in \mathcal{O}_\sim\).
        \item Let \(I\) be an arbitary index set and \(\{U_i\}_{i \in I}\) a family of open sets in \(X / \sim\). It is
        \begin{align}
            \pi^{-1}\left( \bigcup_{i \in I} U_i \right) = \bigcup_{i \in I} \pi^{-1}(U_i) \text{.}
        \end{align}
        Since \(U_i\) is open and \(\pi\) is continuous, \(\pi^{-1}(U_i)\) is open for all \(i \in I\). Thus \(\bigcup_{i \in I} U_i \in \mathcal{O}_\sim\).
        \item Similar as above as preimages preserve unions and intersections.
    \end{enumerate}
\end{proof}
%
\begin{exmbox}
    \begin{example}
        \begin{enumerate}
            \item \(\mathbb{R} / \mathbb{Z}\)

            This space is homeomorph to \(S^{-1}\) and is compact.
            \item \((\mathbb{R} / \mathbb{Q}, \mathcal{O}_{\mathbb{R} / \mathbb{Q}})\) is the trivial topology.
        \end{enumerate}
    \end{example}
\end{exmbox}

\begin{thmbox}
    \begin{proposition}
        \(\mathcal{O}_{X / \sim}\) is the {\color{maththen}finest} {\color{mathobj}topology} in which the {\color{mathif}projection map} \(\pi: X \longrightarrow X / \sim\) is {\color{mathif}continuous}.
    \end{proposition}
\end{thmbox}
\begin{defbox}
    Let \(X\) and \(Y\) be topological spaces and let \(p: X \longrightarrow Y\) be a surjective map. The map is a quotient map (also said strong continuity) if one of the equivalent condition hold.
    \begin{enumerate}
        \item A subset \(U \subset Y\) is open in \(Y\) if and only if the preimage \(p^{-1}(U)\) is open in \(X\).
        \item A subset \(U \subset Y\) is closed in \(Y\) if and only if the preimage \(p^{-1}(U)\) is closed in \(X\).
    \end{enumerate}
\end{defbox}
\begin{rembox}
    \begin{remark}
        Quotient maps are continuous. There are quotient maps that are neither open nor closed maps.
    \end{remark}
\end{rembox}

\begin{thmbox}
    \begin{theorem}
        Let \(Y\) be a topological space. Then the following are equivalent.
        \begin{enumerate}
            \item \(f: X / \sim \longrightarrow Y\) continuous
            \item \(f \circ \pi : X \longrightarrow Y\) is continuous.
        \end{enumerate}

        Moreover, if \(X\) is connected, then \(X / \sim\) is connected. Same is true for path-connectedness and compactness.
    \end{theorem}
\end{thmbox}

\begin{defbox}
    \begin{definition}
        A topological group \(G\) is a topological space that is also a group such that the group operation
        \begin{align*}
            \circ: G \times G \longrightarrow G, (x, y) \mapsto x \circ y
        \end{align*}
        and the inversion map
        \begin{align*}
            ^{-1}: G \longrightarrow G, x \mapsto x^{-1}
        \end{align*}
        are continuous. Here \(G \times G\) is viewed as a topological space with the product topology. Such a topoloy is said to be compatible with the group operations and is called a group topology.
    \end{definition}
\end{defbox}

\begin{rembox}
    \begin{remark}
        About homoemorphism of G-space.
    \end{remark}
\end{rembox}

\begin{defbox}
    \begin{definition}
        Consider a group acting on a set \(X\). The orbit of an element \(x\) in \(X\) is the set of elements in \(X\) to which \(x\) can be moved by the elements of \(G\). The orbit of \(x\) is denoted by \(G \cdot x\).
        \begin{align*}
            G \cdot x := \makeset{g \cdot x}{g \in G} \text{.}
        \end{align*}
    \end{definition}
\end{defbox}

\begin{defbox}
    \begin{definition}
        \(X / G := X / \sim\) such that \(x \sim y\) if and only if there is a \(g \in G\) such that \(x = gy\).
    \end{definition}
\end{defbox}

\begin{defbox}
    \begin{definition}[Hilbert]
        \begin{itemize}
            \item Matrix \(A\) is semi-stable if \(A\) is diagonizable.
            \item Matrix \(A\) is stable, if it is semi-stable and all Eigenvalues are distinct.
        \end{itemize}
    \end{definition}
\end{defbox}

\begin{defbox}
    \begin{definition}[Stabilisator]
        \(x \in X\)

        \(G \supset G_x := \makeset{h}{h \cdot x = x}\)
    \end{definition}
\end{defbox}

\begin{thmbox}
    \begin{lemma}
        \begin{enumerate}
            \item \(G_x \subset G\) is a subgroup.
            \item \(G/G_x \longrightarrow G_x\) is well-defined and \([x] \mapsto gx\) is a continuous bijection (repsective of the quotient topology on \(G / G_x\)).
        \end{enumerate}
    \end{lemma}
\end{thmbox}

\begin{thmbox}
    \begin{corollary}
        If \(G\) is compact and \(X\) is \(T_2\), then \(f: G/G_x \longrightarrow G\) is a homeomorphism.
    \end{corollary}
\end{thmbox}

\begin{defbox}
    \begin{definition}
        
    \end{definition}
\end{defbox}

\newpage
% $$\   $$\            $$\                         
% $$$\  $$ |           $$ |                        
% $$$$\ $$ | $$$$$$\ $$$$$$\    $$$$$$\   $$$$$$$\ 
% $$ $$\$$ |$$  __$$\\_$$  _|  $$  __$$\ $$  _____|
% $$ \$$$$ |$$ /  $$ | $$ |    $$$$$$$$ |\$$$$$$\  
% $$ |\$$$ |$$ |  $$ | $$ |$$\ $$   ____| \____$$\ 
% $$ | \$$ |\$$$$$$  | \$$$$  |\$$$$$$$\ $$$$$$$  |
% \__|  \__| \______/   \____/  \_______|\_______/
\section{Exercises and Notes}
\chapter{Pasting}

\begin{defbox}
    \begin{definition}
        Let \(X_0\) and \(X\) be topological spaces and \(\varphi: X_0 \longrightarrow X\) a continuous function. Set \(X_\varphi := X / \sim\) where \(\sim\) is generated by
        \begin{align*}
            \makeset{x \sim \varphi(x)}{x \in X_0} \text{.}
        \end{align*}
    \end{definition}
\end{defbox}

\begin{exmbox}
    \begin{example}
        idk
    \end{example}
\end{exmbox}

\begin{defbox}
    \begin{definition}
        Abbildungstorus
    \end{definition}
\end{defbox}

\begin{defbox}
    \begin{definition}
        Let \(X\) be a topological space.
        \begin{enumerate}
            \item \(X\) is said to be a first-countable space or to satisfy the first axiom of countability if each point has a countable neighbourhood basis (local base). That is, for each point \(x \in X\) there exists a sequence \(N_1, N_2, \ldots\) of neighbourhoods of \(x\) such that for any neighbourhood \(N\) of \(x\) there exists an integer \(i\) with \(N_i\) contained in \(N\). Since every neighbourhood of any point contains an open neighbourhood of that point, the neighbourhood basis can be chosen without loss of generality to consist of open neighbourhoods.
            \item \(X\) is said to be a second-countable space, also called completely separable space, or to satisfy the second axiom of countability if it has a countable base.
        \end{enumerate}
    \end{definition}
\end{defbox}

\begin{rembox}
    \begin{remark}
        The majority of 'everyday' spaces in mathematics are first-countable.
    \end{remark}
\end{rembox}

\begin{exmbox}
    \begin{example}
        Every metric space is first-countable.
    \end{example}
\end{exmbox}

\begin{proof}
    Let \((M, d)\) be a metric space and fix a point \(x \in M\). The open balls \(\{B_n(x)\}_{n \in \mathbb{N}}\) around \(x\) where
    \begin{align*}
        B_n(x) := \makeset{y \in M}{d(x, y) < \frac{1}{n}}
    \end{align*}
    defines a sequence of neighbourhoods of \(x\). Now, any neighbourhood \(N\) of \(x\) is contained in at least one of these balls. Thus, every metric space is first-countable.
\end{proof}

\begin{exmbox}
    \begin{example}
        Not all metric spaces are second-countable. For example, \(\mathbb{R}\) equipped with the discrete metric is not second-countable.
    \end{example}
\end{exmbox}

\begin{rembox}
    \begin{remark}
        A countable subbase induces a countable base.
    \end{remark}
\end{rembox}

\begin{thmbox}
    \begin{lemma}
        The second-countable axiom implies the first.
    \end{lemma}
\end{thmbox}

\begin{exmbox}
    \begin{example}
        The converse of proposition XXX is not true. For example, the Sorgenfrey line \((\mathbb{R}, \mathcal{O})\) where the topology \(\mathcal{O}\) is generated by all the right-open, left-closed intervals, i.e.
        \begin{align*}
            \mathcal{O} = \{\varnothing\} \cup \makeset{\bigcup_{i \in I} [a_i, b_i) \subset \mathbb{R}}{-\infty < a_i < b_i < \infty}
        \end{align*}
        satisfies the first-countable axiom but not the second.
    \end{example}
\end{exmbox}
%
\begin{proof}
    Let \((\mathbb{R}, \mathcal{O})\) be the Sorgenfrey line as defined abvoe.
    \begin{enumerate}
        \item We show that \((\mathbb{R}, \mathcal{O})\) satisfies the first-countable axiom. Fix a point \(x \in \mathbb{R}\) and set
        \begin{align*}
            I_n(x) := \bigg[x, x + \frac{1}{n}\bigg) \text{.}
        \end{align*}
        Then, \(\{I_n\}_{n \in \mathbb{N}}\) is a sequence of neighbourhoods of \(x\). For any neighbourhood \(N\) of \(x\) there is a \(n \in \mathbb{N}\) large enough such that \(I_n \subset N\), and thus, the Sorgenfrey line \((\mathbb{R}, \mathcal{O})\) is first-countable.
        \item We show that \((\mathbb{R}, \mathcal{O})\) is not second-countable. Consider a point \(x \in \mathbb{R}\). Because \([x, \infty)\) is open, there must be a base element \(B \subset [x, \infty]\). This is true for all points on \(\mathbb{R}\) and all the base elements are distinct. \(\mathbb{R}\) is not countable, thus \((\mathbb{R}, \mathcal{O})\) is not second-countable.
    \end{enumerate}
\end{proof}
%
\begin{thmbox}
    \begin{lemma}
        Is \(X\) a first-countable space, then
        \begin{enumerate}
            \item all sequentially-continuous function is continuous.
            \item all compact spaces are also sequentially compact.
        \end{enumerate}
    \end{lemma}
\end{thmbox}

\begin{defbox}
    \begin{definition}
        A topological mannifold \(\mathcal{M}^n\) of dimension \(n\) is a topological space that is \(T_2\), is a second-countable space, and for each point has a neighbourhood that is homeomorph to a open subset of
        \begin{align*}
            \mathbb{H}^n := \makeset{x = (x_1, \ldots, x_n) \in \mathbb{R}^n}{x_n \geq 0}
        \end{align*}
    \end{definition}
\end{defbox}

\begin{defbox}
    \begin{definition}[Pasting]
        Let \(X\) and \(Y\) be two disjoint topological spaces, \(A \subset X\) a closed subset, and \(f: A \longrightarrow Y\) continuous. Define a equivalence relation on \(X \cup Y\) by
        \begin{align*}
            v \sim w :
            \begin{cases}
                v = w \text{ if } v, w \in X \cup Y \\
                f(v) = f(w) \text{ if } v, w \in A \\
                v = f(w) \text{ if } v \in Y, w \in A \\
                w = f(v) \text{ if } v \in A, w \in Y
            \end{cases}
        \end{align*}
        We write
        \begin{align*}
            Y \cup_f X := X \cup Y / \sim
        \end{align*}
    \end{definition}
\end{defbox}
\begin{example}
    Consider \(X = [0, 1]\) on the \(x\)-axis and \(Y = [0, 1]\).
\end{example}
\begin{example}
    Portals can be thought of a Pasting
\end{example}

\begin{example}
    Let \(D^n := \makeset{x \in \mathbb{R}^n}{||x|| \leq 1}\in \mathbb{R}^n\) be the \(n\)-dimensional ball and define
    \begin{align*}
        id_{}: \partial D^n \longrightarrow \partial D^n, x \mapsto x
    \end{align*}
    It is
    \begin{align*}
        D^n \cup_{\varphi} D^n = D^n \cup D^n / \sim
    \end{align*}
\end{example}

\begin{thmbox}
    \begin{lemma}
        Any open subset of an \(n\)-manifold is an \(n\)-manifold.
    \end{lemma}
\end{thmbox}
\chapter{Manifold}

\begin{defbox}
    \begin{definition}
        A topological space \(M\) is said to be locally Euclidean of dimension \(n\) if one of the following equivalent conditions hold.
        \begin{enumerate}
            \item Every point \(p \in M\) has a neighborhood that is homeomorphic to an open subset of \(\mathbb{R}^n\).
            \item Every point \(p \in M\) has a neighborhood that is homeomorphic to an open ball in \(\mathbb{R}^n\).
            \item if every point \(p \in M\) has a neighborhood that is homeomorphic to \(\mathbb{R}^n\)
        \end{enumerate}
    \end{definition}
\end{defbox}

\begin{defbox}
    \begin{definition}
        An \(n\)-dimensional topological manifold is a second countable Hausdorff space that is locally Euclidean of dimension \(n\).
    \end{definition}
\end{defbox}

\begin{example}
    \begin{enumerate}
        \item The closed unit ball \(D^n \subset \mathbb{R}^n\).
    \end{enumerate}
\end{example}
\chapter{Fundamental Group}

\begin{defbox}
    \begin{definition}
        Let \(X\) and \(Y\) be topological spaces, and let \(f, g: X \longrightarrow Y\) be continuous maps. A homotopy from \(f\) to \(g\) is a continuous map \(H: X \times [0, 1] \longrightarrow Y\) such that for all \(x \in X\),
        \begin{align*}
            H(x, 0) = f(x); \quad H(x, 1) = g(x) \text{.}
        \end{align*}
        If there exists a homotopy from \(f\) to \(g\), we say that \(f\) and \(g\) are homotopic, and write \(f \simeq g\).
    \end{definition}
\end{defbox}

\begin{thmbox}
    \begin{proposition}
        For any topological spaces \(X\) and \(Y\), homotopy is an equivalence relation on the sets of all continuous maps from \(X\) to \(Y\).
    \end{proposition}
\end{thmbox}

\begin{proof}
    Let \(X\) and \(Y\) be two topological spaces and fix three continuous maps \(f, g, h: X \longrightarrow Y\).
    \begin{enumerate}
        \item Reflexivity is clear.
        \item Symmetry is clear.
        \item Let \(F: f \simeq g\) and \(G: g \simeq h\) and define
        \begin{align*}
            H(x, t) = \begin{cases}
                F(x, 2t) \quad 0 \leq t \leq \frac{1}{2}
                G(x, 2t-1) \quad 
            \end{cases}
        \end{align*}
    \end{enumerate}
\end{proof}
\end{document}