\documentclass[a4paper]{book}
\title{Topology}
\author{K}


% ---------------------------------------------------------------------
% P A C K A G E S
% ---------------------------------------------------------------------

% typography and formatting
\usepackage[english]{babel}
\usepackage[utf8]{inputenc}
\usepackage{geometry}
\usepackage{exsheets}
\usepackage{environ}
\usepackage{graphicx}
\usepackage{cutwin}

% mathematics
\usepackage{amsthm} % for theorems, and definitions
\usepackage{amssymb}
\usepackage{amsmath}
\usepackage{textcomp}
% \usepackage{MnSymbol} % for \cupdot

% extra
\usepackage{xcolor}
\usepackage{tikz}

% ---------------------------------------------------------------------
% S E T T I N G
% ---------------------------------------------------------------------

%maybe delete later, for colorbox
\usepackage{tcolorbox}
\newtcolorbox{defbox}{colback=blue!5!white,colframe=blue!75!black}
\newtcolorbox{defboxlight}{colback=cyan!5!white,colframe=cyan!75!black}
\newtcolorbox{thmbox}{colback=orange!5!white,colframe=orange!75!black}
\newtcolorbox{rembox}{colback=purple!5!white,colframe=purple!75!black}
\newtcolorbox{exmbox}{colback=gray!5!white,colframe=gray!75!black}

% typography and formatting
\geometry{margin=3cm}

\SetupExSheets{
  counter-format = ch.qu,
  counter-within = chapter,
  question/print = true,
  solution/print = true,
}

% mathematics
\newcounter{global}

\theoremstyle{definition}
\newtheorem{definition}{Definition}[]
\newtheorem{example}{Example}[definition]

\newtheorem{theorem}[definition]{Theorem}
\newtheorem{corollary}{Corollary}
\newtheorem{lemma}[definition]{Lemma}
\newtheorem{proposition}[definition]{Proposition}

\newtheorem*{remark}{Remark}

% extra
\definecolor{mathif}{HTML}{0000A0} % for conditions
\definecolor{maththen}{HTML}{CC5500} % for consequences
\definecolor{mathrem}{HTML}{8b008b} % for notes
\definecolor{mathobj}{HTML}{008800}

\usetikzlibrary{positioning}
\usetikzlibrary{shapes.geometric, arrows}

% ---------------------------------------------------------------------
% C O M M A N D S
% ---------------------------------------------------------------------

\newcommand{\norm}[1]{\left\lVert#1\right\rVert}
\newcommand{\rank}{\text{rank}}
\newcommand{\Vol}{\text{Vol}}

\newcommand{\set}[1]{\left\{\, #1 \,\right\}}
\newcommand{\makeset}[2]{\left\{\, #1 \mid #2 \,\right\}}

\newcommand*\diff{\mathop{}\!\mathrm{d}}
\newcommand*\Diff{\mathop{}\!\mathrm{D}}

\newcommand\restr[2]{{% we make the whole thing an ordinary symbol
  \left.\kern-\nulldelimiterspace % automatically resize the bar with \right
  #1 % the function
  \vphantom{\big|} % pretend it's a little taller at normal size
  \right|_{#2} % this is the delimiter
  }}

% ---------------------------------------------------------------------
% R E N D E R
% ---------------------------------------------------------------------

%\setlength\parindent{0pt}


\newif\ifshowproof
\showprooftrue

\NewEnviron{Proof}{%
    \ifshowproof%
        \begin{proof}%
            \BODY
        \end{proof}%
    \fi%
}%

\begin{document}
\maketitle
\tableofcontents
%%%%%%%%%%%%%%%%%%%%%%%%%%%%%%%%%%%%%%%%%%%%%%%%%%%%%%%%%%%%%%%%%%%%%%%%%%%%%%%
\chapter{Introduction}
\chapter{Topological Spaces}
\section*{2-1}
``\(\Rightarrow\)'': Let \(f: X_1 \longrightarrow X_2\) be a homeomorphism and fix a subset (not necessarily open) \(U \in \mathcal{T}_1\).
\begin{enumerate}
    \item Assume \(U\) is open in \(X_1\). Because \(f\) is continuous, the image of open subsets are again open, thus \(f(U)\) lies in \(\mathcal{T}_2\).
    \item On the other hand, if \(f(U)\) is open in \(X_2\), then since \(f\) is bijective we have
    \begin{align*}
        f^{-1} \left(f \left(U\right)\right) = U \text{.}
    \end{align*}
    Because \(f\) is continuous, the preimage of open subsets under \(f\) is open. We may therefore conclude \(U\) is open in \(X_1\).
\end{enumerate}
We have shown that if \(f\) is a homeomorphism, then \(f(\mathcal{T}_1) = \mathcal{T}_2\). \\

\noindent ``\(\Leftarrow\)'': Let \(f: X_1 \longrightarrow X_2\) be a bijective map such that \(f(\mathcal{T}_1) = \mathcal{T}_2\). Consider the inverse map \(f^{-1}\). We want to show \(f^{-1}\) is continuous. Fix an open subset \(U \in \mathcal{T}_1\). It is
\begin{align*}
    \left(f^{-1}\right)^{-1} \left(U\right) = f(U)
\end{align*}
because \(f\) is bijective. Since \(f(\mathcal{T}_1) = \mathcal{T}_2\) and \(U\) is open, \(f(U)\) is open as well. Hence the preimage of \(U\) under \(f^{-1}\) is open and \(f^{-1}\) is continuous.

Now we show that \(f\) is also continuous. Again, fix an open subset \(V \in \mathcal{T}_2\). The preimage of \(V\) under \(f\) is just the image of the inverse function. We have already shown that the inverse is continuous. Thus, \(f^{-1}(V)\) is open and \(f\) is continuous. Since \(f\) and \(f^{-1}\) exist and are continuous, \(f\) is a homeomorphism as desired.

\section*{2-2}
\subsection*{a)}
We show that \(\mathcal{T}\) is a topology by verifying the axioms of a topology.
\begin{enumerate}
    \item Since \(\mathcal{T}\) is the collection of all unions of finite intersections of elements of \(\mathcal{B}\), it contains the union of all elements of \(\mathcal{B}\) which is just \(X\). The union of empty collection generates the emptyset so \(\varnothing \in \mathcal{T}\) as well.
    \item Let \(\mathcal{U} \subset \mathcal{T}\) be any subset. The elements of \(\mathcal{U}\) are unions of finite intersections of elements of \(\mathcal{B}\). Thus, \(\bigcup_{U \in \mathcal{U}} U\) is again a union of finite intersections of elements of \(\mathcal{B}\). In other words, \(\mathcal{T}\) is closed under union.
    \item \(\mathcal{T}\) is stable under finite intersections due to distributive property of sets.
\end{enumerate}

\subsection*{b)}

\section*{2-3}
\subsection*{1.}
The collection of subset \(\mathcal{T}_1 = \makeset{U \subset X}{X \setminus U \text{ is finite or is all of } X}\) forms a topology. We show this by verifying the axioms of a topology.
\begin{enumerate}
    \item It is \(X \setminus \varnothing = X\) and \(X \setminus X = \varnothing\) which is finite. Thus, \(X \in \mathcal{T}_1\) and \(\varnothing \in \mathcal{T}_1\).
    \item Let \(\mathcal{U} \subset \mathcal{T}\) be a subset. By De Morgan's laws we have
    \begin{align*}
        X \setminus \left(\bigcup_{U \in \mathcal{U}} U \right) = \bigcap_{U \in \mathcal{U}} \left( X \setminus U \right) \text{.}
    \end{align*}
    Since each \(U \in \mathcal{U}\) lies in \(\mathcal{T}\), the complement \(X \setminus U\) is finite or is all of \(X\). Therefore, the intersection of all \(X \setminus U\) is again finite or all of \(X\), and we may conclude that \(\mathcal{T}\) is stable under arbitary unions.
    \item Use De Morgan's law again.
\end{enumerate}

\subsection*{2.}
The collection of subsets \(\mathcal{T}_2 = \makeset{U \subset X}{X \setminus U \text{ is infinite or is empty}}\) is not a topology. Take \(X = \mathbb{Z}\) for example and consider \(A = \set{1, 2, 3, \ldots}\) and \(B = \set{-1, -2, -3, \ldots}\). \(A\) and \(B\) are open because their complements are the non-positive and the non-negative integers respectively. If \(\mathcal{T}_2\) is a topology, it should contain their union \(A \cup B = \mathbb{Z} \setminus \{0\}\). However,
\begin{align*}
    \mathbb{Z} \setminus (A \cup B) = \mathbb{Z} (\mathbb{Z} \setminus \{0\}) = \{0\}
\end{align*}
which is not infinite and thus doesn't lie in \(\mathcal{T}_2\).

\subsection*{3.}
The collection of subsets \(\mathcal{T}_3 = \makeset{U \subset X}{X \setminus U \text{ is countable or all of } X}\) is a topology PROBABLY.

\section*{2-4}

Already did somewhere else.

\section*{2-5}
\begin{enumerate}
    \item \(\mathrm{id}_1: X \longrightarrow \mathbb{R}^2\) is continuous probably.
    \item \(\mathrm{id}_2: \mathbb{R}^2 \longrightarrow X\) is not continuous probably.
\end{enumerate}

\section*{2-6}
\(f\) is continuous because any preimage of a subset \(U \subset Z\) under \(f\) is open, since any subset in \(X\) is open.

For \(g\), the only preimages to check are the emptyset \(\varnothing\) and \(Y\). Simply, \(g^{-1}(\varnothing) = \varnothing\) and \(g^{-1}(Y) = Z\). Both subsets are open in \(Z\), therefore \(g\) is continuous.

If \(h\) is constant, say \(h(Y) = \{p\}\), then \(h^{-1}(U) = Y\) if \(p \in U\) and \(h^{-1}(U) = \varnothing\) if \(p \in U\). In both cases the preimages are open, thus \(h\) is continuous. Assume \(h\) is continuous but not constant, i.e. there are points \(x_1, x_2 \in Y\) such that \(h(x_1) \neq h(x_2)\). \(Z\) is Hausdorff, so there are disjoint neighbourhoods \(U\) of \(h(x_1)\) and \(V\) of \(h(x_2)\). \(h\) was assumed to be continuous, so \(h^{-1}(U) = Y\) and \(h^{-1}(V) = Y\) which is impossible (REALLY?).

\section*{2-7}
\subsection*{a)}


\subsection*{f)}
\section*{2-8}
Firstly, any element in \(f(\mathcal{B})\) is open because \(f\) is an open map. Fix an open subset \(V\) in \(Y\) and consider its preimage \(f^{-1}(V)\) under \(f\). Because \(f\) is continuous, the preimage is open, thus there are base elements \(B_i\) with \(i \in I\) in \(\mathcal{B}\) such that
\begin{align*}
    f^{-1}(V) = \bigcup_{i \in I} B_i \text{.}
\end{align*}
The surjectivity of \(f\) grants us \(f(f^{-1}(V)) = V\), therefore, we have
\begin{align*}
    f(f^{-1}(V)) = V = f \left( \bigcup_{i \in I} B_i \right) = \bigcup_{i \in I} f(B_i)\text{.}
\end{align*}
Thus, \(f(\mathcal{B})\) is a basis of \(Y\).

\section*{2-9}

\section*{2-10}
Fix a point \(y\) in \(Y\). Since \(f\) is surjective, there is an \(x\) in \(X\) such that \(f(x) = y\). \(X\) is locally Euclidean, thus there is a neighbourhood \(U\) of \(x\) that is homeomorphic to \(\mathbb{R}^n\). Moreover, \(f\) is locally homeomorphic, so there is a neighbourhood \(V\) of \(x\) such that the restriction of \(f\) under \(V\) is a homeomorphism. Then, the intersection \(U \cap V = N\) has both of these properties, i.e. \(N\) is a neighbourhood of \(x\) that is homeomorphic to \(\mathbb{R}^n\) and the restriction of \(f\) under \(V\) is a homeomorphism. \(f(N)\) is a neighbourhood of \(y\) that is homeomorphic to \(\mathbb{R}^n\), therefore \(Y\) is locally Euclidean.

\section*{2-11}

``\(\Rightarrow\)'': Let \(M^0\) be a \(0\)-manifold and consider a point \(p \in M^0\). First, we show that \(M^0\) is discrete. Since \(M^0\) is locally Euclidean, there is a neighbourhood \(U\) of \(p\) such that \(U\) is homeomorphic to an open subset of \(\mathbb{R}^0\). But \(\mathbb{R}^0\) contains only one element, thus the only nonempty open subset is \(\mathbb{R}^0\). Now, a homeomorphism implies bijectivity, we have that \(U = \{p\}\). Every singleton set in \(M^0\) is open, so \(M^0\) is a discrete space.

\(M^0\) is also countable because being a manifold implies that it has a countable base and any base must contain all the singleton sets. \\

\noindent ``\(\Leftarrow\)'': Let \(M^0\) be a countable discrete space. \(M^0\) is second-countable because the set of singletons form a countable base. It is also \(T_2\) since each point has itself as its neighbourhood which clearly does not contain any other points. Now let \(p \in M^0\) be a point. \(\{p\}\) is a neighbourhood of \(p\) and it is homeomorphic to \(\mathbb{R}^0\) by the mapping \(p \mapsto 0\), thus \(M^0\) is locally Euclidean.

\section*{2-12}

\subsection*{a}
It is \(L(b) \cap R(a) = \makeset{c \in X}{c < b \text{ and } c > a} = (a, b)\), thus \((a, b)\) is open. \\

\noindent Moreover, we have \(L(a) \cup R(b) = \makeset{c \in X}{c < a \text{ or } c > b} = X \setminus [a, b]\) which is open, so \([a, b]\) is closed.

\subsection*{b}

Let \(a, b \in X\) be two distinct points and assume without loss of generality \(a < b\). Then, \(L(b)\) is open and contains the point \(a\), while \(R(a)\) is also open and contains the point \(b\), but \(L(b)\) and \(R(a)\) are disjoint. Thus, \(X\) is Hausdorff.

\subsection*{c}

Fix two points \(a, b \in X\). By definition, it is
\begin{align*}
    \overline{(a, b)} = \bigcap \makeset{C \subset X}{(a, b) \subset C \text{ and } C \text{ is closed in } X} \text{.}
\end{align*}
We have shown that \([a, b]\) is closed in \(a)\) and clearly contains \((a, b)\), thus
\begin{align*}
    [a, b] \in \makeset{C \subset X}{(a, b) \subset C \text{ and } C \text{ is closed in } X}
\end{align*}
or in other words
\begin{align*}
    \bigcap \makeset{C \subset X}{(a, b) \subset C \text{ and } C \text{ is closed in } X} \subset [a, b]
\end{align*}
as desired. \\

\noindent When does \(\overline{(a, b)} = [a, b]\) hold? Maybe it is pertinent to ask when does it not hold? The equality does not hold if and only if \((a, b)\) is already closed. That means \(X \setminus (a, b) = (-\infty, a] \cup [b, \infty) \) is open. I'm not sure, maybe \(X\) needs to be countable, finite?

\section*{2-13}

Let \(X\) be a second countable topological space and fix a collection of disjoint open subsets \(\mathcal{S}\), i.e.
\begin{align*}
    \mathcal{S} = \makeset{U \subset X}{U \text{ is open and for all } U, V \in \mathcal{S} \text{ it is } U \cap V = \varnothing } \text{.}
\end{align*}
We want to show \(\mathcal{S}\) is countable. If \(\mathcal{B}\) is a base for \(X\), then for any two members of the collection \(U, V \in \mathcal{S}\), we have
\begin{align*}
    U = \bigcup_{i \in \mathbb{N}} B_i \qquad V = \bigcup_{j \in \mathbb{N}} B_j \text{.}
\end{align*}
Since \(U\) and \(V\) are disjoint, \(B_i\) and \(B_j\) are also disjoint for all \(i, j \in \mathbb{N}\). Thus, any \(U \in \mathcal{S}\) is a union of base elements that is different from any other \(V \in \mathcal{S}\). \(\mathcal{B}\) is countable, therefore \(\mathcal{S}\) must also be.

\section*{2-14}
Let \(X\) be a locally Euclidean space. We show that \(X\) is first-countable. Let \(p \in X\) be a point, then since \(X\) is locally Euclidean, there is a neighbourhood \(N\) of \(p\) such that \(N\) is homeomorphic to \(\mathbb{R}^n\). Thus, we have a sequence of neighbourhoods as \(\mathbb{R}^n\) is first-countable, yada yada yada.

\noindent Let \(M\) be a metric space. I've shown that this is first-countable already.

\section*{2-15}
\subsection*{a)}
Let \(X\) be a second-countable space. We want to show that \(X\) contains a dense subset that is countable.

\chapter{New Spaces from Old}
\section*{3-1}

\chapter{Simplicial Complexes}

\section*{Exercise 5.1}

\begin{definition}[Simplex]
    Given points \(v_0, \ldots, v_k\) in general position in \(\mathbb{R}^n\), simplex spanned by them is the set of all points in \(\mathbb{R}^n\) of the form:
    \begin{align*}
        \sum_{i=0}^k t_i v_i \qquad \text{where } 0 \leq t_1 \leq 1 \text{ and } \sum_{i=0}^k t_i = 1 \text{.} 
    \end{align*}
\end{definition}

\begin{definition}[Convex Hull]
    Let \(X\) be a subset of \(\mathbb{R}^n\), then the convex hull of \(X\) is the intersection of all convex sets containing \(X\).
\end{definition}

\begin{definition}[Convex Set]
    A subset \(X\) of \(\mathbb{R}^n\) is convex if for all \(x, y \in X\) and for all \(t \in [0, 1]\) it is
    \begin{align*}
        (1 - t)x + ty \in X \text{.}
    \end{align*}
\end{definition}

\begin{proof}
    Let \(\sigma\) be a simplex and denote its vertices by \(v_0, \ldots, v_k\). Fix a point \(p \in \sigma\), then by definition
    \begin{align*}
        p = \sum_{i=0}^k t_i v_i
    \end{align*}
    for some \(t \in [0, 1]\) and \(\sum_{i=0}^k t_i = 1\). We show that \(p\) lies in any convex set containing \(v_0, \ldots, v_k\) by induction. In the initial case, \(p = t_i v_i = v_i\) for some \(i\) which clearly lies in all convex set containing \(v_0, \ldots, v_k\). Consider the case where \(n + 1\) number of \(t_0, \ldots, t_k\) is nonzero. Without loss of generality, we may reorder the indecies for notational ease and yield
    \begin{align*}
        p = \sum_{i=0}^{n+1} t_i v_i = t_{n+1} v_{n+1} + \sum_{i=0}^{n} t_i v_i \text{.}
    \end{align*}
    By induction hypothesis, \(\sum_{i=0}^{n} t_i v_i\) is contained in any convex set containing \(v_0, \ldots, v_k\).
\end{proof}

\section*{Exercise 5.2}
\subsection*{a)}

Fix two simplices \(\sigma\) and \(\tau\), and denote the set of their vertices by \(\mathrm{vert}(\sigma)\) and \(\mathrm{vert}(\tau)\) respectively. Let \(f_0: \mathrm{vert}(\sigma) \longrightarrow \mathrm{vert}(\tau)\) be any map and consider a point \(p \in \sigma\). \(p\) may be represented by a linear combination of the vertices, thus
\begin{align*}
    p = \sum_{i=1}^k v_i
\end{align*}
which allows us to define
\begin{align*}
    f(p) := \sum_{i=1}^k f_0(v_i) \text{.}
\end{align*}
Since a simplex is the convex hull of its vertices, \(f(p)\) lies in \(\tau\).

Unsure, but should be the right direction.

\subsection*{b)}

sounds reasonable

\subsection*{c)}

more suprising

\section*{Exercise 5.3}

\subsection*{Example 5.2 a)}

Let \(K\) be the collection of a \(n\)-simplex \(\sigma\) and its faces. Trivially, the faces of \(\sigma\) lies in \(K\), and the faces of its faces are just faces of \(\sigma\), and thus are also members of \(K\). The any intersection of \(\sigma\) and its faces are again faces or empty. (IF I UNDERSTOOD THIS CORRECTLY) since \(K\) is already finite the third condition also applies.

\section*{Exercise 5.4}

\section*{Exercise 5.5}

\begin{defbox}
    \begin{definition}[Simplex]
        Given points \(v_0, \ldots, v_k\) in general position in \(\mathbb{R}^n\), simplex spanned by them is the set of all points in \(\mathbb{R}^n\) of the form:
        \begin{align*}
            \sum_{i=0}^k t_i v_i \qquad \text{where } 0 \leq t_1 \leq 1 \text{ and } \sum_{i=0}^k t_i = 1 \text{.} 
        \end{align*}
    \end{definition}
\end{defbox}

\begin{defbox}
    \begin{definition}[Euclidean Simplicial Complex]
        A Euclidean simplicial complex is a collection \(K\) of simplices in some Euclidean space \(\mathbb{R}^n\) satisfying the following conditions:
        \begin{enumerate}
            \item If \(\sigma \in K\), then every face of \(\sigma\) is in \(K\).
            \item The intersection of any two simplices in \(K\) is either empty or a face of each.
            \item Every point in a simplex of \(K\) has a neighborhood that intersects at most finitely many simplices of \(K\).
        \end{enumerate}
    \end{definition}
\end{defbox}

\begin{defbox}
    \begin{definition}[Abstract Complex]
        A collection \(\mathcal{K}\) of nonempty finite sets is called abstract complex if for all \(\sigma \in \mathcal{K}\) nonempty subsets of \(\sigma\) also lie in \(\mathcal{K}\).
    \end{definition}
\end{defbox}

\begin{defbox}
    \begin{definition}[Vertex Scheme]
        Let \(K\) be an Euclidean simplicial complex and set
        \begin{align*}
            \mathcal{K} := \makeset{\{v_0, \ldots, v_k\} \subset K^{(0)}}{v_0, \ldots, v_k \text{ are vertices of some simplex of K}}
        \end{align*}
        where \(K^{(0)}\) is the \(0\)-skeleton of \(K\), i.e. it is the set of all simplices of dimension \(0\) in \(K\). In this case, \(\mathcal{K}\) is an abstract simplicial complex and it is called the vertex scheme of \(K\).
    \end{definition}
\end{defbox}

Every finite abstract complex is the vertex scheme of a Euclidean simplicial complex.

\begin{proof}
    Let \(\mathcal{K}\) be a finite abstract complex.
\end{proof}

\section*{Exercise 5.6}


\chapter{Homotopy and the Fundamental Group}

\section*{Exercise 7.1.}

\subsection*{a)}
We simply have \(f \sim g \iff f \cdot g^{-1} \sim g \cdot g^{-1} \iff f \cdot g^{-1} \sim c_P\).

\subsection*{b)}
``\(\Rightarrow\)'': Let \(X\) be simply connected and fix two paths \(f\) and \(g\) with the same initial, say \(p\), and terminal point. Since \(X\) is simply connected, \(\pi(X, p)\) is trivial, and therefore, \(f \cdot g^{-1}\) is homotopic to \(c_p\). By \(a)\), \(f \sim g\) as desired. \\

\noindent ``\(\Leftarrow\)'': If any two paths in \(X\) with the same initial and terminal point are path homotopic, then by similar reason as above, \(\pi_1(X)\) is trivial.

\section*{7.4}
First, \(f \cdot g\) and \(h \cdot k\) are well-defined because \(f(1) = F(1, 0) = g(0)\) and \(h(1) = F(0, 1) = k(0)\). The initial and the terminal point of \(f \cdot g\) and \(h \cdot k\) is also the same since \(f(0) = F(0, 0) = h(0)\) and \(g(1) = F(1, 1) = k(1)\).

\section*{7.5}
First, \(\varphi \circ f_0\) and \(\varphi \circ f_1\) have the same initial and terminal points because \(\varphi(f_0(0)) = \varphi(f_1(0))\) and \(\varphi(f_0(1)) = \varphi(f_1(1))\). Let \(F: I \times I \longrightarrow X\) be the homotopy between \(f_0\) and \(f_1\). Set \(H: I \times I \longrightarrow Y\) by \(H(s, t) = \varphi(F(s, t))\). Then, \(H\) is continuous because it is a composition of continuous functions and
\begin{align*}
    H(s, 0) = \varphi(F(s, 0)) = \varphi(f_0(s)) \qquad H(s, 1) = \varphi(F(s, 1)) = \varphi(f_1(s)) \text{.}
\end{align*}

\end{document}