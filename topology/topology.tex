\documentclass[a4paper]{book}
\title{Commutative Ring Theory}
\author{Kei Thoma}


% ---------------------------------------------------------------------
% P A C K A G E S
% ---------------------------------------------------------------------

% typography and formatting
\usepackage[english]{babel}
\usepackage[utf8]{inputenc}
\usepackage{geometry}
\usepackage{exsheets}
\usepackage{environ}
\usepackage{graphicx}
\usepackage{cutwin}
\usepackage{pifont}

% mathematics
\usepackage{xfrac}  
\usepackage{amsthm} % for theorems, and definitions
\usepackage{amssymb}
\usepackage{amsmath}
\usepackage{textcomp}
\usepackage{mathtools}
% \usepackage{MnSymbol} % for \cupdot

% extra
\usepackage{xcolor}
\usepackage{tikz}

% ---------------------------------------------------------------------
% S E T T I N G
% ---------------------------------------------------------------------

%maybe delete later, for colorbox
\usepackage{tcolorbox}
\newtcolorbox{defbox}{colback=blue!5!white,colframe=blue!75!black}
\newtcolorbox{defboxlight}{colback=cyan!5!white,colframe=cyan!75!black}
\newtcolorbox{thmbox}{colback=orange!5!white,colframe=orange!75!black}
\newtcolorbox{rembox}{colback=purple!5!white,colframe=purple!75!black}
\newtcolorbox{exmbox}{colback=gray!5!white,colframe=gray!75!black}
\newtcolorbox{intbox}{colback=violet!5!white,colframe=violet!75!black}

% typography and formatting
\geometry{margin=3cm}

\SetupExSheets{
  counter-format = ch.qu,
  counter-within = chapter,
  question/print = true,
  solution/print = true,
}

% mathematics
\newcounter{global}

\theoremstyle{definition}
\newtheorem{definition}{Definition}[]
\newtheorem{example}{Example}[definition]

\newtheorem{theorem}[definition]{Theorem}
\newtheorem{corollary}{Corollary}
\newtheorem{lemma}[definition]{Lemma}
\newtheorem{proposition}[definition]{Proposition}

\newtheorem*{remark}{Remark}
\newtheorem*{intuition}{Intuition}

% extra
\definecolor{mathif}{HTML}{0000A0} % for conditions
\definecolor{maththen}{HTML}{CC5500} % for consequences
\definecolor{mathrem}{HTML}{8b008b} % for notes
\definecolor{mathobj}{HTML}{008800}

\usetikzlibrary{positioning}
\usetikzlibrary{shapes.geometric, arrows}

% ---------------------------------------------------------------------
% C O M M A N D S
% ---------------------------------------------------------------------

\newcommand{\norm}[1]{\left\lVert#1\right\rVert}
\newcommand{\rank}{\text{rank}}
\newcommand{\Vol}{\text{Vol}}

\newcommand{\set}[1]{\left\{\, #1 \,\right\}}
\newcommand{\makeset}[2]{\left\{\, #1 \mid #2 \,\right\}}

\newcommand*\diff{\mathop{}\!\mathrm{d}}
\newcommand*\Diff{\mathop{}\!\mathrm{D}}

\newcommand\restr[2]{{% we make the whole thing an ordinary symbol
  \left.\kern-\nulldelimiterspace % automatically resize the bar with \right
  #1 % the function
  \vphantom{\big|} % pretend it's a little taller at normal size
  \right|_{#2} % this is the delimiter
  }}

% ---------------------------------------------------------------------
% R E N D E R
% ---------------------------------------------------------------------

\newif\ifshowproof
\showprooftrue

\NewEnviron{Proof}{%
    \ifshowproof%
        \begin{proof}%
            \BODY
        \end{proof}%
    \fi%
}%

\begin{document}
\maketitle
\tableofcontents
\chapter{Introduction and Motivation}
\chapter{Metric Spaces}
\chapter{Topological Spaces}
\chapter{Products, Sequential Continuity, and Nets}

\begin{thmbox}
    \begin{lemma}[Lemma 4.15]
        In any space \(X\), a subset \(A \subset X\) is open if and only if every point \(x \in A\) has a neighbourhood \(\mathcal{V} \subset X\) that is contained in \(A\).
    \end{lemma}
\end{thmbox}
\begin{proof}
    ``\(\Rightarrow\)'': If \(A\) is open, then \(A\) itself can be taken as the desired neighbourhood of every \(x \in A\).
    ``\(\Leftarrow\)'': Let every point \(x \in A\) have a neighbourhood \(\mathcal{V} \subset X\) that is contained in \(A\). Denote the open sets of these neighbourhoods by \(\mathcal{U}_x\). Then, \(A\) is the union of all these open sets \(\mathcal{U}_x\) and thus open.
\end{proof}

\begin{thmbox}
    \begin{lemma}[Lemma 4.16]
        In any first-countable topological space \(X\), a subspace \(A \subset X\) is not open if and only if there exists a point \(x \in A\) and a sequence \(x_n \in X \setminus A\) such that \(x_n \rightarrow x\).
    \end{lemma}
\end{thmbox}
\begin{proof}
    ``\(\Leftarrow\)'': (Proof by contraposition.) If \(A \subset X\) is open, then for every \(x \in A\) and sequence \(x_n \in X\) converging to \(x\), we cannot have \(x_n \in X \setminus A\) for all \(n\) since \(A\) is a neighbourhood of \(x\). This is true so far for all topological spaces, with or without first-countability axiom, but the latter will be needed to prove the converse.

    ``\(\Rightarrow\)'': So suppose now that \(A \subset X\) is not open, which by Lemma 4.15, means there exists a point \(x \in A\) such that no neighbourhood \(\mathcal{V} \subset X\) of \(x\) is contained in \(A\). Fix a countable neighbourhood base \(\mathcal{U}_1, \mathcal{U}_2, \ldots\) for \(x\). XXX

    Observe that since none of the neighbourhoods \(\mathcal{U}_n\) can be contained in \(A\), there exists a sequence of points
    \begin{align*}
        x_n \in \mathcal{U}_n \text{ such that } x_n \not\in A \text{.}
    \end{align*}
    This sequence converges to \(x\) since every neighbourhood \(\mathcal{V} \subset X\) of \(x\) contains one of \(\mathcal{U}_N\), implying that for all \(n \geq N\),
    \begin{align*}
        x_n \in \mathcal{U}_n \subset \mathcal{U}_n \subset \mathcal{V} \text{.}
    \end{align*}
\end{proof}


\begin{defbox}
    \begin{definition}
        A {\color{maththen}directed set} \((I, \prec)\) consists of a set \(I\) with a partial order \(\prec\) such that for every pair \(\alpha, \beta \in I\), there exists an element \(\gamma \in I\) with \(\gamma \prec \alpha\) and \(\gamma \prec \beta\).
    \end{definition}
\end{defbox}

\begin{defbox}
    \begin{definition}
        Given a space \(X\), a net \(\{x_\alpha\}_{\alpha \in I}\) in \(X\) is a function \(I \longrightarrow X: \alpha \mapsto x_\alpha\) where \((I, \prec)\) is a directed set.
    \end{definition}
\end{defbox}

\chapter{Compactness}

\begin{defbox}
    \begin{definition}
        A {\color{mathobj}subset} \(A \subset X\) is {\color{maththen}compact} if every open cover of \(A\) has a finite subcover, i.e. given an arbitary open cover \(\{\mathcal{U}_\alpha\}_{\alpha \in I}\) of \(A\), one can always find a finite subset \(\{\alpha_1, \ldots, \alpha_N\} \subset I\) such that \(A \subset \mathcal{U}_{\alpha_1} \cup \cdots \cup \mathcal{U}_{\alpha_N}\). We say that \(X\) itself is a {\color{maththen}compact space} if \(X\) is compact subset of itself.
    \end{definition}
\end{defbox}
\end{document}