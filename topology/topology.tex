\documentclass[a4paper]{book}
\title{Topology}
\author{K}


% ---------------------------------------------------------------------
% P A C K A G E S
% ---------------------------------------------------------------------

% typography and formatting
\usepackage[english]{babel}
\usepackage[utf8]{inputenc}
\usepackage{geometry}
\usepackage{exsheets}
\usepackage{environ}

% mathematics
\usepackage{amsthm} % for theorems, and definitions
\usepackage{amssymb}
\usepackage{amsmath}
\usepackage{textcomp}
%\usepackage{MnSymbol} % for \cupdot

% extra
\usepackage{xcolor}
\usepackage{tikz}

% ---------------------------------------------------------------------
% S E T T I N G
% ---------------------------------------------------------------------

% typography and formatting
\geometry{margin=3cm}

\SetupExSheets{
  counter-format = ch.qu,
  counter-within = chapter,
  question/print = true,
  solution/print = true,
}

% mathematics
\theoremstyle{definition}
\newtheorem{definition}{Definition}[chapter]
\newtheorem{example}{Example}[definition]

\newtheorem{theorem}{Theorem}[definition]
\newtheorem{corollary}{Corollary}
\newtheorem{lemma}{Lemma}[definition]
\newtheorem{proposition}{Proposition}[definition]

\newtheorem*{remark}{Remark}

% extra
\definecolor{mathif}{HTML}{0000A0} % for conditions
\definecolor{maththen}{HTML}{CC5500} % for consequences
\definecolor{mathrem}{HTML}{8b008b} % for notes

\usetikzlibrary{positioning}
\usetikzlibrary{shapes.geometric, arrows}

% ---------------------------------------------------------------------
% C O M M A N D S
% ---------------------------------------------------------------------

\newcommand{\norm}[1]{\left\lVert#1\right\rVert}
\newcommand{\rank}{\text{rank}}
\newcommand{\Vol}{\text{Vol}}

\newcommand{\set}[1]{\left\{\, #1 \,\right\}}
\newcommand{\makeset}[2]{\left\{\, #1 \mid #2 \,\right\}}


\newcommand*\diff{\mathop{}\!\mathrm{d}}
\newcommand*\Diff{\mathop{}\!\mathrm{D}}

\newcommand\restr[2]{{% we make the whole thing an ordinary symbol
  \left.\kern-\nulldelimiterspace % automatically resize the bar with \right
  #1 % the function
  \vphantom{\big|} % pretend it's a little taller at normal size
  \right|_{#2} % this is the delimiter
  }}

% ---------------------------------------------------------------------
% R E N D E R
% ---------------------------------------------------------------------

\newif\ifshowproof
\showprooftrue

\NewEnviron{Proof}{%
    \ifshowproof%
        \begin{proof}%
            \BODY
        \end{proof}%
    \fi%
}%

\begin{document}
\maketitle
\tableofcontents
%%%%%%%%%%%%%%%%%%%%%%%%%%%%%%%%%%%%%%%%%%%%%%%%%%%%%%%%%%%%%%%%%%%%%%%%%%%%%%%
\begin{definition}[Topological Space]
    A {\color{maththen}topological space} is an ordered pair \((X, \tau)\), where \(X\) is a {\color{mathif}set} and \(\tau\) is a {\color{mathif}collection of subsets} that satisfies the following axioms.
    \begin{enumerate}
        \item The {\color{mathif}empty set} \(\varnothing\) and the {\color{mathif}whole set} \(X\) belongs to \(\tau\).
        \item Any {\color{mathif}arbitary union} of members of \(\tau\) belongs to \(\tau\).
        \item The {\color{mathif}intersection of finite number} of members of \(\tau\) belongs to \(\tau\).
    \end{enumerate}
    The {\color{mathif}collection} \(\tau\) is called a {\color{maththen}topology} on \(X\) and the {\color{mathif}elements} of \(\tau\) are called {\color{maththen}open sets}. A {\color{mathif}subset} \(A \subset X\) is said to be {\color{maththen}closed} if its {\color{mathif}complement} \(X \setminus A\) is {\color{mathif}open}.
\end{definition}

%TODO: There are other equivalent definitions

\begin{example}
    Let \(X\) be a set.
    \begin{enumerate}
        \item \(\tau = \mathcal{P}(X)\) is called the {\color{maththen}discrete topology}. In this case, \((X, \tau)\) is called the {\color{maththen}discrete space}. It is the finest topology. (One can define an ordering of topologies.)
        \item \(\tau = \{\varnothing, \mathcal{P}(X)\}\) is called the {\color{maththen}trivial topology}.
        \item Let \((X, d)\) be a {\color{mathif}metric space}. Set
        \begin{equation}
            \tau_d := \makeset{U \in X}{U \text{ is a open subset in the metric space } (X, d)} \text{.}
        \end{equation}
        Recall that \(U\) being an open subset in the metric space \((X, d)\) means that for all \(x \in U\) there is an \(r > 0\) such that \(B_d(x, r)\) is contained in \(U\).

        Here, \(\tau\) is a topology. In other words, a metric induces a topology.

        (Proof as homework.)
        \item The Zariski-topology.
    \end{enumerate}

    \begin{definition}[Continuous Maps]
        Let \((X, \tau_X)\) and \((Y, \tau_Y)\) be {\color{mathif}topological spaces}. A {\color{mathif}map} \(f: X \longrightarrow Y\) is said to be {\color{maththen}continuous} if the preimage of an open subset is again open, i.e.
        \begin{equation}
            \text{for all } U \in \tau_Y \text{ it is } f^{-1}(U) \in \tau_X \text{.}
        \end{equation}
    \end{definition}

    \begin{lemma}
        The different definitions of continuity in a topological space and a metric space are equivalent, i.e. if \(X\) and \(Y\) are metric spaces, then \(f: X \longrightarrow Y\) is \(\epsilon\)-\(\delta\)-continuous if and only if \(f\) is continuous.
    \end{lemma}

    \begin{definition}[Homeomorphism]
        Let \(X\) and \(Y\) be topological spaces. A map \(f: X \longrightarrow Y\) is a homeomorphism if it has the following properties.

        \begin{enumerate}
            \item \(f\) is {\color{mathif}bijective}.
            \item \(f\) is {\color{mathif}continuous}.
            \item The inverse map \(f^{-1}\) is {\color{mathif}continuous}.
        \end{enumerate}

        If such function exists, \(X\) and \(Y\) are said to be {\color{maththen}homeomorphic}.

        We denote the set of all homeomorphisms from \(X\) to \(Y\) by \(\mathrm{Homeo}(X, Y)\). The set of all homeomorphisms of \(X\) to itself \(\mathrm{Homeo}(X)\) is a group with composition as its operation.
    \end{definition}
\end{example}
\end{document}