\documentclass[a4paper]{book}
\title{Topology}
\author{K}


% ---------------------------------------------------------------------
% P A C K A G E S
% ---------------------------------------------------------------------

% typography and formatting
\usepackage[english]{babel}
\usepackage[utf8]{inputenc}
\usepackage{geometry}
\usepackage{exsheets}
\usepackage{environ}

% mathematics
\usepackage{amsthm} % for theorems, and definitions
\usepackage{amssymb}
\usepackage{amsmath}
\usepackage{textcomp}
% \usepackage{MnSymbol} % for \cupdot

% extra
\usepackage{xcolor}
\usepackage{tikz}

% ---------------------------------------------------------------------
% S E T T I N G
% ---------------------------------------------------------------------

%maybe delete later, for colorbox
\usepackage{tcolorbox}
\newtcolorbox{defbox}{colback=orange!5!white,colframe=orange!75!black}

% typography and formatting
\geometry{margin=3cm}

\SetupExSheets{
  counter-format = ch.qu,
  counter-within = chapter,
  question/print = true,
  solution/print = true,
}

% mathematics
\theoremstyle{definition}
\newtheorem{definition}{Definition}[chapter]
\newtheorem{example}{Example}[definition]

\newtheorem{theorem}{Theorem}[definition]
\newtheorem{corollary}{Corollary}
\newtheorem{lemma}{Lemma}[definition]
\newtheorem{proposition}{Proposition}[definition]

\newtheorem*{remark}{Remark}

% extra
\definecolor{mathif}{HTML}{0000A0} % for conditions
\definecolor{maththen}{HTML}{CC5500} % for consequences
\definecolor{mathrem}{HTML}{8b008b} % for notes
\definecolor{mathobj}{HTML}{008800}

\usetikzlibrary{positioning}
\usetikzlibrary{shapes.geometric, arrows}

% ---------------------------------------------------------------------
% C O M M A N D S
% ---------------------------------------------------------------------

\newcommand{\norm}[1]{\left\lVert#1\right\rVert}
\newcommand{\rank}{\text{rank}}
\newcommand{\Vol}{\text{Vol}}

\newcommand{\set}[1]{\left\{\, #1 \,\right\}}
\newcommand{\makeset}[2]{\left\{\, #1 \mid #2 \,\right\}}

\newcommand*\diff{\mathop{}\!\mathrm{d}}
\newcommand*\Diff{\mathop{}\!\mathrm{D}}

\newcommand\restr[2]{{% we make the whole thing an ordinary symbol
  \left.\kern-\nulldelimiterspace % automatically resize the bar with \right
  #1 % the function
  \vphantom{\big|} % pretend it's a little taller at normal size
  \right|_{#2} % this is the delimiter
  }}

% ---------------------------------------------------------------------
% R E N D E R
% ---------------------------------------------------------------------

\newif\ifshowproof
\showprooftrue

\NewEnviron{Proof}{%
    \ifshowproof%
        \begin{proof}%
            \BODY
        \end{proof}%
    \fi%
}%

\begin{document}
\maketitle
\tableofcontents
%%%%%%%%%%%%%%%%%%%%%%%%%%%%%%%%%%%%%%%%%%%%%%%%%%%%%%%%%%%%%%%%%%%%%%%%%%%%%%%
\chapter{Topological Space}
\begin{defbox}
    \begin{definition}[Topological Space]
        A {\color{maththen}topological space} is an {\color{mathobj}ordered pair} \((X, \tau)\), where \(X\) is a {\color{mathif}set} and \(\tau\) is a {\color{mathif}collection of subsets} that satisfies the following {\color{mathrem}axioms}.
        \begin{enumerate}
            \item The {\color{mathif}empty set} \(\varnothing\) and the {\color{mathif}entire set} \(X\) belongs to \(\tau\).
            \item Any \textbf{arbitary} {\color{mathif}union} of members of \(\tau\) belongs to \(\tau\).
            \item The {\color{mathif}intersection} of \textbf{finite number} of members of \(\tau\) belongs to \(\tau\).
        \end{enumerate}
        The {\color{mathobj}collection} \(\tau\) is called a {\color{maththen}topology} on \(X\) and the {\color{mathobj}elements} of \(\tau\) are called {\color{maththen}open sets}. A {\color{mathobj}subset} \(A \subset X\) is said to be {\color{maththen}closed} if its {\color{mathif}complement} \(X \setminus A\) is {\color{mathif}open}.
    \end{definition}
\end{defbox}
%TODO: There are other equivalent definitions
%TODO: write something about the notation

\begin{example}
    Let \(X\) be a {\color{mathif}set}.
    \begin{enumerate}
        \item \(\tau = \mathcal{P}(X)\) is called the {\color{maththen}discrete topology}. In this case, \((X, \tau)\) is called the {\color{maththen}discrete space}. It is the {\color{mathrem}finest topology} that can be defined on a set. (The set of all possible topologies on a given set forms a partially ordered set.)
        \item \(\tau = \{\varnothing, \mathcal{P}(X)\}\) is called the {\color{maththen}trivial topology}.
        \item Let \((X, d)\) be a {\color{mathif}metric space}. Set
        \begin{equation}
            \tau_d := \makeset{U \in X}{U \text{ is a open subset in the metric space } (X, d)} \text{.}
        \end{equation}
        Recall that \(U\) being an open subset in the metric space \((X, d)\) means that for all \(x \in U\) there is an \(r > 0\) such that \(B_d(x, r)\) is contained in \(U\).

        Here, \(\tau\) is a topology. In other words, a metric induces a topology.

        (Proof as homework.)
        \item The Zariski-topology.
    \end{enumerate}
\end{example}
\begin{defbox}
    \begin{definition}[Continuous Maps]
        Let \((X, \tau_X)\) and \((Y, \tau_Y)\) be {\color{mathif}topological spaces}. A {\color{mathif}map} \(f: X \longrightarrow Y\) is said to be {\color{maththen}continuous} if the preimage of an open subset is again open, i.e.
        \begin{equation}
            \text{for all } U \in \tau_Y \text{ it is } f^{-1}(U) \in \tau_X \text{.}
        \end{equation}
    \end{definition}
\end{defbox}

    \begin{lemma}
        The different definitions of continuity in a topological space and a metric space are equivalent, i.e. if \(X\) and \(Y\) are metric spaces, then \(f: X \longrightarrow Y\) is \(\epsilon\)-\(\delta\)-continuous if and only if \(f\) is continuous.
    \end{lemma}

\begin{defbox}
    \begin{definition}[Homeomorphism]
        Let \(X\) and \(Y\) be {\color{mathif}topological spaces}.
        \begin{enumerate}
            \item A {\color{mathobj}map} \(f: X \longrightarrow Y\) is a {\color{maththen}homeomorphism} if it has the following properties.

            \begin{enumerate}
                \item \(f\) is {\color{mathif}bijective}.
                \item \(f\) and the {\color{mathif}inverse map} \(f^{-1}\) is {\color{mathif}continuous}.
            \end{enumerate}

            \item Two topological spaces \(X\) and \(Y\) are said to be {\color{maththen}homeomorphic} if a homeomorphism exists.

            \item We denote the set of all homeomorphisms from \(X\) to \(Y\) by \(\mathrm{Homeo}(X, Y)\). If \(Y = X\) we also write \(\mathrm{Homeo}(X)\).
        \end{enumerate}
    \end{definition}
\end{defbox}
\begin{remark}
    The set of all homeomorphisms of \(X\) to itself \(\mathrm{Homeo}(X)\) is a group with composition as its operation.
\end{remark}

\begin{defbox}
    \begin{definition}
        Let \((X, \tau)\) a {\color{mathif}topological space}.
        \begin{enumerate}
            \item \(\mathcal{B} \subset \mathcal{O}\) is a {\color{maththen}basis} of the topology, if any member of \(\mathcal{O}\) is the {\color{mathif}union of subsets} from \(\mathcal{B}\).
            \item \(\mathcal{S} \subset \mathcal{O}\) is a {\color{maththen}subbasis} of the topology, if any member of \(\mathcal{O}\) is the {\color{mathif}union of finite intersections of subsets} from \(\mathcal{S}\).
        \end{enumerate}
        We say that \(\mathcal{B}\) and \(\mathcal{S}\) {\color{maththen}generates} \(\mathcal{O}\) and write \(\overline{\mathcal{S}} = \overline{\mathcal{B}} = \mathcal{O}\).
    \end{definition}
\end{defbox}

    \begin{lemma}
        Let \(\mathcal{S} \subset \mathcal{P}(X)\), then there exists exactly one topology \(\mathcal{O} \subset \mathcal{P}(X)\) of \(X\) such that
        \begin{enumerate}
            \item \(\mathcal{S} \subset \mathcal{O}\)
        \end{enumerate}
    \end{lemma}
    %%% lecture 2 missing
    Note about product topology: \(\makeset{U \times V}{U \in \mathcal{O}_X, V \in \mathcal{O}_Y}\); often \(W \subset X \times Y \iff \forall (x, y) \in W \exists U_X \in \mathcal{O}_X, V_Y \in \mathcal{O}_Y, x \in U_X, y \in V_Y\)

    \chapter{Connected Spaces and Sets}
    \begin{defbox}
        \begin{definition}[Def 9]
            A {\color{mathobj}topological space} \(X\) is said to be {\color{maththen}connected}, if one of the following {\color{mathrem}equivalent} conditions is met.
            \begin{enumerate}
                \item \(X\) is \textbf{not} a {\color{mathif}union} of two {\color{mathif}disjoint} sets.
                \item The \textbf{only} {\color{mathif}subsets} of \(X\) that are \textbf{both} {\color{mathif}open} and {\color{mathif}closed} ({\color{mathrem}clopen}) are the emptyset \(\varnothing\) and the entire set \(X\).
                % commented out as the other 3 conditions are not mentioned in the lecture
                % \item The \textbf{only} {\color{mathif}subsets} of \(X\) with empty {\color{mathif}boundary} are the emptyset \(\varnothing\) and the entire set \(X\).
                % \item All {\color{mathif}continuous} maps from \(X\) to the two point space \(\{0, 1\}\) endowed with the {\color{mathif}discrete} topology is {\color{mathif}constant}. 
            \end{enumerate}
        \end{definition}
    \end{defbox}
    % proof missing

    \begin{lemma}
        Any {\color{mathif}interval} \(I \subset \mathbb{R}\) is {\color{maththen}connected}.
    \end{lemma}

    \begin{proof}
        Let \(I = A \cup B\) with \(A\) and \(B\) being nonempty disjoint sets in \(\mathbb{R}\) that are open, and let \(a \in A\) and \(b \in B\). Without loss of generality, assume \(a < b\). If we set
        \begin{align}
            s := \inf \makeset{x \in B}{a < x}
        \end{align}
        then \(s \in [a, b] \subset I\) because \(I\) is an interval. %proof missing
    \end{proof}

    \begin{example}
        The general linear group \(\mathrm{GL}_n(K)\) for a field \(K\) and \(n \in \mathbb{N}\) is not connected for \(K = \mathbb{R}\) and \(K = \mathbb{C}\).
    \end{example}

    \begin{defbox}
        \begin{definition}
            A connected component of a topological space is a maximally connected subset \(X_0 \subseteq X\), i.e. \(X_0\) connected and for all \(X_0 \subsetneq X_1\) then \(X_1\) is not connected.
        \end{definition}
    \end{defbox}

    \begin{remark}
        Let \(f: X \longrightarrow Y\) be continuous and \(X\) be connected, then \(f(X) \subset Y\) is connected.
    \end{remark}
    \begin{proof}
        Let \(f(X) = A \sqcup B\) with \(A\) and \(B\) being two open disjoint sets. \(f^{-1}(A)\) and \(f^{-1}(B)\) are open since \(f\) is continuous. We also have \(f^{-1}(A) \cap f^{-1}B = f^{-1}(A \cap B) = \varnothing\) so \(f^{-1}(A) = \varnothing\) or \(f^{-1}(B) = \varnothing\), so \(A = \varnothing\) or \(B = \varnothing\) and we are done.
    \end{proof}
    \begin{proposition}
        Connected components are closed subsets.
    \end{proposition}
    \begin{proof}
        % proof missing
    \end{proof}
    \begin{example}
        For \(\mathbb{Q} \subset \mathbb{R}\) the connected components are points and those are not open.
    \end{example}
    \begin{lemma}[Lemma 11]
        Let \(X\) be connected and \(f: X \longrightarrow Y\) and locally constant, i.e. for all \(x \in X\) there exists a \(U_x \in \mathcal{O}_X\), \(x \in U_x\) such that \(f\) restricted on \(U_x\) is identical to \(f(x)\)., then \(f\) is constant.
    \end{lemma}
    \begin{proof}
        Locally constant implies continuous with regards to the discrete topology on \(Y\). Let \(x \in X\), \(X = f^{-1}(f(x)) \cup f^{-1}(Y \setminus \{f(x)\})\) is a disjoint union and since \(X\) is connected \(f^{-1}(Y \setminus \{f(x)\}) = \varnothing\). Conclude \(f\) is identical to \(f(x)\).
    \end{proof}

    \textbf{Application:} \(f: X \longrightarrow \{0, 1\}\), \(X\) is connected, \(f\) locally constant, there is a \(x \in X\) such that \(f(x) = 1\), then \(f\) is identical to \(1\).

    \begin{defbox}
        \begin{definition}
            \(X\) is said to be {\color{maththen}path connected}, if for every pair of points \(x\) and \(x_0\) in \(X\) there is a continuous map (called path) \(\gamma: [0, 1] \longrightarrow X\) with \(\gamma(0) = x_0\) and \(\gamma(1) = x\).
        \end{definition}
    \end{defbox}

    \begin{lemma}
        If \(X\) is path connected, then it is also connected.
    \end{lemma}
    \begin{proof}
        Let \(A\) and \(B\) two disjoint open sets such that \(A \sqcup B = X\), and let \(a \in A\) and \(b \in B\). Let \(\gamma: [0, 1] \longrightarrow X\) be continuous path with \(\gamma(0) = x_0\) and \(\gamma(1) = x_1\). We have that \(\gamma^{-1}\)
    \end{proof}

    \begin{remark}
        The converse statement is not true in general.
    \end{remark}

    \begin{example}
        \(X = \makeset{(x, \sin(\frac{1}{x}))}{x > 0} \cup \{0\} \times [-1, 1] \subset \mathbb{R}^2\) is connected but not path connected.
    \end{example}
    \begin{proof}
        Homework
    \end{proof}
    \begin{remark}
        missing
    \end{remark}

    \chapter{Trennungsaxiome}
    Literature: Groessere Liste in Sten, Seibeck

    \begin{defbox}
        \begin{definition}
            \((X, \tau)\) fullfills
            \begin{enumerate}
                \item For all \(x \in X\) and \(y \in X\) with \(x \neq y\) there is a subset \(U \in X\) open such that \(y \in U\) but \(x \not\in U\).
                \item Hausdorff
            \end{enumerate}
        \end{definition}
    \end{defbox}
    \begin{lemma}
        \begin{enumerate}
            \item \(X\) is from type 1 if and only if \(\{x\}\) is closed.
        \end{enumerate}
    \end{lemma}
    \begin{remark}
        The type 1 and type 2 properties are inherited to subspaces, topological sums and products.
        Metric spaces are from type 2.
    \end{remark}
\end{document}