\documentclass[a4paper]{book}
\title{Commutative Ring Theory}
\author{Kei Thoma}


% ---------------------------------------------------------------------
% P A C K A G E S
% ---------------------------------------------------------------------

% typography and formatting
\usepackage[english]{babel}
\usepackage[utf8]{inputenc}
\usepackage{geometry}
\usepackage{exsheets}
\usepackage{environ}
\usepackage{graphicx}
\usepackage{cutwin}
\usepackage{pifont}

% mathematics
\usepackage{xfrac}  
\usepackage{amsthm} % for theorems, and definitions
\usepackage{amssymb}
\usepackage{amsmath}
\usepackage{textcomp}
\usepackage{mathtools}
% \usepackage{MnSymbol} % for \cupdot

% extra
\usepackage{xcolor}
\usepackage{tikz}

% ---------------------------------------------------------------------
% S E T T I N G
% ---------------------------------------------------------------------

%maybe delete later, for colorbox
\usepackage{tcolorbox}
\newtcolorbox{defbox}{colback=blue!5!white,colframe=blue!75!black}
\newtcolorbox{defboxlight}{colback=cyan!5!white,colframe=cyan!75!black}
\newtcolorbox{thmbox}{colback=orange!5!white,colframe=orange!75!black}
\newtcolorbox{rembox}{colback=purple!5!white,colframe=purple!75!black}
\newtcolorbox{exmbox}{colback=gray!5!white,colframe=gray!75!black}
\newtcolorbox{intbox}{colback=violet!5!white,colframe=violet!75!black}

% typography and formatting
\geometry{margin=3cm}

\SetupExSheets{
  counter-format = ch.qu,
  counter-within = chapter,
  question/print = true,
  solution/print = true,
}

% mathematics
\newcounter{global}

\theoremstyle{definition}
\newtheorem{definition}{Definition}[]
\newtheorem{example}{Example}[definition]

\newtheorem{theorem}[definition]{Theorem}
\newtheorem{corollary}{Corollary}
\newtheorem{lemma}[definition]{Lemma}
\newtheorem{proposition}[definition]{Proposition}

\newtheorem*{remark}{Remark}
\newtheorem*{intuition}{Intuition}

% extra
\definecolor{mathif}{HTML}{0000A0} % for conditions
\definecolor{maththen}{HTML}{CC5500} % for consequences
\definecolor{mathrem}{HTML}{8b008b} % for notes
\definecolor{mathobj}{HTML}{008800}

\usetikzlibrary{positioning}
\usetikzlibrary{shapes.geometric, arrows}

% ---------------------------------------------------------------------
% C O M M A N D S
% ---------------------------------------------------------------------

\newcommand{\norm}[1]{\left\lVert#1\right\rVert}
\newcommand{\rank}{\text{rank}}
\newcommand{\Vol}{\text{Vol}}

\newcommand{\set}[1]{\left\{\, #1 \,\right\}}
\newcommand{\makeset}[2]{\left\{\, #1 \mid #2 \,\right\}}

\newcommand*\diff{\mathop{}\!\mathrm{d}}
\newcommand*\Diff{\mathop{}\!\mathrm{D}}

\newcommand\restr[2]{{% we make the whole thing an ordinary symbol
  \left.\kern-\nulldelimiterspace % automatically resize the bar with \right
  #1 % the function
  \vphantom{\big|} % pretend it's a little taller at normal size
  \right|_{#2} % this is the delimiter
  }}

% ---------------------------------------------------------------------
% R E N D E R
% ---------------------------------------------------------------------

\newif\ifshowproof
\showprooftrue

\NewEnviron{Proof}{%
    \ifshowproof%
        \begin{proof}%
            \BODY
        \end{proof}%
    \fi%
}%

\begin{document}
\maketitle
\tableofcontents
\chapter{Introduction and Motivation}
\chapter{Metric Spaces}

\begin{defbox}
    \begin{definition}[Definition 2.1]
        Metric Space. Metric.
    \end{definition}
\end{defbox}

\begin{defbox}
    \begin{definition}[Pseudometric Space]
        Unlike a metric space, points in a pseudometric space need not be distinguishable; that is, one may have \(d(x, y) = 0\) for distinct values \(x \neq y\).
    \end{definition}
\end{defbox}

\begin{defbox}
    \begin{definition}[Open Ball]
        In any metric space \((X, d)\), one can define the open ball of radius \(r > 0\) about a given point \(x \in X\) as
        \begin{align*}
            B_r(x) := \makeset{y \in X}{d(x, y) < r}
        \end{align*}
    \end{definition}
\end{defbox}

\begin{intbox}
    \begin{intuition}
        The axiom of triangle inequality in the definition of a metric might seem arbitary. But it is needed to have for example two desireable properties.
        \begin{enumerate}
            \item Open balls are open themselves.
            \item The function \(d: X \times X \longrightarrow [0, \infty)\) is continuous.
        \end{enumerate}
    \end{intuition}
\end{intbox}
\begin{proof}
    \begin{enumerate}
        \item Let \((X, d)\) be a metric space and \(B_r(x)\) be an open ball. For any \(y \in B_r(x)\) choose \(0 < \epsilon < r - d(x, y)\). 
    \end{enumerate}
\end{proof}
\begin{intbox}
    \begin{intuition}
        Without the fifth axiom in the definition of a metric, a singleton \(\{x\}\) need not be closed.
    \end{intuition}
\end{intbox}
\begin{proof}
    Let \((X, d)\) be a metric space and \(x \in X\) a point. Consider the complement \(X \setminus \{x\}\). We want to show \(X \setminus \{x\}\) is open. Indeed, for any \(y \in X \setminus \{x\}\) and for all \(\epsilon < d(x, y)\) the ball \(B_\epsilon(y)\) is contained in \(X \setminus \{x\}\).
\end{proof}

\begin{exmbox}
    \begin{example}[Pseudometric]
        
    \end{example}
\end{exmbox}

\begin{example}
    Show that for the pseudometric space \(X\) in Example 2.2, \([(0, 0)] \subset X\) is not a closed subset.
\end{example}
\begin{proof}[Solution]
    Assume \(\{[(0, 0)]\} \subset X\) is a closed subset. Then, \(X \setminus \{[(0, 0)]\}\) must be open, that is, every point in \(X \setminus \{[(0, 0)]\}\) has an open ball centered around it that is contained in \(X \setminus \{[(0, 0)]\}\). Consider the point \([(0, 1)]\). We have that \(d([0, 0], [0, 1]) = 0\), thus \(X \setminus \{[(0, 0)]\}\) cannot be open.
\end{proof}

\begin{defbox}
    \begin{definition}[Definition 2.4]
        Convergence of a sequence.
    \end{definition}
\end{defbox}

\begin{defbox}
    \begin{definition}[Definition 2.5]
        For two metric spaces \((X, d_X)\) and \((Y, d_Y)\), a map \(f: X \longrightarrow Y\) is called continuous if it satisfies any of the following equivalent conditions:
        \begin{enumerate}
            \item epsilon delta
            \item For every open subset \(\mathcal{U} \subset Y\), the preimage
            \begin{align*}
                f^{-1}(\mathcal{U}) := \makeset{x \in X}{f(x) \in \mathcal{U}}
            \end{align*}
            is an open subset.
            \item For every convergent sequence \(x_n \in X\), \(x_n \rightarrow x\) implies \(f(x_n) \rightarrow f(x)\).
        \end{enumerate}
    \end{definition}
\end{defbox}
\begin{proof}
    ``\(2. \Rightarrow 3.\)'': Let \(x_n \rightarrow x\) be a sequence and \(\mathcal{U}\) a neighbourhood of \(f(x)\).
\end{proof}
\begin{rembox}
    \begin{remark}
        The direction 3. to 2. requires the metric spaces because 2. and 3. are not equivalent in arbitary topological spaces.
    \end{remark}
\end{rembox}

\begin{defbox}
    \begin{definition}[Homeomorphism]
        
    \end{definition}
\end{defbox}

\begin{exmbox}
    \begin{example}
        Any open ball in \(\mathbb{R}^n\) with the standard Euclidean metric \(d_E\) is homeomorphic to \((\mathbb{R}^n, d_E)\). It follows that any open ball in \(\mathbb{R}^n\) are homeomorphic to each other.
    \end{example}
\end{exmbox}
\begin{proof}
    We begin by constructing a homeomorphism. Choose any continuous, increasing, bijective function \(f:[0, r) \longrightarrow [0, \infty)\) and define \(F: B_r(x) \longrightarrow \mathbb{R}^n\) by
    \begin{align*}
        F(\textbf{x}) = \textbf{x} \text{ and } F(\textbf{x} + \textbf{y}) = \textbf{x} + f(|\textbf{y}|) \frac{\textbf{y}}{|\textbf{y}|} \text{ for all } \textbf{y} \in B_r(0) \setminus \{0\} \subset \mathbb{R}^n
    \end{align*}
    \(F\) and \(F^{-1}\) is continuous and bijective.
\end{proof}

\begin{defbox}
    \begin{definition}[Compactness]
        
    \end{definition}
\end{defbox}

\begin{thmbox}
    \begin{theorem}
        If \(f: X \longrightarrow Y\) is continuous and \(K \subset X\) is compact, then so is \(f(K) \subset Y\).
    \end{theorem}
\end{thmbox}
\begin{intbox}
    \begin{intuition}
        The above theorem means that compactness is a topologically invariant property.
    \end{intuition}
\end{intbox}
\begin{proof}
    Let \(f: X \longrightarrow Y\) be continuous, \(K \subset X\) compact, and \(\{\mathcal{U}_\alpha\}_{\alpha \in I}\) be a open cover of \(f(K)\). Since \(f\) is continuous, \(f^{-1}(\mathcal{U}_\alpha)\) is open for all \(\alpha \in I\). Thus, \(\{f^{-1}(\mathcal{U})_\alpha\}_{\alpha \in I}\) is an open cover of \(K\). \(K\) was compact, so this open cover has a finite subcover \(\{f^{-1}(\mathcal{U})_\alpha\}_{\alpha \in J}\) with \(J \subset I\) finite. Then \(\{\mathcal{U}_\alpha\}_{\alpha \in J}\) is a finite subcover of \(f(K)\).
\end{proof}

\chapter{Topological Spaces}
\chapter{Products, Sequential Continuity, and Nets}

\begin{thmbox}
    \begin{lemma}[Lemma 4.15]
        In any space \(X\), a subset \(A \subset X\) is open if and only if every point \(x \in A\) has a neighbourhood \(\mathcal{V} \subset X\) that is contained in \(A\).
    \end{lemma}
\end{thmbox}
\begin{proof}
    ``\(\Rightarrow\)'': If \(A\) is open, then \(A\) itself can be taken as the desired neighbourhood of every \(x \in A\).
    ``\(\Leftarrow\)'': Let every point \(x \in A\) have a neighbourhood \(\mathcal{V} \subset X\) that is contained in \(A\). Denote the open sets of these neighbourhoods by \(\mathcal{U}_x\). Then, \(A\) is the union of all these open sets \(\mathcal{U}_x\) and thus open.
\end{proof}

\begin{thmbox}
    \begin{lemma}[Lemma 4.16]
        In any first-countable topological space \(X\), a subspace \(A \subset X\) is not open if and only if there exists a point \(x \in A\) and a sequence \(x_n \in X \setminus A\) such that \(x_n \rightarrow x\).
    \end{lemma}
\end{thmbox}
\begin{proof}
    ``\(\Leftarrow\)'': (Proof by contraposition.) If \(A \subset X\) is open, then for every \(x \in A\) and sequence \(x_n \in X\) converging to \(x\), we cannot have \(x_n \in X \setminus A\) for all \(n\) since \(A\) is a neighbourhood of \(x\). This is true so far for all topological spaces, with or without first-countability axiom, but the latter will be needed to prove the converse.

    ``\(\Rightarrow\)'': So suppose now that \(A \subset X\) is not open, which by Lemma 4.15, means there exists a point \(x \in A\) such that no neighbourhood \(\mathcal{V} \subset X\) of \(x\) is contained in \(A\). Fix a countable neighbourhood base \(\mathcal{U}_1, \mathcal{U}_2, \ldots\) for \(x\). XXX

    Observe that since none of the neighbourhoods \(\mathcal{U}_n\) can be contained in \(A\), there exists a sequence of points
    \begin{align*}
        x_n \in \mathcal{U}_n \text{ such that } x_n \not\in A \text{.}
    \end{align*}
    This sequence converges to \(x\) since every neighbourhood \(\mathcal{V} \subset X\) of \(x\) contains one of \(\mathcal{U}_N\), implying that for all \(n \geq N\),
    \begin{align*}
        x_n \in \mathcal{U}_n \subset \mathcal{U}_n \subset \mathcal{V} \text{.}
    \end{align*}
\end{proof}


\begin{defbox}
    \begin{definition}
        A {\color{maththen}directed set} \((I, \prec)\) consists of a set \(I\) with a partial order \(\prec\) such that for every pair \(\alpha, \beta \in I\), there exists an element \(\gamma \in I\) with \(\gamma \prec \alpha\) and \(\gamma \prec \beta\).
    \end{definition}
\end{defbox}

\begin{defbox}
    \begin{definition}
        Given a space \(X\), a net \(\{x_\alpha\}_{\alpha \in I}\) in \(X\) is a function \(I \longrightarrow X: \alpha \mapsto x_\alpha\) where \((I, \prec)\) is a directed set.
    \end{definition}
\end{defbox}

\chapter{Compactness}

\begin{defbox}
    \begin{definition}
        A {\color{mathobj}subset} \(A \subset X\) is {\color{maththen}compact} if every open cover of \(A\) has a finite subcover, i.e. given an arbitary open cover \(\{\mathcal{U}_\alpha\}_{\alpha \in I}\) of \(A\), one can always find a finite subset \(\{\alpha_1, \ldots, \alpha_N\} \subset I\) such that \(A \subset \mathcal{U}_{\alpha_1} \cup \cdots \cup \mathcal{U}_{\alpha_N}\). We say that \(X\) itself is a {\color{maththen}compact space} if \(X\) is compact subset of itself.
    \end{definition}
\end{defbox}
\end{document}