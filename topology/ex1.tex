\documentclass[a4paper]{article}
\title{Topology}
\author{K}


% ---------------------------------------------------------------------
% P A C K A G E S
% ---------------------------------------------------------------------

% typography and formatting
\usepackage[english]{babel}
\usepackage[utf8]{inputenc}
\usepackage{geometry}
\usepackage{exsheets}
\usepackage{environ}

% mathematics
\usepackage{amsthm} % for theorems, and definitions
\usepackage{amssymb}
\usepackage{amsmath}
\usepackage{textcomp}
%\usepackage{MnSymbol} % for \cupdot

% extra
\usepackage{xcolor}
\usepackage{tikz}

% ---------------------------------------------------------------------
% S E T T I N G
% ---------------------------------------------------------------------

% typography and formatting
\geometry{margin=3cm}

\SetupExSheets{
  counter-format = ch.qu,
  counter-within = chapter,
  question/print = true,
  solution/print = true,
}

% mathematics

% extra
\definecolor{mathif}{HTML}{0000A0} % for conditions
\definecolor{maththen}{HTML}{CC5500} % for consequences
\definecolor{mathrem}{HTML}{8b008b} % for notes

\usetikzlibrary{positioning}
\usetikzlibrary{shapes.geometric, arrows}

% ---------------------------------------------------------------------
% C O M M A N D S
% ---------------------------------------------------------------------

\newcommand{\norm}[1]{\left\lVert#1\right\rVert}
\newcommand{\rank}{\text{rank}}
\newcommand{\Vol}{\text{Vol}}

\newcommand{\set}[1]{\left\{\, #1 \,\right\}}
\newcommand{\makeset}[2]{\left\{\, #1 \mid #2 \,\right\}}


\newcommand*\diff{\mathop{}\!\mathrm{d}}
\newcommand*\Diff{\mathop{}\!\mathrm{D}}

\newcommand\restr[2]{{% we make the whole thing an ordinary symbol
  \left.\kern-\nulldelimiterspace % automatically resize the bar with \right
  #1 % the function
  \vphantom{\big|} % pretend it's a little taller at normal size
  \right|_{#2} % this is the delimiter
  }}

% ---------------------------------------------------------------------
% R E N D E R
% ---------------------------------------------------------------------

\newif\ifshowproof
\showprooftrue

\NewEnviron{Proof}{%
    \ifshowproof%
        \begin{proof}%
            \BODY
        \end{proof}%
    \fi%
}%

\begin{document}
    Let \((X, d)\) be a metric space. Prove that the set of subsets
    \begin{equation}
        \mathcal{O}(d) := \makeset{U \subset X}{\forall x \, \exists \epsilon > 0 \text{ with } B_d(x, \epsilon) \subset U}
    \end{equation}
    defines a topology.
    \newline
    \textit{Proof.} We verify that \(\mathcal{O}(d)\) fullfills the axioms of a topology.
    \begin{enumerate}
        \item \(X \in \mathcal{O}(d)\) since any ball of a point \(x\) is contained in \(X\). \(\varnothing \in \mathcal{O}(d)\) is true vacuously.
        \item Let \(I\) be an arbitary index set and \(\{A_i\}_{i \in I}\) be a family of subsets belong to \(\mathcal{O}(d)\). Consider the union \(\bigcup_{i \in I}A_i\). If a point \(x\) is in \(\bigcup_{i \in I}A_i\), then there is an \(A_i\) where this point \(x\) is contained. Since \(A_i\) is in \(\mathcal{O}(d)\), there exists an \(\epsilon\) such that \(B_d(x, \epsilon) \subset A_i \subset \bigcup_{i \in I}A_i\). Therefore, we have that \(\bigcup_{i \in I}A_i\) belongs to \(\mathcal{O}(d)\).
        \item Let \(I\) be a finite index set and \(A_i\) with \(i \in I\) be subsets in \(\mathcal{O}(d)\). Consider the intersection \(\bigcap_{i \in I}A_i\). If a point \(x\) is in \(\bigcap_{i \in I}A_i\), then \(x\) is included in each \(A_i\). Again, \(A_i\) is in \(\mathcal{O}(d)\), so there is an \(\epsilon_i\) such that \(B_d(x, \epsilon_i) \subset A_i\). Choose the smallest (accordig to the metric \(d\)) among all \(\epsilon_i \in I\) and denote it as \(\epsilon\). We have \(B_d(x, \epsilon) \subset B_d(x, \epsilon_i) \subset A_i\) for all \(i \in I\). This means \(B_d(x, \epsilon) \subset \bigcap_{i \in I}A_i\) as desired.
    \end{enumerate}

    Show that any ball \(B_d(x, r) \in \mathcal{O}(d)\) for all \(x \in X\) and for all \(r > 0\).
    \begin{proof}
        Fix an \(p \in B_d(x, r)\). Set \(\epsilon := (r - d(x, p)) / 2\) (dividing it by two might only be for good measure). Then \(B_d(p, \epsilon) \subset B_d(x, r)\), so \(B_d(x, r) \in \mathcal{O}(d)\).
    \end{proof}

    Let \(d_1\) and \(d_2\) be equivalent metrics on \(X\). Show that \(\mathcal{O}(d_1) = \mathcal{O}(d_2)\).

    \begin{proof}
        We will show \(\mathcal{O}(d_1) \subseteq \mathcal{O}(d_2)\). Symmetry will take care of the other side. Let \(A \in \mathcal{O}(d_1)\) and fix a point \(x \in A\). We have that there exists an \(\epsilon_1\) such that \(B_{d_1}(x, \epsilon_1) \subset A\). Set \(\epsilon_2 := c \epsilon_1\) and consider \(B_{d_2}(x, \epsilon_2)\). Let \(y \in B_{d_2}(x, \epsilon_2)\). We have
        \begin{align}
            d_2(x, y) < \epsilon_2 & \iff d_2(x, y) < c \cdot \epsilon_1 \\
            & \iff
        \end{align}
    \end{proof}

    Let \(f: (X, d_X) \longrightarrow (X, d_Y)\) be a map that is \(\epsilon\)-\(\delta\)-continuous in the sense of metric spaces. Show that \(f\) is continuous with respect to the topologies \(\mathcal{O}_{d_X}\) and \(\mathcal{O}_{d_Y}\).

    
\end{document}