\documentclass[a4paper]{article}
\title{Topology}
\author{K}


% ---------------------------------------------------------------------
% P A C K A G E S
% ---------------------------------------------------------------------

% typography and formatting
\usepackage[english]{babel}
\usepackage[utf8]{inputenc}
\usepackage{geometry}
\usepackage{exsheets}
\usepackage{environ}

% mathematics
\usepackage{amsthm} % for theorems, and definitions
\usepackage{amssymb}
\usepackage{amsmath}
\usepackage{textcomp}
%\usepackage{MnSymbol} % for \cupdot

% extra
\usepackage{xcolor}
\usepackage{tikz}

% ---------------------------------------------------------------------
% S E T T I N G
% ---------------------------------------------------------------------

% typography and formatting
\geometry{margin=3cm}

\SetupExSheets{
  counter-format = ch.qu,
  counter-within = chapter,
  question/print = true,
  solution/print = true,
}

% mathematics

% extra
\definecolor{mathif}{HTML}{0000A0} % for conditions
\definecolor{maththen}{HTML}{CC5500} % for consequences
\definecolor{mathrem}{HTML}{8b008b} % for notes

\usetikzlibrary{positioning}
\usetikzlibrary{shapes.geometric, arrows}

% ---------------------------------------------------------------------
% C O M M A N D S
% ---------------------------------------------------------------------

\newcommand{\norm}[1]{\left\lVert#1\right\rVert}
\newcommand{\rank}{\text{rank}}
\newcommand{\Vol}{\text{Vol}}

\newcommand{\set}[1]{\left\{\, #1 \,\right\}}
\newcommand{\makeset}[2]{\left\{\, #1 \mid #2 \,\right\}}


\newcommand*\diff{\mathop{}\!\mathrm{d}}
\newcommand*\Diff{\mathop{}\!\mathrm{D}}

\newcommand\restr[2]{{% we make the whole thing an ordinary symbol
  \left.\kern-\nulldelimiterspace % automatically resize the bar with \right
  #1 % the function
  \vphantom{\big|} % pretend it's a little taller at normal size
  \right|_{#2} % this is the delimiter
  }}

% ---------------------------------------------------------------------
% R E N D E R
% ---------------------------------------------------------------------

\newif\ifshowproof
\showprooftrue

\NewEnviron{Proof}{%
    \ifshowproof%
        \begin{proof}%
            \BODY
        \end{proof}%
    \fi%
}%

\begin{document}
\section*{Exercise 3 b)}
Suppose \(\mathcal{B}\) is a subbasis for a topology \(\mathcal{T}\) on a set \(X\). Given another topological space \(Y\), show that a map \(f: Y \longrightarrow X\) is continuous if and only if for every \(\mathcal{U} \in \mathcal{B}\), \(f^{-1}(\mathcal{U})\) is open in \(Y\).

\begin{proof}[Solution]
    Denote the topology of \(Y\) by \(\mathcal{S}\).

    ``\(\Rightarrow\)'': Let \(f: Y \longrightarrow X\) be continuous and fix an \(\mathcal{U} \in \mathcal{B}\). Since \(\mathcal{B}\) is subbasis, all its elements are open subsets, thus \(\mathcal{U}\) is open. Then by definition of continuous maps, the preimage \(f^{-1}(\mathcal{U})\) is also open in \(Y\). As we have fixed an arbitary \(\mathcal{U} \in \mathcal{B}\), we may conclude the desired result.

    ``\(\Leftarrow\)'': On the other hand, let for every \(\mathcal{U} \in \mathcal{B}\) the preimage \(f^{-1}(\mathcal{U})\) be open in \(Y\). Consider an arbitary open subset \(\mathcal{V} \in \mathcal{T}\). By the definition of a subbasis, \(\mathcal{V}\) is a finite intersection of members of \(\mathcal{B}\), i.e.
    \begin{align*}
        \mathcal{V} = \mathcal{U}_1 \cap \cdots \cap \mathcal{U}_n
    \end{align*}
    with \(n \in \mathbb{N}\). The preimage of \(\mathcal{V}\) therefore is
    \begin{align*}
        f^{-1}(\mathcal{V}) &= f^{-1}(\mathcal{U}_1 \cap \ldots \cap \mathcal{U}_n) \\
        &= f^{-1}(\mathcal{U}_1) \cap \ldots \cap f^{-1}(\mathcal{U}_n)
    \end{align*}
    where we applied the lemma on the last step. Now, \(f^{-1}(\mathcal{U}_i)\) are open subsets for all \(1 \leq i \leq n\). By the definition of topological spaces, finite intersections of open subsets are also open, hence \(f^{-1}(\mathcal{V})\) is open. Thus, \(f\) is continuous.
\end{proof}


\end{document}