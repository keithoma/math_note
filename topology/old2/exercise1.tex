\documentclass[a4paper]{article}
\title{Series 1}
\author{K}


% ---------------------------------------------------------------------
% P A C K A G E S
% ---------------------------------------------------------------------

% typography and formatting
\usepackage[english]{babel}
\usepackage[utf8]{inputenc}
\usepackage{geometry}
\usepackage{exsheets}
\usepackage{environ}

% mathematics
\usepackage{amsthm} % for theorems, and definitions
\usepackage{amssymb}
\usepackage{amsmath}
\usepackage{textcomp}
%\usepackage{MnSymbol} % for \cupdot

% extra
\usepackage{xcolor}
\usepackage{tikz}

% ---------------------------------------------------------------------
% S E T T I N G
% ---------------------------------------------------------------------

% typography and formatting
\geometry{margin=3cm}

\SetupExSheets{
  counter-format = ch.qu,
  counter-within = chapter,
  question/print = true,
  solution/print = true,
}

% mathematics

% extra
\definecolor{mathif}{HTML}{0000A0} % for conditions
\definecolor{maththen}{HTML}{CC5500} % for consequences
\definecolor{mathrem}{HTML}{8b008b} % for notes

\usetikzlibrary{positioning}
\usetikzlibrary{shapes.geometric, arrows}

\theoremstyle{definition}
\newtheorem*{lemma}{Lemma}

% ---------------------------------------------------------------------
% C O M M A N D S
% ---------------------------------------------------------------------

\newcommand{\norm}[1]{\left\lVert#1\right\rVert}
\newcommand{\rank}{\text{rank}}
\newcommand{\Vol}{\text{Vol}}

\newcommand{\set}[1]{\left\{\, #1 \,\right\}}
\newcommand{\makeset}[2]{\left\{\, #1 \mid #2 \,\right\}}

\newcommand{\equivcls}[1]{%
  #1/\!{\sim}%
}

\newcommand*\diff{\mathop{}\!\mathrm{d}}
\newcommand*\Diff{\mathop{}\!\mathrm{D}}

\newcommand\restr[2]{{% we make the whole thing an ordinary symbol
  \left.\kern-\nulldelimiterspace % automatically resize the bar with \right
  #1 % the function
  \vphantom{\big|} % pretend it's a little taller at normal size
  \right|_{#2} % this is the delimiter
  }}
%---------------------------------------------------------------------
% R E N D E R
% ---------------------------------------------------------------------

\newif\ifshowproof
\showprooftrue

\NewEnviron{Proof}{%
    \ifshowproof%
        \begin{proof}%
            \BODY
        \end{proof}%
    \fi%
}%
\begin{document}
\maketitle
\section*{Problem 1}
Suppose \((X, d_X)\) is a metric space and \(\sim\) is an equivalence relation on \(X\), with the set of equivalence classes denoted by \(\equivcls{X}\). For the equivalence classes \([x], [y] \in \equivcls{X}\) represented by elements \(x, y \in X\), define
\begin{align*}
    d([x], [y]) := \inf \makeset{d_X(x, y)}{x \in [x], \, y \in [y]}
\end{align*}
\subsection*{(a)}
Show that \(d\) is a metric on \(\equivcls{X}\) if the following assumption is added: for every triple \([x], [y], [z] \in \equivcls{X}\), there exist representatives \(x \in [x], y \in [y]\) and \(z \in [z]\) such that
\begin{align*}
    d_X(x, y) = d([x], [y]) \quad \text{and} \quad d_X(y, z) = d([y], [z]) \text{.}
\end{align*}
\begin{proof}[Solution]
    Clearly, \(d\) is a map from \(\equivcls{X} \times \equivcls{X} \longrightarrow\) to \(\mathbb{R}\). We verify the axioms of a metric one by one.
    \begin{enumerate}
        \item ``\(d([x], [y]) \geq 0\)'': This is simply inherited from \(d_X\).
        \item ``\(d([x], [x]) = 0\)'': Again, inherited from \(d_X\).
        \item ``\(d([x], [y]) = d([y], [x])\)'': Also inherited from \(d_X\).
        \item ``\(d([x], [z]) \leq d(x, y) + d(y, z)\)'':
        \item ``\(d([x], [y]) > 0\) whenever \(x \neq y\)'': By the added assumption, we are guranteed that the value \(d([x], [y])\) is taken by some \(x \in [x]\) and \(y \in [y]\). In other words, \(d([x], [y])\) is not just an infimum of \(d_X(x, y)\), but also a minimum. Thus, this property is also inherited by \(d_X\).
    \end{enumerate}
\end{proof}
\end{document}