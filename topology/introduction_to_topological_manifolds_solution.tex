\documentclass[a4paper]{book}
\title{Topology}
\author{K}


% ---------------------------------------------------------------------
% P A C K A G E S
% ---------------------------------------------------------------------

% typography and formatting
\usepackage[english]{babel}
\usepackage[utf8]{inputenc}
\usepackage{geometry}
\usepackage{exsheets}
\usepackage{environ}
\usepackage{graphicx}
\usepackage{cutwin}

% mathematics
\usepackage{amsthm} % for theorems, and definitions
\usepackage{amssymb}
\usepackage{amsmath}
\usepackage{textcomp}
% \usepackage{MnSymbol} % for \cupdot

% extra
\usepackage{xcolor}
\usepackage{tikz}

% ---------------------------------------------------------------------
% S E T T I N G
% ---------------------------------------------------------------------

%maybe delete later, for colorbox
\usepackage{tcolorbox}
\newtcolorbox{defbox}{colback=blue!5!white,colframe=blue!75!black}
\newtcolorbox{defboxlight}{colback=cyan!5!white,colframe=cyan!75!black}
\newtcolorbox{thmbox}{colback=orange!5!white,colframe=orange!75!black}
\newtcolorbox{rembox}{colback=purple!5!white,colframe=purple!75!black}
\newtcolorbox{exmbox}{colback=gray!5!white,colframe=gray!75!black}

% typography and formatting
\geometry{margin=3cm}

\SetupExSheets{
  counter-format = ch.qu,
  counter-within = chapter,
  question/print = true,
  solution/print = true,
}

% mathematics
\newcounter{global}

\theoremstyle{definition}
\newtheorem{definition}{Definition}[]
\newtheorem{example}{Example}[definition]

\newtheorem{theorem}[definition]{Theorem}
\newtheorem{corollary}{Corollary}
\newtheorem{lemma}[definition]{Lemma}
\newtheorem{proposition}[definition]{Proposition}

\newtheorem*{remark}{Remark}

% extra
\definecolor{mathif}{HTML}{0000A0} % for conditions
\definecolor{maththen}{HTML}{CC5500} % for consequences
\definecolor{mathrem}{HTML}{8b008b} % for notes
\definecolor{mathobj}{HTML}{008800}

\usetikzlibrary{positioning}
\usetikzlibrary{shapes.geometric, arrows}

% ---------------------------------------------------------------------
% C O M M A N D S
% ---------------------------------------------------------------------

\newcommand{\norm}[1]{\left\lVert#1\right\rVert}
\newcommand{\rank}{\text{rank}}
\newcommand{\Vol}{\text{Vol}}

\newcommand{\set}[1]{\left\{\, #1 \,\right\}}
\newcommand{\makeset}[2]{\left\{\, #1 \mid #2 \,\right\}}

\newcommand*\diff{\mathop{}\!\mathrm{d}}
\newcommand*\Diff{\mathop{}\!\mathrm{D}}

\newcommand\restr[2]{{% we make the whole thing an ordinary symbol
  \left.\kern-\nulldelimiterspace % automatically resize the bar with \right
  #1 % the function
  \vphantom{\big|} % pretend it's a little taller at normal size
  \right|_{#2} % this is the delimiter
  }}

% ---------------------------------------------------------------------
% R E N D E R
% ---------------------------------------------------------------------

%\setlength\parindent{0pt}


\newif\ifshowproof
\showprooftrue

\NewEnviron{Proof}{%
    \ifshowproof%
        \begin{proof}%
            \BODY
        \end{proof}%
    \fi%
}%

\begin{document}
\maketitle
\tableofcontents
%%%%%%%%%%%%%%%%%%%%%%%%%%%%%%%%%%%%%%%%%%%%%%%%%%%%%%%%%%%%%%%%%%%%%%%%%%%%%%%
\chapter{Introduction}
\chapter{Topological Spaces}
\section*{2-1}
``\(\Rightarrow\)'': Let \(f: X_1 \longrightarrow X_2\) be a homeomorphism and fix a subset (not necessarily open) \(U \in \mathcal{T}_1\).
\begin{enumerate}
    \item Assume \(U\) is open in \(X_1\). Because \(f\) is continuous, the image of open subsets are again open, thus \(f(U)\) lies in \(\mathcal{T}_2\).
    \item On the other hand, if \(f(U)\) is open in \(X_2\), then since \(f\) is bijective we have
    \begin{align*}
        f^{-1} \left(f \left(U\right)\right) = U \text{.}
    \end{align*}
    Because \(f\) is continuous, the preimage of open subsets under \(f\) is open. We may therefore conclude \(U\) is open in \(X_1\).
\end{enumerate}
We have shown that if \(f\) is a homeomorphism, then \(f(\mathcal{T}_1) = \mathcal{T}_2\). \\

\noindent ``\(\Leftarrow\)'': Let \(f: X_1 \longrightarrow X_2\) be a bijective map such that \(f(\mathcal{T}_1) = \mathcal{T}_2\). Consider the inverse map \(f^{-1}\). We want to show \(f^{-1}\) is continuous. Fix an open subset \(U \in \mathcal{T}_1\). It is
\begin{align*}
    \left(f^{-1}\right)^{-1} \left(U\right) = f(U)
\end{align*}
because \(f\) is bijective. Since \(f(\mathcal{T}_1) = \mathcal{T}_2\) and \(U\) is open, \(f(U)\) is open as well. Hence the preimage of \(U\) under \(f^{-1}\) is open and \(f^{-1}\) is continuous.

Now we show that \(f\) is also continuous. Again, fix an open subset \(V \in \mathcal{T}_2\). The preimage of \(V\) under \(f\) is just the image of the inverse function. We have already shown that the inverse is continuous. Thus, \(f^{-1}(V)\) is open and \(f\) is continuous. Since \(f\) and \(f^{-1}\) exist and are continuous, \(f\) is a homeomorphism as desired.

\section*{2-2}
\subsection*{a)}
We show that \(\mathcal{T}\) is a topology by verifying the axioms of a topology.
\begin{enumerate}
    \item Since \(\mathcal{T}\) is the collection of all unions of finite intersections of elements of \(\mathcal{B}\), it contains the union of all elements of \(\mathcal{B}\) which is just \(X\). The union of empty collection generates the emptyset so \(\varnothing \in \mathcal{T}\) as well.
    \item Let \(\mathcal{U} \subset \mathcal{T}\) be any subset. The elements of \(\mathcal{U}\) are unions of finite intersections of elements of \(\mathcal{B}\). Thus, \(\bigcup_{U \in \mathcal{U}} U\) is again a union of finite intersections of elements of \(\mathcal{B}\). In other words, \(\mathcal{T}\) is closed under union.
    \item \(\mathcal{T}\) is stable under finite intersections due to distributive property of sets.
\end{enumerate}

\subsection*{b)}

\section*{2-3}
\subsection*{1.}
The collection of subset \(\mathcal{T}_1 = \makeset{U \subset X}{X \setminus U \text{ is finite or is all of } X}\) forms a topology. We show this by verifying the axioms of a topology.
\begin{enumerate}
    \item It is \(X \setminus \varnothing = X\) and \(X \setminus X = \varnothing\) which is finite. Thus, \(X \in \mathcal{T}_1\) and \(\varnothing \in \mathcal{T}_1\).
    \item Let \(\mathcal{U} \subset \mathcal{T}\) be a subset. By De Morgan's laws we have
    \begin{align*}
        X \setminus \left(\bigcup_{U \in \mathcal{U}} U \right) = \bigcap_{U \in \mathcal{U}} \left( X \setminus U \right) \text{.}
    \end{align*}
    Since each \(U \in \mathcal{U}\) lies in \(\mathcal{T}\), the complement \(X \setminus U\) is finite or is all of \(X\). Therefore, the intersection of all \(X \setminus U\) is again finite or all of \(X\), and we may conclude that \(\mathcal{T}\) is stable under arbitary unions.
    \item Use De Morgan's law again.
\end{enumerate}

\subsection*{2.}
The collection of subsets \(\mathcal{T}_2 = \makeset{U \subset X}{X \setminus U \text{ is infinite or is empty}}\) is not a topology. Take \(X = \mathbb{Z}\) for example and consider \(A = \set{1, 2, 3, \ldots}\) and \(B = \set{-1, -2, -3, \ldots}\). \(A\) and \(B\) are open because their complements are the non-positive and the non-negative integers respectively. If \(\mathcal{T}_2\) is a topology, it should contain their union \(A \cup B = \mathbb{Z} \setminus \{0\}\). However,
\begin{align*}
    \mathbb{Z} \setminus (A \cup B) = \mathbb{Z} (\mathbb{Z} \setminus \{0\}) = \{0\}
\end{align*}
which is not infinite and thus doesn't lie in \(\mathcal{T}_2\).

\subsection*{3.}
The collection of subsets \(\mathcal{T}_3 = \makeset{U \subset X}{X \setminus U \text{ is countable or all of } X}\) is a topology PROBABLY.

\section*{2-4}

Already did somewhere else.

\section*{2-5}
\begin{enumerate}
    \item \(\mathrm{id}_1: X \longrightarrow \mathbb{R}^2\) is continuous probably.
    \item \(\mathrm{id}_2: \mathbb{R}^2 \longrightarrow X\) is not continuous probably.
\end{enumerate}

\section*{2-6}
\(f\) is continuous because any preimage of a subset \(U \subset Z\) under \(f\) is open, since any subset in \(X\) is open.

For \(g\), the only preimages to check are the emptyset \(\varnothing\) and \(Y\). Simply, \(g^{-1}(\varnothing) = \varnothing\) and \(g^{-1}(Y) = Z\). Both subsets are open in \(Z\), therefore \(g\) is continuous.

If \(h\) is constant, say \(h(Y) = \{p\}\), then \(h^{-1}(U) = Y\) if \(p \in U\) and \(h^{-1}(U) = \varnothing\) if \(p \in U\). In both cases the preimages are open, thus \(h\) is continuous. Assume \(h\) is continuous but not constant, i.e. there are points \(x_1, x_2 \in Y\) such that \(h(x_1) \neq h(x_2)\). \(Z\) is Hausdorff, so there are disjoint neighbourhoods \(U\) of \(h(x_1)\) and \(V\) of \(h(x_2)\). \(h\) was assumed to be continuous, so \(h^{-1}(U) = Y\) and \(h^{-1}(V) = Y\) which is impossible (REALLY?).

\section*{2-7}
\subsection*{a)}


\subsection*{f)}
\section*{2-8}
Firstly, any element in \(f(\mathcal{B})\) is open because \(f\) is an open map. Fix an open subset \(V\) in \(Y\) and consider its preimage \(f^{-1}(V)\) under \(f\). Because \(f\) is continuous, the preimage is open, thus there are base elements \(B_i\) with \(i \in I\) in \(\mathcal{B}\) such that
\begin{align*}
    f^{-1}(V) = \bigcup_{i \in I} B_i \text{.}
\end{align*}
The surjectivity of \(f\) grants us \(f(f^{-1}(V)) = V\), therefore, we have
\begin{align*}
    f(f^{-1}(V)) = V = f \left( \bigcup_{i \in I} B_i \right) = \bigcup_{i \in I} f(B_i)\text{.}
\end{align*}
Thus, \(f(\mathcal{B})\) is a basis of \(Y\).

\section*{2-9}

\section*{2-10}
Fix a point \(y\) in \(Y\). Since \(f\) is surjective, there is an \(x\) in \(X\) such that \(f(x) = y\). \(X\) is locally Euclidean, thus there is a neighbourhood \(U\) of \(x\) that is homeomorphic to \(\mathbb{R}^n\). Moreover, \(f\) is locally homeomorphic, so there is a neighbourhood \(V\) of \(x\) such that the restriction of \(f\) under \(V\) is a homeomorphism. Then, the intersection \(U \cap V = N\) has both of these properties, i.e. \(N\) is a neighbourhood of \(x\) that is homeomorphic to \(\mathbb{R}^n\) and the restriction of \(f\) under \(V\) is a homeomorphism. \(f(N)\) is a neighbourhood of \(y\) that is homeomorphic to \(\mathbb{R}^n\), therefore \(Y\) is locally Euclidean.
\end{document}