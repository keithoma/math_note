\begin{theorem}[Important Identities]
    \begin{align}
        & \int x^\alpha \diff x = \frac{1}{\alpha + 1}x^{\alpha + 1} + c & \text{For all \(\alpha \in \mathbb{N}\)} \label{thm:ii:pow}\\
        & \int \frac{1}{x} \diff x = \ln |x| + c & \text{If \(x \neq 0\).} \\
        & \int e^x \diff x = e^x + c & \\
        & \int \cos x \diff x = \sin x + c& \\
        & \int \sin x \diff x = -\cos x + c& \\
        & \int \frac{1}{1 + x^2} \diff x = \arctan(x) + c & \\
        & \int 
    \end{align}
\end{theorem}

\begin{question}
    \begin{align}
        \int (1 - t)^9 \diff x
    \end{align}
\end{question}
\begin{solution}
    Substitute \(1 - t = u \Rightarrow -1 \diff t = \diff u\), then we have
    \begin{align}
        \int (1 - t)^9 \diff x &= - \int u^9 \diff u & \text{(Substitution.)}\\
        &= -\frac{u ^{10}}{10} + c & \text{(Important identities: \ref{thm:ii:pow})} \\
        &= -\frac{(1 - t)^{10}}{10} + c \text{.} & \text{(\(u = 1 - t\).)}
    \end{align}
\end{solution}
\begin{question}
    \begin{align}
        \int (x^2 + 1)^2 \diff x
    \end{align}
\end{question}
\begin{solution}
    Substitute \(x^2 + 1 = u \Rightarrow 2x \diff x = \diff u\).
    \begin{align}
        \int (x^2 + 1)^2 \diff x &= \int x^4 + 2x^2 + 1 \diff x \\
        &= \int x^4 \diff x + 2 \int x^2 \diff x + \int 1 \diff x \\
        &= \frac{1}{5} x^5 + \frac{2}{3} x^3 + x
    \end{align}
\end{solution}
\begin{question}
    \begin{align}
        \int \frac{1}{x^2 -x + 1} \diff x
    \end{align}
\end{question}
\begin{solution}
    Substitute \(x - \frac{1}{2} = u \Rightarrow 1 \diff x = \diff u\).
    \begin{align}
        \int \frac{1}{x^2 -x + 1} \diff x &= \int \frac{1}{(x - \frac{1}{2})^2 + \frac{3}{4}} \diff x\\
        &= \int \frac{1}{u^2 + \frac{3}{4}} \diff x \\
        &= \frac{3}{4} \int \frac{1}{\frac{4}{3} u^2 + 1} \diff u
    \end{align}
    The last step was done in order to force the integrand to be in a form similar to
    \begin{align}
        \frac{\diff}{\diff x} \arctan x = \frac{1}{x^2 + 1}
    \end{align}
    Lastly, substitute in the equation above \(\frac{2}{\sqrt{3}} u = v \Rightarrow \frac{2}{\sqrt{3}} \diff u = \diff v\). It is \(u = \frac{\sqrt{3}}{2} v\) and \(\diff u = \frac{\sqrt{3}}{2} \diff v\), therefore, we have
    \begin{align}
        \frac{3}{4} \int \frac{1}{\frac{4}{3} u^2 + 1} \diff u &= \frac{3}{4} \int \frac{1}{\frac{4}{3} (\frac{\sqrt{3}}{2}v)^2 + 1} \frac{\sqrt{3}}{2} \diff v \\
        &= \frac{3}{4} \frac{\sqrt{3}}{2} \int \frac{1}{v^2 + 1} \diff v \\
        &= \frac{3 \sqrt{3}}{8} \arctan v  \\
        &= \frac{3 \sqrt{3}}{8} \arctan \left( \frac{\sqrt{3}}{2}u \right) \\
        &= \frac{3 \sqrt{3}}{8} \arctan \left( \frac{\sqrt{3}}{2}(x - \frac{1}{2})\right)
    \end{align}
\end{solution}