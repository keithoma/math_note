\documentclass[a4paper]{book}
\title{Integration and Integration}
\author{K}


% ---------------------------------------------------------------------
% P A C K A G E S
% ---------------------------------------------------------------------

% typography and formatting
\usepackage[english]{babel}
\usepackage[utf8]{inputenc}
\usepackage{geometry}
\usepackage{exsheets}
\usepackage{environ}

% mathematics
\usepackage{amsthm} % for theorems, and definitions
\usepackage{amssymb}
\usepackage{amsmath}
\usepackage{textcomp}
\usepackage{dsfont}
%\usepackage{MnSymbol} % for \cupdot

% extra
\usepackage{xcolor}
\usepackage{tikz}

% ---------------------------------------------------------------------
% S E T T I N G
% ---------------------------------------------------------------------

% typography and formatting
\geometry{margin=3cm}

\SetupExSheets{
  counter-format = ch.qu,
  counter-within = chapter,
  question/print = true,
  solution/print = true,
}

% mathematics
\theoremstyle{definition}
\newtheorem{definition}{Definition}
\newtheorem{example}{Example}[definition]

\newtheorem{theorem}{Theorem}[definition]
\newtheorem{corollary}{Corollary}
\newtheorem{lemma}{Lemma}[definition]
\newtheorem{proposition}{Proposition}[definition]

\newtheorem*{remark}{Remark}

% extra
\definecolor{mathif}{HTML}{0000A0} % for conditions
\definecolor{maththen}{HTML}{CC5500} % for consequences
\definecolor{mathrem}{HTML}{8b008b} % for notes

\usetikzlibrary{positioning}
\usetikzlibrary{shapes.geometric, arrows}

% ---------------------------------------------------------------------
% C O M M A N D S
% ---------------------------------------------------------------------

\newcommand{\norm}[1]{\left\lVert#1\right\rVert}
\newcommand{\rank}{\text{rank}}
\newcommand{\Vol}{\text{Vol}}
\newcommand*\diff{\mathop{}\!\mathrm{d}}
\newcommand*\Diff{\mathop{}\!\mathrm{D}}

\newcommand\restr[2]{{% we make the whole thing an ordinary symbol
  \left.\kern-\nulldelimiterspace % automatically resize the bar with \right
  #1 % the function
  \vphantom{\big|} % pretend it's a little taller at normal size
  \right|_{#2} % this is the delimiter
  }}



  \newcounter{problem}
  \newcounter{solution}
  
  \newcommand\Problem{%
    \stepcounter{problem}%
    \textbf{\theproblem.}~%
    \setcounter{solution}{0}%
  }
  
  \newcommand\TheSolution{%
    \textbf{Solution:}\\%
  }
  
  \newcommand\ASolution{%
    \stepcounter{solution}%
    \textbf{Solution \thesolution:}\\%
  }
  \parindent 0in
  \parskip 1em
% ---------------------------------------------------------------------
% R E N D E R
% ---------------------------------------------------------------------

\newif\ifshowproof
\showprooftrue

\NewEnviron{Proof}{%
    \ifshowproof%
        \begin{proof}%
            \BODY
        \end{proof}%
    \fi%
}%

\begin{document}
\chapter{Interpolation}
% \section{Lagrange Interpolation}

\begin{example}
    Consider the domain \([2, 10]\) partitioned into \(5\) points, i.e. \(\{2, 4, 6, 8, 10\}\) and a function \(f: [0, 10] \rightarrow \mathbb{R}, \, x \mapsto f(x) = \ln(x)\). The y-values then are
    \begin{equation}
        \ln(2) \approx 0.6931 \quad \ln(4) \approx 1.3862 \quad \ln(6) \approx 1.7917 \quad \ln(8) \approx 2.0794 \quad \ln(10) \approx 2.3025 \text{.}
    \end{equation}
    Computing the Lagrange polynomials gives
    \begin{align}
        L_1 (x) &= \ln(2) \cdot \frac{x - 4}{2 - 4} \cdot \frac{x - 6}{2 - 6} \cdot \frac{x - 8}{2 - 8} \cdot \frac{x - 10}{2 - 10} \\
        &= 5 \ln(2) - \frac{77}{24} x \ln(2) + \frac{71}{96} x^2 \ln(2) - \frac{7}{96} x^3 \ln(2) + \frac{1}{384} x^4 \ln(2) \\
        L_2 (x) &= \ln(2) \cdot \frac{x - 2}{4 - 2} \cdot \frac{x - 6}{4 - 6} \cdot \frac{x - 8}{4 - 8} \cdot \frac{x - 10}{4 - 10} \\
        &= -10 \ln(2) + \frac{107}{12} x \ln(2) - \frac{59}{24} x^2 \ln(2) + \frac{13}{48} x^3 \ln(2) - \frac{1}{96} x^4 \ln(2)
    \end{align}
\end{example}

\begin{example}
    Let \(f(x) = x^8\). We want to interpolate \(f\) on the grid points \(\{-3, -2, -1, 0, 1, 2, 3\}\). The Lagrange polynomials are
    \begin{align}
        L_1(x) &= -6561 \cdot \frac{x + 2}{-3 + 2} \cdot \frac{x + 1}{-3 + 1} \cdot \frac{x + 0}{-3 + 0} \cdot \frac{x - 1}{-3 - 1}
    \end{align}
\end{example}

\begin{example}
    We interpolate \(\log_2(x)\) on the points \(\{16, 32, 64\}\). It is
    \begin{align}
        \log_2(16) L_1 (x) &= \log_2(16) \cdot \frac{x - 32}{16 - 32} \cdot \frac{x - 64}{16 - 64} \\
        &= \frac{1}{192} x^2 - \frac{1}{2} x + \frac{32}{3} \\
        \log_2(32) L_2 (x) &= \log_2(32) \cdot \frac{x - 16}{32 - 16} \cdot \frac{x - 64}{32 - 64} \\
        &= -\frac{5}{512} x^2 + \frac{25}{32} x -10 \\
        \log_2(64) L_3 (x) &= \log_2(64) \cdot \frac{x - 16}{64 - 16} \cdot \frac{x - 32}{64 - 32} \\
        &= \frac{1}{256}x^2 - \frac{3}{16}x + 2 \text{.}
        \intertext{Summing up yields}
        p(x) &= -\frac{1}{1536}x^2 + \frac{3}{32}x + \frac{8}{3} \text{.}
    \end{align}
\end{example}

\section{Spline Interpolation}
\chapter{Finite Element}
% \begin{example}
    Let \(K = [0, 1]\), \(\mathcal{P}\) be the set of linear polynomials and \(\mathcal{L} = \{L_1, L_2\}\) where \(L_1(p) = p(0)\) and \(L_2(p) = p(1)\) for all \(p \in \mathcal{L}\). Then \((K, \mathcal{P}, \mathcal{L})\) is a finite element.
\end{example}
\begin{proof}[First proof by verifying linearity]
    Just check
    \begin{align*}
        \lambda_1 L_1 + \lambda_2 L_2 = 0
    \end{align*}
\end{proof}
\begin{proof}[Second proof by construction of a nodal basis]
    We construct the nodal basis \(\{\phi_1, \phi_2\}\) of \(\mathcal{P}\) explicitly. Since \(\phi_j\) must fulfill \(L_i (\phi_j) = \delta_{ij}\), we have
    \begin{align*}
        L_1 (\phi_1) = 1 & \iff a_1 \cdot 0 + b_1 = 1 \\
        & \iff b_1 = 1 \\
        L_2 (\phi_1) = 0 & \iff a_1 \cdot 1 + b_1 = 0 \\
        & \iff a_1 = -1 \\
        L_1 (\phi_2) = 0 & \iff a_2 \cdot 0 + b_2 = 0 \\
        & \iff b_2 = 0 \\
        L_2 (\phi_2) = 1 & \iff a_2 \cdot 1 + b_2 = 1 \\
        & \iff a_2 = 1
    \end{align*}
    Set \(\phi_1(x) = - x + 1\) and \(\phi_2(x) = x\), then \(\{\phi_1, \phi_2\}\) is a nodal basis of \(\mathcal{P}\) and \((K, \mathcal{P}, \mathcal{L})\) is a finite element.
\end{proof}

\begin{example}[Counter Example to Nonconform \(P_2\)-FE]
    Denote
    \begin{equation}
        P_1 = \begin{pmatrix}
            x_1 \\ y_1
        \end{pmatrix} \qquad
        P_2 = \begin{pmatrix}
            x_2 \\ y_2
        \end{pmatrix} \qquad
        P_3 = \begin{pmatrix}
            x_3 \\ y_3
        \end{pmatrix} \qquad
    \end{equation}
    and write \(p(x) = a_0 + a_1 x + a_2 y + a_3 x^2 + a_4 y^2 + a_5 xy\). Then, for \(L_1\) we have
    \begin{align}
        L_1 (p) &= p(Q_1) \\
        &= p(\mu P_2 + (1 - \mu)P_3) \\
        &= \mu p (P_2) + p (P_3) - \mu p (P_3) \\
        \begin{split}
            &= \mu \left( a_0 + a_1 x + a_2 y + a_3 x^2 + a_4 y^2 + a_5 xy \right) \\
            & + \left( a_0 + a_1 x + a_2 y + a_3 x^2 + a_4 y^2 + a_5 xy \right) \\
            & - \mu \left( a_0 + a_1 x + a_2 y + a_3 x^2 + a_4 y^2 + a_5 xy \right) \\
        \end{split}
    \end{align}
    \begin{align}
        \mu
    \end{align}
\end{example}

\begin{example}
    Let \(K\) be any rectangle, \(\mathcal{P} = \mathcal{Q}_k\) and \(\mathcal{N}\) denote point evaluations at \(\{(t_i, t_j) \mid 0 \leq i, j \leq k\}\) where \(0 = t_0 < t_1 < \cdots < t_{k-1} < t_k = 1\). Then \((K, \mathcal{P}, \mathcal{N})\) is a finite element.
\end{example}
\chapter{Newton and other algorithms}
\end{document}