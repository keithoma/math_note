\begin{definition}[Newton's Method]
    Let \(D \subset \mathbb{K}^n\) an nonempty open domain and \(x_0 \in D\). For \(j \in \mathbb{N}\) Newton's Method is recursively defined as
    \begin{equation}
        x_{j+1} = x_j - \frac{f(x_j)}{f'(x_j)}
    \end{equation}
    and stops if \(x_{j+1} \not\in D\) or \(f^\prime (x_j) = 0\).
\end{definition}

\begin{theorem}
    Let \(x^* \in D\) be a root of \(f \in C^1 \left(D, \mathbb{K}^n\right)\) with \(f'(x^*)\) regular (if \(f\) is one-dimensional, then \(f^\prime (x^*) \neq 0\)), then the following hold.
    \begin{enumerate}
        \item There exists a \(\epsilon > 0\) such that the Newton's Method does not break for all \(x_0 \in \overline{B(x^*, \epsilon)}\). Moreover, we have the convergence
        \begin{equation}
            \lim_{j \rightarrow \infty} x_j = x^*
        \end{equation}
        \item Superlinear convergence. For all \(j \in \mathbb{N}\) and a null sequence \(q_j\)
    \end{enumerate}
\end{theorem}