\begin{definition}
    Let
    \begin{enumerate}
        \item \(K \subset \mathbb{R}^n\) be a {\color{mathif}bounded closed set} with {\color{mathif}nonempty interior} and {\color{mathif}piecewise smooth boundary} (the {\color{mathrem}element domain}),
        \item \(\mathcal{P}\) be a {\color{mathif}finite-dimensional space of function} on \(K\) (the space of {\color{mathrem}shape functions}), and
        \item \(\mathcal{L} = \{L_1, L_2, \ldots, L_k\}\) be a {\color{mathif}basis} for \(\mathcal{P}^\prime\) (the set of {\color{mathrem}nodal variables}).
    \end{enumerate}
    Then \((K, \mathcal{P}, \mathcal{L})\) is called a {\color{maththen}finite element}.
\end{definition}

\begin{definition}
    Let \((K, \mathcal{P}, \mathcal{L})\) be a finite element. The basis \(\{\phi_1, \phi_2, \ldots, \phi_k\}\) of \(\mathcal{P}\) dual to \(\mathcal{L}\) (i.e. \(N_i(\phi_j) = \delta_{ij}\)) is called the nodal basis of \(\mathcal{P}\).
\end{definition}

\begin{lemma}
    Let \(\mathcal{P}\) be a \(k\)-dimensional vector space and let \(\{L_1, L_2, \ldots, L_k\}\) be a subset of the dual space \(\mathcal{P}^\prime\). Then the following two statements are equivalent.
    \begin{enumerate}
        \item \(\{L_1, L_2, \ldots, L_k\}\) is a basis for \(\mathcal{P}^\prime\).
        \item Given \(p \in \mathcal{P}\) with \(L_i(p) = 0\) for all \(i \in \{1, 2, \ldots, k\}\), then \(p = 0\).
    \end{enumerate}
\end{lemma}

\begin{example}
    Let \(K = [0, 1]\), \(\mathcal{P}\) be the set of linear polynomials and \(\mathcal{L} = \{L_1, L_2\}\) where \(L_1(p) = p(0)\) and \(L_2(p) = p(1)\) for all \(p \in \mathcal{L}\). Then \((K, \mathcal{P}, \mathcal{L})\) is a finite element.
\end{example}
\begin{proof}[First proof by verifying linearity]
    Just check
    \begin{align*}
        \lambda_1 L_1 + \lambda_2 L_2 = 0
    \end{align*}
\end{proof}
\begin{proof}[Second proof by construction of a nodal basis]
    We construct the nodal basis \(\{\phi_1, \phi_2\}\) of \(\mathcal{P}\) explicitly. Since \(\phi_j\) must fulfill \(L_i (\phi_j) = \delta_{ij}\), we have
    \begin{align*}
        L_1 (\phi_1) = 1 & \iff a_1 \cdot 0 + b_1 = 1 \\
        & \iff b_1 = 1 \\
        L_2 (\phi_1) = 0 & \iff a_1 \cdot 1 + b_1 = 0 \\
        & \iff a_1 = -1 \\
        L_1 (\phi_2) = 0 & \iff a_2 \cdot 0 + b_2 = 0 \\
        & \iff b_2 = 0 \\
        L_2 (\phi_2) = 1 & \iff a_2 \cdot 1 + b_2 = 1 \\
        & \iff a_2 = 1
    \end{align*}
    Set \(\phi_1(x) = - x + 1\) and \(\phi_2(x) = x\), then \(\{\phi_1, \phi_2\}\) is a nodal basis of \(\mathcal{P}\) and \((K, \mathcal{P}, \mathcal{L})\) is a finite element.
\end{proof}
\begin{proof}[Third proof with lemma]
    Denote \(p \in \mathcal{P}\) with \(L_1(p) = L_2(p) = 0\) as \(p(x) = ax+b\). We have \(L_1(p) = b\) and \(L_2(p) = a + b\), so \(a = b = 0\) and \(p = 0\). By lemma XXX \(\{L_1, L_2\}\) is a basis for \(\mathcal{P}^\prime\).
\end{proof}


We can generalize the previous example.

\begin{example}
    Let \(K = [a, b]\), \(\mathcal{P}_k\) be the set of all polynomials of degree less than or equal to \(k\), and \(\mathcal{L} = \{L_0, L_1, \ldots, L_k\}\) where
    \begin{equation}
        L_i (p) = p \left( a + \frac{(b-a) i}{k} \right) \qquad \text{for all \(p \in \mathcal{P}_k\) and \(i \in \{0, 1, \ldots, k\}\).}
    \end{equation}
    Then \(K, \mathcal{P}_k, \mathcal{N}_k\) is a finite element.
\end{example}

\begin{proof}
    Let \(p \in \mathcal{P}_k\) with \(L_1 (p) = L_2 (p) = \cdots = L_k (p) = 0\). So \(p\) has \(k + 1\) roots but is only of degree \(k\), therefore \(p\) must be the zero-polynomial.
\end{proof}

\begin{example}[Counter Example to Nonconform \(P_2\)-FE]
    Denote
    \begin{equation}
        P_1 = \begin{pmatrix}
            x_1 \\ y_1
        \end{pmatrix} \qquad
        P_2 = \begin{pmatrix}
            x_2 \\ y_2
        \end{pmatrix} \qquad
        P_3 = \begin{pmatrix}
            x_3 \\ y_3
        \end{pmatrix} \qquad
    \end{equation}
    and write \(p(x) = a_0 + a_1 x + a_2 y + a_3 x^2 + a_4 y^2 + a_5 xy\). Then, for \(L_1\) we have
    \begin{align}
        L_1 (p) &= p(Q_1) \\
        &= p(\mu P_2 + (1 - \mu)P_3) \\
        &= \mu p (P_2) + p (P_3) - \mu p (P_3) \\
        \begin{split}
            &= \mu \left( a_0 + a_1 x + a_2 y + a_3 x^2 + a_4 y^2 + a_5 xy \right) \\
            & + \left( a_0 + a_1 x + a_2 y + a_3 x^2 + a_4 y^2 + a_5 xy \right) \\
            & - \mu \left( a_0 + a_1 x + a_2 y + a_3 x^2 + a_4 y^2 + a_5 xy \right) \\
        \end{split}
    \end{align}
    \begin{align}
        \mu
    \end{align}
\end{example}

\begin{example}
    Let \(K\) be any rectangle, \(\mathcal{P} = \mathcal{Q}_k\) and \(\mathcal{N}\) denote point evaluations at \(\{(t_i, t_j) \mid 0 \leq i, j \leq k\}\) where \(0 = t_0 < t_1 < \cdots < t_{k-1} < t_k = 1\). Then \((K, \mathcal{P}, \mathcal{N})\) is a finite element.
\end{example}