\section{Lagrange Interpolation}

\begin{definition}
    For \(n \in \mathbb{N}_0\), \(n + 1\) distinct \(x_0, \ldots, x_n \in \mathbb{K}\) and the index \(j \in \{0, \ldots, n\}\) the \(j\)-th Lagrange polynomials respective to the nodes \(x_0, \ldots, x_n\) are defined as
    \begin{equation}
        L_j = \prod_{\substack{k=0 \\ k \neq j}}^n \frac{x - x_k}{x_j - x_k} \in \mathbb{K}[x] \text{.}
    \end{equation}
    If \(f(x_0), \ldots, f(x_n)\) are now the data respective to the nodes \(x_0, \ldots, x_n\) then the Lagrange interpolation \(p\) is given by
    \begin{equation}
        p(x) = \sum_{j=0}^n f(x_j) L_j(x) \text{.}
    \end{equation}
\end{definition}

\begin{example}
    Consider the domain \([2, 10]\) partitioned into \(5\) points, i.e. \(\{2, 4, 6, 8, 10\}\) and a function \(f: [0, 10] \rightarrow \mathbb{R}, \, x \mapsto f(x) = \ln(x)\). The y-values then are
    \begin{equation}
        \ln(2) \approx 0.6931 \quad \ln(4) \approx 1.3862 \quad \ln(6) \approx 1.7917 \quad \ln(8) \approx 2.0794 \quad \ln(10) \approx 2.3025 \text{.}
    \end{equation}
    Computing the Lagrange polynomials gives
    \begin{align}
        L_1 (x) &= \ln(2) \cdot \frac{x - 4}{2 - 4} \cdot \frac{x - 6}{2 - 6} \cdot \frac{x - 8}{2 - 8} \cdot \frac{x - 10}{2 - 10} \\
        &= 5 \ln(2) - \frac{77}{24} x \ln(2) + \frac{71}{96} x^2 \ln(2) - \frac{7}{96} x^3 \ln(2) + \frac{1}{384} x^4 \ln(2) \\
        L_2 (x) &= \ln(2) \cdot \frac{x - 2}{4 - 2} \cdot \frac{x - 6}{4 - 6} \cdot \frac{x - 8}{4 - 8} \cdot \frac{x - 10}{4 - 10} \\
        &= -10 \ln(2) + \frac{107}{12} x \ln(2) - \frac{59}{24} x^2 \ln(2) + \frac{13}{48} x^3 \ln(2) - \frac{1}{96} x^4 \ln(2)
    \end{align}
\end{example}

\begin{example}
    Let \(f(x) = x^8\). We want to interpolate \(f\) on the grid points \(\{-3, -2, -1, 0, 1, 2, 3\}\). The Lagrange polynomials are
    \begin{align}
        L_1(x) &= -6561 \cdot \frac{x + 2}{-3 + 2} \cdot \frac{x + 1}{-3 + 1} \cdot \frac{x + 0}{-3 + 0} \cdot \frac{x - 1}{-3 - 1}
    \end{align}
\end{example}

\begin{example}
    We interpolate \(\log_2(x)\) on the points \(\{16, 32, 64\}\). It is
    \begin{align}
        \log_2(16) L_1 (x) &= \log_2(16) \cdot \frac{x - 32}{16 - 32} \cdot \frac{x - 64}{16 - 64} \\
        &= \frac{1}{192} x^2 - \frac{1}{2} x + \frac{32}{3} \\
        \log_2(32) L_2 (x) &= \log_2(32) \cdot \frac{x - 16}{32 - 16} \cdot \frac{x - 64}{32 - 64} \\
        &= -\frac{5}{512} x^2 + \frac{25}{32} x -10 \\
        \log_2(64) L_3 (x) &= \log_2(64) \cdot \frac{x - 16}{64 - 16} \cdot \frac{x - 32}{64 - 32} \\
        &= \frac{1}{256}x^2 - \frac{3}{16}x + 2 \text{.}
        \intertext{Summing up yields}
        p(x) &= -\frac{1}{1536}x^2 + \frac{3}{32}x + \frac{8}{3} \text{.}
    \end{align}
\end{example}

\section{Spline Interpolation}

\begin{definition}
    A function \(s \in C^q([a, b], \mathbb{K})\) is called a spline of degree \(p\) and the smoothness \(q\) respective to the partition \(\triangle\) if \(s\) is a polynomial on the subinterval \([x_{j-1}, x_j]\) with a degree equal or less than \(q\). We denote \(s \in \mathcal{S}_q^p(\triangle)\).
\end{definition}

\begin{example}
    \begin{itemize}
        \item The space of linear spline functions is \(\mathcal{S}_0^1(\triangle)\). We have \(s \in C^0 ([a, b], \mathbb{K})\) so \(s\) is continuous but not necessarily differentiatable.
        \item The space of cubic spline functions is \(\mathcal{S}_2^3(\triangle)\).
    \end{itemize}
\end{example}

\begin{theorem}
    \(\mathcal{S}_{m-1}^m(\triangle)\) is a \(\mathbb{R}\)-vector space. In particular, it contains all polynomials of degree \(\leq m\). Moreover, the dimension of \(\mathcal{S}_{m-1}^m (x_0, \ldots, x_n)\) is \(m + n\). 
\end{theorem}

% move this
\begin{example}
    Let \([-1, 1]\) be a domain partitioned into \([-1, 0] \cup [0, 1]\). Consider the function
    \begin{equation}
        f(x) = \cos \left( \frac{\pi}{2} x \right) \qquad x \in [-1, 1] \text{.}
    \end{equation}
\end{example}

\subsection{Linear Splines}

\subsection{Cubic Splines}

\begin{definition}
    \begin{enumerate}
        \item hermitian
        \item natural
        \item periodic
    \end{enumerate}
\end{definition}

\begin{theorem}
    If \(\triangle = (x_0, \ldots, x_n)\) is a partition of the interval \([a, b]\) and \(y_0, \ldots, y_n\) the respective data then there is exactly one interpolating natural cubic spline \(s \in \mathcal{S}_2^3(\triangle)\) with \(s^{\prime\prime} (a) = s^{\prime\prime}(b) = 0\).
\end{theorem}

\begin{example}
    Given the data set \((0, 0)\), \((1, 0.5)\), \((2, 2)\) and \((3, 1.5)\) we want to find the interpolating cubic spline \(s(x)\) satisfying \(s^\prime (0) = 0.2\) and \(s^\prime (3) = -1\).

    We start by finding \(s_i^{\prime\prime}\). Denote \(s_i^{\prime\prime}(x_i) = M_i\). With Lagrange interpolation we get the general formula
    \begin{equation}
        s_i^{\prime\prime} (x) = M_{i - 1} \frac{x - x_i}{x_{i - 1} - x_i} + M_i \frac{x - x_{x-1}}{x_i - x_{i-1}} \text{.}
    \end{equation}
    Plugging in the values for \(x_i\) gives
    \begin{align}
        s_1^{\prime\prime}(x) &= M_0 \frac{x - 1}{0 - 1} + M_1 \frac{x - 0}{1 - 0} \\
        &= M_0 + (M_1 - M_0) x \\
        s_2^{\prime\prime}(x) &= M_1 \frac{x - 2}{1 - 2} + M_2 \frac{x - 1}{2 - 1} \\
        &= 2 M_1 - M_2 + (M_2 - M_1) x \\
        s_3^{\prime\prime}(x) &= M_2 \frac{x - 3}{2 - 3} + M_3 \frac{x - 2}{3 - 2} \\
        &= 3 M_2 -2 M_3 + (M_3 - M_2) x \text{.}
        \intertext{Now we integrate.}
        \int s_1^{\prime\prime}(x) \diff x = s_1^\prime (x) &= C_1^\prime + M_0 x + \frac{M_1 - M_0}{2} x^2 \\
        \int s_2^{\prime\prime}(x) \diff x = s_2^\prime(x) &= C_2^\prime + (2M_1 - M_2) x + \frac{M_2 - M_1}{2} x^2 \\
        \int s_3^{\prime\prime}(x) \diff x = s_3^\prime (x) &= C_3^\prime + (3M_2 - 2M_3) x + \frac{M_3 - M_2}{2} x^2 \text{.}
        \intertext{One more time to determine the constants.}
        \int s_1^\prime (x) = s_1 (x) &= C_1 + C_1^\prime x + \frac{M_0}{2} x^2 + \frac{M_1 - M_0}{6} x^3 \\
        \int s_2^\prime (x) = s_2 (x) &= C_2 + C_2^\prime x + \frac{2M_1 - M_2}{2} x^2 + \frac{M_2 - M_1}{6} x^3 \\
        \int s_3^\prime (x) = s_3 (x) &= C_3 + C_3^\prime x + \frac{3M_2 - 2M_3}{2} x^2 + \frac{M_3 - M_2}{6} x^3
        \intertext{With the given values \((0, 0)\), \((1, 0.5)\), \((2, 2)\), \((3, 1.5)\) we have}
        s_1(0) = 0 &= C_1 \\
        s_1(1) = 0.5 & = C_1 + \frac{1}{2} C_1^\prime + \frac{M_0}{8} + \frac{M_1 - M_0}{48} \\
        s_2(1) = 0.5 &= FUCK IT WE KNOW HOW THIS WORKS
        %
        %
        %
        %
        \intertext{With the given values, we get}
        s_1^\prime(0) = 0.2 &= C_1 \\
        s_3^\prime(3) = -1 &= C_3 + (3M_2 - 2M_3) \cdot 3+ \frac{M_3 - M_2}{2} \cdot 3^2 \\
        &= C_3 + \frac{9}{2} M_2 - \frac{3}{2} M_3 \text{.}
    \end{align}
    We also have
    \begin{align}
        s_1^\prime (1) = s_2^\prime(1) & \iff C_1 + M_0 + \frac{M_1 - M_0}{2} = C_2 + (2 M_1 - M_2) + \frac{M_2 - M_1}{2} \\
        & \iff 0 = C_1 - C_2 + M_0 + \frac{1}{2} (M_1 - M_0) - \frac{3}{2} (M_1 - M_2) \\
        s_2^\prime (2) = s_3^\prime(2) & \iff C_2 + 2(2 M_1 - M_2) + 2(M_2 - M_1) = C_3 + 2(3M_2 - 2M_3) + 2(M_3 - M_2)
    \end{align}

    We start by finding \(s_1\). It is \(s_1(x) = a + bx + cx^2 + dx^3\) and \(s_1^\prime (x) = b + cx + dx^2\) for some \(a, b, c, d \in \mathbb{R}\). Plugging in the conditions gives
    \begin{align}
        s_1(0) &= a = 0 \\
        s_1(1) &= a + b + c + d = 0.5 \\
        s_1^\prime(0) &= b = 0.2
    \end{align}
\end{example}

\begin{example}
    Given the data set \((0, 0)\), \((1, 0.5)\), \((2, 2)\) and \((3, 1.5)\) we want to find the interpolating cubic spline \(s(x)\) satisfying \(s^\prime (0) = 0.2\) and \(s^\prime (3) = -1\).
\end{example}