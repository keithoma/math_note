\documentclass[a4paper]{article}
\title{About Set Operations}
\author{K}


% ---------------------------------------------------------------------
% P A C K A G E S
% ---------------------------------------------------------------------

% typography and formatting
\usepackage[english]{babel}
\usepackage[utf8]{inputenc}
\usepackage{geometry}
\usepackage{exsheets}
\usepackage{environ}

% mathematics
\usepackage{amsthm} % for theorems, and definitions
\usepackage{amssymb}
\usepackage{amsmath}
\usepackage{textcomp}
%\usepackage{MnSymbol} % for \cupdot

% extra
\usepackage{xcolor}
\usepackage{tikz}

% ---------------------------------------------------------------------
% S E T T I N G
% ---------------------------------------------------------------------

% typography and formatting
\geometry{margin=3cm}

\SetupExSheets{
  counter-format = ch.qu,
  counter-within = chapter,
  question/print = true,
  solution/print = true,
}

% mathematics
\theoremstyle{definition}
\newtheorem{definition}{Definition}[section]
\newtheorem{example}{Example}[definition]

\newtheorem{theorem}{Theorem}[definition]
\newtheorem{corollary}{Corollary}
\newtheorem{lemma}{Lemma}[definition]
\newtheorem{proposition}{Proposition}[definition]

\newtheorem*{remark}{Remark}

% extra
\definecolor{mathif}{HTML}{0000A0} % for conditions
\definecolor{maththen}{HTML}{CC5500} % for consequences
\definecolor{mathrem}{HTML}{8b008b} % for notes

\usetikzlibrary{positioning}
\usetikzlibrary{shapes.geometric, arrows}

% ---------------------------------------------------------------------
% C O M M A N D S
% ---------------------------------------------------------------------

\newcommand{\norm}[1]{\left\lVert#1\right\rVert}
\newcommand{\rank}{\text{rank}}
\newcommand{\Vol}{\text{Vol}}
\newcommand*\diff{\mathop{}\!\mathrm{d}}
\newcommand*\Diff{\mathop{}\!\mathrm{D}}

\newcommand\restr[2]{{% we make the whole thing an ordinary symbol
  \left.\kern-\nulldelimiterspace % automatically resize the bar with \right
  #1 % the function
  \vphantom{\big|} % pretend it's a little taller at normal size
  \right|_{#2} % this is the delimiter
  }}

% ---------------------------------------------------------------------
% R E N D E R
% ---------------------------------------------------------------------

\newif\ifshowproof
\showprooftrue

\NewEnviron{Proof}{%
    \ifshowproof%
        \begin{proof}%
            \BODY
        \end{proof}%
    \fi%
}%

\begin{document}
\begin{enumerate}
  \item true
  \item \(A \uparrow B\)
\end{enumerate}
%
\section{Sheffer Stroke}
\begin{definition}
  \textit{\(A \uparrow B\) if \(x\) is not in the intersection of \(A\) and \(B\).}
  \begin{align}
    A \uparrow B = X \setminus (A \cap B) = (X \setminus A) \cup (X \setminus B)
  \end{align}
\end{definition}
%
\begin{proposition}
  \begin{enumerate}
    \item Sheffer stroke is associative.
    \item Sheffer stroke is commutative.
    \item For \(X\) the only identity element is \(\varnothing\).
  \end{enumerate}
\end{proposition}
\begin{Proof}
  Let \(A, B, C \subset X\). We have
  \begin{align}
    (A \uparrow B) \uparrow C =& (X \setminus (A \uparrow B)) \cup X \setminus C \\
    =& (X \setminus A) \cup (X \setminus B) \cup (X \setminus C) \\
    =& (X \setminus A) \cup ((X \setminus B) \cup (X \setminus C)) \\
    =& A \uparrow ((X \setminus B) \cup (X \setminus C)) \\
    =& A \uparrow (B \uparrow C)
  \end{align}
  \begin{align}
    A \uparrow B =& (X \setminus A) \cup (X \setminus B) \\
    =& (X \setminus B) \cup (X \setminus A) \\
    =& B \uparrow A
  \end{align}
  \begin{align}
    X \uparrow \varnothing =& X \setminus (X \cap \varnothing) \\
    =& X \setminus \varnothing \\
    =& X
  \end{align}
\end{Proof}
\begin{corollary}
  Sheffer stroke does not form a group on the power set of \(X\).
\end{corollary}
%
\end{document}
