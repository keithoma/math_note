\documentclass[a4paper]{article}
\title{Strong Factorial Conjecture}


% ---------------------------------------------------------------------
% P A C K A G E S
% ---------------------------------------------------------------------

% typography and formatting
\usepackage[english]{babel}
\usepackage[utf8]{inputenc}
\usepackage{geometry}
\usepackage{exsheets}
\usepackage{environ}
\usepackage{graphicx}
\usepackage{cutwin}
\usepackage{pifont}

% mathematics
\usepackage{xfrac}  
\usepackage{amsthm} % for theorems, and definitions
\usepackage{amssymb}
\usepackage{amsmath}
\usepackage{textcomp}
\usepackage{mathtools}
\usepackage{mleftright} % for scaling mid bar in sets
% \usepackage{MnSymbol} % for \cupdot

% extra
\usepackage{xcolor}
\usepackage{tikz}

% ---------------------------------------------------------------------
% S E T T I N G
% ---------------------------------------------------------------------

%maybe delete later, for colorbox
\usepackage{tcolorbox}
\newtcolorbox{defbox}{colback=blue!5!white,colframe=blue!75!black}
\newtcolorbox{defboxlight}{colback=cyan!5!white,colframe=cyan!75!black}
\newtcolorbox{thmbox}{colback=orange!5!white,colframe=orange!75!black}
\newtcolorbox{rembox}{colback=purple!5!white,colframe=purple!75!black}
\newtcolorbox{exmbox}{colback=gray!5!white,colframe=gray!75!black}
\newtcolorbox{intbox}{colback=violet!5!white,colframe=violet!75!black}

% typography and formatting
\geometry{margin=3cm}

\SetupExSheets{
  counter-format = ch.qu,
  counter-within = chapter,
  question/print = true,
  solution/print = true,
}

% mathematics
\newcounter{global}

\theoremstyle{definition}
\newtheorem{definition}{Definition}[]
\newtheorem{example}{Example}[definition]

\newtheorem{theorem}[definition]{Theorem}
\newtheorem{corollary}{Corollary}
\newtheorem{lemma}[definition]{Lemma}
\newtheorem{proposition}[definition]{Proposition}

\newtheorem*{remark}{Remark}
\newtheorem*{intuition}{Intuition}

% extra
\definecolor{mathif}{HTML}{0000A0} % for conditions
\definecolor{maththen}{HTML}{CC5500} % for consequences
\definecolor{mathrem}{HTML}{8b008b} % for notes
\definecolor{mathobj}{HTML}{008800}

\usetikzlibrary{positioning}
\usetikzlibrary{shapes.geometric, arrows}

% ---------------------------------------------------------------------
% C O M M A N D S
% ---------------------------------------------------------------------

\newcommand{\norm}[1]{\left\lVert#1\right\rVert}
\newcommand{\rank}{\text{rank}}
\newcommand{\Vol}{\text{Vol}}

\newcommand{\set}[1]{\mleft\{\, #1 \,\mright\}}
\newcommand{\makeset}[2]{\mleft\{\, #1 \; \middle| \; #2 \,\mright\}}

\newcommand*\diff{\mathop{}\!\mathrm{d}}
\newcommand*\Diff{\mathop{}\!\mathrm{D}}

\newcommand\restr[2]{{% we make the whole thing an ordinary symbol
  \left.\kern-\nulldelimiterspace % automatically resize the bar with \right
  #1 % the function
  \vphantom{\big|} % pretend it's a little taller at normal size
  \right|_{#2} % this is the delimiter
  }}

% ---------------------------------------------------------------------
% R E N D E R
% ---------------------------------------------------------------------

\newif\ifshowproof
\showprooftrue

\NewEnviron{Proof}{%
    \ifshowproof%
        \begin{proof}%
            \BODY
        \end{proof}%
    \fi%
}%

\begin{document}
% \maketitle
% \tableofcontents

My notes on "The Strong Factorial Conjecture" by Eric Edo and Arno van den Essen. See: https://arxiv.org/abs/1304.3956

\begin{theorem}[Conjecture 2.13]
    Let \(a(X) \in \mathbb{C}[X]\) be a polynomial of degree less or equal to \(m + 1 \in \mathbb{N}_+\) such that \(a(X) \equiv X \mod{X^2}\). If the first \(m\) consecutive coefficient of the compositional inverse \(a^{-1}(X)\) vanish, then \(a(X) = X\).
\end{theorem}

\begin{theorem}[Conjecture 2.14]
    Let \(a(X) \in \mathbb{C}[X]\) be a polynomial of degree less or equal to \(m + 1 \in \mathbb{N}_+\) such that \(a(X) \equiv X \mod{X^2}\). If the coefficients of \(X^{n+1}, \ldots, X^{n+m}\) of the compositional inverse vanish, then \(a(X) = X\).
\end{theorem}

\begin{remark}
    \(R(m)\) if and only if \(R(m)_n\) for all \(n \in \mathbb{N}_+\).
\end{remark}

\begin{proof}
    Let \(R(m)\) be true for a \(m \in \mathbb{N}_0\).

    Then \(R(m)_1\) is true, i.e. if \(\text{deg}(a) \leq m + 1\) and if the
\end{proof}

\begin{remark}
    If we denote the polynomial \(a(X)\) by \(\sum_{k \in \mathbb{N}_0}a_k X^k\) for some \(a_k \in \mathbb{C}\) for all \(k \in \mathbb{N}_0\), then the condition \(a(X) \equiv X \mod{X^2}\) amounts to \(a_0 = 0\) and \(a_1 = 1\).

    Moreover, we have this:

    A power series has a compositional inverse if and only if \(a_1 \neq 0\). In that case, the inverse is unique.

    See

    https://www.amazon.com/dp/B00HMUGS4S

    https://math.stackexchange.com/questions/2520744/finding-compositional-inverses-for-formal-power-series

    My questions:

    \begin{enumerate}
        \item What if \(a_0 \neq 0\)? Pick \(a_0 = 3\).
        
        Let \(f \in \mathbb{C}[X]\) be a polynomial with \(a_0 \neq 0\). Then we may write \(f(X) = g(X) + a_0\) where \(g\) has a compositional inverse. Thus it it
        \begin{align*}
            g^{-1}(g(X) + a_0) &= g^{-1}(g(X)) + g^{-1}(a_0) \\
            &= X + g^{-1}(a_0) \\
        \end{align*}
        \begin{align*}
            h(X) &= g^{-1}(X) + g^{-1}(a_0) \\
            h(f(X)) &= h(g(X) + a_0) \\
            &= g^{-1}(g(X) + a_0) + g^{-1}(a_0) \\
            &= X
        \end{align*}

        Let \(f \in \mathbb{C}[X]\) be a polynomial with \(a_1 \neq 1\) and \(a_1 \neq 0\). Then we may write \(f(X) =\)
    \end{enumerate}
\end{remark}

https://www.math.uwaterloo.ca/~dgwagner/co430I.pdf

proof



\begin{proposition}
    \begin{enumerate}
        \item The polynomial \(a(X)\) is invertible for the composition.
        \item For all \(i \in \set{1, \ldots, \text{deg}(a - 1)}\), the coefficient \(a_i\) is nilpotent element in \(A\).
        I just don't see this ...
    \end{enumerate}
\end{proposition}

\begin{lemma}[Lagrange Inversion Formula]
    Let \(K\) be a field of charateristic
\end{lemma}

\begin{example}[See 5.4.4]
    \(f(X) = X e^{-X} = X \sum_{k=0}^\infty \frac{(-1)^k}{k!} X^k\)
    \begin{align*}
        [X^n]f^{-1}(X) = \frac{1}{n} [X^{n-1}] e^{nX}
    \end{align*}
\end{example}


\begin{lemma}[Lemma 2.20 (Additive Inversion Formula)]
    Let \(\alpha_1, \ldots, \alpha_m \in \mathbb{C}\) be complex numbers. The formal inverse of \(a(X) = X(1 - (\alpha_1 X + \cdots + \alpha_m X^m))\) is given by the following formula
    \begin{align*}
        a^{-1}(X) &= X \mleft( 1 + \frac{1}{n + 1} \sum_{n \geq 1} u_n X^n \mright)
        \intertext{where}
        u_n &= \frac{1}{n!} \sum_{j_1 + 2 j_2 + \cdots + m j_m = n} \frac{(n + j_1 + \cdots + j_m)!}{j_1 ! \cdots j_m!} \alpha_1^{j_1} \cdots \alpha_m^{j_m}
    \end{align*}
\end{lemma}

\begin{proposition}[Proposition 2.23]
    Let \(\alpha_1, \ldots, \alpha_m \in \mathbb{C}\) be complex numbers and let \((u_n)_{n \in \mathbb{N}_+}\) be a sequence defined by AIF in Lemma 2.20. For all \(n \in \mathbb{N}_+\), the Rigidity Conjecture \(R(m)_n\) is equivalent to the following implication: If \(u_n = \cdots = u_{n + m - 1} = 0\) then \(\alpha_1 = \cdots = \alpha_m = 0\).
\end{proposition}

\begin{proof}
    
\end{proof}

\begin{theorem}
    \begin{enumerate}
        \item The inclusion \(E^{[m]} \subset F_n^{[m]}\) implies \(R(m)_n\)
    \end{enumerate}
\end{theorem}

\begin{definition}
    \begin{align*}
        E^{[m]} & = \makeset{X_1 \cdots X_m (\mu_1 X_1 + \cdots + \mu_m X_m)}{\mu_1, \ldots, \mu_m \in \mathbb{C}} \subset \\
        F^{[m]}_n &= \makeset{f \in \mathbb{C}^{[m]} \setminus \{0\}}{\mathcal{L}(f^k) \neq 0 \text{ for some } n \leq k \leq \mathcal{N}(f) - 1} \cup \{0\}
    \end{align*}
\end{definition}

\end{document}