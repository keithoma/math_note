\documentclass[a4paper]{article}
\title{Strong Factorial Conjecture}


% ---------------------------------------------------------------------
% P A C K A G E S
% ---------------------------------------------------------------------

% typography and formatting
\usepackage[english]{babel}
\usepackage[utf8]{inputenc}
\usepackage{geometry}
\usepackage{exsheets}
\usepackage{environ}
\usepackage{graphicx}
\usepackage{cutwin}
\usepackage{pifont}

% mathematics
\usepackage{xfrac}  
\usepackage{amsthm} % for theorems, and definitions
\usepackage{amssymb}
\usepackage{amsmath}
\usepackage{textcomp}
\usepackage{mathtools}
% \usepackage{MnSymbol} % for \cupdot

% extra
\usepackage{xcolor}
\usepackage{tikz}

% ---------------------------------------------------------------------
% S E T T I N G
% ---------------------------------------------------------------------

%maybe delete later, for colorbox
\usepackage{tcolorbox}
\newtcolorbox{defbox}{colback=blue!5!white,colframe=blue!75!black}
\newtcolorbox{defboxlight}{colback=cyan!5!white,colframe=cyan!75!black}
\newtcolorbox{thmbox}{colback=orange!5!white,colframe=orange!75!black}
\newtcolorbox{rembox}{colback=purple!5!white,colframe=purple!75!black}
\newtcolorbox{exmbox}{colback=gray!5!white,colframe=gray!75!black}
\newtcolorbox{intbox}{colback=violet!5!white,colframe=violet!75!black}

% typography and formatting
\geometry{margin=3cm}

\SetupExSheets{
  counter-format = ch.qu,
  counter-within = chapter,
  question/print = true,
  solution/print = true,
}

% mathematics
\newcounter{global}

\theoremstyle{definition}
\newtheorem{definition}{Definition}[]
\newtheorem{example}{Example}[definition]

\newtheorem{theorem}[definition]{Theorem}
\newtheorem{corollary}{Corollary}
\newtheorem{lemma}[definition]{Lemma}
\newtheorem{proposition}[definition]{Proposition}

\newtheorem*{remark}{Remark}
\newtheorem*{intuition}{Intuition}

% extra
\definecolor{mathif}{HTML}{0000A0} % for conditions
\definecolor{maththen}{HTML}{CC5500} % for consequences
\definecolor{mathrem}{HTML}{8b008b} % for notes
\definecolor{mathobj}{HTML}{008800}

\usetikzlibrary{positioning}
\usetikzlibrary{shapes.geometric, arrows}

% ---------------------------------------------------------------------
% C O M M A N D S
% ---------------------------------------------------------------------

\newcommand{\norm}[1]{\left\lVert#1\right\rVert}
\newcommand{\rank}{\text{rank}}
\newcommand{\Vol}{\text{Vol}}

\newcommand{\set}[1]{\left\{\, #1 \,\right\}}
\newcommand{\makeset}[2]{\left\{\, #1 \mid #2 \,\right\}}

\newcommand*\diff{\mathop{}\!\mathrm{d}}
\newcommand*\Diff{\mathop{}\!\mathrm{D}}

\newcommand\restr[2]{{% we make the whole thing an ordinary symbol
  \left.\kern-\nulldelimiterspace % automatically resize the bar with \right
  #1 % the function
  \vphantom{\big|} % pretend it's a little taller at normal size
  \right|_{#2} % this is the delimiter
  }}

% ---------------------------------------------------------------------
% R E N D E R
% ---------------------------------------------------------------------

\newif\ifshowproof
\showprooftrue

\NewEnviron{Proof}{%
    \ifshowproof%
        \begin{proof}%
            \BODY
        \end{proof}%
    \fi%
}%

\begin{document}
% \maketitle
% \tableofcontents

My notes on "The Strong Factorial Conjecture" by Eric Edo and Arno van den Essen. See: https://arxiv.org/abs/1304.3956

\begin{theorem}[Conjecture]
    Let \(a(X) \in \mathbb{C}[X]\) be a polynomial of degree less or equal to \(m + 1 \in \mathbb{N}_+\) such that \(a(X) \equiv X \mod{X^2}\). If the first \(m\) consecutive coefficient of the compositional inverse \(a^{-1}(X)\) vanish, then \(a(X) = X\).
\end{theorem}
\begin{remark}
    If we denote the polynomial \(a(X)\) by \(\sum_{k \in \mathbb{N}_0}a_k X^k\) for some \(a_k \in \mathbb{C}\) for all \(k \in \mathbb{N}_0\), then the condition \(a(X) \equiv X \mod{X^2}\) amounts to \(a_0 = 0\) and \(a_1 = 1\).

    Moreover, we have this:

    A power series has a compositional inverse if and only if \(a_1 \neq 0\). In that case, the inverse is unique.

    See

    https://www.amazon.com/dp/B00HMUGS4S

    https://math.stackexchange.com/questions/2520744/finding-compositional-inverses-for-formal-power-series

    My questions:

    \begin{enumerate}
        \item What if \(a_0 \neq 0\)? Pick \(a_0 = 3\).
        
        Let \(f \in \mathbb{C}[X]\) be a polynomial with \(a_0 \neq 0\). Then we may write \(f(X) = g(X) + a_0\) where \(g\) has a compositional inverse. Thus it it
        \begin{align*}
            g^{-1}(g(X) + a_0) &= g^{-1}(g(X)) + g^{-1}(a_0) \\
            &= X + g^{-1}(a_0) \\
        \end{align*}
        \begin{align*}
            h(X) &= g^{-1}(X) + g^{-1}(a_0) \\
            h(f(X)) &= h(g(X) + a_0) \\
            &= g^{-1}(g(X) + a_0) + g^{-1}(a_0) \\
            &= X
        \end{align*}

        Let \(f \in \mathbb{C}[X]\) be a polynomial with \(a_1 \neq 1\) and \(a_1 \neq 0\). Then we may write \(f(X) =\)
    \end{enumerate}
\end{remark}

\item we can probably modify the inverse such that we don't really need to care for the other cases not covered in the conjecture above

\end{document}