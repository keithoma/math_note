\documentclass[a4paper]{article}
\title{Strong Factorial Conjecture}


% ---------------------------------------------------------------------
% P A C K A G E S
% ---------------------------------------------------------------------

% typography and formatting
\usepackage[english]{babel}
\usepackage[utf8]{inputenc}
\usepackage{geometry}
\usepackage{exsheets}
\usepackage{environ}
\usepackage{graphicx}
\usepackage{cutwin}
\usepackage{pifont}

% mathematics
\usepackage{xfrac}  
\usepackage{amsthm} % for theorems, and definitions
\usepackage{amssymb}
\usepackage{amsmath}
\usepackage{textcomp}
\usepackage{mathtools}
\usepackage{mleftright} % for scaling mid bar in sets
% \usepackage{MnSymbol} % for \cupdot

% extra
\usepackage{xcolor}
\usepackage{tikz}

% ---------------------------------------------------------------------
% S E T T I N G
% ---------------------------------------------------------------------

%maybe delete later, for colorbox
\usepackage{tcolorbox}
\newtcolorbox{defbox}{colback=blue!5!white,colframe=blue!75!black}
\newtcolorbox{defboxlight}{colback=cyan!5!white,colframe=cyan!75!black}
\newtcolorbox{thmbox}{colback=orange!5!white,colframe=orange!75!black}
\newtcolorbox{rembox}{colback=purple!5!white,colframe=purple!75!black}
\newtcolorbox{exmbox}{colback=gray!5!white,colframe=gray!75!black}
\newtcolorbox{intbox}{colback=violet!5!white,colframe=violet!75!black}

% typography and formatting
\geometry{margin=3cm}

\SetupExSheets{
  counter-format = ch.qu,
  counter-within = chapter,
  question/print = true,
  solution/print = true,
}

% mathematics
\newcounter{global}

\theoremstyle{definition}
\newtheorem{definition}{Definition}[]
\newtheorem{example}{Example}[definition]

\newtheorem{theorem}[definition]{Theorem}
\newtheorem{corollary}{Corollary}
\newtheorem{lemma}[definition]{Lemma}
\newtheorem{proposition}[definition]{Proposition}

\newtheorem*{remark}{Remark}
\newtheorem*{intuition}{Intuition}

% extra
\definecolor{mathif}{HTML}{0000A0} % for conditions
\definecolor{maththen}{HTML}{CC5500} % for consequences
\definecolor{mathrem}{HTML}{8b008b} % for notes
\definecolor{mathobj}{HTML}{008800}

\usetikzlibrary{positioning}
\usetikzlibrary{shapes.geometric, arrows}

% ---------------------------------------------------------------------
% C O M M A N D S
% ---------------------------------------------------------------------

\newcommand{\norm}[1]{\left\lVert#1\right\rVert}
\newcommand{\rank}{\text{rank}}
\newcommand{\Vol}{\text{Vol}}

\newcommand{\set}[1]{\mleft\{\, #1 \,\mright\}}
\newcommand{\makeset}[2]{\mleft\{\, #1 \; \middle| \; #2 \,\mright\}}

\newcommand*\diff{\mathop{}\!\mathrm{d}}
\newcommand*\Diff{\mathop{}\!\mathrm{D}}

\newcommand\restr[2]{{% we make the whole thing an ordinary symbol
  \left.\kern-\nulldelimiterspace % automatically resize the bar with \right
  #1 % the function
  \vphantom{\big|} % pretend it's a little taller at normal size
  \right|_{#2} % this is the delimiter
  }}

% ---------------------------------------------------------------------
% R E N D E R
% ---------------------------------------------------------------------

\newif\ifshowproof
\showprooftrue

\NewEnviron{Proof}{%
    \ifshowproof%
        \begin{proof}%
            \BODY
        \end{proof}%
    \fi%
}%

\begin{document}

\section{Rigidity Conjecture}
\begin{remark}
    When studying compositions of formal power series, we require that the inner power series \(f(X)\) to have no constant term, i.e., \(f(0) = 0\).
    \begin{align*}
        f(g(X)) &= a_0 + a_1 (b_0 + b_1 X + b_2 X^2 + \cdots) + a_2 (b_0 + b_1 X + b_2 X^2 + \cdots)^2 + \cdots \\
        &= a_0 + (a_1 b_0 + a_1 b_1 X + a_1 b_2 X^2 + \cdots) + (a_2 b_0^2 + 2 a_2 b_0 b_1 X + (2 a_2 b_0 b_2 + a_2 b_1^2) X^2 + \cdots) \\
        &= a_0 + a_1 b_0 + a_2 b_0^2 + \cdots
    \end{align*}
\end{remark}
\begin{definition}
    Let \(f(X) \in \mathbb{C}[[X]]\) be a power series. We call a power series \(f^{-1}(X) \in \mathbb{C}[[X]]\) the compositional inverse of \(f\), if it satisfies \(f(f^{-1}(X)) = f^{-1}(f(X)) = X\). 
\end{definition}
\begin{proposition}
    A power series \(f(X) = a_0 + a_1 X + \cdots\in \mathbb{C}[[X]]\) has a compositional inverse if and only if \(a_0 = 0\) and \(a_1 \neq 0\). Moreover, if the compositional inverse exists, then it is unique. 
\end{proposition}
\begin{proof}
    Assume \(f\) has a compositional inverse and denote the compositional inverse by \(f^{-1}(X) = b_0 + b_1 X + b_2 X^2 + \cdots\). Writing out \(f(f^{-1}(X)) = X\) using multinomial thoerem gives
    \begin{align*}
        X &= a_0 + a_1 (b_0 + b_1 X + b_2 X^2 + \cdots) + a_2 (b_0 + b_1 X + b_2 X^2 + \cdots)^2 + \cdots \\
        &= a_0 + (a_1b_0 + a_1 b_1 X + a_1 b_2 X^2 + \cdots) + (a_2 b_0^2 + 2 a_ 2 b_0 b_1 X + \cdots) \text{.}
    \end{align*}
    Equating the coefficients on both sides yields a linear system of equations.
    \begin{align*}
        0 &= a_0 + a_1 b_0 + a _2 b_0^2 + \cdots \\
        1 &= a_1 b_1 +
    \end{align*}
\end{proof}

\end{document}