\documentclass[a4paper]{article}
\title{Strong Factorial Conjecture}


% ---------------------------------------------------------------------
% P A C K A G E S
% ---------------------------------------------------------------------

% typography and formatting
\usepackage[english]{babel}
\usepackage[utf8]{inputenc}
\usepackage{geometry}
\usepackage{exsheets}
\usepackage{environ}
\usepackage{graphicx}
\usepackage{cutwin}
\usepackage{pifont}

% mathematics
\usepackage{xfrac}  
\usepackage{amsthm} % for theorems, and definitions
\usepackage{amssymb}
\usepackage{amsmath}
\usepackage{textcomp}
\usepackage{mathtools}
\usepackage{mleftright} % for scaling mid bar in sets
% \usepackage{MnSymbol} % for \cupdot

% extra
\usepackage{xcolor}
\usepackage{tikz}

% ---------------------------------------------------------------------
% S E T T I N G
% ---------------------------------------------------------------------

%maybe delete later, for colorbox
\usepackage{tcolorbox}
\newtcolorbox{defbox}{colback=blue!5!white,colframe=blue!75!black}
\newtcolorbox{defboxlight}{colback=cyan!5!white,colframe=cyan!75!black}
\newtcolorbox{thmbox}{colback=orange!5!white,colframe=orange!75!black}
\newtcolorbox{rembox}{colback=purple!5!white,colframe=purple!75!black}
\newtcolorbox{exmbox}{colback=gray!5!white,colframe=gray!75!black}
\newtcolorbox{intbox}{colback=violet!5!white,colframe=violet!75!black}

% typography and formatting
\geometry{margin=3cm}

\SetupExSheets{
  counter-format = ch.qu,
  counter-within = chapter,
  question/print = true,
  solution/print = true,
}

% mathematics
\newcounter{global}

\theoremstyle{definition}
\newtheorem{definition}{Definition}[]
\newtheorem{example}{Example}[definition]

\newtheorem{theorem}[definition]{Theorem}
\newtheorem{corollary}{Corollary}
\newtheorem{lemma}[definition]{Lemma}
\newtheorem{proposition}[definition]{Proposition}

\newtheorem*{remark}{Remark}
\newtheorem*{intuition}{Intuition}

% extra
\definecolor{mathif}{HTML}{0000A0} % for conditions
\definecolor{maththen}{HTML}{CC5500} % for consequences
\definecolor{mathrem}{HTML}{8b008b} % for notes
\definecolor{mathobj}{HTML}{008800}

\usetikzlibrary{positioning}
\usetikzlibrary{shapes.geometric, arrows}

% ---------------------------------------------------------------------
% C O M M A N D S
% ---------------------------------------------------------------------

\newcommand{\norm}[1]{\left\lVert#1\right\rVert}
\newcommand{\rank}{\text{rank}}
\newcommand{\Vol}{\text{Vol}}

\newcommand{\set}[1]{\mleft\{\, #1 \,\mright\}}
\newcommand{\makeset}[2]{\mleft\{\, #1 \; \middle| \; #2 \,\mright\}}

\newcommand*\diff{\mathop{}\!\mathrm{d}}
\newcommand*\Diff{\mathop{}\!\mathrm{D}}

\newcommand\restr[2]{{% we make the whole thing an ordinary symbol
  \left.\kern-\nulldelimiterspace % automatically resize the bar with \right
  #1 % the function
  \vphantom{\big|} % pretend it's a little taller at normal size
  \right|_{#2} % this is the delimiter
  }}

% ---------------------------------------------------------------------
% R E N D E R
% ---------------------------------------------------------------------

\newif\ifshowproof
\showprooftrue

\NewEnviron{Proof}{%
    \ifshowproof%
        \begin{proof}%
            \BODY
        \end{proof}%
    \fi%
}%

\begin{document}

\section{Rigidity Conjecture}
\begin{proposition}[Division]
    A formal power series \(f(X) = \sum_{k \geq 1} a_k X^k\in \mathbb{C}[[X]]\) is invertible if and only if its constant coefficient \(a_0\) is nonzero.
\end{proposition}
\begin{lemma}
    Given two formal power series \(f(X) = \sum_{n \geq 0} a_k X^k \in \mathbb{C}[[X]]\) and \(g(X) = \sum_{n \geq 0} b_k X^k \in \mathbb{C}[[X]]\) with \(b_0 \neq 0\), we may compute their quotient
    \begin{align}
        \frac{\sum_{n \geq 0} a_k X^k}{\sum_{n \geq 0} b_k X^k} = \sum_{n \geq 0} c_k X^k
    \end{align}
    by
    \begin{align}
        c_n = \frac{1}{b_0} \left(a_n - \sum_{k \geq 1} b_k c_{n - k}\right) \text{.}
    \end{align}
\end{lemma}
\begin{proof}
    \#MISSING
\end{proof}
\begin{remark}
    When studying compositions of formal power series, we require that the inner power series \(f(X)\) has no constant term, i.e., \(f(0) = 0\). This condition ensures that the resulting composition is well-defined in the ring of formal power series \(\mathbb{C}[[X]]\), as it prevents infinite contributions to the coefficients.

    Consider \(f(X) = \sum_{k \geq 1} a_k X^k\) and \(g(X) = \sum_{k \geq 0} b_k X^k\). The composition \(g(f(X))\) is given by substituting \(f(X)\) into \(g(X)\):
    \begin{align*}
        g(f(X)) &= b_0 + b_1 f(X) + b_2 f(X)^2 + \cdots \\
        &= b_0 + b_1 (a_1 X + a_2 X^2 + \cdots) + b_2 (a_1 X + a_2 X^2 + \cdots)^2 + \cdots \\
        &= b_0 + b_1 a_1 X + (b_1 a_2 + b_2 a_1^2) X^2 + \cdots,
    \end{align*}
    where we grouped the terms by powers of \(X\) in the last step. We observe that the coefficients of \(X^n\) in \(g(f(X))\) depend only on a finite number of coefficients of \(f(X)\) and \(g(X)\). This is because, with \(f(0) = 0\), each power \(f(X)^k\) introduces terms of degree at least \(k\), ensuring that lower-degree terms do not contribute infinitely to higher-order coefficients.

    On the other hand, if \(f(0) \neq 0\), we write \(f(X) = a_0 + \sum_{k \geq 1} a_k X^k\), where \(a_0 = f(0)\). In this case,
    \begin{align*}
        f(X)^k = (a_0 + a_1 X + a_2 X^2 + \cdots)^k
    \end{align*}
    produces a constant term \(a_0^k \neq 0\). Consequently, the constant term of \(g(f(X))\) depends on infinitely many terms of \(g(X)\), and the composition \(g(f(X))\) is no longer a formal power series.

    Since we are interested in the compositional inverse, it is necessary to extend the condition \(f(0) = 0\) to both power series. This ensures that the inverse series \(f^{-1}(X)\), when substituted into \(f(X)\), results in the identity series \(X\), with no contributions from constant terms that would otherwise make the series ill-defined.
\end{remark}

The following proposition and lemma are taken from Enumerative Combinatorics by Richard P. Stanley and Sergey Fomin.

\begin{definition}
    Let \(f(X) \in \mathbb{C}[[X]]\) be a power series with no constant term. We call a power series \(f^{-1}(X) \in \mathbb{C}[[X]]\) the compositional inverse of \(f\), if it satisfies \(f(f^{-1}(X)) = f^{-1}(f(X)) = X\). 
\end{definition}
\begin{proposition}
    A power series \(f(X) = a_1 X + a_2 X^2 + \cdots\in \mathbb{C}[[X]]\) has a compositional inverse if and only if \(a_1 \neq 0\). Moreover, if the compositional inverse exists, then it is unique. 
\end{proposition}
\begin{proof}
    Assume \(f\) has a compositional inverse and denote the compositional inverse by \(f^{-1}(X) = b_1 X + b_2 X^2 + \cdots\). Writing out \(f(f^{-1}(X)) = X\) using multinomial thoerem gives
    \begin{align*}
        X &= a_1 (b_1 X + b_2 X^2 + \cdots) + a_2 (b_1 X + b_2 X^2 + \cdots)^2 + \cdots \\
        &= (a_1 b_1 X + a_1 b_2 X^2 + a_1 b_3 X^3 + \cdots) + (a_2 b_1^2 X^2 + 2 a_2 b_1 b_2 X^3 + \cdots) \\
        &= (a_1 b_1) X + (a_1 b_2 + a_2 b_1^2) X^2 + (a_1b_3 + 2 a_2 b_1 b_2 + a_3 b_1^3) X^3 \cdots\text{.}
    \end{align*}
    Equating the coefficients on both sides yields a linear system of equations.
    \begin{align*}
        1 &= a_1 b_1 \\
        0 &= a_1 b_2 + a_2 b_1^2 \\
        0 &= a_1b_3 + 2 a_2 b_1 b_2 + a_3 b_1^3 \\
        & \quad\vdots
    \end{align*}
    The first equation has a solution if and only if \(a_1 \neq 0\). In that case, the solution is unique. Then, the second equation can be solved uniquely for \(b_2\). By this process, we are able to solve the third equation for \(b_3\), the fourth for \(b_4\) and so on. Thus, \(f^{-1}(X)\) exists if and only if \(a_1 \neq 0\) and in that case, \(f^{-1}(X)\) is unique.
\end{proof}
\begin{lemma}[Lagrange Inversion Formula]
    Let \(f(X) = \sum_{k \geq 1} a_k X^k \in \mathbb{C}[[X]]\) be a power series with \(a_1 \neq 0\) and denote its composition inverse by \(f^{-1}(X) = \sum_{k \geq 1}b_k X^k \in \mathbb{C}[[X]]\). The coefficients of the inverse is given by the following formula.
    \begin{align*}
        b_k = \frac{1}{k} [X^{n - 1}] \left(\frac{X}{f(X)}\right)^k
    \end{align*}
\end{lemma}
\begin{proof}
    We begin by substituting \(f(X)\) into \(f^{-1}(X)\). It is
    \begin{align*}
        X = f^{-1}(f(X)) = \sum_{k \geq 1} b_k f(X)^k \text{.}
    \end{align*}
    Differentiating and subsequently taking the quotient with \(f(X)^n\) for \(n \in \mathbb{N}\) on both sides yields
    \begin{align*}
        1 &= \sum_{k \geq 1} k \cdot b_k \cdot f(X)^{k - 1} \cdot f'(X) \\
        \Rightarrow \quad \frac{1}{f(X)^n} &= \sum_{k \geq 1} k \cdot b_k \cdot \frac{f(X)^{k}}{f(X)^{n+1}} \cdot f'(X) \text{.}
    \end{align*}
    We want to take the coefficient of \(X^{-1}\) on both sides. For that, first notice that for \(k \neq n\) it is
    \begin{align*}
        \frac{1}{k - n} \frac{d}{dX} f(X)^{k - n} = f(X)^{k - n -1} f'(X) = \frac{f(X)^{k}}{f(X)^{n + 1}} f'(X) \text{.}
    \end{align*}
    For any Laurent series, its derivative has no \(X^{-1}\) term. Thus, for \(k \neq n\), it is
    \begin{align*}
        [X^{-1}]\frac{f(X)^{k}}{f(X)^{n + 1}} f'(X) = [X^{-1}] \frac{1}{k - n} \frac{d}{dX} f(X)^{k - n} = 0 \text{.}
    \end{align*}
    If we now take the coefficient of \(X^{-1}\) in \#REFMISSING, we get
    \begin{align}
        [X^{-1}] \frac{1}{f(X)^n} &= [X^{-1}] \sum_{k \geq 1} k \cdot b_k \cdot \frac{f(X)^{k}}{f(X)^{n+1}} \cdot f'(X) \\
        &= \sum_{k \geq 1} k \cdot b_k \cdot [X^{-1}] \frac{f(X)^{k}}{f(X)^{n+1}} \cdot f'(X) \\
        &= n \cdot b_n [X^{-1}] \frac{f(X)^n}{f(X)^{n+1}} \cdot f'(X) \\
        &= n \cdot b_n [X^{-1}] \frac{f'(X)}{f(X)} \\
        &= n \cdot b_n [X^{-1}] \frac{a_1 + 2 a_2 X + 3 a_3 X^2 + \cdots}{a_1 X + a_2 X^2 + a_3 X^3 + \cdots} \\
        &= n \cdot b_n [X^{-1}] \frac{1}{X} \frac{a_1 + 2 a_2 X + 3 a_3 X^2 + \cdots}{a_1 + a_2 X + a_3 X^2 + \cdots} \\
        &= n \cdot b_n
    \end{align}
    where we used the formula for power series division given in \#REFMISSING to compute the constant term of the quotient.
    \begin{align}
        \frac{1}{a_1} \left(a_1 - 0\right) = 1
    \end{align}
    Now, by shifting the power of the coefficient to be extracted, we get
    \begin{align}
        [X^{-1}] \frac{1}{f(X)^n} = [X^{n-1}] \frac{X^n}{f(X)^n} = [X^{n-1}] \left( \frac{X}{f(X)} \right)^n \text{.}
    \end{align}
    Finally, continuing from \#REFMIISING, we get
    \begin{align}
        n \cdot b_n &= [X^{-1}] \frac{1}{f(X)^n} = [X^{n-1}] \left( \frac{X}{f(X)} \right)^n \\
        \Rightarrow \quad b_n &= \frac{1}{n} [X^{n-1}] \left( \frac{X}{f(X)} \right)^n
    \end{align}
    as desired.
\end{proof}
% ==================================
\begin{lemma}[Additive Inversion Formula]
    For some \(n \in \mathbb{N}_+\), let \(a(X) = X(1 - (\alpha_1 X + \cdots + \alpha_m X^m)) \in \mathbb{C}[X]\) be a polynomial. The compositional inverse is given by
    \begin{align}
        a^{-1}(X) = X \left(1 + \sum_{n \geq 1} \frac{1}{n + 1} u_n X^n\right)
    \end{align}
    where
    \begin{align}
        u_n = \frac{1}{n!} \sum_{k_1 + 2k_2 + \cdots + m k_m = n} \frac{(n + k_1 + \cdots + k_m)!}{k_1! \cdots k_m!} \alpha_1^{k_1} \cdots \alpha_m^{k_m} \text{.}
    \end{align}
\end{lemma}
\begin{proof}
    We start with the expression for \(u_n\) given by the Lagrange inversion formula.
    \begin{align}
        u_n &= [X^n] \left(\frac{X}{a(X)} \right)^{n+1}  \label{eq:step1} \\
        &= [X^n] \left(\frac{1}{1 - (\alpha_1 X + \cdots + \alpha_m X^m)}\right)^{n+1} \label{eq:step2} \\
        &= [X^n] \sum_{k \geq 0} \binom{n+k}{n} \left(\alpha_1 X + \cdots + \alpha_m X^m \right)^k \label{eq:step3} \\
        &= [X^n] \sum_{k \geq 0} \binom{n+k}{n} \sum_{k_1 + \cdots k_m = k} \frac{k!}{k_1! \cdots k_m!} (\alpha_1 X)^{k_1} \cdots (\alpha_m X^m)^{k_m} \label{eq:step4} \\
        &= \frac{1}{n!} [X^n] \sum_{k \geq 0} (n + k)! \sum_{k_1 + \cdots k_m = k} \frac{1}{k_1! \cdots k_m!} \alpha_1^{k_1} \cdots \alpha_m^{k_m} X^{k_1 + \cdots m k_m} \label{eq:step5} \\
        &= \frac{1}{n!} \sum_{k_1 + 2k_2 + \cdots + m k_m = n} \frac{(n + k_1 + \cdots + k_m)!}{k_1! \cdots k_m!} \alpha_1^{k_1} \cdots \alpha_m^{k_m} \label{eq:step6}
    \end{align}
    We have substituted \(a(X)\) in \eqref{eq:step2}, expanded the fraction using the binominal series formula for \((1-z)^{-(n+1)}\) (see Concrete Mathematics 5.56) in \eqref{eq:step3}, used the multinominal theorem in \eqref{eq:step4}, and finally collected terms for \([X^n]\) in \eqref{eq:step6}.
\end{proof}
%
\begin{lemma}[Multiplicative Inversion Formula]
    For some \(n \in \mathbb{N}_+\), let \(a(X) = X(1 - \mu_1 X) \cdots (1 - \mu_m X^m) \in \mathbb{C}[X]\) be a polynomial. The compositional inverse is given by
    \begin{align}
        a^{-1}(X) = X \left(1 + \sum_{n \geq 1} \frac{1}{n + 1} u_n X^n\right)
    \end{align}
    where
    \begin{align}
        u_n = \frac{1}{(n!)^m} \sum_{k_1 + \cdots + k_m = n} \frac{(n + k_1)! \cdots (n+k_m)!}{k_1! \cdots k_m!} \mu_1^{k_1} \cdots \mu_m^{k_m} \text{.}
    \end{align}
\end{lemma}
%
\begin{proof}
    \begin{align}
        u_n &= [X^n] \left(\frac{X}{a(X)}\right)^{n+1} \\
        &= [X^n] \left(\frac{1}{(1 - \mu_1 X) \cdots (1 - \mu_m X)}\right)^{n+1} \\
        &= [X^n] \prod_{i=1}^m \sum_{k_i \geq 0} \binom{n + k_i}{k_i} \mu_i^{k_i} X^{k_i} \\
        &= [X^n] \sum_{k_1, k_2, \ldots, k_m \geq 0} \left(\prod_{i=1}^m \binom{n + k_i}{k_i} \mu_i^{k_i}\right) X^{k_1 + k_2 + \cdots k_m} \\
        &= \sum_{k_1 + k_2 + \cdots k_m = n} \prod_{i=1}^m \binom{n+k_i}{k_i} \mu_i^{k_i} \\
        &= \frac{1}{(n!)^m} \sum_{k_1 + \cdots + k_m = n} \frac{(n + k_1)! \cdots (n+k_m)!}{k_1! \cdots k_m!} \mu_1^{k_1} \cdots \mu_m^{k_m} 
    \end{align}
    \begin{enumerate}
        \item substituted
        \item multinomial thoerem
        \item distribution law (not checked)
        \item extracting coefficients (not checked)
        \item simplyfing binom (not checked)
    \end{enumerate}
\end{proof}
\end{document}