\documentclass[refman]{scrartcl}

\usepackage[english]{babel}
\usepackage[utf8]{inputenc}

\usepackage{fancyhdr}
\usepackage{graphicx}
\usepackage[dvipsnames]{xcolor} 
\usepackage{epigraph}


\usepackage{amsmath}

\usepackage{listings}

\pagestyle{fancy}

\lstdefinestyle{mystyle}{
    backgroundcolor=\color{white},   
    commentstyle=\color{codegreen},
    keywordstyle=\color{Fuchsia},
    numberstyle=\tiny\color{gray},
    stringstyle=\color{codepurple},
    basicstyle=\footnotesize,
    breakatwhitespace=false,         
    breaklines=true,                 
    captionpos=b,                    
    keepspaces=true,                 
    numbers=left,                    
    numbersep=5pt,                  
    showspaces=false,                
    showstringspaces=false,
    showtabs=false,                  
    tabsize=2
}
 
\lstset{style=mystyle}


\newcommand{\mymod}{\text{\ \ mod\ \ }}

\begin{document}

% ----------------------------------------------------------------------------------------------------------
% PREAMBLE
% ----------------------------------------------------------------------------------------------------------

\begin{titlepage}
	\centering
	\includegraphics[width=0.15\textwidth]{graphics/huberlin_logo}\par\vspace{1cm}
	{\scshape\LARGE Humboldt University of Berlin \par}
	\vspace{1cm}
	{\scshape\Large Einf{\"u}hrung in das wissenschaftliche Rechnen \par}
	\vspace{1.5cm}
	{\huge\bfseries Documentation of Fraction Application Programming Interface and Command Line Interface Calculator\par}
	\vspace{2cm}
	{\Large\itshape Christian Parpart \& Kei Thoma \par}
	\vfill

	\vfill

% Bottom of the page
	{\large \today\par}
\end{titlepage}

\tableofcontents
\newpage

\section{User Manual}

\section{Documentation}

\subsection{tools3.py}

In this module, we have the two functions to compute the greatest common divisor and the least common multiple. Here, there are no classes, just free functions.

\begin{itemize}
    \item \texttt{ggt(arg1, arg2)}
    computes the greatest common divisor via Euclidean algorithm.
    \begin{itemize}
        \item Arguments
        \begin{enumerate}
            \item \texttt{arg1} (int): first integer
            \item \texttt{arg2} (int): second integer
        \end{enumerate}
        \item Returns (int): greatest common divisor of the first and second integer
    \end{itemize}
    \item \texttt{kgv(arg1,arg2)}
    determines the least common multiple, utilizing the greatest common divisor, computed by the function \texttt{ggt(arg1, arg2)}.
    \begin{itemize}
        \item Arguments
        \begin{enumerate}
            \item \texttt{arg1} (int): first integer
            \item \texttt{arg2} (int): second integer
        \end{enumerate}
        \item Returns (int): least common multiple of the first and second integer.
    \end{itemize}
    \item \texttt{main()} for testing purposes. Takes no arguments and returns none.
\end{itemize}

\subsection{bruch.py}

In this module, we have implemented the class Bruch that represents fractions.

\subsubsection{class Bruch()}

The objects of this class represent fractions.

\noindent\textbf{Attributes}

\begin{itemize}
    \item zaehler (int): the numerator
    \item nenner (int): the denominator
\end{itemize}

\noindent\textbf{Methodes}

\begin{itemize}
    \item \texttt{kuerzen(self)} reduces the fraction. Takes no arguments except for self and returns none.
    \item \texttt{\_\_add\_\_(self, other)} adds two fractions together via finding the greatest common divisor and reduces afterwards. The result is a new Bruch object.
    \begin{itemize}
        \item Arguments
        \begin{enumerate}
            \item other (Bruch): another fraction
        \end{enumerate}
        \item Returns (Bruch): the sum of the two fractions
    \end{itemize}
    \item \texttt{\_\_repr\_\_(self)} returns a printable string.
    \begin{itemize}
        \item Arguments: none except for self
        \item Returns (str): printable string
    \end{itemize}
    \item \texttt{check\_validity(self)} checks the fraction for validity. Returns false if the denominator is \(0\).
    \begin{itemize}
        \item Arguments: none except for self
        \item Returns (boolean): false if the denominator is \(0\), in any other case true
    \end{itemize}
\end{itemize}

\subsubsection{Free Functions}

\begin{itemize}
    \item \texttt{addiere(bruch\_1, bruch\_2)} adds two fractions into a new fractions
    \begin{itemize}
        \item Arguments
        \begin{enumerate}
            \item \texttt{bruch\_1} (Bruch): first summand
            \item \texttt{bruch\_2} (Bruch): second summand
        \end{enumerate}
        \item Returns (Bruch): the sum of the two fractions
    \end{itemize}
\end{itemize}

\begin{thebibliography}{9}
	\bibitem{bosch} 
	Bosch, Siegfried. 
	\textit{Algebra}. 
	Springer-Verlag Berlin Heidelberg, 7th Edition, 2009.
	
	\bibitem{knuth} 
	Knuth, Donald. 
	\textit{The Art of Computer Programming Volume 2}. 
	Prentice Hall, 3rd Edition, 1997.
	
\end{thebibliography}

\end{document}
