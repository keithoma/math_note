\documentclass[13pt]{scrartcl}
\usepackage[ngerman]{babel}
\usepackage{tikz}
\usepackage{endnotes}
\usetikzlibrary{shapes.geometric, arrows}
\tikzstyle{startstop} = [rectangle, rounded corners, minimum width=3cm, minimum height=1cm,text centered, draw=black, fill=red!30]
\tikzstyle{io} = [trapezium, trapezium left angle=70, trapezium right angle=110, minimum width=3cm, minimum height=1cm, text centered, draw=black, fill=blue!30]
\tikzstyle{process} = [rectangle, minimum width=3cm, minimum height=1cm, text centered, draw=black, fill=orange!30]
\tikzstyle{decision} = [diamond, minimum width=3cm, minimum height=1cm, text centered, draw=black, fill=green!30]
\tikzstyle{arrow} = [thick,->,>=stealth]
\title{Bruch}
\author{Robert Bedard}
\begin{document}


\section{Der Euklidische Algorithmus als Flussbild}
\begin{tikzpicture}[node distance=4cm]
\node (start) [startstop] {Start};
\node (in1) [io, below of=start] {a, b};
\node (pro1) [process, below of=in1] {a mod b = c};
\node (dec1) [decision, right of=pro1, xshift=4cm] {if c = 0};
\node (out1) [io, below of=dec1] {c};
\node (out2) [io, above of=dec1] {a = b , b = c};
\node (stop) [startstop, below of=out1] {Stop};
\draw [arrow] (start) -- (in1);
\draw [arrow] (in1) -- (pro1);
\draw [arrow] (pro1) -- (dec1);
\draw [arrow] (dec1) -- node[anchor=east] {no}(out2);
\draw [arrow] (out2) -- (in1);
\draw [arrow] (dec1) -- node[anchor=east] {yes}(out1);
\draw [arrow] (out1) -- (stop);
\end{tikzpicture}
\end{document}