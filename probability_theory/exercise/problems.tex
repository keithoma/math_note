\documentclass[a4paper]{article}
\title{Integration and Integration}
\author{K}


% ---------------------------------------------------------------------
% P A C K A G E S
% ---------------------------------------------------------------------

% typography and formatting
\usepackage[english]{babel}
\usepackage[utf8]{inputenc}
\usepackage{geometry}
\usepackage{exsheets}
\usepackage{environ}

% mathematics
\usepackage{amsthm} % for theorems, and definitions
\usepackage{amssymb}
\usepackage{amsmath}
\usepackage{textcomp}
\usepackage{dsfont}
%\usepackage{MnSymbol} % for \cupdot

% extra
\usepackage{xcolor}
\usepackage{tikz}

% ---------------------------------------------------------------------
% S E T T I N G
% ---------------------------------------------------------------------

% typography and formatting
\geometry{margin=3cm}

\SetupExSheets{
  counter-format = ch.qu,
  counter-within = chapter,
  question/print = true,
  solution/print = true,
}

% mathematics
\theoremstyle{definition}
\newtheorem{definition}{Definition}
\newtheorem{example}{Example}[definition]

\newtheorem{theorem}{Theorem}[definition]
\newtheorem{corollary}{Corollary}
\newtheorem{lemma}{Lemma}[definition]
\newtheorem{proposition}{Proposition}[definition]

\newtheorem*{remark}{Remark}

% extra
\definecolor{mathif}{HTML}{0000A0} % for conditions
\definecolor{maththen}{HTML}{CC5500} % for consequences
\definecolor{mathrem}{HTML}{8b008b} % for notes

\usetikzlibrary{positioning}
\usetikzlibrary{shapes.geometric, arrows}

% ---------------------------------------------------------------------
% C O M M A N D S
% ---------------------------------------------------------------------

\newcommand{\norm}[1]{\left\lVert#1\right\rVert}
\newcommand{\rank}{\text{rank}}
\newcommand{\Vol}{\text{Vol}}
\newcommand*\diff{\mathop{}\!\mathrm{d}}
\newcommand*\Diff{\mathop{}\!\mathrm{D}}

\newcommand\restr[2]{{% we make the whole thing an ordinary symbol
  \left.\kern-\nulldelimiterspace % automatically resize the bar with \right
  #1 % the function
  \vphantom{\big|} % pretend it's a little taller at normal size
  \right|_{#2} % this is the delimiter
  }}



  \newcounter{problem}
  \newcounter{solution}
  
  \newcommand\Problem{%
    \stepcounter{problem}%
    \textbf{\theproblem.}~%
    \setcounter{solution}{0}%
  }
  
  \newcommand\TheSolution{%
    \textbf{Solution:}\\%
  }
  
  \newcommand\ASolution{%
    \stepcounter{solution}%
    \textbf{Solution \thesolution:}\\%
  }
  \parindent 0in
  \parskip 1em
% ---------------------------------------------------------------------
% R E N D E R
% ---------------------------------------------------------------------

\newif\ifshowproof
\showprooftrue

\NewEnviron{Proof}{%
    \ifshowproof%
        \begin{proof}%
            \BODY
        \end{proof}%
    \fi%
}%

\begin{document}
\Problem{
  Compute the values of the following sums.
  \begin{enumerate}
      \item \(\sum_{k=0}^\infty x^k\), for \(|x| < 1\),
      \item \(\sum_{k=0}^\infty \frac{x^k}{k!}\),
      \item \(\sum_{k=0}^n \binom{n}{k} a^k b^{n-k}\)
  \end{enumerate}
}

\TheSolution{
  \begin{enumerate}
      \item The sum submits to the ratio test and converges to some number. Denote this number with \(S\). We have
      \begin{align}
          S = 
      \end{align}
  \end{enumerate}
}

\Problem{
  For \(\lambda > 0\) let \(X \sim \text{Exp}(\lambda)\) and let
  \begin{equation}
    Y := \lceil{X} \rceil := \min \{\, n \in \mathbb{N} \, \mid \, n \geq X \, \}
  \end{equation}
  Show that for the parameter \(p = 1 - e^{-\lambda}\) holds \(Y \sim \text{Geo}(p)\).
}

\TheSolution{
  For the random variable \(X\) we have the distribution \(f^X (x) = \lambda e^{-\lambda x} \mathds{1}_{[0, \infty)}(x)\) that generates the probability measure via
  \begin{align}
    \mathbb{P}( X \in (a, b]) = \int_a^b \lambda e^{-\lambda x} \mathds{1}_{[0, \infty)}(x) \diff x, \qquad \text{for all } a, b \in \mathbb{R} \text{.}
  \end{align}
}
  If \(Y = \lceil{X} \rceil\), then we have
  \begin{align}
    \mathbb{P}( Y \in (a, b]) &= \mathbb{P} (\lceil{X} \rceil \in (a, b])
  \end{align}
\end{document}