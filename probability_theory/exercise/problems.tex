\documentclass[a4paper]{book}
\title{Integration and Integration}
\author{K}


% ---------------------------------------------------------------------
% P A C K A G E S
% ---------------------------------------------------------------------

% typography and formatting
\usepackage[english]{babel}
\usepackage[utf8]{inputenc}
\usepackage{geometry}
\usepackage{exsheets}
\usepackage{environ}

% mathematics
\usepackage{amsthm} % for theorems, and definitions
\usepackage{amssymb}
\usepackage{amsmath}
\usepackage{textcomp}
\usepackage{dsfont}
%\usepackage{MnSymbol} % for \cupdot

% extra
\usepackage{xcolor}
\usepackage{tikz}

% ---------------------------------------------------------------------
% S E T T I N G
% ---------------------------------------------------------------------

% typography and formatting
\geometry{margin=3cm}

\SetupExSheets{
  counter-format = ch.qu,
  counter-within = chapter,
  question/print = true,
  solution/print = true,
}

% mathematics
\theoremstyle{definition}
\newtheorem{definition}{Definition}
\newtheorem{example}{Example}[definition]

\newtheorem{theorem}{Theorem}[definition]
\newtheorem{corollary}{Corollary}
\newtheorem{lemma}{Lemma}[definition]
\newtheorem{proposition}{Proposition}[definition]

\newtheorem*{remark}{Remark}

% extra
\definecolor{mathif}{HTML}{0000A0} % for conditions
\definecolor{maththen}{HTML}{CC5500} % for consequences
\definecolor{mathrem}{HTML}{8b008b} % for notes

\usetikzlibrary{positioning}
\usetikzlibrary{shapes.geometric, arrows}

% ---------------------------------------------------------------------
% C O M M A N D S
% ---------------------------------------------------------------------

\newcommand{\norm}[1]{\left\lVert#1\right\rVert}
\newcommand{\rank}{\text{rank}}
\newcommand{\Vol}{\text{Vol}}
\newcommand*\diff{\mathop{}\!\mathrm{d}}
\newcommand*\Diff{\mathop{}\!\mathrm{D}}

\newcommand\restr[2]{{% we make the whole thing an ordinary symbol
  \left.\kern-\nulldelimiterspace % automatically resize the bar with \right
  #1 % the function
  \vphantom{\big|} % pretend it's a little taller at normal size
  \right|_{#2} % this is the delimiter
  }}



  \newcounter{problem}
  \newcounter{solution}
  
  \newcommand\Problem{%
    \stepcounter{problem}%
    \textbf{\theproblem.}~%
    \setcounter{solution}{0}%
  }
  
  \newcommand\TheSolution{%
    \textbf{Solution:}\\%
  }
  
  \newcommand\ASolution{%
    \stepcounter{solution}%
    \textbf{Solution \thesolution:}\\%
  }
  \parindent 0in
  \parskip 1em
% ---------------------------------------------------------------------
% R E N D E R
% ---------------------------------------------------------------------

\newif\ifshowproof
\showprooftrue

\NewEnviron{Proof}{%
    \ifshowproof%
        \begin{proof}%
            \BODY
        \end{proof}%
    \fi%
}%

\begin{document}

\chapter{Probability Space}

\Problem{
  \parskip 0em
  Decide if the following statements are true or false.
  \begin{enumerate}
    \item If \(F^X\) is a CDF of a random variable \(X\) that has only values in \((0, 1)\), then \(F^X (0) = 0\).
    \item Let \(\Omega = \{1, 2, \ldots, 6\}^6\) be the sample space that describes a six consequtive dice roles. The event \(\{\omega \in \Omega \mid \omega_1 < \omega_2 < \ldots < \omega_6\}\) is precisely the outcome in which a street \(1, 2, \ldots, 6\) is rolled.
    \item Every probability measure is \(\sigma\)-finite.
  \end{enumerate}
}

\Problem{
  For \(\lambda > 0\) let \(X \sim \text{Exp}(\lambda)\) and let
  \begin{equation}
    Y := \lceil{X} \rceil := \min \{\, n \in \mathbb{N} \, \mid \, n \geq X \, \}
  \end{equation}
  Show that for the parameter \(p = 1 - e^{-\lambda}\) holds \(Y \sim \text{Geo}(p)\).
}

\TheSolution{
  For the distribution of \(Y\) we have for all \(k \in \mathbb{N}_0\)
  \begin{align}
    p^Y (k) = \mathbb{P}(Y = k) = \mathbb{P}(\lceil{X} \rceil = k) = \mathbb{P}(k - 1 < X \leq k) = F^X (k) - F^X(k - 1) \text{.}
  \end{align}
  On the other hand, the CDF of \(X\) is
  \begin{equation}
    F^X (x) = \int_{-\infty}^x f^X(y) \diff y = \int_{-\infty}^x \lambda e^{-\lambda y} \mathds{1}_{[0, \infty)}(y) \diff y = 1 - e^{-\lambda x} \text{.}
  \end{equation}
  Therefore we have
  \begin{align}
    p^Y (k) &= F^X (k) - F^X(k - 1) \\
    &= (1 - e^{-\lambda k}) - (1 - e^{-\lambda (k - 1)}) \\
    &= - e^{-\lambda k} + e^{-\lambda (k - 1)} \\
    &= - e^{-\lambda} \cdot e^{-\lambda (k - 1)} + e^{-\lambda (k - 1)} \\
    &= e^{-\lambda (k - 1)} (1 - e^{-\lambda}) \\
    &= \left(1 - \left(1 - e^{-\lambda}\right) \right)^{k - 1} (1 - e^{-\lambda})
  \end{align}
  Setting \(1 - e^{-\lambda} = p\) yields the desired result \(p^Y (k) = \left(1-p\right)^{k-1} p\).
}

\Problem{
  Let \(X \sim N(\mu, \sigma^2)\) and \(Y = aX + b\) for some \(a, b \in \mathbb{R}\) with \(a \neq 0\).
  \begin{enumerate}
    \item Show that \(Y \sim N(a \mu + b, a^2 \sigma^2)\).
    \item How must \(a\) and \(b\) be chosen so that \(Y \sim N(0, 1)\) holds?
  \end{enumerate}
}

\TheSolution{
  \begin{enumerate}
    \item Just use Transformationformula.
    \item \(a = \sigma^{-1}\) and \(b = -\mu \sigma^{-1}\).
  \end{enumerate}
}

\Problem{
  Let \(\Phi = \Phi_{0, 1}\) the cumulative mass function of \(N(0, 1)\)-distribution and let \(\Phi^{-1}\) be its inverse. Show that the following equation hold.
  \begin{equation}
    \Phi^{-1} (p) = - \Phi^{-1} (1 - p), \qquad p \in (0, 1) \text{.}
  \end{equation}
}

\TheSolution{
Consider the PMF of the standard normal distribution \(\Phi\) and let \(x \in \mathbb{R}^+_0\). Because of its symmetry on the \(y\)-axis, we have
\begin{align}
  1 &= \int_{-\infty}^\infty \phi(y) \diff y \\
  &= \int_{-\infty}^{x} \phi(y) \diff y + \int_{x}^{\infty} \phi(y) \diff y \\
  &= \int_{-\infty}^{x} \phi(y) \diff y + \int_{-\infty}^{-x} \phi(y) \diff y \\
  &= \Phi(x) + \Phi(-x) \text{.}
\end{align}
Hence \(\Phi(-x) = 1 - \Phi(x)\). Using this and the bijectivity of \(\Phi\) we get
\begin{align}
  \Phi^{-1} (p) = - \Phi^{-1} (1 - p) &\iff \Phi \left( \Phi^{-1} (p) \right) = \Phi \left( - \Phi^{-1} (1 - p) \right) \\
  &\iff p = 1 - \Phi(\Phi^{-1} (1 - p)) \\
  &\iff p = p
\end{align}

\Problem{
  Blatt 3 Aufgabe 3, Fabrik frage, easy
}

\chapter{Independence}

\Problem{
  Let \(X_1, \ldots, X_n\) be real-valued independent random variables with CDFs \(F_1, \ldots, F_n\).
  \begin{enumerate}
    \item Show that the CDFs of \(M = \max(X_1, \ldots, X_n)\) and \(m = \min(X_1, \ldots, X_n)\) are given by
    \begin{align}
      F^M (x) = \prod_{i=1}^n F_i(x) \qquad \text{and} \qquad F^m(x) = 1 - \prod_{i=1}^n \left(1 - F_i (x) \right) \text{.}
    \end{align}
  \end{enumerate}
}

\TheSolution{
  \begin{align}
  \{X_i \in A_i\}
  \end{align}
}
\end{document}