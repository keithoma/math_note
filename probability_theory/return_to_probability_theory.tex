\documentclass{book}
\usepackage{graphicx} % Required for inserting images
\usepackage{amsthm} % for theorems, definitions, etc.
\usepackage{amsmath} % for align
\usepackage{amssymb} % for varnothing

\theoremstyle{definition}
\newtheorem{definition}{Definition}[]







\title{Return to Probability Theory}
\author{Kei Thoma }
\date{April 2024}

\begin{document}

\maketitle

\section{Basics?}

\begin{definition}[\(\sigma\)-Algebra]
    Let \(X\) be a set and denote its power set by \(\mathcal{P}(X)\). Then a subset \(\Sigma \subset \mathcal{P}(X)\) is called a \(\sigma\)-algebra if it satisfies the following three axioms:
    \begin{enumerate}
        \item \(X \in \Sigma\)
        \item \(\Sigma\) is closed under complementation.
        \item \(\Sigma\) is closed under finite unions.
    \end{enumerate}
    Elements of the \(\sigma\)-algebra are called measurable sets. The set \(X\) with its \(\sigma\)-algebra, denoted \((X, \Sigma)\) is called measurable space.
\end{definition}





\begin{definition}[Measure]
    Let \((X, \Sigma\) be a measure space. A function
    \begin{align*}
        \mu: \Sigma \longrightarrow \mathbb{R} \cup \{-\infty, \infty\}
    \end{align*}
    is called a measure if it satisfies the following conditions:
    \begin{enumerate}
        \item Non-negativity.
        \item \(\mu(\varnothing) = 0\)
        \item Countable additivity.
    \end{enumerate}
\end{definition}


\begin{definition}[Probability Measure]
    
\end{definition}

\begin{definition}[Probability Space]
    A probability space is a triple \( (\Omega, \mathcal{F}, P) \) consisting of
    \begin{enumerate}
        \item the sample space \(\Omega\); a set
        \item the event space \(\mathcal{F}\); a \(\sigma\)-algebra of \(\Omega\)
        \item the probability measure \(P: \mathcal{F} \longrightarrow [0, 1]\)
    \end{enumerate}
\end{definition}

\begin{definition}[Measurable Function]
    Let \((X, \Sigma_X)\) and \((Y, \Sigma_Y)\) be measurable spaces. A function \(f: X \longrightarrow Y\) is said to be measurable if for every \(T \in \Sigma_Y\) the preimage of \(T\) under \(f\) is in \(\Sigma_X\).
\end{definition}

\begin{definition}[Random Variable]
    A random variable \(X\) is a measurable function \(X: \Omega \longrightarrow E\) from a sample space to a measurable space \(E\).
\end{definition}

\begin{definition}[Expected Value]
    
\end{definition}

\begin{definition}[Variance]
    
\end{definition}

\begin{definition}[Covariance]
    
\end{definition}

\begin{definition}[Almost Everywhere]
    Let \((X, \Sigma\) be a measurable space. A property \(P\) is said to hold almost everywhere in \(X\) if there exists a set \(N \in \Sigma \) with \(\mu(N) = 0\), and all \(x \in X \setminus N\) have the property \(P\).
\end{definition}

\end{document}
