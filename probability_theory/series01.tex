\documentclass[a4paper]{book}
\title{Integration and Integration}
\author{K}


% ---------------------------------------------------------------------
% P A C K A G E S
% ---------------------------------------------------------------------

% typography and formatting
\usepackage[english]{babel}
\usepackage[utf8]{inputenc}
\usepackage{geometry}
\usepackage{exsheets}
\usepackage{environ}

% mathematics
\usepackage{amsthm} % for theorems, and definitions
\usepackage{amssymb}
\usepackage{amsmath}
\usepackage{textcomp}
%\usepackage{MnSymbol} % for \cupdot

% extra
\usepackage{xcolor}
\usepackage{tikz}

% ---------------------------------------------------------------------
% S E T T I N G
% ---------------------------------------------------------------------

% typography and formatting
\geometry{margin=3cm}

\SetupExSheets{
  counter-format = ch.qu,
  counter-within = chapter,
  question/print = true,
  solution/print = true,
}

% mathematics
\theoremstyle{definition}
\newtheorem{definition}{Definition}[chapter]
\newtheorem{example}{Example}[definition]

\newtheorem{theorem}{Theorem}[definition]
\newtheorem{corollary}{Corollary}
\newtheorem{lemma}{Lemma}[definition]
\newtheorem{proposition}{Proposition}[definition]

\newtheorem*{remark}{Remark}

% extra
\definecolor{mathif}{HTML}{0000A0} % for conditions
\definecolor{maththen}{HTML}{CC5500} % for consequences
\definecolor{mathrem}{HTML}{8b008b} % for notes

\usetikzlibrary{positioning}
\usetikzlibrary{shapes.geometric, arrows}

% ---------------------------------------------------------------------
% C O M M A N D S
% ---------------------------------------------------------------------

\newcommand{\norm}[1]{\left\lVert#1\right\rVert}
\newcommand{\rank}{\text{rank}}
\newcommand{\Vol}{\text{Vol}}
\newcommand*\diff{\mathop{}\!\mathrm{d}}
\newcommand*\Diff{\mathop{}\!\mathrm{D}}

\newcommand\restr[2]{{% we make the whole thing an ordinary symbol
  \left.\kern-\nulldelimiterspace % automatically resize the bar with \right
  #1 % the function
  \vphantom{\big|} % pretend it's a little taller at normal size
  \right|_{#2} % this is the delimiter
  }}

% ---------------------------------------------------------------------
% R E N D E R
% ---------------------------------------------------------------------

\newif\ifshowproof
\showprooftrue

\NewEnviron{Proof}{%
    \ifshowproof%
        \begin{proof}%
            \BODY
        \end{proof}%
    \fi%
}%

\begin{document}
\noindent\textbf{Problem 01.2}
\newline
\newline
Let \((\Omega, \mathcal{A}, \mathbb{P})\) a probability space and \((M, \mathcal{F})\) a measurable space. Moreover, let \(X: \Omega \rightarrow M\) a \((\mathcal{A}, \mathcal{F})\)-measurable random variable. Show that
\begin{align}
    \mathbb{P}^X (B) := \mathbb{P}(X \in B) = \mathbb{P} (X^{-1}(B)), \qquad B \in \mathcal{F}
\end{align}
defines a probability measure on \((M, \mathcal{F})\).
\newline
\newline
\textbf{Solution}
\begin{enumerate}
    \item We have
    \begin{align}
        \mathbb{P}^X(M) &\stackrel{\text{def.}}{=} \mathbb{P}(X \in M) \\
        &\stackrel{\text{def.}}{=} \mathbb{P} (\{\omega \in M \mid X(\omega) \in M\}) \\
        &= \mathbb{P} (\{\omega \in M\}) \label{eq1}\\
        &= \mathbb{P} (M) \\
        &\stackrel{\text{def.}}{=} 1 \text{.}
    \end{align}
    In (\ref{eq1}), we used that the codomain of \(X\) is \(M\) and in the last step, we used the normed property of the probability measure \(\mathbb{P}\).
    \item Let \(A_i \in \mathcal{F}\) with \(i \in \mathbb{N}\) disjoint subsets. We have
    \begin{align}
        \mathbb{P}^X \left( \bigcup_{i = 1}^\infty A_i \right) &\stackrel{\text{def.}}{=} \mathbb{P} \left( X \in \bigcup_{i = 1}^\infty A_i \right) \\
        &\stackrel{\text{def.}}{=} \mathbb{P} \left( \left\{ \omega \in M \, \middle| \, X(\omega) \in \bigcup_{i = 1}^\infty  A_i \right\}\right) \text{.}
        \intertext{As \(A_i\) are disjoint, \(X(\omega)\) is included in one and only one \(A_i\). Therefore with the \(\sigma\)-additivity of \(\mathbb{P}\), we have}
        &= \mathbb{P} \left( \bigcup_{i = 1}^\infty \left\{ \omega \in M \, \middle| \, X(\omega) \in A_i \right\}\right) \\
        &\stackrel{\text{def.}}{=} \sum_{i = 1}^\infty \mathbb{P} \{\omega \in M \mid X(\omega) \in A_i \} \\
        &\stackrel{\text{def.}}{=} \sum_{i = 1}^\infty \mathbb{P} (X \in A_i) \\
        &\stackrel{\text{def.}}{=} \sum_{i = 1}^\infty \mathbb{P}^X (A_i) \text{.}
    \end{align}
    In short, \(\mathbb{P}^X\) is \(\sigma\)-additive.
\end{enumerate}
From above, it follows that \(\mathbb{P}^X\) is a probability measure.
\newpage
\noindent\textbf{Problem 01.3}
Let \((\Omega, \mathcal{A}, \mathbb{P})\) be a probability space, \(n \in \mathbb{N}\) and \(A_k \in \mathcal{A}\) for all \(k \in \{1, \dots, n\}\). Prove the following formula.
\begin{align}
    \mathbb{P} \left( \bigcup_{k=1}^n A_k \right) = \sum_{k = 1}^n \left( (-1)^{k-1} \sum_{I \subset \{1, \dots, n\}, |I| = k } \mathbb{P} \left( \bigcap_{i \in I} A_i \right) \right)
\end{align}
\newline
\newline
\textbf{Solution}
\end{document}