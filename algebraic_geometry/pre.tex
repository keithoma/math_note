TODO

\part{Pre}
\begin{enumerate}
    \item Prove \(R\) int domain, then \(R[X]\) is int domain.
\end{enumerate}

\begin{proposition}
    If \(R\) is an integral domain, then the polynomial ring \(R[X]\) is again an integral domain.
\end{proposition}
\begin{proof}
    \begin{enumerate}
        \item Since \(1 \in R \subset R[X]\), the polynomial ring \(R[X]\) is nonempty.
        \item Let \(f, g \in R[X]\) be two nonzero polynomials with
        \begin{align*}
            f = \sum_{i = 0}^{m} a_i X^i \quad \text{and} \quad g = \sum_{j = 0}^{n} b_j X^j \text{.}
        \end{align*}
        Consider its product
        \begin{align*}
            f \cdot g = \sum_{k = 0}^{m + n} c_j X^k
        \end{align*}
        and suppose \(f \cdot g = 0\).
        \item Since the leading coefficient of the product \(c_{m+n}\) is obtained by multiplying the leading coefficients of \(f\) and \(g\), we have \(c_{m+n} = a_m \cdot b_n\).
        \item We had \(f \cdot g = 0\), thus \(c_{m+n} = a_m \cdot b_n = 0\).
        \item \(R\) is an integral domain, therefore \(a_m \cdot b_n = 0\) means \(a_m = 0\) or \(b_n = 0\).
        \item This contradicts that \(f\) and \(g\) were nonzero polynomials.
    \end{enumerate}
\end{proof}