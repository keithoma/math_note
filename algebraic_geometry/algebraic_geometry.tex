\documentclass[a4paper]{book}
\title{Algebraic Geometry}
\author{Kei Thoma}


% ---------------------------------------------------------------------
% P A C K A G E S
% ---------------------------------------------------------------------

% typography and formatting
\usepackage[english]{babel}
\usepackage[utf8]{inputenc}
\usepackage{geometry}
\usepackage{exsheets}
\usepackage{environ}
\usepackage{graphicx}
\usepackage{cutwin}
\usepackage{pifont}

% mathematics
\usepackage{xfrac}  
\usepackage{amsthm} % for theorems, and definitions
\usepackage{amssymb}
\usepackage{amsmath}
\usepackage{textcomp}
\usepackage{mathtools}
% \usepackage{MnSymbol} % for \cupdot

% extra
\usepackage{xcolor}
\usepackage{tikz}

% ---------------------------------------------------------------------
% S E T T I N G
% ---------------------------------------------------------------------

%maybe delete later, for colorbox
\usepackage{tcolorbox}
\newtcolorbox{defbox}{colback=blue!5!white,colframe=blue!75!black}
\newtcolorbox{defboxlight}{colback=cyan!5!white,colframe=cyan!75!black}
\newtcolorbox{thmbox}{colback=orange!5!white,colframe=orange!75!black}
\newtcolorbox{rembox}{colback=purple!5!white,colframe=purple!75!black}
\newtcolorbox{exmbox}{colback=gray!5!white,colframe=gray!75!black}
\newtcolorbox{intbox}{colback=violet!5!white,colframe=violet!75!black}

% typography and formatting
\geometry{margin=3cm}

\SetupExSheets{
  counter-format = ch.qu,
  counter-within = chapter,
  question/print = true,
  solution/print = true,
}

% mathematics
\newcounter{global}

\theoremstyle{definition}
\newtheorem{definition}{Definition}[]
\newtheorem{example}{Example}[definition]

\newtheorem{theorem}[definition]{Theorem}
\newtheorem{corollary}{Corollary}
\newtheorem{lemma}[definition]{Lemma}
\newtheorem{proposition}[definition]{Proposition}

\newtheorem*{remark}{Remark}
\newtheorem*{intuition}{Intuition}

% extra
\definecolor{mathif}{HTML}{0000A0} % for conditions
\definecolor{maththen}{HTML}{CC5500} % for consequences
\definecolor{mathrem}{HTML}{8b008b} % for notes
\definecolor{mathobj}{HTML}{008800}

\usetikzlibrary{positioning}
\usetikzlibrary{shapes.geometric, arrows}

% ---------------------------------------------------------------------
% C O M M A N D S
% ---------------------------------------------------------------------

\newcommand{\norm}[1]{\left\lVert#1\right\rVert}
\newcommand{\rank}{\text{rank}}
\newcommand{\Vol}{\text{Vol}}

\newcommand{\set}[1]{\left\{\, #1 \,\right\}}
\newcommand{\makeset}[2]{\left\{\, #1 \mid #2 \,\right\}}

\newcommand*\diff{\mathop{}\!\mathrm{d}}
\newcommand*\Diff{\mathop{}\!\mathrm{D}}

\newcommand\restr[2]{{% we make the whole thing an ordinary symbol
  \left.\kern-\nulldelimiterspace % automatically resize the bar with \right
  #1 % the function
  \vphantom{\big|} % pretend it's a little taller at normal size
  \right|_{#2} % this is the delimiter
  }}

% ---------------------------------------------------------------------
% R E N D E R
% ---------------------------------------------------------------------

\newif\ifshowproof
\showprooftrue

\NewEnviron{Proof}{%
    \ifshowproof%
        \begin{proof}%
            \BODY
        \end{proof}%
    \fi%
}%

\begin{document}
\maketitle
\tableofcontents
%%%%%%%%%%%%%%%%%%%%%%%%%%%%%%%%%%%%%%%%%%%%%%%%%%%%%%%%%%%%%%%%%%%%%%%%%%%%%%%

\newpage
\chapter*{Housekeeping}
Notes taken from
\begin{itemize}
    \item my course
\end{itemize}

\textbf{to-do}
\begin{enumerate}
    \item
\end{enumerate}




\chapter{Basics?}

Let \(K\) be an arbitary field.

\begin{defbox}
    \begin{definition}[Algebraic Subset]
        For a subset \(M \subset K[X_1, \ldots, X_n]\), we define
        \begin{align*}
            V(M) = \makeset{p \in K^n}{\text{ for all polynomials }f \in M \text{ it is } f(p) = 0}
        \end{align*}
        called the algebraic subset in \(K^n\) or an affine algebraic set over \(K\).
    \end{definition}
\end{defbox}
\begin{intbox}
    \begin{intuition}
        Imagine the ring of polynomials \(K[X_1, \ldots, X_n]\) to be a set of locks and the kartesian product of the field \(K^n\) to be a set of keys. Then, a subset \(M\) of \(K[X_1, \ldots, X_n]\) is some combination of locks and the algebraic subset \(V(M)\) are the set of keys that open all the locks in \(M\).
    \end{intuition}
\end{intbox}
\begin{example}
    \begin{enumerate}
        \item Let \(\mathbb{K} = \mathbb{C}\) and \(A = \mathbb{C}[X]\).
        \begin{enumerate}
            \item If we set \(M = \set{X^2 - 1} \subset \mathbb{C}[X]\), then \(V(\set{X^2 - 1}) = \set{-1, 1} \in \mathbb{C}\) is the algebraic subset.
            \item Now, if we append the set above to \(M = \set{X^2 - 1, X - 1} \subset \mathbb{C}[X]\), we have \begin{align*}
                V(\set{X^2 - 1, X - 1}) = \set{1} \in \mathbb{C}
            \end{align*}
            instead. This example illustrates that appending the set of polynomials makes its algebraic subset smaller.
            \item In general, the finite subsets of \(\mathbb{K}\) are precisely the algebraic subsets of \(\mathbb{K}\).
        \end{enumerate}
        \item Let \(\mathbb{K} = \mathbb{C}\) and \(A = \mathbb{C}[X, Y]\).
        \begin{enumerate}
            \item If \(M = \set{X - 1, Y - 1} \subset \mathbb{C}[X, Y]\), then \(V(\set{X - 1, Y - 1}) = \set{(1, 1)} \subset \mathbb{C}^2\).
            \item If we remove the second polynomial, we get \(M = \set{X - 1} \subset \mathbb{C}[X, Y]\) and \begin{align*}
                V(\set{X - 1}) = \makeset{(1, y)}{y \in \mathbb{C}}\text{.}
            \end{align*}
            It is the first example of an infinite algebraic subset.
        \end{enumerate}
    \end{enumerate}
\end{example}
\begin{thmbox}
    \begin{lemma}
            If \(M, N \subset K[X_1, \ldots, X_n]\) with \(N \subset M\), then it is \(V(N) \supset V(M)\).
    \end{lemma}
\end{thmbox}
\begin{intbox}
    \begin{intuition}
            \(N\) has less locks then \(M\). If \(N\) has less locks, then more keys are able to open all locks in \(N\).
    \end{intuition}
\end{intbox}
\begin{proof}
    Let \(p \in V(M)\). For all \(f \in M\), we have \(f(p) = 0\). Now, \(N \subset M\), thus for all \(g \in N\) it is \(g \in M\) and therefore \(g(p) = 0\). That means \(V(M) \subset V(N)\).
\end{proof}
\begin{thmbox}
    \begin{proposition}
        \begin{enumerate}
            \item The empty set \(\varnothing\) is an algebraic subset of \(\mathbb{K}^n\).
            \item The whole set \(\mathbb{K}^n\) is an algebraic subset of \(\mathbb{K}^n\).
            \item An arbitary intersection of algebraic subsets of \(\mathbb{K}^n\) is an algebraic subset of \(\mathbb{K}^n\).
            \item A finite union of algebraic subsets of \(\mathbb{K}^n\) is an algebraic subset of \(\mathbb{K}^n\).
        \end{enumerate}
    \end{proposition}
\end{thmbox}
\begin{exmbox}
    \begin{example}
    The union of non-finite union of algebraic subsets is not algebraic. Take \(\mathbb{K} = \mathbb{C}\) and \(A = \mathbb{C}[X]\). As we have seen before, the singleton sets \(\{x\}\) are algebraic for each \(x \in \mathbb{N}\), but their union
    \begin{align*}
        \bigcup_{x \in \mathbb{N}} \{x\} = \mathbb{N}
    \end{align*}
    is not.
    \end{example}
\end{exmbox}

The above proposition justifies the following definition.

\begin{defbox}
    \begin{definition}
        The unique topology on \(K^n\) whose closed subsets are the algebraic subsets are called the Zariski topology on \(K^n\).
    \end{definition}
\end{defbox}

\#\#missing I think we didn't prove the uniqueness yet.
\begin{thmbox}    
    \begin{lemma}
        Let \(M \subset K[X_1, \ldots, X_n]\) be a subset. Define \(I\) to be the ideal generated\footnote{By Hilbert's Basis Theorem the polynomial ring is Noetherian and thus this ideal is finitely generated.} by \(M\). Then we have \(V(M) = V(I)\).
    \end{lemma}
\end{thmbox}
Thus, \(V(M)\) may be expressed by finite amount of generators \(V(M) = V(f_1, \ldots, f_n)\).
\begin{example}
    \begin{enumerate}
        \item Let \(\mathbb{K} = \mathbb{C}\) and \(A = \mathbb{C}[X]\).
        \begin{enumerate}
            \item With the same example as above, set \(M = \set{X^2 - 1} \subset \mathbb{C}[X]\), then \(V(\set{X^2 - 1}) = \set{-1, 1} \in \mathbb{C}\) is the algebraic subset. Consider the ideal \((X^2 - 1)\). Some elements in this ideal are
            \begin{align*}
                3 (X^2 - 1), \qquad X(X^2 - 1), \qquad (X + 1)(X^2 - 1)
            \end{align*}
            which all still have the roots \(\set{-1, 1}\).
        \end{enumerate}
    \end{enumerate}
\end{example}

\begin{defbox}    
    \begin{definition}[Radical of an Ideal]

    \end{definition}
\end{defbox}
\begin{thmbox}
    \begin{proposition}
        Let \(I\) be an ideal in \(A\), then:
        \begin{enumerate}
            \item \(\sqrt{I} = \sqrt{\sqrt{I}}\)
            \item \(I = \sqrt{I} \iff A / I\) is a reduced ring, i.e. there are no nilpotent elements in \(A / I\).
            \item \(A \overset{\pi}{\longrightarrow} A_\text{red} := A / \sqrt{I}\) has the following universial property:
        \end{enumerate}
    \end{proposition}
\end{thmbox}
\begin{proof}
    
\end{proof}
\begin{thmbox}
    \begin{lemma}
        \(V(I) = V(\sqrt{I})\)
    \end{lemma}
\end{thmbox}
This implies the surjectivity of the following map:
\begin{align*}
    \set{\text{radical ideals in } K[X_1, \ldots, X_n]} \rightarrow \set{\text{algebraic sets of }\mathbb{A}^n(K)}
\end{align*}
(because let \(V\) be an algebraic set i.e. \(V = V(M)\)
then \(V(M) = V((M)) = V(\sqrt{(M)})\))

In general not injective. \#\#missingexample


\begin{thmbox}
    \begin{theorem}[Hilbert Nullstellensatz]
        If \(K\) is algebraically closed, then the above map is a bijection with inverse \(V \mapsto \text{Ann}(V)\).
    \end{theorem}
\end{thmbox}

From now on, assume \(K = \overline{K}\).

\begin{corollary}
    For any proper ideal \(I \subset K[X_1, \ldots, X_n]\) there is a \(p \in K^n\) with \(f(p) = 0\) for all \(p \in I\).
\end{corollary}

\begin{defbox}
    \begin{definition}[A-algebra]
        Let \(A\) be a ring. An \(A\)-algebra is a ring \(B\) together with a rign homomorphism \(f: A \longrightarrow B\), making \(B\) into an \(A\)-module such that scalar multiplication and the product on \(B\) are compatible.

        If \(A = K\) is a field, \(f\) is injective, so a \(K\)-algebra is a ring containing \(K\) as a subring, for example \(K[X_1, \ldots, X_n]\).

        \#\#missing something about two \(A\)-algebras
    \end{definition}
\end{defbox}

\begin{defbox}
    \begin{definition}[Coordinate Ring]
        Let \(V \subset K^n\) be an algebraic subset. Definite the coordinate ring to be \(\mathcal{O}(V) := K[X_1, \ldots, X_n] / Ann(V)\) which is a reduced, finitely generated \(K\)-algebra (underlying ring is reduced).
    \end{definition}
\end{defbox}

\begin{defbox}
    \begin{definition}
        Suppose we have two algebraic sets \(X \subset \mathbb{A}^n\) and \(Y \subset \mathbb{A}^m\). A morphism between those two is a map \(f: X \longrightarrow Y\), \(p \mapsto (f_1(p), \ldots, f_m(p))\) where \(f_i \in \mathcal{O}(X)\) for all \(1 \leq i \leq m\).
    \end{definition}
\end{defbox}

\#\#missing blue text

\end{document}