\documentclass[11pt]{book}

% Language and font encoding
\usepackage[utf8]{inputenc}
\usepackage[T1]{fontenc}
\usepackage[english]{babel}

\usepackage{csquotes}

% Mathematics packages
\usepackage{amsmath, amssymb, amsthm, amsfonts, mathtools}
\usepackage{mathrsfs} % For script fonts

% Theorem environments
\newtheorem{theorem}{Theorem}[section]
\newtheorem{proposition}[theorem]{Proposition}
\newtheorem{lemma}[theorem]{Lemma}
\newtheorem{corollary}[theorem]{Corollary}

\newtheorem{exercise}{Exercise}[section]
\newenvironment{solution}
  {\renewcommand{\qedsymbol}{}\begin{proof}[Solution]}
  {\end{proof}}

\theoremstyle{definition}
\newtheorem{definition}[theorem]{Definition}
\newtheorem{example}[theorem]{Example}
\newtheorem{remark}[theorem]{Remark}

% Definition & Theorem Boxes
\usepackage{tcolorbox}
\newtcolorbox{defbox}{colback=blue!5!white,colframe=blue!75!black}
\newtcolorbox{defboxlight}{colback=cyan!5!white,colframe=cyan!75!black}
\newtcolorbox{thmbox}{colback=orange!5!white,colframe=orange!75!black}
\newtcolorbox{rembox}{colback=purple!5!white,colframe=purple!75!black}
\newtcolorbox{exmbox}{colback=gray!5!white,colframe=gray!75!black}
\newtcolorbox{intbox}{colback=violet!5!white,colframe=violet!75!black}

% Numbering and cross-referencing
\numberwithin{equation}{section}
\usepackage{hyperref}
\hypersetup{
    colorlinks=true,
    linkcolor=blue,
    filecolor=magenta,      
    urlcolor=cyan,
    citecolor=blue,
    pdftitle={Algebraic Geometry Notes},
    pdfpagemode=FullScreen,
}

% Commutative diagrams
\usepackage{tikz-cd}

% Margins and spacing
\usepackage[a4paper, left=1in, right=1in, top=1in, bottom=1in]{geometry}
\usepackage{setspace}
\onehalfspacing

% Custom commands
\newcommand{\C}{\mathbb{C}}
\newcommand{\R}{\mathbb{R}}
\newcommand{\Q}{\mathbb{Q}}
\newcommand{\Z}{\mathbb{Z}}
\newcommand{\N}{\mathbb{N}}
\newcommand{\A}{\mathbb{A}}
\newcommand{\Pp}{\mathbb{P}}
\newcommand{\OO}{\mathcal{O}}
\newcommand{\Spec}{\text{Spec}}
\newcommand{\Proj}{\text{Proj}}
\newcommand{\Hom}{\text{Hom}}
\newcommand{\Ext}{\text{Ext}}
\newcommand{\Tor}{\text{Tor}}
\newcommand{\End}{\text{End}}
\newcommand{\sheafhom}{\mathscr{H}\kern -.5pt om}
\newcommand{\sheafext}{\mathscr{E}\kern -.5pt xt}

\newcommand{\set}[1]{\left\{\, #1 \,\right\}}
\newcommand{\makeset}[2]{\left\{\, #1 \mathrel{\mid} #2 \,\right\}}

% For including graphics
\usepackage{graphicx}

% Packages for better tables
\usepackage{booktabs}
\usepackage{array}

% Bibliography
\usepackage[backend=biber, style=alphabetic, citestyle=numeric]{biblatex}

% Specify your .bib file
\addbibresource{references.bib}

% Title and author information
\title{Notes on Algebraic Geometry}
\author{Kei Thoma}
\date{\today}

\begin{document}

\maketitle
\tableofcontents

\cite{hartshorne1977}

\begin{defbox}
    \begin{definition}
        Let \( k \) be a fixed algebraically closed field. We define affine \( n \)-space over \( k \), denoted \( \mathbb{A}^n \) or simply \( \mathbb{A}^n \), to be the set of all \( n \)-tuples of elements of \( k \). An element \( P \in \mathbb{A}^n \) will be called a point, and if \( P = (a_1, a_2, \ldots, a_n) \) with \( a_i \in k \), then the \( a_i \) will be called the coordinates of \( P \).
    \end{definition}
\end{defbox}

\begin{defbox}
    \begin{definition}
        For a subset \(T\) of the polynomial ring \(A = k[X_1, \ldots, X_n]\), we define the zero set of \(T\) to be the common zeros of all the elements of \(T\), i.e.
        \begin{align*}
            Z(T) = \makeset{P \in \mathbb{A}^n}{f(P) = 0 \text{ for all } f \in T} \text{.}
        \end{align*}
    \end{definition}
\end{defbox}

\begin{defbox}
    \begin{definition}
        A subset \(Y\) of \(\mathbb{A}^n\) is an algebraic set if there exists a subset \(T \subset A = k[X_1, \ldots, X_n]\) such that \(Y = Z(T)\).
    \end{definition}
\end{defbox}

\begin{defbox}
    \begin{definition}
        \label{def:affine_coordinate_ring}
        If \(Y \subset \mathbb{A}^n\) is an affine algebraic set, we define the affine coordinate ring \(A(Y)\) of \(Y\), to be \(A / I(Y)\). 
    \end{definition}
\end{defbox}

\begin{exercise}
    Let \( Y \) be the plane curve \( y = x^2 \) (i.e., \( Y \) is the zero set of the polynomial \( f = y - x^2 \)). Show that \( A(Y) \) is isomorphic to a polynomial ring in one variable over \( k \).
\end{exercise}
\begin{solution}
    By definition ~\ref{def:affine_coordinate_ring}, we simply have \(A(Y) = k[X, Y] / (Y - X^2)\). The isomorphism follows from the isomorphism theorem and the map \(f: k[X, Y] \rightarrow k[X]\) where we set \(f(Y) = X^2\).
\end{solution}
\begin{exercise}
    Let \(Z\) be the plane curve \(xy = 1\). Show that \(A(Z)\) is not isomorphic to a polynomial ring in one variable over \(k\). 
\end{exercise}
\begin{solution}
    \(A(Z) = k[X, Y] / (XY - 1)\)

    We know \(A(Z)\) is an \(k\)-algebra (see remark). Consider \(f: k[X, Y] \longrightarrow k[T]\). We must have \(\ker{f} = (XY - 1)\), thus \(f(XY - 1) = 0\), so \(f(X) = 1 / f(Y)\)

    I'll think about the rigorous details later, but basically \(A(Z) \cong k[X, X^{-1}]\)
\end{solution}
\begin{exercise}
    Let \(f\) be any irreducible quadratic polynomial in \(k[X, Y]\), and let \(W\) be the conic defined by \(f\). Show that \(A(W)\) is isomorphic to \(A(Y)\) or \(A(Z)\). Which one is it when?
\end{exercise}
\begin{solution}
    Let \(f\) be irreducible.

    \(A(W) = k[X, Y] / (f)\)
    
\end{solution}
\printbibliography
\end{document}