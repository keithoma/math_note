\begin{exercise}
    Let \( Y \) be the plane curve \( y = x^2 \) (i.e., \( Y \) is the zero set of the polynomial \( f = y - x^2 \)). Show that \( A(Y) \) is isomorphic to a polynomial ring in one variable over \( k \).
\end{exercise}
\begin{solution}
    By definition ~\ref{def:affine_coordinate_ring}, we simply have \(A(Y) = k[X, Y] / (Y - X^2)\). The isomorphism follows from the isomorphism theorem and the map \(f: k[X, Y] \rightarrow k[X]\) where we set \(f(Y) = X^2\).
\end{solution}
\begin{exercise}
    Let \(Z\) be the plane curve \(xy = 1\). Show that \(A(Z)\) is not isomorphic to a polynomial ring in one variable over \(k\). 
\end{exercise}
\begin{solution}
    \(A(Z) = k[X, Y] / (XY - 1)\)

    We know \(A(Z)\) is an \(k\)-algebra (see remark). Consider \(f: k[X, Y] \longrightarrow k[T]\). We must have \(\ker{f} = (XY - 1)\), thus \(f(XY - 1) = 0\), so \(f(X) = 1 / f(Y)\)

    I'll think about the rigorous details later, but basically \(A(Z) \cong k[X, X^{-1}]\)
\end{solution}
\begin{exercise}
    Let \(f\) be any irreducible quadratic polynomial in \(k[X, Y]\), and let \(W\) be the conic defined by \(f\). Show that \(A(W)\) is isomorphic to \(A(Y)\) or \(A(Z)\). Which one is it when?
\end{exercise}
\begin{solution}
    Let \(f\) be irreducible.

    \(A(W) = k[X, Y] / (f)\)
    
    isn't this kinda clear ...? I'll come back to write it down rigorously, but in general ...
\end{solution}

\begin{exercise}
    Let \(V \subset \mathbb{A}^3\) be the set \(Y = \makeset{(x, x^2, x^3) \in \mathbb{A}^3}{x \in K}\).
    \begin{enumerate}
        \item Show that \(V\) is an affine variety of dimension \(1\).
    \end{enumerate}
\end{exercise}
\begin{solution}
    \begin{enumerate}
        \item We show that \(V\) is an algebraic set.
        \begin{enumerate}
            \item Consider the ideal \((Y - X^2, Z - X^3) \subset K[X, Y]\) and it's zero set \(Z(Y - X^2, Z - X^3)\).
            \item Writing out the definition of the zero set gives
            \begin{align*}
                Z(Y - X^2, Z - X^3) &= \makeset{(x, y, z) \in \mathbb{A}^3}{y - x^2 = 0, \; z - x^3 = 0} \\
                &= \makeset{(x, y, z) \in \mathbb{A}^3}{y = x^2, \; z = x^3} \\
                &= \makeset{(x, x^2, x^3) \in \mathbb{A}^3}{x \in K} \text{.}
            \end{align*}
            Thus, \(V\) is the zero set of the ideal \((Y - X^2, Z - X^3)\).
            \item Hence, by definition, \(V = Z(Y - X^2, Z - X^3)\) is an algebraic set.
        \end{enumerate}
        \item Here, we prove that \(V\) is irreducible.
        \begin{enumerate}
            \item Consider the quotient \(K[X, Y, Z] / (Y - X^2, Z - X^3)\).
            \item By substitution, we get the isomorphism
            \begin{align*}
                K[X, Y, Z] / (Y - X^2, Z - X^3) \cong K[X, X^2, X^3] = K[X] \text{.}
            \end{align*}
            \item Since \(K\) is a field it in particular an integral domain and so is \(K[X]\).
            \item Thus, \((Y - X^2, Z - X^3)\) is prime in \(K[X, Y, Z]\).
            \item With corollary \ref{cor:ideal_correspondence} we may conclude the variety \(V = Z(Y - X^2, Z - X^3)\) is irreducible.
        \end{enumerate}
    \end{enumerate}
\end{solution}


\begin{exercise}
    Let \(Y\) be the algebraic set in \(\mathbb{A}^3\) defined by two polynomials \(x^2 - yz\) and \(xz - x\). Show that \(Y\) is a union of three irreducible components. Describe them and find their prime ideals.
\end{exercise}
\begin{solution}
    \(Y = Z(x^2 - yz, xz - x)\)

    If \(z = 0\), then \(x = 0\) and \(y\) can be any thing, so one irreducible component is the \(y\)-axis. This is described by \((x, z)\).

    If \(x = 0\), then \(yz = 0\). If \(z = 0\), then see above. \(y = 0\) gives the \(z\)-axis as above.

    If \(z \neq 0\), then \(x^2 = yz\) and \(x(z - 1) = 0\). And we have a parabola ...
\end{solution}

    

\begin{exercise}
    If we identify \(\mathbb{A}^2\) with \(\mathbb{A}^1 \times \mathbb{A}^1\) in the natural way, show that the Zariski topology on \(\mathbb{A^2}\) is not the product topology of the Zariski topologies on the two copies of \(\mathbb{A}^1\).
\end{exercise}
\begin{solution}
    In \(\mathbb{A}^1\), the only closed sets are finite sets and \(\mathbb{A}^1\) itself. Take \(\set{p_1, p_2} \subset \mathbb{A}^1\). Consider
    \begin{align*}
        \set{(0, 0); (0, 1); (1, 0); (1, 1)}
    \end{align*}
    No, this is probably okay

    Rather consider this example: \(Z(X^2 - Y)\), then the preimages of the projections give the x-axis and the non-negative y-axis, but the latter is not an algebraic set.
\end{solution}