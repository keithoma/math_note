\begin{exercise}
    Let \( Y \) be the plane curve \( y = x^2 \) (i.e., \( Y \) is the zero set of the polynomial \( f = y - x^2 \)). Show that \( A(Y) \) is isomorphic to a polynomial ring in one variable over \( k \).
\end{exercise}
\begin{solution}
    By definition ~\ref{def:affine_coordinate_ring}, we simply have \(A(Y) = k[X, Y] / (Y - X^2)\). The isomorphism follows from the isomorphism theorem and the map \(f: k[X, Y] \rightarrow k[X]\) where we set \(f(Y) = X^2\).
\end{solution}
\begin{exercise}
    Let \(Z\) be the plane curve \(xy = 1\). Show that \(A(Z)\) is not isomorphic to a polynomial ring in one variable over \(k\). 
\end{exercise}
\begin{solution}
    \(A(Z) = k[X, Y] / (XY - 1)\)

    We know \(A(Z)\) is an \(k\)-algebra (see remark). Consider \(f: k[X, Y] \longrightarrow k[T]\). We must have \(\ker{f} = (XY - 1)\), thus \(f(XY - 1) = 0\), so \(f(X) = 1 / f(Y)\)

    I'll think about the rigorous details later, but basically \(A(Z) \cong k[X, X^{-1}]\)
\end{solution}
\begin{exercise}
    Let \(f\) be any irreducible quadratic polynomial in \(k[X, Y]\), and let \(W\) be the conic defined by \(f\). Show that \(A(W)\) is isomorphic to \(A(Y)\) or \(A(Z)\). Which one is it when?
\end{exercise}
\begin{solution}
    Let \(f\) be irreducible.

    \(A(W) = k[X, Y] / (f)\)
    
    isn't this kinda clear ...? I'll come back to write it down rigorously, but in general ...
\end{solution}

\begin{exercise}
    Let \(Y \subset \mathbb{A}^3\) be the set \(Y = \makeset{t, t^2, t^3}{t \in k}\).
    \begin{enumerate}
        \item Show that \(Y\) is an affine variety of dimension \(1\).
    \end{enumerate}
\end{exercise}
\begin{solution}
    By definition \ref{def:algebraic_variety}, affine algebraic varieties are irreducible closed subsets of \(\mathbb{A}^n\).
    \begin{enumerate}
        \item The topology on \(\mathbb{A}^n\) is the Zariski topology which is defined through closed sets. Closed sets are algebraic sets. Algebraic sets are zeros of some polynomials. Choose \((Y - X^2, Z - X^3)\).
        \item By corollary \ref{cor:ideal_correspondence},
        \item 
    \end{enumerate}
\end{solution}

\begin{exercise}
    Let \(Y\) be the algebraic set in \(\mathbb{A}^3\) defined by two polynomials \(x^2 - yz\) and \(xz - x\). Show that \(Y\) is a union of three irreducible components. Describe them and find their prime ideals.
    \end{exercise}
    \begin{solution}
        \(Y = Z(x^2 - yz, xz - x)\)
    
        If \(z = 0\), then \(x = 0\) and \(y\) can be any thing, so one irreducible component is the \(y\)-axis. This is described by \((x, z)\).
    
        If \(x = 0\), then \(yz = 0\). If \(z = 0\), then see above. \(y = 0\) gives the \(z\)-axis as above.
    
        If \(z \neq 0\), then \(x^2 = yz\) and \(x(z - 1) = 0\). And we have a parabola ...
    \end{solution}
    