\documentclass[a4paper]{article}
\title{Strong Factorial Conjecture}


% ---------------------------------------------------------------------
% P A C K A G E S
% ---------------------------------------------------------------------

% typography and formatting
\usepackage[english]{babel}
\usepackage[utf8]{inputenc}
\usepackage{geometry}
\usepackage{exsheets}
\usepackage{environ}
\usepackage{graphicx}
\usepackage{cutwin}
\usepackage{pifont}

% mathematics
\usepackage{xfrac}  
\usepackage{amsthm} % for theorems, and definitions
\usepackage{amssymb}
\usepackage{amsmath}
\usepackage{textcomp}
\usepackage{mathtools}
\usepackage{mleftright} % for scaling mid bar in sets
% \usepackage{MnSymbol} % for \cupdot

% extra
\usepackage{xcolor}
\usepackage{tikz}

% ---------------------------------------------------------------------
% S E T T I N G
% ---------------------------------------------------------------------

%maybe delete later, for colorbox
\usepackage{tcolorbox}
\newtcolorbox{defbox}{colback=blue!5!white,colframe=blue!75!black}
\newtcolorbox{defboxlight}{colback=cyan!5!white,colframe=cyan!75!black}
\newtcolorbox{thmbox}{colback=orange!5!white,colframe=orange!75!black}
\newtcolorbox{rembox}{colback=purple!5!white,colframe=purple!75!black}
\newtcolorbox{exmbox}{colback=gray!5!white,colframe=gray!75!black}
\newtcolorbox{intbox}{colback=violet!5!white,colframe=violet!75!black}

% typography and formatting
\geometry{margin=3cm}

\SetupExSheets{
  counter-format = ch.qu,
  counter-within = chapter,
  question/print = true,
  solution/print = true,
}

% mathematics
\newcounter{global}

\theoremstyle{definition}
\newtheorem{definition}{Definition}[]
\newtheorem{example}{Example}[definition]

\newtheorem{theorem}[definition]{Theorem}
\newtheorem{corollary}{Corollary}
\newtheorem{lemma}[definition]{Lemma}
\newtheorem{proposition}[definition]{Proposition}

\newtheorem*{remark}{Remark}
\newtheorem*{intuition}{Intuition}

% extra
\definecolor{mathif}{HTML}{0000A0} % for conditions
\definecolor{maththen}{HTML}{CC5500} % for consequences
\definecolor{mathrem}{HTML}{8b008b} % for notes
\definecolor{mathobj}{HTML}{008800}

\usetikzlibrary{positioning}
\usetikzlibrary{shapes.geometric, arrows}

% ---------------------------------------------------------------------
% C O M M A N D S
% ---------------------------------------------------------------------

\newcommand{\norm}[1]{\left\lVert#1\right\rVert}
\newcommand{\rank}{\text{rank}}
\newcommand{\Vol}{\text{Vol}}

\newcommand{\set}[1]{\mleft\{\, #1 \,\mright\}}
\newcommand{\makeset}[2]{\mleft\{\, #1 \; \middle| \; #2 \,\mright\}}

\newcommand*\diff{\mathop{}\!\mathrm{d}}
\newcommand*\Diff{\mathop{}\!\mathrm{D}}

\newcommand\restr[2]{{% we make the whole thing an ordinary symbol
  \left.\kern-\nulldelimiterspace % automatically resize the bar with \right
  #1 % the function
  \vphantom{\big|} % pretend it's a little taller at normal size
  \right|_{#2} % this is the delimiter
  }}

% ---------------------------------------------------------------------
% R E N D E R
% ---------------------------------------------------------------------

\newif\ifshowproof
\showprooftrue

\NewEnviron{Proof}{%
    \ifshowproof%
        \begin{proof}%
            \BODY
        \end{proof}%
    \fi%
}%

\begin{document}
\section{Question 1}
Let \(\sum_{k=1}^\infty a_k\) and \(\sum_{k=1}^\infty b_k\) be two absolutely convergent series. Does \(\sum_{k=1}^{\infty} a_k b_k\) converge absolutely as well?
\begin{proof}[Solution]
    Yes. Let \(\sum_{k=1}^\infty a_k\) and \(\sum_{k=1}^\infty b_k\) be two absolutely convergent series. We aim to show that \(\sum_{k=1}^\infty a_k b_k\) converges absolutely as well.
    
    Since \(\sum_{k=1}^\infty b_k\) converges absolutely, the sequence \(\{b_k\}\) is bounded, i.e., there exists \(M > 0\) such that \(|b_k| \leq M\) for all \(k \in \mathbb{N}\). Similarly, the absolute convergence of \(\sum_{k=1}^\infty a_k\) implies that \(\sum_{k=1}^\infty |a_k| < \infty\).
    
    Consider the series
    \[
    \sum_{k=1}^\infty |a_k b_k| = \sum_{k=1}^\infty |a_k| \cdot |b_k|.
    \]
    Using the boundedness of \(\{b_k\}\), we have
    \[
    |a_k b_k| \leq |a_k| \cdot M.
    \]
    Thus,
    \[
    \sum_{k=1}^\infty |a_k b_k| \leq M \sum_{k=1}^\infty |a_k|.
    \]
    Since \(\sum_{k=1}^\infty |a_k| < \infty\), it follows that \(\sum_{k=1}^\infty |a_k b_k|\) converges. Hence, \(\sum_{k=1}^\infty a_k b_k\) converges absolutely.
    \end{proof}
\section{Question 2}
Let \(\sum_{k=1}^\infty a_k\) and \(\sum_{k=1}^\infty b_k\) be two convergent series. Moreover, let \(a_k\) and \(b_k\) be two null sequences. Does \(\sum_{k=1}^\infty a_k b_k\) converge?
\begin{proof}[Solution]
    No. Consider
    \begin{align*}
        \sum_{k=1}^{\infty} a_k = \sum_{k=1}^{\infty} b_k = \sum_{k=1}^{\infty} \frac{(-1)^k}{\sqrt{k}} \text{.}
    \end{align*}
    \(a_k\) and \(b_k\) are clearly null sequences, and by alternating series test, both series converge. However,
    \begin{align*}
        \sum_{k=1}^\infty a_k b_k = \sum_{k=1}^\infty \frac{(-1)^k}{\sqrt{k}} \cdot  \frac{(-1)^k}{\sqrt{k}} = \sum_{k=1}^\infty \frac{1}{k}
    \end{align*}
    which is the harmonic series and hence does not converge.
\end{proof}
\section{Question 3}
Let \(\sum_{k=1}^\infty a_k\) and \(\sum_{k=1}^\infty b_k\) be two convergent series. Moreover, let \(a_k\) and \(b_k\) be monotonic sequences. Does \(\sum_{k=1}^{\infty}a_k b_k\) converge absolutely?
\begin{proof}[Solution]
    Yes. If \(a_k\) is monotonic and \(\sum_{k=1}^{\infty} a_k\) converge, then \(\sum_{k=1}^\infty a_k\) converges absolutely. Similarly, \(|b_k|\) is bounded by \(b_1\). Now, we have
    \begin{align*}
        \sum_{k=1}^{\infty} |a_k b_k | = \sum_{k=1}^{\infty} |a_k || b_k | \leq b_1 \sum_{k=1}^{\infty} |a_k |
    \end{align*}
    Thus, \(\sum_{k=1}^\infty a_k b_k\) converges absolutely.
\end{proof}
\section{Question 4}
Let \(\sum_{k=1}^\infty a_k\) converge and let \(\sum_{k=1}^\infty b_k\) converge absolutely. Does \(\sum_{k=1}^\infty a_k b_k\) converge absolutely?
\begin{proof}[Solution]
    Yes. If \(\sum_{k=1}^\infty a_k\) converges, then the sequence \(|a_k|\) converges to \(0\), thus it is bounded by say \(M\). We have
    \begin{align*}
        \sum_{k=1}^{\infty} |a_k b_k| = \sum_{k=1}^{\infty} |a_k | | b_k| \leq M \sum_{k=1}^{\infty} |b_k| \text{.}
    \end{align*}
    Hence \(\sum_{k=1}^{\infty} a_k b_k\) converges absolutely.
\end{proof}

\section{Question 5}

\end{document}