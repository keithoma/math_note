\begin{definition}[Symmetric difference]
    Let {\color{mathif}\(A, B\)} be {\color{mathif}sets}. The binary set operation {\color{maththen}symmetric difference} is defined as
    \begin{align}
        A \triangle B := (A \setminus B) \cup (B \setminus A) \text{.}
    \end{align}
    In other words, \(x \in A \triangle B\) implies \(x\) is either in \(A\) or \(B\), but not in both.
\end{definition}
%
\begin{proposition}[Properties of Symmetric Difference]
    Let \(A, B, C, X\) and \(Y\) be {\color{mathif}sets}. Moreover, let \(A_i\) and \(X_i\) be {\color{mathif}sets} with an {\color{mathif}arbitary non-empty index set} \(i \in I\). Then, the following {\color{maththen}identities} hold.
    \begin{enumerate}
        \item \( A \triangle B = (A \cup B) \setminus (A \cap B) \).
        \item \( ( A \triangle B) \triangle C = A \triangle (B \triangle C) \). (Symmetric difference is {\color{mathrem}associative}.) 
        \item \( A \triangle B = B \triangle A \). (Symmetric difference is {\color{mathrem}commutative}.)
        \item \( A \triangle \emptyset = A\) and \( A \triangle A = \emptyset\)
        \item \( (A \triangle B) \cup C = (A \cup C) \triangle (B \cup C)\).
        \item \( A \cap B = \emptyset \Rightarrow A \triangle B = A \cup B \).
        \item \( B \subset A \Rightarrow A \triangle B = A \setminus B \).
        \item \(X \cap Y = \emptyset \Rightarrow A \cap B \subset (X \triangle A) \cup (Y \triangle B) \).
        \item \( (\bigcup_{i \in I} X_i) \triangle (\bigcup_{i \in I} A_i) \subset \bigcup_{i \in I} (X_i \triangle A_i) \)
    \end{enumerate}
\end{proposition}
%
\begin{Proof}
    Let \(A, B, C, X\) and ...
\end{Proof}