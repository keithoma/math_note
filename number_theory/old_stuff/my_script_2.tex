\documentclass[a4paper]{book}
\title{Integration and Integration}
\author{K}


% ---------------------------------------------------------------------
% P A C K A G E S
% ---------------------------------------------------------------------

% typography and formatting
\usepackage[english]{babel}
\usepackage[utf8]{inputenc}
\usepackage{geometry}
\usepackage{exsheets}
\usepackage{environ}

% mathematics
% \usepackage{amsthm} % for theorems, and definitions
\usepackage{amssymb}
\usepackage{amsmath}
\usepackage{textcomp}
%\usepackage{MnSymbol} % for \cupdot

\usepackage[framemethod=tikz]{mdframed} 
\usepackage{ntheorem}

% extra
\usepackage{xcolor}
\usepackage{tikz}
\usepackage[sfdefault,light]{roboto}  %% Option 'sfdefault' only if the base font of the document is to be sans serif
\usepackage[T1]{fontenc}

\renewcommand*{\rmdefault}{roboto}


% ---------------------------------------------------------------------
% S E T T I N G
% ---------------------------------------------------------------------

% typography and formatting
\geometry{margin=3cm}

\SetupExSheets{
  counter-format = ch.qu,
  counter-within = chapter,
  question/print = true,
  solution/print = true,
}

% mathematics
% extra
\definecolor{mathif}{HTML}{0000A0} % for conditions
\definecolor{maththen}{HTML}{CC5500} % for consequences
\definecolor{mathrem}{HTML}{8b008b} % for notes
\definecolor{defbg}{HTML}{fde5d6}

\theoremstyle{break}
% \theoremheaderstyle{\rmfamily}
\theorembodyfont{\sffamily}

\newmdtheoremenv[
    ntheorem=true,
    hidealllines=true,
    backgroundcolor=defbg,
    splittopskip=2\baselineskip,
    middleextra={\node[anchor=north west,font=\bfseries,inner xsep=0pt,xshift=10pt] at (P-|O) {Example~\thetest\ (Continued)};},
    secondextra={\node[anchor=north west,font=\bfseries,inner xsep=0pt,xshift=10pt] at (P-|O) {Example~\thetest\ (Continued)};}
]{definition}{Definition}

\newtheorem{example}{Example}

\theoremstyle{plain}

\newtheorem{theorem}{Theorem}[definition]
\newtheorem{corollary}{Corollary}
\newtheorem{lemma}{Lemma}[definition]
\newtheorem{proposition}{Proposition}[definition]

\newtheorem{remark}{Remark}



\usetikzlibrary{positioning}
\usetikzlibrary{shapes.geometric, arrows}

% ---------------------------------------------------------------------
% C O M M A N D S
% ---------------------------------------------------------------------

\newcommand{\norm}[1]{\left\lVert#1\right\rVert}
\newcommand{\rank}{\text{rank}}
\newcommand{\Vol}{\text{Vol}}
\newcommand*\diff{\mathop{}\!\mathrm{d}}
\newcommand*\Diff{\mathop{}\!\mathrm{D}}

\newcommand\restr[2]{{% we make the whole thing an ordinary symbol
  \left.\kern-\nulldelimiterspace % automatically resize the bar with \right
  #1 % the function
  \vphantom{\big|} % pretend it's a little taller at normal size
  \right|_{#2} % this is the delimiter
  }}

% ---------------------------------------------------------------------
% R E N D E R
% ---------------------------------------------------------------------

\newif\ifshowproof
\showprooftrue

\NewEnviron{Proof}{%
    \ifshowproof%
        \begin{proof}%
            \BODY
        \end{proof}%
    \fi%
}%

\begin{document}
% taken from here: https://math.mit.edu/classes/18.785/2015fa/LectureNotes6.pdf
% https://math.mit.edu/classes/18.785/2015fa/LectureNotes7.pdf
\section{Ideal norms and the Dedekind-Kummer thoerem}

\begin{theorem}[Dedekind-Kummer Theorem]
    Assume \(AKLB\) with \(L = K(\alpha)\) and \(\alpha \in B\), let \(f \in A[X]\) be the minimal polynomial of \(\alpha\) and assume \(B = A[\alpha]\). Suppose \(g_1, \ldots, g_r \in A[X]\) are monic polynomials for which
    \begin{equation}
        \overline{f} = \overline{g_1}\,^{e_1} \cdots \overline{g_r}\,^{e_r}
    \end{equation}
    is a complete factorization of \(\overline{f} \in (A / \mathfrak{p}) [X]\), where \(\overline{\cdot}\) denotes reduction modulo \(\mathfrak{p}\), and let \(\mathfrak{q}_i := (\mathfrak{p}, g_i(\alpha))\) be the \(B\)-ideal genereated by \(\mathfrak{p}\) and \(g_i (\alpha)\). Then
    \begin{equation}
        pB = \mathfrak{q}_1^{e_1} \cdots \mathfrak{q}_r^{e^r} \text{,}
    \end{equation}
    is the prime factorization of \(\mathfrak{p}B\) in \(B\) and the residue degree of \(\mathfrak{q}_i\) is \(f_i := \deg{g_i}\).
\end{theorem}

\begin{definition}[Conductor]
    Let \(S/R\) be an {\color{mathif}extension of rings}. The {\color{maththen}conductor} \(\mathfrak{c}\) of \(R\) in \(S\) is the {\color{mathif}largest \(S\)-ideal} that is also an {\color{mathif}\(R\)-ideal}, i.e.,
    \begin{equation}
        \mathfrak{c} := \mathfrak{c}_{S / R} := \{ \, \alpha \in S \, \mid \, \alpha S \subseteq R \, \} = \{ \, \alpha \in R \, \mid \, a S \subseteq R \, \} \text{.}
    \end{equation}
\end{definition}

\begin{proposition}
    The assumption \(B = A[\alpha]\) in the Dedekind-Kummer theorem can be replaced with the assumption that \(\mathfrak{p}A[\alpha]\) is prime to the conductor of \(A[\alpha]\) in \(B\), i.e,
    \begin{equation}
        asdf
    \end{equation}
\end{proposition}

\section{Orders in Dedekind domains, primes in Galois extensions}

\begin{lemma}
    Let \(R\) be a noetherian domain. The conductor of \(R\) in its integral closure \(S\) is nonzero if and only if \(S\) is finitely generated as an \(R\)-module.
\end{lemma}

\begin{definition}[Order]
    An order \(\mathcal{O}\) is a noetherian domain of dimension one whose conductor is nonzero, equivalently, whose integral closure is finitely generated as an \(\mathcal{O}\)-module.
\end{definition}

\begin{lemma}
    Let \(\mathcal{O}\) be an order with integral closure \(B\) and conductor \(\mathfrak{c}\). A prime \(\mathfrak{q}\) of \(B\) contains \(\mathfrak{c}\) if and only if the prime \(\mathfrak{p} = \mathfrak{q} \cap \mathcal{O}\) of \(\mathfrak{O}\) contains \(\mathfrak{c}\). In particular, only finitely many primes \(\mathfrak{p}\) of \(\mathcal{O}\) contain \(\mathfrak{c}\).
\end{lemma}



\end{document}