\documentclass[a4paper]{book}
\title{Integration and Integration}
\author{K}


% ---------------------------------------------------------------------
% P A C K A G E S
% ---------------------------------------------------------------------

% typography and formatting
\usepackage[english]{babel}
\usepackage[utf8]{inputenc}
\usepackage{geometry}
\usepackage{exsheets}
\usepackage{environ}

% mathematics
% \usepackage{amsthm} % for theorems, and definitions
\usepackage{amssymb}
\usepackage{amsmath}
\usepackage{textcomp}
%\usepackage{MnSymbol} % for \cupdot

\usepackage[framemethod=tikz]{mdframed} 
\usepackage{ntheorem}

% extra
\usepackage{xcolor}
\usepackage{tikz}
\usepackage[sfdefault,light]{roboto}  %% Option 'sfdefault' only if the base font of the document is to be sans serif
\usepackage[T1]{fontenc}

\renewcommand*{\rmdefault}{roboto}


% ---------------------------------------------------------------------
% S E T T I N G
% ---------------------------------------------------------------------

% typography and formatting
\geometry{margin=3cm}

\SetupExSheets{
  counter-format = ch.qu,
  counter-within = chapter,
  question/print = true,
  solution/print = true,
}

% mathematics
% extra
\definecolor{mathif}{HTML}{0000A0} % for conditions
\definecolor{maththen}{HTML}{CC5500} % for consequences
\definecolor{mathrem}{HTML}{8b008b} % for notes
\definecolor{defbg}{HTML}{fde5d6}

\theoremstyle{break}
% \theoremheaderstyle{\rmfamily}
\theorembodyfont{\sffamily}

\newmdtheoremenv[
    ntheorem=true,
    hidealllines=true,
    backgroundcolor=defbg,
    splittopskip=2\baselineskip,
    middleextra={\node[anchor=north west,font=\bfseries,inner xsep=0pt,xshift=10pt] at (P-|O) {Example~\thetest\ (Continued)};},
    secondextra={\node[anchor=north west,font=\bfseries,inner xsep=0pt,xshift=10pt] at (P-|O) {Example~\thetest\ (Continued)};}
]{definition}{Definition}

\newtheorem{example}{Example}

\theoremstyle{plain}

\newtheorem{theorem}{Theorem}[definition]
\newtheorem{corollary}{Corollary}
\newtheorem{lemma}{Lemma}[definition]
\newtheorem{proposition}{Proposition}[definition]

\newtheorem{remark}{Remark}



\usetikzlibrary{positioning}
\usetikzlibrary{shapes.geometric, arrows}

% ---------------------------------------------------------------------
% C O M M A N D S
% ---------------------------------------------------------------------

\newcommand{\norm}[1]{\left\lVert#1\right\rVert}
\newcommand{\rank}{\text{rank}}
\newcommand{\Vol}{\text{Vol}}
\newcommand*\diff{\mathop{}\!\mathrm{d}}
\newcommand*\Diff{\mathop{}\!\mathrm{D}}

\newcommand\restr[2]{{% we make the whole thing an ordinary symbol
  \left.\kern-\nulldelimiterspace % automatically resize the bar with \right
  #1 % the function
  \vphantom{\big|} % pretend it's a little taller at normal size
  \right|_{#2} % this is the delimiter
  }}

% ---------------------------------------------------------------------
% R E N D E R
% ---------------------------------------------------------------------

\newif\ifshowproof
\showprooftrue

\NewEnviron{Proof}{%
    \ifshowproof%
        \begin{proof}%
            \BODY
        \end{proof}%
    \fi%
}%

\begin{document}
%\maketitle
%\tableofcontents

\chapter{Rings}

\begin{definition}[Ring]
    A {\color{maththen}ring} is a {\color{mathif}set} equipped with two {\color{mathif}binary operations} "\(+\)" ({\color{mathrem}addition}) and "\(\cdot\)" ({\color{mathrem}multiplication}) satisfying the following three sets of {\color{mathif}axioms}, called the {\color{mathrem}ring axioms}.
\end{definition}

\begin{remark}
    \begin{itemize}
        \item A nonzero commutative ring in which every nonzero element has a multiplicative inverse is a field.
        \item A structure with the same axiomatic definition but omitting the requirement of a multiplicative identity is called a rng.
    \end{itemize}
\end{remark}

\begin{example}
    \begin{enumerate}
        \item \((\mathbb{Z}, +, \cdot)\)
        \item All fields, such as \((\mathbb{Q}, +, \cdot)\), \((\mathbb{R}, +, \cdot)\), and \((\mathbb{C}, +, \cdot)\), are rings.
        \item The zero ring, denoted \(\{0\}\) with the operations \(0 + 0 = 0\) and \(0 \cdot 0 = 0\) is a commutative ring.
        \item Let \(R\) be a commutative ring, then \(R[X]\), the set of polynomials with coefficients in \(R\), is again a ring, e.g. \(\mathbb{Z}[X]\), \(\mathbb{Q}[X]\), and \(\mathbb{R}[X]\).
        \item For any ring \(R\) and for any \(n \in \mathbb{N}\), the set of all square \(n\)-by-\(n\) matricies with entries from \(R\), forms a ring with matrix addition and matrix multiplication as operations. If \(n = 1\), this matrix ring is isomorphic to \(R\) itself. For \(n > 1\) (and \(R\) not a zero ring), this matrix is noncommutative. More concretely, \(\text{Mat}_{3 \times 3}(\mathbb{R})\) is a noncommutative ring. 
    \end{enumerate}
\end{example}

\newpage

\section{Integral Domain}

\textit{Integral domains are generalization of the ring of integers and provide a natural setting for studying divisibility. In an integral domain, every nonzero element \(a\) has the cancellation property, that is, if \(a \neq 0\), an equality \(ab = ac\) implies \(b = c\).}

\begin{definition}
    An {\color{maththen}integral domain} \(R\) is a {\color{mathif}nonzero commutative ring} in which the product of any two nonzero elements is nonzero, i.e. for all \(a, b \in R \setminus \{0\}\) it is \(a \cdot b \neq 0\). Equivalently:
    \begin{enumerate}
        % stop making nonzero blue
        \item An {\color{maththen}integral domain} \(R\) is a {\color{mathif}nonzero commutative ring} with no nonzero {\color{mathif}zero divisors}, i.e. there exists no element \(a \in R \setminus \{0\}\) such that \(a \cdot x = 0\) for some \(x \in R\).
        \item An {\color{maththen}integral domain} \(R\) is a {\color{mathif}commutative ring} in which the {\color{mathif}zero ideal} \(\{0\}\) is a {\color{mathif}prime ideal}.
        \item An {\color{maththen}integral domain} \(R\) is a {\color{mathif}nonzero commutative ring} for which every nonzero element is {\color{mathif}cancellable under multiplication}, i.e. if \(a \in R \setminus \{0\}\), an equality \(ab = ac\) implies \(b = c\).
        \item An {\color{maththen}integral domain} \(R\) is a {\color{mathif}ring} for which the {\color{mathif}set of nonzero elements} is a {\color{mathif}commutative monoid} under multiplication.
        \item An {\color{maththen}integral domain} \(R\) is a {\color{mathif}nonzero commutative ring} in which for every nonzero element \(r\), the {\color{mathif}function} that maps each element \(x\) of the ring to the product \(xr\) is {\color{mathif}injective}. Elements \(r\) with this property are called {\color{mathrem}regular}, so it is equivalent to require that every nonzero element of the ring be regular.
        \item An {\color{maththen}integral domain} \(R\) is a {\color{mathif}ring} that is {\color{mathif}isomorphic} to a {\color{mathif}subring} of a {\color{mathif}field}.
    \end{enumerate}
    % proof should be elementary.
\end{definition}

\begin{example}
    \begin{enumerate}
        \item \(\mathbb{Z}\).
        \item Every field such as \(\mathbb{Q}\), \(\mathbb{R}\), or \(\mathbb{C}\) are integral domains.
    \end{enumerate}
\end{example}

\section{Unique Factorization Domain}

\textit{A unique factorization domain (UFD) or factorial ring is an integral domain in which every nonzero non-unit element can be written as a product of prime elements, uniquely up to order and units. Therefore, in a unique factorization domain a statement analogous to the fundamental theorem of arithmetic holds.}

\begin{definition}
    A unique factorization domain is an integral domain \(R\) in which every nonzero element can be written as a product of unit and prime elements of \(R\).
\end{definition}

\section{Noetherian Ring}

\section{Dedekind Domain}

\section{Principal Ideal Domain}

\textit{A principal ideal domain (PID) is an in which every ideal is principal, i.e., can be generated by a single element. Thus, principal ideal domains are structures that behave somewhat like the integers, with respect to divisibility as firstly, any element of a principal ideal domain has a unique decomposition into prime elements, and secondly, any two elements of a principal ideal domain have a greatest common divisor.}


\chapter{something}

\section{No idea yet}

\begin{definition}[Fractional Ideal]
    Let \(A\) be an {\color{mathif}integral domain}.
    \begin{enumerate}
        \item A {\color{maththen}fractional ideal} of \(A\) is an {\color{mathif}\(A\)-submodule} \(I \subset \text{Quot}(A)\) such that \(d I \subset A\) for some {\color{mathrem}denominator} \(d \in A \setminus \{0\}\).
        \item A {\color{maththen}principal fractional ideal} is a {\color{mathif}fractional ideal} of the form \((r) = rA = \{ar \mid a \in A\}\)
    \end{enumerate}
\end{definition}

\begin{example}
    \begin{itemize}
        \item All ordinary ideals \(I \subset A\) are also fractional ideals with denominator \(d = 1\), and are often referred to as {\color{mathrem}integral ideals}.
        \item The subset
        \begin{equation}
            \frac{3}{25}\mathbb{Z} = \left\{\, \frac{3n}{25} \in \mathbb{Q} \,\middle|\, n \in \mathbb{Z} \,\right\} \subset \mathbb{Q}
        \end{equation}
        is a principal fractional ideal of \(\mathbb{Z}\)
    \end{itemize}
\end{example}

\begin{example}
    The subset
    \begin{equation}
        \mathbb{Z}\left[\frac{1}{2}\right] = \left\{\, a_0 + a_1 \frac{1}{2} + a_2 \frac{1}{2^2} + \dots + a_n \frac{1}{2^n} \,\middle|\, a_0, \dots, a_n \in \mathbb{Z} \subset \mathbb{Q} \,\right\}
    \end{equation}
    is not a fractional ideal, because the denominators are not bounded.
\end{example}

\begin{lemma}
    If \(I \subset \text{Quot}(A)\) is an \(A\)-submodule and \(d \in \text{Quot}(A)\), then \(dI \subset \text{Quot}(A)\) is also an \(A\)-module. Thus \(I \subset K\) is a fractional ideal if and only if \(I = \frac{1}{d} J\) for some \(d \in A \setminus \{0\}\) and some integral ideal \(J \subset A\) (just take \(d\) a denominator of \(I\) and \(J = dI\)).
\end{lemma}

\begin{lemma}
    Let \(A\) be an integral domain and denote its field of fraction with \(\text{Quot}(A) = K\).
    \begin{enumerate}
        \item If \(I \subset K\) is a finitely generated \(A\)-submodule, then \(I\) is a fractional ideal.
        \item Conversely, if \(A\) is noetherian and \(I \subset K\) is a fractional ideal, then \(I\) is a finitely generated \(A\)-module.
        \item If \(I, J \subset K\) are fractional ideals, then \(I \cap J, I + J, IJ, \subset K\) are also fractional ideals.
        \item If \(I, J \subset K\) are fractional ideals and \(J \neq {0}\), then the generalized ideal quotient
        \begin{equation}
            (I : J) := \{\, x \in K \, \mid \, xJ \subset I \,\}
        \end{equation}
        is also a fractional ideal. Moreover, it satisfies \((I : J)J \subset I\).
    \end{enumerate}
\end{lemma}
% proof

The nonzero fractional ideals form an abelian semigroup with neutral element \(A\) with respect to the multiplication. We will now show that, if \(A\) is a Dedekind domain, every nonzero fractional ideal has an inverse hence they forme an abelian group \(Id (A)\).

\begin{definition}
    Let \(A\) be an {\color{mathif}integral domain}. A {\color{mathif}fractional ideal} \(I \subset K\) is {\color{maththen}invertible} if \(IJ = A\) for some fractional ideal \(J\) called the {\color{mathrem}inverse} of \(I\).
\end{definition}

The following result shows characterizes invertible fractional ideals and their inverses (which are unique).

\begin{lemma}
    A fractional ideal \(I\) is invertible if and only if \(I (A : I) = A\), in which case \(I^{-1} := (A : I)\) is the unique inverse.
\end{lemma}

% proof

The main result of this section is to prove that, in a Dedekind domain, every nonzero ideal is invertible. To this aim we need first a technical result.

\begin{lemma}
    Let \(A\) be a {\color{mathif}Dedekind domain} and \(I \subset A\) a {\color{mathif}nonzero integral ideal}. Then there are not necessarily distinct {\color{maththen}nonzero prime ideals} \(\mathfrak{p}_1, \dots, \mathfrak{p}_n \subset A\) such that \(\mathfrak{p}_1 \cdot \dots \cdot \mathfrak{p}_n \subset I\).
\end{lemma}

% proof

Let
\begin{equation}
    \Sigma = \{\, I \neq \{0\} \, \mid \, I \subset A \text{ ideal. \(I\) does not contain any product of nonzero prime ideals.} \, \} \text{.}
\end{equation}

If \(\Sigma \neq \varnothing\), let \(I \in \Sigma\) be a maximal element which must exist since \(A\) is noetherian. In particular, \(I\) is not prime, i.e. there exists \(a, b \in A \setminus I\) with \(a \cdot b \in I\). % why is I not prime?

Because of the maximility of \(I\), the ideals \(I + (a), I + (b) \supsetneq I\) don't lie in \(I\), i.e. there exists nonzero prime ideals \(\mathfrak{p}_1, \dots, \mathfrak{p}_n, \mathfrak{q}_1, \dots, \mathfrak{q}_m\) such that
\begin{align}
    \mathfrak{p}_1, \dots, \mathfrak{p}_n &\subseteq I + (a) \\
    \mathfrak{q}_1, \dots, \mathfrak{q}_n &\subseteq I + (b) \text{.}
\end{align}
We have
\begin{equation}
    \mathfrak{p}_1 \cdot \dots \cdot \mathfrak{p}_n \cdot \mathfrak{q}_1 \cdot \dots \cdot \mathfrak{q}_m \subseteq (I + (a))(I + (b)) \subseteq I
\end{equation}
which is a contradiction. Hence \(\Sigma = \varnothing\).

\begin{theorem}
    Let \(A\) be a Dedekind domain, \(I\) a nonzero ideal, and \(\mathfrak{p}\) a prime ideal such that \(I \subseteq \mathfrak{p}\). Set
    \begin{equation}
        \mathfrak{p}^{-1} := (A : \mathfrak{p}) = \{x \in \text{Quot}(A) \mid x \mathfrak{p} \subseteq A\} \text{.}
    \end{equation}
    Then, \(I \subsetneq \mathfrak{p}^{-1} I \subseteq A\). In particular, \(A \subsetneq \mathfrak{p}^{-1}\) and \(\mathfrak{p}^{-1}\mathfrak{p} = A\), i.e. \(\mathfrak{p}\) is invertible.
\end{theorem}

\begin{corollary}
    Let \(A\) be a Dedekind domain and
    \begin{equation}
        Id(A) = \{I \subseteq K \mid I \text{ is a nonzero fractional ideal.}\} \text{.}
    \end{equation}
    \begin{enumerate}
        \item Every nonzero fractional ideal \(I \in Id(A)\) is invertible. In particular, \(Id(A)\) is an abelian group with respect to the product of ideals, and the trivial ideal \((1) = A\) as neutral element.
        \item Moreover, the map
        \begin{equation}
            \varphi: K^* \rightarrow Id(A), \quad \frac{a}{b} \mapsto \left(\frac{a}{b}\right) = \left\{\frac{ac}{b} \middle| c \in A \right\} \subseteq K \text{,}
        \end{equation}
        is a group homomorphism, whose image is the subgroup \(P_A\) of nonzero principal fractional ideals.
    \end{enumerate}
\end{corollary}

\newpage

\begin{definition}
    The {\color{maththen}(ideal) class group} of a {\color{mathif}Dedekind domain} \(A\) is the quotient \(Cl(A) = Id(A) / P_A\) which is naturally an abelian group.
\end{definition}

\begin{remark}
    Two crucial objects in the study of a Dedekind domain \(A\) are the group of units \(A^*\) and the class group \(Cl(A)\).
    \begin{enumerate}
        \item For example, \(A\) is a principal ideal domain if and only if the class group is trivial.
        \item In general, it is immediate that the kernel of \(\varphi\) is the set of units \(A^*\). Hence there is an exact sequence of abelian groups
        \begin{equation}
            1 \rightarrow A^* \rightarrow K^* \rightarrow Id(A) \rightarrow Cl(A) \rightarrow 0 \text{.}
        \end{equation}
    \end{enumerate}
\end{remark}

\section{Divisibility and unique factorization of ideals}

\begin{theorem}
    Let \(I \subseteq K = \text{Quot}(A)\) be a nonzero fractional ideal of \(A\).
    \begin{enumerate}
        \item There exist an integer \(n \in \mathbb{N}_0\), distinct nonzero prime ideals \(\mathfrak{p}_1, \ldots, \mathfrak{p}_n \subseteq A\), and integers \(r_1, \ldots, r_n \in \mathbb{Z} \setminus \{0\}\) such that
        \begin{equation}
            I = \mathfrak{p}_1^{r_1} \cdot \ldots \cdot \mathfrak{p}_n^{r_n}
        \end{equation}
        with the convention that the empty product \(n = 0\) is \(A\), and \(\mathfrak{p}^{-r} := \left(\mathfrak{p}^{-1}\right)^r\) for any nonzero \(r \in \mathbb{N}\).
        \item The decomposition is unique up to permutation of the factors.
        \item \(I \subseteq A\) if and only if \(r_1, \ldots, r_n \geq 0\).
    \end{enumerate}
\end{theorem}

\begin{corollary}
    the chinese remainder theorm.
\end{corollary}

\begin{definition}
    For every nonzero prime ideal \(\mathfrak{p} \subseteq A\), we define \(v_\mathfrak{p}(I) \in \mathbb{Z}\) as the exponent of \(\mathfrak{p}\) in the unique factorization of \(I\) into a product prime ideals.
\end{definition}

\section{The case of local Dedekind domains}

\begin{definition}
    A ring \(A\) is called local if it contains a unique maximal ideal \(\mathfrak{m}\). Sometimes one says that the pair \((A, \mathfrak{m})\) is a local ring.
\end{definition}

\section{Chapter 5}

How to compute the prime factorization \(I = \mathfrak{p}_1^{r_2} \cdot \ldots \cdot \mathfrak{p}_n^{r_n}\) of a nonzero ideal in a Dedekind domain \(I \subseteq A\)?

One idea is to find a smaller Dedekind subring \(A^\prime \subseteq A\) where we can compute these factorizations and then

\begin{enumerate}
    \item Factorize \(I \cap A^\prime \subseteq A^\prime \Rightarrow I \cap A^\prime = \tilde{\mathfrak{p}}_1^{s_1}, \cdot \ldots \cdot \tilde{\mathfrak{p}}_k^{s_k}\).
    \item Factorize \(\tilde{\mathfrak{p}}_i^{s_i} \cdot A \subset A \Rightarrow \tilde{\mathfrak{p}}_1^{s_1} \cdot A = \prod_{j=1}^{N_i} \mathfrak{p}_{i, j}^{e_{i, j}}\).
    \item For each \(\mathfrak{p}_{i, j}\) find the right exponent, i.e. smallest \(k\) such that \(I \subseteq \mathfrak{p}_{i, j}^k\) (\(k \leq s_i \cdot e_{i, j}\)).
\end{enumerate}

Another approch is

\begin{enumerate}
    \item list all prime ideals \(\mathfrak{p} \subseteq A\), \(\mathfrak{p}_1, \mathfrak{p}_2, \mathfrak{p}_3, \ldots\).
    \item localize at \(\mathfrak{p}_1\), compute \(r_1 = v_{\mathfrak{p}_1} (I \cdot A_{\mathfrak{p}_1})\) check if \(I = \mathfrak{p}_2^{r_1}\)
    \item If not, then compute again
    \item jadajadajada
\end{enumerate}

\begin{definition}
    The spectrum of a ring \(A\) is
    \begin{equation}
        \text{Spec}(A) = \{\mathfrak{p} \subseteq A \mid \mathfrak{p} \text{ prime ideal}\}.
    \end{equation}
\end{definition}

\begin{lemma}
    In the \(AKLB\)-setting, let \(\mathfrak{p} \subseteq A\) and \(\mathfrak{q} \subseteq B\) be prime ideals. Then the following holds.
    \begin{enumerate}
        \item \(\mathfrak{q}\) divides \(\mathfrak{p}B\) if and only if \(\mathfrak{p} = \mathfrak{q} \cap A\).
        \item Given \(\mathfrak{p}\), there is always such a \(\mathfrak{q}\).
    \end{enumerate}
\end{lemma}

\begin{enumerate}
    \item We have
    \begin{align}
        & \mathfrak{q} \mid \mathfrak{p}B \\
        \iff & \mathfrak{p}B \subseteq \mathfrak{q} & \text{there is a lemma for this} \\
        \iff & \mathfrak{p} \subseteq \mathfrak{p} B \cap A \subseteq \mathfrak{q} \cap A & \\
        \iff & \mathfrak{p} = \mathfrak{q} \cap A &
    \end{align}
\end{enumerate}

\newpage

\begin{definition}
    Let \(A\) be a {\color{mathif}Dedekind domain}, \(K = \text{Quot}(A)\) its {\color{mathif}field of fraction}, \(L / K\) a {\color{mathif}finite separable field extension}, and \(B = \overline{A}\) the {\color{mathif}integral closure} of \(A\) in \(L\).
    
    Moreover, let \(\mathfrak{p} \subset A\) and \(\mathfrak{q} \subset B\) be two {\color{mathif}prime ideals}. We say that \(\mathfrak{q}\) {\color{mathrem}lies over} \(\mathfrak{p}\) if \(\mathfrak{q} \, | \, \mathfrak{p}B\), i.e. \(\mathfrak{q} \cap A = \mathfrak{p}\). In this case, define

    \begin{enumerate}
        \item \(e_{\mathfrak{q} | \mathfrak{p}} = v_\mathfrak{q}(\mathfrak{p}B) \in \mathbb{Z}_{>0}\) the {\color{maththen}ramification index} of \(\mathfrak{q}\) over \(\mathfrak{p}\).
    \end{enumerate}
\end{definition}

\begin{example}
    Consider \(A = \mathbb{Z}\), \(K = \mathbb{Q}\), \(L = \mathbb{Q}(i)\), so that \(B = \mathcal{O}_L = \mathbb{Z}[i]\). For a nonzero prime ideal \(\mathfrak{p} = (p) \subseteq \mathbb{Z}\).
    \begin{enumerate}
        \item \(p \mathbb{Z}[i] = \mathfrak{q}^2 = (1 + i)^2\) for \(p = 2\), i.e. \((2) \subseteq \mathbb{Z}\) is ramified (with ramification index \(e_{\mathfrak{q} | \mathfrak{p}} = 2\)). The residue class field \(\mathbb{F}_\mathfrak{q} \cong \mathbb{F}_2\), hence
    \end{enumerate}
\end{example}

\begin{example}
    Let \(\alpha := \sqrt[3]{2}\). Consider a Dedekind domain \(A := \mathbb{Z}\), \(K := \text{Quot}(A)\), \(L := \mathbb{Q}(\alpha)\), and \(B := \mathcal{O}_K\) the integral closure of \(\mathbb{Z}\) in \(\mathbb{Q}(\alpha)\).

    Take a prime ideal \((2) \subseteq A\), then \((2) \mathcal{O}_K\)
\end{example}


\begin{theorem}
    % A integral domain?
    Let \(A\) be a ring and let \(B = A[\alpha]\), and let \(f(X) \in A[X]\) be the minimal polynomial of \(\alpha\). Moreover, let \(\mathfrak{p} \subseteq A\) be a nonzero prime ideal and \(g_1(X), \ldots, g_r(X) \in A[X]\) monic such that
    \begin{equation}
        \overline{f(X)} = \overline{g_1(X)}^{e_1} \cdot \ldots \cdot \overline{g_r(X)}^{e_r} \mod{p} \in A / \mathfrak{p} [X] = \mathbb{F}_p [X] \text{.}
    \end{equation}
    Then,
    \begin{equation}
        \mathfrak{p}B = \prod_{i=1}^r Q_i^{e_i} \qquad \text{with } Q_i = (\mathfrak{p}, g_i(\alpha)) \subseteq B
    \end{equation}
    is the prime factorization of \(\mathfrak{p}B\).
\end{theorem}

\begin{example}
    Let \(D \in \mathbb{Z}\) be squarefree with \(D \equiv 2, 3 \mod{4}\) and \(L = \mathbb{Q}(\sqrt{D})\), such that \(B = O_L = \mathbb{Z}[\sqrt{D}]\) with the minimal polynomial \(f(X) = X^2 - D \in \mathbb{Z}[X]\).

    Let \(p \in \mathbb{Z}\) be a prime number and look for the factorization of \(pB = p \mathcal{O}_L = p \mathbb{Z}[\sqrt{D}]\).

    \noindent \textbf{Case A:} If \(p \neq 2\) consider the factorization of \(X^2 - D \in \mathbb{Z} / p \mathbb{Z} [X] = \mathbb{F}_p [X]\).

    \noindent \textbf{Case A1:} If \(p \, | \, D\) then \(\overline{f(X)} = X^2\), so \(pB = (p, \sqrt{D})^2\), with \(B / (p, \sqrt{p}) \cong \mathbb{F}_p[X]/(X) \cong \mathbb{F}_p\).
\end{example}

\begin{example}
    Denote \(\alpha = \sqrt[3]{2}\) and let \(A := \mathbb{Z}\), 
\end{example}

\newpage

\section{Ramification}

\begin{definition}[Ramification Index]
    Let \(K\) be an algebraic number field of degree \(n\). Let \(\mathfrak{p}\) be a prime ideal of \(\mathcal{O}_K\). Let \(p\) be a rational prime lying below \(P\). Then the unique positive integer \(e\) such that
    \begin{equation}
        \mathfrak{p}^e \mid (p), \, \mathfrak{p}^{e + 1} \neq\mid (p)
    \end{equation}
    is called the ramification index of \(\mathfrak{p}\) in \(K\) and is written \(e_K(\mathfrak{p})\).
\end{definition}

\end{document}
