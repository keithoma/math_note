\documentclass[a4paper]{article}
\title{Integration and Integration}
\author{K}


% ---------------------------------------------------------------------
% P A C K A G E S
% ---------------------------------------------------------------------

% typography and formatting
\usepackage[english]{babel}
\usepackage[utf8]{inputenc}
\usepackage{geometry}
\usepackage{exsheets}
\usepackage{environ}

% mathematics
\usepackage{amsthm} % for theorems, and definitions
\usepackage{amssymb}
\usepackage{amsmath}
\usepackage{textcomp}
%\usepackage{MnSymbol} % for \cupdot

% extra
\usepackage{xcolor}
\usepackage{tikz}

% ---------------------------------------------------------------------
% S E T T I N G
% ---------------------------------------------------------------------

% typography and formatting
\geometry{margin=3cm}

\SetupExSheets{
  counter-format = ch.qu,
  counter-within = chapter,
  question/print = true,
  solution/print = true,
}

% mathematics
\theoremstyle{definition}
\newtheorem{definition}{Definition}
\newtheorem{example}{Example}[definition]

\newtheorem{theorem}{Theorem}[definition]
\newtheorem{corollary}{Corollary}
\newtheorem{lemma}{Lemma}[definition]
\newtheorem{proposition}{Proposition}[definition]

\newtheorem*{remark}{Remark}

% extra
\definecolor{mathif}{HTML}{0000A0} % for conditions
\definecolor{maththen}{HTML}{CC5500} % for consequences
\definecolor{mathrem}{HTML}{8b008b} % for notes

\usetikzlibrary{positioning}
\usetikzlibrary{shapes.geometric, arrows}

% ---------------------------------------------------------------------
% C O M M A N D S
% ---------------------------------------------------------------------

\newcommand{\norm}[1]{\left\lVert#1\right\rVert}
\newcommand{\rank}{\text{rank}}
\newcommand{\Vol}{\text{Vol}}
\newcommand*\diff{\mathop{}\!\mathrm{d}}
\newcommand*\Diff{\mathop{}\!\mathrm{D}}

\newcommand\restr[2]{{% we make the whole thing an ordinary symbol
  \left.\kern-\nulldelimiterspace % automatically resize the bar with \right
  #1 % the function
  \vphantom{\big|} % pretend it's a little taller at normal size
  \right|_{#2} % this is the delimiter
  }}

% ---------------------------------------------------------------------
% R E N D E R
% ---------------------------------------------------------------------

\newif\ifshowproof
\showprooftrue

\NewEnviron{Proof}{%
    \ifshowproof%
        \begin{proof}%
            \BODY
        \end{proof}%
    \fi%
}%

\begin{document}
\begin{center}
    \noindent\textbf{Exercise Sheet 2}
\end{center}
\noindent\textbf{Exercise 1}

\noindent Which of the following rings are Dedekind domains?

\begin{enumerate}
    \item \(\mathbb{Z} \times \mathbb{Z}\).
    \item \(\mathbb{Z}[X] / (X^2 + 3)\).
    \item \(\mathbb{F}_{11}[X]\).
    \item \(\mathbb{R}[X, Y]\).
    \item \(\mathbb{C}[X, Y] / (X^5 + Y - 13)\).
\end{enumerate}

\noindent\textbf{Solution}

\begin{enumerate}
    \item \(\mathbb{Z}\times\mathbb{Z}\) is not a Dedekind domain as it is not even an integral domain. Take \((1, 0) \in \mathbb{Z}\times\mathbb{Z}\) and \((0, 1) \in \mathbb{Z}\times\mathbb{Z}\) for example. \((1, 0) \cdot (0, 1) = (0, 0)\) even though we chose nonzero elements.
    \item \(\mathbb{Z}[X] / (X^2 + 3)\) is not a Dedekind domain as it is not integrally closed.

    First, define a ring homomorphism \(\varphi: \mathbb{Z}[X] \rightarrow \mathbb{Z}\) that substitues \(X\) with \(\sqrt{-3}\). We have \(\varphi(\mathbb{Z}[X]) = \mathbb{Z}[\sqrt{-3}]\) and \(\text{ker} (\varphi) = (X^2 + 3)\). With the first isomorphism theorem for rings, we have \(\mathbb{Z}[X] / (X^2 + 3) \cong \mathbb{Z}[\sqrt{-3}]\).

    Consider

    \begin{equation}
        \alpha := \frac{1}{2} + \frac{1}{2} \sqrt{-3} \in \text{Quot}(\mathbb{Z}[\sqrt{-3}]) \cong \mathbb{Q}(\sqrt{-3}) \text{.}
    \end{equation}

    From example 3.2.5. (script), we know that

    \begin{equation}
        \mathcal{O}_{\mathbb{Q}(\sqrt{-3})} = \mathbb{Z}[\alpha] \text{.}
    \end{equation}
    
    Therefore, \(\alpha\) is integral over \(\mathbb{Z}\) and hence over \(\mathbb{Z}[\sqrt{-3}]\) as well, but it does not lie in \(\mathbb{Z}[\sqrt{-3}]\). We conclude \(\mathbb{Z}[\sqrt{-3}]\) and with it \(\mathbb{Z}[X] / (X^2 + 3)\) are not integrally closed.
    % We have \(\mathbb{Z}[X] / (X^2 + 3) \cong \mathbb{Z}[\sqrt{-3}]\); therefore, \(\mathbb{Z}[X] / (X^2 + 3)\) is an integral domain.
    %
    % \(\mathbb{Z}[X] / (X^2 + 3)\) is also noetherian as \(\mathbb{Z}\) is noetherian and therefore it's polynomial ring and the quotient ring. 
    % why are they isomorphic? use isomorphism theorem 
    % it is also noetherian
    % is integrally closed in Quot(A)?
    % all above doesn't matter, this is not a Dedekind domain
    % it is integrally closed
    % it does not satisfy the last condition though
    \item \(\mathbb{F}_{11}[X]\) is a Dedekind domain.
    
    From remark 1.0.3. / 2. (script), we have that the ring of polynomials in one variable over a field is a Euclidean ring, so \(\mathbb{F}_{11}[X]\) is a Euclidean ring. Every Euclidean ring is a principal ideal domain (remark 1.0.3. / 3. from the script) and every principal ideal domain is a Dedekind domain (example 4.1.10. from the script). Hence, \(\mathbb{F}_{11}[X]\) is a Dedekind domain.
    % as fields are integral domain and polynomial rings over integral domains are integral domain, this is integral domain
    % same as above, field noetherian, polynomial rings over noetherian rings are noetherian
    % as it is a principal ideal domain, it is integrally closed domain, so it is integral domain
    % this is a Dedekind domain as it is a PID!!
    \item \(\mathbb{R}[X, Y]\) is not a Dedekind domain because not all nonzero prime ideals are maximal.

    Consider the ideal \((X^2 - Y)\) and the quotient generated \(\mathbb{R}[X, Y] / (X^2 - Y)\). If we can show that the quotient is an integral domain, but not a field, then the ideal \((X^2 - Y)\) is prime, but not maximal.

    Define a ring homomorphism \(\varphi: \mathbb{R}[X, Y] \rightarrow \mathbb{R}[X]\) that substitues the variable \(Y\) with \(X^2\). We have that \(\varphi(\mathbb{R}[X, Y]) = \mathbb{R}[X]\) and \(\text{ker}(\varphi) = (X^2 - Y)\). From the first isomorphism theorem for rings, we get \(\mathbb{R}[X, Y] / (X^2 - Y) \cong \mathbb{R}[X]\).

    As \(\mathbb{R}[X]\) is a integral domain (polynomial rings over integral domains are again an integral domain), but not a field, the ideal \((X^2 - Y)\) is a non-maximal prime ideal. Hence \(\mathbb{R}[X, Y]\) is not a Dedekind domain.
    % R[X, Y] is a UFD => prime == irreducible => (X) is pricipal == (X) is prime ideal
    % (X, Y)
    \item \(\mathbb{C}[X, Y] / (X^5 + Y - 13)\) is a Dedekind domain.
    
    As we did in 2., define a ring homomorphism \(\varphi: \mathbb{C}[X, Y] \rightarrow \mathbb{C}[X]\) as

    \begin{equation}
        \varphi(p(X, Y)) := p(X, X^5 - 13) \text{.} 
    \end{equation}

    \(\varphi\) is surjective and the kernel is \((X^5 + Y - 13)\). With the first isomorphism theorem for rings, we have the isomorphism \(\mathbb{C}[X, Y] / (X^5 + Y - 13) \cong \mathbb{C}[X]\).

    Using similar argument as in 3., \(\mathbb{C}[X]\) is a principal ideal domain and hence \(\mathbb{C}[X, Y] / (X^5 + Y - 13)\) is a Dedekind domain.
    % integral domain (x^5 + Y - 13) is prime because its principal
    % noetherian because C[X, Y] is noetherian quotient is also neotherian
    % 
    % the point is if this is a integrally
\end{enumerate}

\end{document}