\documentclass[a4paper]{article}
\title{Integration and Integration}
\author{K}


% ---------------------------------------------------------------------
% P A C K A G E S
% ---------------------------------------------------------------------

% typography and formatting
\usepackage[english]{babel}
\usepackage[utf8]{inputenc}
\usepackage{geometry}
\usepackage{exsheets}
\usepackage{environ}

% mathematics
\usepackage{amsthm} % for theorems, and definitions
\usepackage{amssymb}
\usepackage{amsmath}
\usepackage{textcomp}
%\usepackage{MnSymbol} % for \cupdot

% extra
\usepackage{xcolor}
\usepackage{tikz}

% ---------------------------------------------------------------------
% S E T T I N G
% ---------------------------------------------------------------------

% typography and formatting
\geometry{margin=3cm}

\SetupExSheets{
  counter-format = ch.qu,
  counter-within = chapter,
  question/print = true,
  solution/print = true,
}

% mathematics
\theoremstyle{definition}
\newtheorem{definition}{Definition}
\newtheorem{example}{Example}[definition]

\newtheorem{theorem}{Theorem}[definition]
\newtheorem{corollary}{Corollary}
\newtheorem{lemma}{Lemma}[definition]
\newtheorem{proposition}{Proposition}[definition]

\newtheorem*{remark}{Remark}

% extra
\definecolor{mathif}{HTML}{0000A0} % for conditions
\definecolor{maththen}{HTML}{CC5500} % for consequences
\definecolor{mathrem}{HTML}{8b008b} % for notes

\usetikzlibrary{positioning}
\usetikzlibrary{shapes.geometric, arrows}

% ---------------------------------------------------------------------
% C O M M A N D S
% ---------------------------------------------------------------------

\newcommand{\norm}[1]{\left\lVert#1\right\rVert}
\newcommand{\rank}{\text{rank}}
\newcommand{\Vol}{\text{Vol}}
\newcommand*\diff{\mathop{}\!\mathrm{d}}
\newcommand*\Diff{\mathop{}\!\mathrm{D}}

\newcommand\restr[2]{{% we make the whole thing an ordinary symbol
  \left.\kern-\nulldelimiterspace % automatically resize the bar with \right
  #1 % the function
  \vphantom{\big|} % pretend it's a little taller at normal size
  \right|_{#2} % this is the delimiter
  }}

% ---------------------------------------------------------------------
% R E N D E R
% ---------------------------------------------------------------------

\newif\ifshowproof
\showprooftrue

\NewEnviron{Proof}{%
    \ifshowproof%
        \begin{proof}%
            \BODY
        \end{proof}%
    \fi%
}%

\begin{document}
\begin{center}
    \noindent\textbf{Exercise Sheet 1}
\end{center}
\noindent\textbf{Exercise 2}

Let \(k \in \mathbb{Z}_{>0}\).
\begin{enumerate}
    \item Show that \(k = a^2 + b^2\) for some \(a, b \in \mathbb{Z}\) if and only if for every prime \(p \equiv 3 \mod 4\), the exponent of \(p\) in the prime decomposition of \(k\) (in \(\mathbb{Z}\)) is even.
    \item In this case, describe how to obtain all solutions \((a, b) \in \mathbb{Z}^2\).
\end{enumerate}

\noindent\textbf{Solution}

\noindent\underline{1.}

\noindent Let \(k = a^2 + b^2\) for some \(a, b \in \mathbb{Z}\). We show that for for every prime \(p \equiv 3 \mod 4\), the exponent of \(p\) in the prime decomposition of \(k\) is even.

First, let \(p \in \mathbb{Z}\) be a prime number and consider \(p\) in \(\mathbb{Z}[i]\). We have \(N(p) = p\) and from the multiplicative property of the norm it follows that \(p\) is also irreducible in \(\mathbb{Z}[i]\).

Now, we have \(k = a^2 + b^2 = (a + ib)(a - ib)\) and let \(p\) be a prime in the decomposition of \(k\). As \(p\) divides \(k\) and is a irreducible element in \(\mathbb{Z}[i]\), it divides \((a + ib)\) or \((a - ib)\). But since \(p\) is real it has to divide both \((a + ib)\) and \((a - ib)\). Let \(p^n\) be the highest exponent that divides \((a + ib)\), then \(p^{2n}\) divides \((a + ib)(a - ib) = k\). We conclude that every prime in the prime decomposition of \(k\) have even exponents.
\end{document}