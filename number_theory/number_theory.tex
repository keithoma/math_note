\documentclass[a4paper]{book}
\title{Number Theory}
\author{K}


% ---------------------------------------------------------------------
% P A C K A G E S
% ---------------------------------------------------------------------

% typography and formatting
\usepackage[english]{babel}
\usepackage[utf8]{inputenc}
\usepackage{geometry}
\usepackage{exsheets}
\usepackage{environ}

% mathematics
\usepackage{amsthm} % for theorems, and definitions
\usepackage{amssymb}
\usepackage{amsmath}
\usepackage{textcomp}
% \usepackage{MnSymbol} % for \cupdot

% extra
\usepackage{xcolor}
\usepackage{tikz}

% ---------------------------------------------------------------------
% S E T T I N G
% ---------------------------------------------------------------------

%maybe delete later, for colorbox
\usepackage{tcolorbox}
\newtcolorbox{defbox}{colback=blue!5!white,colframe=blue!75!black}
\newtcolorbox{thmbox}{colback=orange!5!white,colframe=orange!75!black}
\newtcolorbox{rembox}{colback=purple!5!white,colframe=purple!75!black}

% typography and formatting
\geometry{margin=3cm}

\SetupExSheets{
  counter-format = ch.qu,
  counter-within = chapter,
  question/print = true,
  solution/print = true,
}

% mathematics
\newcounter{global}

\theoremstyle{definition}
\newtheorem{definition}{Definition}[]
\newtheorem{example}{Example}[definition]

\newtheorem{theorem}[definition]{Theorem}
\newtheorem{corollary}{Corollary}
\newtheorem{lemma}[definition]{Lemma}
\newtheorem{proposition}[definition]{Proposition}

\newtheorem*{remark}{Remark}

% extra
\definecolor{mathif}{HTML}{0000A0} % for conditions
\definecolor{maththen}{HTML}{CC5500} % for consequences
\definecolor{mathrem}{HTML}{8b008b} % for notes
\definecolor{mathobj}{HTML}{008800}

\usetikzlibrary{positioning}
\usetikzlibrary{shapes.geometric, arrows}

% ---------------------------------------------------------------------
% C O M M A N D S
% ---------------------------------------------------------------------

\newcommand{\norm}[1]{\left\lVert#1\right\rVert}
\newcommand{\rank}{\text{rank}}
\newcommand{\Vol}{\text{Vol}}

\newcommand{\set}[1]{\left\{\, #1 \,\right\}}
\newcommand{\makeset}[2]{\left\{\, #1 \mid #2 \,\right\}}

\newcommand*\diff{\mathop{}\!\mathrm{d}}
\newcommand*\Diff{\mathop{}\!\mathrm{D}}

\newcommand\restr[2]{{% we make the whole thing an ordinary symbol
  \left.\kern-\nulldelimiterspace % automatically resize the bar with \right
  #1 % the function
  \vphantom{\big|} % pretend it's a little taller at normal size
  \right|_{#2} % this is the delimiter
  }}

% ---------------------------------------------------------------------
% R E N D E R
% ---------------------------------------------------------------------

\newif\ifshowproof
\showprooftrue

\NewEnviron{Proof}{%
    \ifshowproof%
        \begin{proof}%
            \BODY
        \end{proof}%
    \fi%
}%

\begin{document}
\maketitle
\tableofcontents
%%%%%%%%%%%%%%%%%%%%%%%%%%%%%%%%%%%%%%%%%%%%%%%%%%%%%%%%%%%%%%%%%%%%%%%%%%%%%%%
We have the following tree of inclusion.
\begin{figure}[h]
    \center
    \begin{tikzpicture}[node distance=2cm and 1cm]
        \tikzstyle{box} = [draw=none, minimum width=4cm]
        \tikzstyle{arrow} = [thick,-,>=stealth]
        \node [draw=none]                 (ring of sets) {ring of sets};
        \coordinate[below=of ring of sets] (c);
        \node [box, left=of c]      (algebra of sets) {algebra of sets};
        \node [box] at (c)     (s-ring)     {\(\sigma\)-ring};
        \node [box, right=of c] (monotone class)     {monotone class};
        \node [box, below=of c]     (s-algebra)    {\(\sigma\)-algebra};
        \node [box, below=of s-algebra] (b-algebra) {Borel \(\sigma\)-algebra};
        \draw [arrow] (ring of sets) -- (algebra of sets);
        \draw [arrow] (ring of sets) -- (s-ring);
        \draw [arrow] (algebra of sets) -- (s-algebra);
        \draw [arrow] (s-ring) -- (s-algebra);
        \draw [arrow] (monotone class) -- (s-algebra);
        \draw [arrow] (s-algebra) -- (b-algebra);
    \end{tikzpicture}
\end{figure}
%
\begin{definition}[Collection]
    \textit{A collection is an assortment of elements.}

    Given a set \(X\), a collection \(A\) of elements in \(X\) is subset of \(X\).
\end{definition}
%
\begin{definition}[Family]
    \textit{A family is an indexed collection.}

    Given two sets \(X\) and \(I\), a family of elements in \(X\) indexed by \(I\) is a function \(f: I \rightarrow X\). In this document, we will denote such a family by \(\{A_i\}_{i \in I}\) where \(A_i := f(i)\) for every \(i \in I\).
\end{definition}
\begin{exmbox}
    \begin{example}
        Let \(d \in \mathbb{Z}\) square-free, and let \(K = \mathbb{Q}(\sqrt{d})\). The ring of integers of \(K\) is \(\mathcal{O}_K = \mathbb{Z}[\alpha]\) where
        \begin{align*}
            \alpha := \begin{cases}
                \sqrt{d} & \text{if } d \equiv 2, 3 \mod{4}\\
                \frac{1 + \sqrt{d}}{2} & \text{if } d \equiv 1 \mod{4} \text{.}
            \end{cases}
        \end{align*}
    \end{example}
\end{exmbox}
\begin{proof}
    By definition, the ring of integers is
    \begin{align*}
        \mathcal{O}_K = \makeset{a + b \sqrt{d} \in K}{a, b \in \mathbb{Q}, \, p(a + b \sqrt{d}) = 0} \text{,}
    \end{align*}
    where \(p \in \mathbb{Z}[X]\) is a monic polynomial with coefficients in \(\mathbb{Z}\). Consider \(a + b \sqrt{d}\in \mathbb{Q}(\sqrt{d})\) with \(a, b \in \mathbb{Q}\) and \(b \neq 0\). The minimal polynomial of \(a + b \sqrt{d}\) is of degree \(2\). It is
    \begin{align*}
        & (a + b \sqrt{d})^2 = 2 ab \sqrt{d} + a^2 + b^2 d \text{.}
    \end{align*}
    In the first step, we want to cancel \(\sqrt{d}\) from \(2 ab \sqrt{d}\) by adding a suitable multiple of \(a + b \sqrt{d}\), i.e. if \(n \in \mathbb{Z}\), then
    \begin{align*}
        2ab \sqrt{d} + n (a + b \sqrt{d}) \in \mathbb{Q} \text{.}
    \end{align*}
    Choose \(n := -2a\), then
    \begin{align*}
        2ab \sqrt{d} - 2a (a + b \sqrt{d}) = -2 a^2 \text{.}
    \end{align*}
    Going back, we have
    \begin{align*}
        (a + b \sqrt{d})^2 - 2a (a + b \sqrt{d}) = -a^2 + b^2 d \text{,}
    \end{align*}
    thus, it is
    \begin{align*}
        (a + b \sqrt{d})^2 - 2a (a + b \sqrt{d}) + a^2 - b^2 d = 0 \text{.}
    \end{align*}
    This gives us the minimal polynomial of \(a + b \sqrt{d}\) is \(m(X) = X^2 - 2a X + a^2 - b^2 d\). However, \(a, b\) where in \(\mathbb{Q}\), but \(-2a\) and \(a^2 - b^2d\) must lie in \(\mathbb{Z}\) in order for \(a + b \sqrt{d}\) to be in the ring of integers \(\mathcal{O}_K\).

    Firstly, if \(2a\) is odd, then we may write \(2a = 2k + 1\) for some \(k \in \mathbb{Z}\), and we have
    \begin{align*}
        a^2 - b^2 d = \frac{(2a)^2 - 4 b^2d}{4} = \frac{4k^2 + 4k + 1 - 4b^2 d}{4} \text{.}
    \end{align*}
    In the equation above, \(b\) cannot be an integer, so it must be \(2b = 2i + 1\), and it is
    \begin{align*}
        = \frac{4k^2 + 4k + 1 - (2b)^2 d}{4} = \frac{4k^2 + 4k + 1 - (2i + 1)^2 d}{4} = \frac{4k^2 + 4k + 1 - 4i^2d - 4id - d}{4}
    \end{align*}
    Not done, but something like this.
\end{proof}

\begin{exmbox}
    \begin{example}
        Let \(\)
    \end{example}
\end{exmbox}
\chapter{Ring of Integers}
% $$$$$$\  $$\                            $$\     
% $$  __$$\ $$ |                           $$ |    
% $$ /  \__|$$$$$$$\   $$$$$$\   $$$$$$\ $$$$$$\   
% $$ |      $$  __$$\ $$  __$$\ $$  __$$\\_$$  _|  
% $$ |      $$ |  $$ |$$$$$$$$ |$$$$$$$$ | $$ |    
% $$ |  $$\ $$ |  $$ |$$   ____|$$   ____| $$ |$$\ 
% \$$$$$$  |$$ |  $$ |\$$$$$$$\ \$$$$$$$\  \$$$$  |
%  \______/ \__|  \__| \_______| \_______|  \____/
\section{Definitions and Theorems}
%
\newpage
% $$$$$$$\                                 $$$$$$\  
% $$  __$$\                               $$  __$$\ 
% $$ |  $$ | $$$$$$\   $$$$$$\   $$$$$$\  $$ /  \__|
% $$$$$$$  |$$  __$$\ $$  __$$\ $$  __$$\ $$$$\     
% $$  ____/ $$ |  \__|$$ /  $$ |$$ /  $$ |$$  _|    
% $$ |      $$ |      $$ |  $$ |$$ |  $$ |$$ |      
% $$ |      $$ |      \$$$$$$  |\$$$$$$  |$$ |      
% \__|      \__|       \______/  \______/ \__|
\section{Proofs, Remarks, and Examples}

%
\newpage
% $$\   $$\            $$\                         
% $$$\  $$ |           $$ |                        
% $$$$\ $$ | $$$$$$\ $$$$$$\    $$$$$$\   $$$$$$$\ 
% $$ $$\$$ |$$  __$$\\_$$  _|  $$  __$$\ $$  _____|
% $$ \$$$$ |$$ /  $$ | $$ |    $$$$$$$$ |\$$$$$$\  
% $$ |\$$$ |$$ |  $$ | $$ |$$\ $$   ____| \____$$\ 
% $$ | \$$ |\$$$$$$  | \$$$$  |\$$$$$$$\ $$$$$$$  |
% \__|  \__| \______/   \____/  \_______|\_______/
\section{Exercises and Notes}
\end{document}