\documentclass[a4paper]{article}
\title{Integration and Integration}
\author{K}


% ---------------------------------------------------------------------
% P A C K A G E S
% ---------------------------------------------------------------------

% typography and formatting
\usepackage[english]{babel}
\usepackage[utf8]{inputenc}
\usepackage{geometry}
\usepackage{exsheets}
\usepackage{environ}

% mathematics
\usepackage{amsthm} % for theorems, and definitions
\usepackage{amssymb}
\usepackage{amsmath}
\usepackage{textcomp}
%\usepackage{MnSymbol} % for \cupdot

% extra
\usepackage{xcolor}
\usepackage{tikz}

% ---------------------------------------------------------------------
% S E T T I N G
% ---------------------------------------------------------------------

% typography and formatting
\geometry{margin=3cm}

\SetupExSheets{
  counter-format = ch.qu,
  counter-within = chapter,
  question/print = true,
  solution/print = true,
}

% mathematics
\theoremstyle{definition}
\newtheorem{definition}{Definition}
\newtheorem{example}{Example}[definition]

\newtheorem{theorem}{Theorem}[definition]
\newtheorem{corollary}{Corollary}
\newtheorem{lemma}{Lemma}[definition]
\newtheorem{proposition}{Proposition}[definition]

\newtheorem*{remark}{Remark}

% extra
\definecolor{mathif}{HTML}{0000A0} % for conditions
\definecolor{maththen}{HTML}{CC5500} % for consequences
\definecolor{mathrem}{HTML}{8b008b} % for notes

\usetikzlibrary{positioning}
\usetikzlibrary{shapes.geometric, arrows}

% ---------------------------------------------------------------------
% C O M M A N D S
% ---------------------------------------------------------------------

\newcommand{\norm}[1]{\left\lVert#1\right\rVert}
\newcommand{\rank}{\text{rank}}
\newcommand{\Vol}{\text{Vol}}
\newcommand*\diff{\mathop{}\!\mathrm{d}}
\newcommand*\Diff{\mathop{}\!\mathrm{D}}

\newcommand\restr[2]{{% we make the whole thing an ordinary symbol
  \left.\kern-\nulldelimiterspace % automatically resize the bar with \right
  #1 % the function
  \vphantom{\big|} % pretend it's a little taller at normal size
  \right|_{#2} % this is the delimiter
  }}

% ---------------------------------------------------------------------
% R E N D E R
% ---------------------------------------------------------------------

\newif\ifshowproof
\showprooftrue

\NewEnviron{Proof}{%
    \ifshowproof%
        \begin{proof}%
            \BODY
        \end{proof}%
    \fi%
}%

\begin{document}
\begin{center}
    \noindent\textbf{Exercise Sheet 7}
\end{center}
\noindent\textbf{Exercise 3}

\noindent\textbf{Solution 1.}

\noindent Let \(D \in \mathbb{Z}\) be square-free integer with \(D \equiv 1 \mod{4}\) and denote \(L := \mathbb{Q}(\sqrt{D})\). Then, according to example 3.2.5. (script) we have 
\begin{equation}
    \mathcal{O}_L = \mathbb{Z}\left[\frac{1 + \sqrt{D}}{2}\right] =: \mathbb{Z}[\alpha] \text{.}
\end{equation}
We want to apply the theorem from the lecture. First, we find the minimal polynomial of \(\alpha\). It is
\begin{equation}
    \left(\frac{1 + \sqrt{D}}{2}\right)^2 = \left(\frac{D - 1}{4}\right) + \left(\frac{1 + \sqrt{D}}{2}\right)\text{.}
\end{equation}
Thus the minimal polynomial is
\begin{equation}
    f_\alpha(X) = X^2 - X - \frac{D - 1}{4} \in \mathbb{Z}[X]
\end{equation}
as \(D \equiv 1 \mod{4}\). Now, we will aplly the theorem.

\bigskip

\noindent Let \(p \in \mathbb{Z}\) be an odd prime.

\noindent \underline{Case 1:} Let \(p \mid D\). Then,

\begin{align}
    X^2 - X - \frac{D - 1}{4} &\equiv X^2 + (p - 1)X + \frac{1}{4} \mod{p} \\
    &\equiv \left(X + \frac{p-1}{2}\right)^2 \mod{p}
\end{align}

\noindent So we have \(p\mathcal{O}_L = (p, \frac{p +\sqrt{D}}{2})^2\). Finally, we want to show \((p, \frac{p + \sqrt{D}}{2}) = (p, \sqrt{D})\). Clearly, it is \((p, \frac{p + \sqrt{D}}{2}) \subseteq (p, \sqrt{D})\). For the other side, we have
\begin{equation}
    p \cdot \alpha - \sqrt{D} \cdot \frac{p - 1}{2} = \frac{p + \sqrt{D}}{2} \text{.}
\end{equation}
We conclude \(p \mathcal{O}_L = (p, \sqrt{D})^2\).

\bigskip

\noindent \underline{Case 2:} Let \(p \nmid D\) but \(D \equiv m^2 \mod{p}\). We have \(D = m^2 + pn \equiv m^2 \mod{p}\) and hence \(\overline{f(X)} = X^2 - m^2 = (X + m)(X - m)\). With the theorem, we have \(p \mathcal{O}_L = (p, \sqrt{D} + m)(p, \sqrt{D} - m)\).

\bigskip

\noindent \underline{Case 3:} Otherwise, we have \(D \not\equiv m^2 \mod{p}\) for any \(m \in \mathbb{Z}\). We have \(\overline{f(X)} = X^2 - D\)

\bigskip

\noindent \textbf{Solution 2.}

\noindent Now let \(p = 2\). We apply the same theorem used above. If \(D \equiv 1 \mod{8}\), then we have for some \(n \in \mathbb{Z}\)
\begin{align}
    f_\alpha(X) &= X^2 - X - \frac{8n + 1 - 1}{4} \\
    &= X^2 - X - 2n \\
    &\equiv \overline{X(X +  1)} \mod{2}
\end{align}
hence \(2\mathcal{O}_L = (2, \alpha) (2, 1 + \alpha)\). On the other hand, if \(D \equiv 5 \mod{8}\), then we have for some \(n \in \mathbb{Z}\)
\begin{align}
    f_\alpha(X) &= X^2 - X - \frac{8n + 5 - 1}{4} \\
    &= X^2 - X - 2n - 1 \\
    &\equiv \overline{X^2 + X + 1} \mod{2} \text{.}
\end{align}
As \(f_\alpha\) here is irreducible, we have \(2 \mathcal{O}_L = (2, \alpha^2 + \alpha + 1)\).





\end{document}