\documentclass[a4paper]{article}
\title{Integration and Integration}
\author{K}


% ---------------------------------------------------------------------
% P A C K A G E S
% ---------------------------------------------------------------------

% typography and formatting
\usepackage[english]{babel}
\usepackage[utf8]{inputenc}
\usepackage{geometry}
\usepackage{exsheets}
\usepackage{environ}

% mathematics
\usepackage{amsthm} % for theorems, and definitions
\usepackage{amssymb}
\usepackage{amsmath}
\usepackage{textcomp}
%\usepackage{MnSymbol} % for \cupdot

% extra
\usepackage{xcolor}
\usepackage{tikz}

% ---------------------------------------------------------------------
% S E T T I N G
% ---------------------------------------------------------------------

% typography and formatting
\geometry{margin=3cm}

\SetupExSheets{
  counter-format = ch.qu,
  counter-within = chapter,
  question/print = true,
  solution/print = true,
}

% mathematics
\theoremstyle{definition}
\newtheorem{definition}{Definition}
\newtheorem{example}{Example}[definition]

\newtheorem{theorem}{Theorem}[definition]
\newtheorem{corollary}{Corollary}
\newtheorem{lemma}{Lemma}[definition]
\newtheorem{proposition}{Proposition}[definition]

\newtheorem*{remark}{Remark}

% extra
\definecolor{mathif}{HTML}{0000A0} % for conditions
\definecolor{maththen}{HTML}{CC5500} % for consequences
\definecolor{mathrem}{HTML}{8b008b} % for notes

\usetikzlibrary{positioning}
\usetikzlibrary{shapes.geometric, arrows}

% ---------------------------------------------------------------------
% C O M M A N D S
% ---------------------------------------------------------------------

\newcommand{\norm}[1]{\left\lVert#1\right\rVert}
\newcommand{\rank}{\text{rank}}
\newcommand{\Vol}{\text{Vol}}
\newcommand*\diff{\mathop{}\!\mathrm{d}}
\newcommand*\Diff{\mathop{}\!\mathrm{D}}

\newcommand\restr[2]{{% we make the whole thing an ordinary symbol
  \left.\kern-\nulldelimiterspace % automatically resize the bar with \right
  #1 % the function
  \vphantom{\big|} % pretend it's a little taller at normal size
  \right|_{#2} % this is the delimiter
  }}

% ---------------------------------------------------------------------
% R E N D E R
% ---------------------------------------------------------------------

\newif\ifshowproof
\showprooftrue

\NewEnviron{Proof}{%
    \ifshowproof%
        \begin{proof}%
            \BODY
        \end{proof}%
    \fi%
}%

\begin{document}
\begin{center}
    \noindent\textbf{Exercise Sheet 8}
\end{center}
\noindent\textbf{Exercise 1}

\noindent\textbf{Solution to 1.}

\noindent According to Theorem 4.1.11. (script), \(B\) is a Dedekind domain and with Remark 3.1.5.2 (script) it is also an \(A\)-module. As \(B\) is in particular noetherian, it is according to Corollary 4.1.5.1. finitely generated, i.e. there is a generating set

\begin{equation}
    \{b_1, \ldots, b_n\} \subseteq B \text{ for some \(n \in \mathbb{N}\).}
\end{equation}

\noindent For each \(i \in \mathbb{N}\), we have

\begin{equation}
    b_i = a_{i, 0} + a_{i, 1} \alpha + \cdots + a_{i, m} \alpha^m
\end{equation}

\noindent with \(a_{i, j} \in K\) and \(j, m \in \mathbb{N}\). As \(K\) is a quotient field, we can write \(a_{i,j} = p_{i,j}q_{i,j}^{-1}\) with \(p_{i,j}, q_{i,j} \in A\). From the equation above, we get

\begin{align}
    b_i \prod_{j = 0}^m q_{i, j} = p_{i, 0} + p_{i, 1} \alpha + \cdots + p_{i, m} \alpha^m \text{.}
\end{align}

\noindent Set \(\prod_{j=0}^m q_{i,j} = C_i \in A\). As \(p_{i, j} \in A\), we have that \(b_i C_i \in A[\alpha]\) for all \(1 \leq i \leq m\). Now set \(\prod_{i=1}^n C_i = C \in A\) and we have \(b_i C \in A[\alpha]\).

We found a factor \(C\) so that the generating set of \(B\) lies in \(A[\alpha]\). So for every \(b \in B\) we have that \(Cb \in A[\alpha]\).

We have \(C \in I_\alpha \neq \varnothing\).

\bigskip

\noindent\textbf{Solution to 2.}

\noindent We first show \(B / \mathfrak{p} B \cong \mathbb{F}_{\mathfrak{p}}[T] / (\overline{f(T)})\).

\bigskip

Define a homomorphism \(\varphi: A[\alpha] \rightarrow B / \mathfrak{p}B\), \(x \mapsto \varphi(x) \equiv y \mod{\mathfrak{p}}\). Its kernel is simply \(\mathfrak{p}A[\alpha]\). From the condition, we have \(\mathfrak{p} + I_\alpha = A\) and hence also \(\mathfrak{p}B + I_\alpha = B\) (as \(A \subseteq B\)). With \(I_\alpha \in A[\alpha]\) we have \(\mathfrak{p}B + A[\alpha] = B\). Therefore, \(\varphi\) is surjective and we have with the isomorphism theorem for rings \(B / \mathfrak{p}B \cong A[\alpha] / \mathfrak{p}A[\alpha]\).

\bigskip

Again define a natural homomorphism \(\psi: A[T] \rightarrow \mathbb{F}_\mathfrak{p} / (\overline{f(T)})\). This mapping is already surjective. For the kernel, we have the ideal generated by \((\mathfrak{p}, \overline{f(T)})\). Moreover, we have that \(A[\alpha] = A[T] / (f(T))\). With the isomorphism theorem for rings we have \(\mathbb{F}_{\mathfrak{p}}[T] / (\overline{f(T)}) \cong A[T] / (\mathfrak{p}, \overline{f(T)}) \cong A[\alpha] /\mathfrak{p}A[\alpha] \cong B / \mathfrak{p} B \).

To conclude that the condition \(\mathfrak{p} + I_\alpha = A\) can be substituted for \(B = A[\alpha]\), we follow the proof from the lecture. In the proof, \(\mathfrak{p} + I_\alpha = A\) is used to show

\begin{equation}
    B / \mathfrak{p} B \cong A[T] / (f(T), \mathfrak{p}) \cong \mathbb{F}_\mathfrak{p}[T] / (\overline{f(T)}) \text{.}
\end{equation}

which we have already shown. So the prime factorization of \(\mathfrak{p}B\) corresponds to the factorization as in the case \(A[\alpha] = B\).
\end{document}