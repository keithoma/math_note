\documentclass[a4paper]{article}
\title{Integration and Integration}
\author{K}


% ---------------------------------------------------------------------
% P A C K A G E S
% ---------------------------------------------------------------------

% typography and formatting
\usepackage[english]{babel}
\usepackage[utf8]{inputenc}
\usepackage{geometry}
\usepackage{exsheets}
\usepackage{environ}

% mathematics
\usepackage{amsthm} % for theorems, and definitions
\usepackage{amssymb}
\usepackage{amsmath}
\usepackage{textcomp}
%\usepackage{MnSymbol} % for \cupdot

% extra
\usepackage{xcolor}
\usepackage{tikz}

% ---------------------------------------------------------------------
% S E T T I N G
% ---------------------------------------------------------------------

% typography and formatting
\geometry{margin=3cm}

\SetupExSheets{
  counter-format = ch.qu,
  counter-within = chapter,
  question/print = true,
  solution/print = true,
}

% mathematics
\theoremstyle{definition}
\newtheorem{definition}{Definition}
\newtheorem{example}{Example}[definition]

\newtheorem{theorem}{Theorem}[definition]
\newtheorem{corollary}{Corollary}
\newtheorem{lemma}{Lemma}[definition]
\newtheorem{proposition}{Proposition}[definition]

\newtheorem*{remark}{Remark}

% extra
\definecolor{mathif}{HTML}{0000A0} % for conditions
\definecolor{maththen}{HTML}{CC5500} % for consequences
\definecolor{mathrem}{HTML}{8b008b} % for notes

\usetikzlibrary{positioning}
\usetikzlibrary{shapes.geometric, arrows}

% ---------------------------------------------------------------------
% C O M M A N D S
% ---------------------------------------------------------------------

\newcommand{\norm}[1]{\left\lVert#1\right\rVert}
\newcommand{\rank}{\text{rank}}
\newcommand{\Vol}{\text{Vol}}
\newcommand*\diff{\mathop{}\!\mathrm{d}}
\newcommand*\Diff{\mathop{}\!\mathrm{D}}

\newcommand\restr[2]{{% we make the whole thing an ordinary symbol
  \left.\kern-\nulldelimiterspace % automatically resize the bar with \right
  #1 % the function
  \vphantom{\big|} % pretend it's a little taller at normal size
  \right|_{#2} % this is the delimiter
  }}

% ---------------------------------------------------------------------
% R E N D E R
% ---------------------------------------------------------------------

\newif\ifshowproof
\showprooftrue

\NewEnviron{Proof}{%
    \ifshowproof%
        \begin{proof}%
            \BODY
        \end{proof}%
    \fi%
}%

\begin{document}
\begin{center}
    \noindent\textbf{Exercise Sheet 8}
\end{center}
\noindent\textbf{Exercise 1}

\noindent\textbf{Solution to 1.}

\noindent We want to show that \(I_\alpha\) is a nonzero ideal. First we reformulate the statement to be proven. Let \(n \in \mathbb{N}\) the smallest integer such that \(\alpha^{n + 1} \in A\). We have

\begin{align}
    & I_\alpha = \{a \in A \mid aB \subseteq A[\alpha]\} \neq \{0\} \\
    \iff & \exists \, a \in A \setminus \{0\} : aB \subseteq A[\alpha] \\
    \iff & \exists \, a \in A \setminus \{0\}, \, \exists \, b \in B, \, \exists \, \lambda_1, \ldots, \lambda_n \in A : a b = \sum_{k=0}^n \lambda_k \alpha^k \\
    \iff & \exists \, a \in A \setminus \{0\}, \, \exists \, b \in B, \, \exists \, \lambda_1, \ldots, \lambda_n \in A : b = \sum_{k=0}^n \frac{\lambda_k}{a} \alpha^k \text{.}
\end{align}

According to Theorem 4.1.11. (script), \(B\) is a Dedekind domain and with Remark 3.1.5.2 (script) it is also an \(A\)-module. As \(B\) is in particular noetherian, it is according to Corollary 4.1.5.1. finitely generated, i.e. there is a generating set

\begin{equation}
    \{b_1, \ldots, b_n\} \subseteq B \text{ for some \(n \in \mathbb{N}\).}
\end{equation}

For each \(i \in \mathbb{N}\), we have

\begin{equation}
    b_i = a_{i, 0} + a_{i, 1} \alpha + \cdots + a_{i, m} \alpha^m
\end{equation}

for \(a_{i, j} \in K\) and \(j, m \in \mathbb{N}\). As \(K\) is a quotient field, we can write \(a_{i,j} = \frac{p_{i,j}}{q_{i,j}}\) with \(p_{i,j}, q_{i,j} \in A\).

\end{document}