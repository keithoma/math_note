\documentclass[a4paper]{article}
\title{Number Theory}
\author{K}


% ---------------------------------------------------------------------
% P A C K A G E S
% ---------------------------------------------------------------------

% typography and formatting
\usepackage[english]{babel}
\usepackage[utf8]{inputenc}
\usepackage{geometry}
\usepackage{exsheets}
\usepackage{environ}

% mathematics
\usepackage{amsthm} % for theorems, and definitions
\usepackage{amssymb}
\usepackage{amsmath}
\usepackage{textcomp}
\usepackage{mathtools}
%\usepackage{MnSymbol} % for \cupdot

% extra
\usepackage{xcolor}
\usepackage{tikz}

% ---------------------------------------------------------------------
% S E T T I N G
% ---------------------------------------------------------------------

% typography and formatting
\geometry{margin=3cm}

\SetupExSheets{
  counter-format = ch.qu,
  counter-within = chapter,
  question/print = true,
  solution/print = true,
}

% mathematics

% extra
\definecolor{mathif}{HTML}{0000A0} % for conditions
\definecolor{maththen}{HTML}{CC5500} % for consequences
\definecolor{mathrem}{HTML}{8b008b} % for notes

\usetikzlibrary{positioning}
\usetikzlibrary{shapes.geometric, arrows}

% ---------------------------------------------------------------------
% C O M M A N D S
% ---------------------------------------------------------------------

\newcommand{\norm}[1]{\left\lVert#1\right\rVert}
\newcommand{\rank}{\text{rank}}
\newcommand{\Vol}{\text{Vol}}

\newcommand{\set}[1]{\left\{\, #1 \,\right\}}
\newcommand{\makeset}[2]{\left\{\, #1 \mid #2 \,\right\}}
\newcommand{\bigslant}[2]{{\raisebox{.2em}{$#1$}\left/\raisebox{-.2em}{$#2$}\right.}}

\newcommand*\diff{\mathop{}\!\mathrm{d}}
\newcommand*\Diff{\mathop{}\!\mathrm{D}}

\newcommand\restr[2]{{% we make the whole thing an ordinary symbol
  \left.\kern-\nulldelimiterspace % automatically resize the bar with \right
  #1 % the function
  \vphantom{\big|} % pretend it's a little taller at normal size
  \right|_{#2} % this is the delimiter
  }}

% ---------------------------------------------------------------------
% R E N D E R
% ---------------------------------------------------------------------

\newif\ifshowproof
\showprooftrue

\NewEnviron{Proof}{%
    \ifshowproof%
        \begin{proof}%
            \BODY
        \end{proof}%
    \fi%
}%

\begin{document}
\section*{Exercise 2.1}
Let \(d \in \mathbb{Z}\) be a square-free integer and consider \(K = \mathbb{Q}(\sqrt{d})\).
\begin{enumerate}
  \item Find an integral basis for \(K\).
  \begin{proof}
    From exercise 1.2.2. we have that \(\mathcal{O}_K = \mathbb{Z}[\alpha]\) where
    \begin{equation*}
      \alpha = \begin{cases}
        \frac{1 + \sqrt{d}}{2} &\qquad d \equiv 1 \mod{4} \\
        \sqrt{d} &\qquad d \not\equiv 1 \mod{4} \text{.}
      \end{cases}
    \end{equation*}
    so the integral basis is \(\mathcal{B} = \set{1, \alpha}\).
  \end{proof}
  \item Using the basis, compute the discriminant of \(K / \mathbb{Q}\).
  \begin{proof}
    We have
    \begin{align*}
      \Delta_K = \det \begin{pmatrix}
        \sigma_1(b_1) & \sigma_1(b_2) \\
        \sigma_2(b_1) & \sigma_2(b_2)
      \end{pmatrix}^2
    \end{align*}
    where \(\sigma_1\) and \(\sigma_2\) are the set of embeddings of \(K\) onto the complex numbers, and \(b_1\) and \(b_2\) are the integral basis of \(\mathcal{O}_K\). If \(d \equiv 1 \mod{4}\), we have
    \begin{align*}
      \Delta_K = \det \begin{pmatrix}
        \sigma_1(b_1) & \sigma_1(b_2) \\
        \sigma_2(b_1) & \sigma_2(b_2)
      \end{pmatrix}^2
      = \det \begin{pmatrix}
        1 & \frac{1 + \sqrt{d}}{2} \\
        1 & \frac{1 - \sqrt{d}}{2}
      \end{pmatrix}^2
      = (-\sqrt{d})^2 = d \text{.}
    \end{align*}
    If \(d \not\equiv 1 \mod{4}\), we have
    \begin{align*}
      \Delta_K = \det \begin{pmatrix}
        \sigma_1(b_1) & \sigma_1(b_2) \\
        \sigma_2(b_1) & \sigma_2(b_2)
      \end{pmatrix}^2
      = \det \begin{pmatrix}
        1 & \sqrt{d} \\
        1 & -\sqrt{d}
      \end{pmatrix}^2
      = (-2\sqrt{d})^2 = 4d \text{.}
    \end{align*}
    So the discriminant is
    \begin{equation}
      \Delta_K = \begin{cases}
        d & \qquad d \equiv 1 \mod{4}\\
        4d & \qquad d \not\equiv 1 \mod{4}
      \end{cases}
    \end{equation}
  \end{proof}
\end{enumerate}
\section*{Exercise 2.2}
Let \(K = \mathbb{Q}(\sqrt{-5})\), so \(\mathcal{O}_K = \mathbb{Z}[\sqrt{-5}]\). Consider the ideals \(\mathfrak{p}: = (2, 1 + \sqrt{-5})\) and \(\mathfrak{q} = (3, 1 + \sqrt{-5})\) in \(\mathcal{O}_K\) and let \(\overline{\mathfrak{p}}\) and \(\overline{\mathfrak{q}}\) denote the ideals obtained by elementwise complex conjugation.
\begin{enumerate}
  \item Show that \(\mathfrak{p}\) and \(\mathfrak{q}\) are prime but not principal. Prove \(\mathfrak{p} = \overline{\mathfrak{p}}\).
  \begin{proof}
    Consider \(\mathbb{Z}[\sqrt{-5}] / (2, 1 + \sqrt{-5})\). We want to show that this is an integral domain. We have
    \begin{align*}
      \mathbb{Z}[\sqrt{-5}] / (2, 1 + \sqrt{-5}) &\simeq (\mathbb{Z}[X] / (2, 1 + X)) / (X^2 + 5) \\
      & \simeq (\mathbb{Z} / (2)) [X] / (X + 1, X^2 + 5) \\
      & \simeq (\mathbb{Z} / (2)) [X] / (X + 1, X^2 + 1) \\
      & \simeq (\mathbb{Z} / (2)) [X] / (X + 1) \\
      & \simeq (\mathbb{Z} / (2)) [X] / (X) \\
      & \simeq (\mathbb{Z} / (2))
    \end{align*}
    And the last expression is an integral domain.
  \end{proof}
  \item Verify that \(\mathfrak{p}\mathfrak{q} = (1 + \sqrt{-5})\) and \(\mathfrak{p}\overline{\mathfrak{q}} = 1 - \sqrt{-5}\).
  \item Show that \(\mathfrak{p}^2 \mathfrak{q}\overline{\mathfrak{q}} = (6)\).
\end{enumerate}

\section*{Excersie 2.3}
Let \(K\) be a field suppose \(L = K(\alpha)\) is a separable extension such that the minimal polynomial of \(\alpha\) has the form \(f = T^3 + aT + b\) for some \(a, b \in K\). Compute \(D(1, \alpha, \alpha^2)\) in terms of \(a\) and \(b\).
\begin{proof}
  We have
  \begin{align*}
    D(1, \alpha, \alpha^2) = (1 - \alpha)^2 (1 - \alpha^2)^2 (\alpha - \alpha^2)^2
  \end{align*}
\end{proof}
\end{document}