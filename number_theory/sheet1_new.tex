\documentclass[a4paper]{article}
\title{Number Theory}
\author{K}


% ---------------------------------------------------------------------
% P A C K A G E S
% ---------------------------------------------------------------------

% typography and formatting
\usepackage[english]{babel}
\usepackage[utf8]{inputenc}
\usepackage{geometry}
\usepackage{exsheets}
\usepackage{environ}

% mathematics
\usepackage{amsthm} % for theorems, and definitions
\usepackage{amssymb}
\usepackage{amsmath}
\usepackage{textcomp}
\usepackage{mathtools}
%\usepackage{MnSymbol} % for \cupdot

% extra
\usepackage{xcolor}
\usepackage{tikz}

% ---------------------------------------------------------------------
% S E T T I N G
% ---------------------------------------------------------------------

% typography and formatting
\geometry{margin=3cm}

\SetupExSheets{
  counter-format = ch.qu,
  counter-within = chapter,
  question/print = true,
  solution/print = true,
}

% mathematics

% extra
\definecolor{mathif}{HTML}{0000A0} % for conditions
\definecolor{maththen}{HTML}{CC5500} % for consequences
\definecolor{mathrem}{HTML}{8b008b} % for notes

\usetikzlibrary{positioning}
\usetikzlibrary{shapes.geometric, arrows}

% ---------------------------------------------------------------------
% C O M M A N D S
% ---------------------------------------------------------------------

\newcommand{\norm}[1]{\left\lVert#1\right\rVert}
\newcommand{\rank}{\text{rank}}
\newcommand{\Vol}{\text{Vol}}

\newcommand{\set}[1]{\left\{\, #1 \,\right\}}
\newcommand{\makeset}[2]{\left\{\, #1 \mid #2 \,\right\}}
\newcommand{\bigslant}[2]{{\raisebox{.2em}{$#1$}\left/\raisebox{-.2em}{$#2$}\right.}}

\newcommand*\diff{\mathop{}\!\mathrm{d}}
\newcommand*\Diff{\mathop{}\!\mathrm{D}}

\newcommand\restr[2]{{% we make the whole thing an ordinary symbol
  \left.\kern-\nulldelimiterspace % automatically resize the bar with \right
  #1 % the function
  \vphantom{\big|} % pretend it's a little taller at normal size
  \right|_{#2} % this is the delimiter
  }}

% ---------------------------------------------------------------------
% R E N D E R
% ---------------------------------------------------------------------

\newif\ifshowproof
\showprooftrue

\NewEnviron{Proof}{%
    \ifshowproof%
        \begin{proof}%
            \BODY
        \end{proof}%
    \fi%
}%

\begin{document}
\section*{Exercise 1.1}
Let \(A \subseteq B\) be an integral extension of rings and assume that \(B\) is an integral domain. Suppose \(\mathfrak{q} \subset B\) is a prime ideal and let \(\mathfrak{p} := \mathfrak{q} \cap A \subseteq A\).
\begin{enumerate}
    \item Prove that \(A\) is a field if and only if \(B\) is a field.
    \begin{proof}
        \begin{enumerate}
            \item ``\(\Rightarrow\)'': Let \(A\) be a field, and \(b \in B\) an element. Since \(B\) is an integral extension of rings, \(b\) is integral over \(A\), thus we have for some \(a_0, \ldots, a_{n-1}\) that
            \begin{align*}
                & b^n + a_{n-1} b^{n-1} + \cdots + a_1 b + a_0 = 0 \\
                \iff& b^n + a_{n-1} b^{n-1} + \cdots + a_1 b = - a_0 \\
                \iff& b (b^{n-1} + a_{n-1} b^{n-2} + \cdots + a_1) = - a_0 \\
                \iff& b \left(\frac{-(b^{n-1} + a_{n-1} b^{n-2} + \cdots + a_1)}{a_0}\right)  = 1 \text{.}
            \end{align*}
            The last step was possible because if \(a_0 = 0\), then the above would not have been the minimal polynomial. The equation above shows \(b\) is invertible, hence \(B\) is a field.
            \item ``\(\Leftarrow\)'': Let \(B\) be a field, and \(y \in A\) an element. Since \(B\) is a field, \(y\) has an inverse \(x\) in \(B\) that is integral over \(A\), i.e.
            \begin{align}
                & x^n + a_{n-1} x^{n-1} + \cdots + a_1 x + a_0 = 0 \\
                \iff & x + a_{n-1} + \cdots + a_1 y^{n-2} + a_0 y^{n-1} = 0 \\
                \iff x = -(a_{n-1} + \cdots + a_1 y^{n-2} + a_0 y^{n-1})
            \end{align}
            so \(x \in A\) and \(A\) is a field.
        \end{enumerate}
    \end{proof}
    \item Show that \(\mathfrak{p}\) is a prime ideal of \(A\) and that \(A / \mathfrak{p}\) can be viewed as a subring of \(B / \mathfrak{q}\).
    \begin{proof}
        
    \end{proof}
\end{enumerate}
\end{document}