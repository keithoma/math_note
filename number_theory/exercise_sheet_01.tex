\documentclass[a4paper]{article}
\title{Integration and Integration}
\author{K}


% ---------------------------------------------------------------------
% P A C K A G E S
% ---------------------------------------------------------------------

% typography and formatting
\usepackage[english]{babel}
\usepackage[utf8]{inputenc}
\usepackage{geometry}
\usepackage{exsheets}
\usepackage{environ}

% mathematics
\usepackage{amsthm} % for theorems, and definitions
\usepackage{amssymb}
\usepackage{amsmath}
\usepackage{textcomp}
%\usepackage{MnSymbol} % for \cupdot

% extra
\usepackage{xcolor}
\usepackage{tikz}

% ---------------------------------------------------------------------
% S E T T I N G
% ---------------------------------------------------------------------

% typography and formatting
\geometry{margin=3cm}

\SetupExSheets{
  counter-format = ch.qu,
  counter-within = chapter,
  question/print = true,
  solution/print = true,
}

% mathematics
\theoremstyle{definition}
\newtheorem{definition}{Definition}
\newtheorem{example}{Example}[definition]

\newtheorem{theorem}{Theorem}[definition]
\newtheorem{corollary}{Corollary}
\newtheorem{lemma}{Lemma}[definition]
\newtheorem{proposition}{Proposition}[definition]

\newtheorem*{remark}{Remark}

% extra
\definecolor{mathif}{HTML}{0000A0} % for conditions
\definecolor{maththen}{HTML}{CC5500} % for consequences
\definecolor{mathrem}{HTML}{8b008b} % for notes

\usetikzlibrary{positioning}
\usetikzlibrary{shapes.geometric, arrows}

% ---------------------------------------------------------------------
% C O M M A N D S
% ---------------------------------------------------------------------

\newcommand{\norm}[1]{\left\lVert#1\right\rVert}
\newcommand{\rank}{\text{rank}}
\newcommand{\Vol}{\text{Vol}}
\newcommand*\diff{\mathop{}\!\mathrm{d}}
\newcommand*\Diff{\mathop{}\!\mathrm{D}}

\newcommand\restr[2]{{% we make the whole thing an ordinary symbol
  \left.\kern-\nulldelimiterspace % automatically resize the bar with \right
  #1 % the function
  \vphantom{\big|} % pretend it's a little taller at normal size
  \right|_{#2} % this is the delimiter
  }}

% ---------------------------------------------------------------------
% R E N D E R
% ---------------------------------------------------------------------

\newif\ifshowproof
\showprooftrue

\NewEnviron{Proof}{%
    \ifshowproof%
        \begin{proof}%
            \BODY
        \end{proof}%
    \fi%
}%

\begin{document}
\begin{center}
    \noindent\textbf{Exercise Sheet 1}
\end{center}
\noindent\textbf{Exercise 1}

Use the Norm to find factorizations of \(3 + 7i\) and \(23 + 14i\) into irreducible elements in \(\mathbb{Z}[i]\).

\noindent\textbf{Solution}

\noindent\underline{\(3 + 7i\)}

We want to find \(a, b \in \mathbb{Z}[i]\) such that \(ab = 3 + 7i\). We have \(N(3 + 7i) = 9 + 49 = 58\) and the prime factorization is \(58 = 2 \cdot 29\). Because of the multiplicative property of the Norm, we have \(N(3 + 7i) = N(a)N(b) = 2 \cdot 29\) or in other words, \(N(a) = 2\) and \(N(b) = 29\).

Set \(a = 1 + i\) which is irreducible as \(N(a) = 2\). Then, we have \(b = (3 + 7i) \cdot (1 + i)^{-1} = 5 + 2i\) which again is irreducible as \(N(b) \equiv 1 \mod{4}\).

Thus the solution is

\begin{align}
    3 + 7i = (1 + i) (5 + 2i) \text{.}
\end{align}

\noindent\underline{\(23 + 14i\)}

Similarly, we have \(N(23 + 14i) = 725 = 5^2 \cdot 29\). Try out some \(a \in \mathbb{Z}[i]\) with \(N(a) = 29\) until \((23 + 14i)a^{-1}\) is in \(\mathbb{Z}[i]\).

\begin{align}
    (23 + 14i)(5 + 2i)^{-1} &\notin \mathbb{Z}[i] \\
    (23 + 14i)(-5 + 2i)^{-1} = -3-4i &\in \mathbb{Z}[i]
\end{align}

It is \((23 + 14i)(-5+2i)^{-1} = -3-4i\). Now try out some \(b \in \mathbb{Z}[i]\) with \(N(b) = 5\) until \((-3-4i)b^{-1}\) is in \(\mathbb{Z}[i]\).

\begin{align}
    (-3 -4i)(1 + 2i)^{-1} &\notin \mathbb{Z}[i] \\
    (-3 -4i)(-1 + 2i)^{-1} = -1 + 2i&\in \mathbb{Z}[i] \\
\end{align}

As before, \((-5 + 2i)\) and \(-1 + 2i)\) are irreducible because the Norm is equivalent to \(1\) in mod \(4\). All together, we have

\begin{align}
    23 + 14i = (-1 + 2i)^2 (-5 + 2i) \text{.}
\end{align}

\end{document}