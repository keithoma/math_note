\documentclass[a4paper]{article}
\title{Integration and Integration}
\author{K}


% ---------------------------------------------------------------------
% P A C K A G E S
% ---------------------------------------------------------------------

% typography and formatting
\usepackage[english]{babel}
\usepackage[utf8]{inputenc}
\usepackage{geometry}
\usepackage{exsheets}
\usepackage{environ}

% mathematics
\usepackage{amsthm} % for theorems, and definitions
\usepackage{amssymb}
\usepackage{amsmath}
\usepackage{textcomp}
%\usepackage{MnSymbol} % for \cupdot

% extra
\usepackage{xcolor}
\usepackage{tikz}

% ---------------------------------------------------------------------
% S E T T I N G
% ---------------------------------------------------------------------

% typography and formatting
\geometry{margin=3cm}

\SetupExSheets{
  counter-format = ch.qu,
  counter-within = chapter,
  question/print = true,
  solution/print = true,
}

% mathematics
\theoremstyle{definition}
\newtheorem{definition}{Definition}
\newtheorem{example}{Example}[definition]

\newtheorem{theorem}{Theorem}[definition]
\newtheorem{corollary}{Corollary}
\newtheorem{lemma}{Lemma}[definition]
\newtheorem{proposition}{Proposition}[definition]

\newtheorem*{remark}{Remark}

% extra
\definecolor{mathif}{HTML}{0000A0} % for conditions
\definecolor{maththen}{HTML}{CC5500} % for consequences
\definecolor{mathrem}{HTML}{8b008b} % for notes

\usetikzlibrary{positioning}
\usetikzlibrary{shapes.geometric, arrows}

% ---------------------------------------------------------------------
% C O M M A N D S
% ---------------------------------------------------------------------

\newcommand{\norm}[1]{\left\lVert#1\right\rVert}
\newcommand{\rank}{\text{rank}}
\newcommand{\Vol}{\text{Vol}}
\newcommand*\diff{\mathop{}\!\mathrm{d}}
\newcommand*\Diff{\mathop{}\!\mathrm{D}}

\newcommand\restr[2]{{% we make the whole thing an ordinary symbol
  \left.\kern-\nulldelimiterspace % automatically resize the bar with \right
  #1 % the function
  \vphantom{\big|} % pretend it's a little taller at normal size
  \right|_{#2} % this is the delimiter
  }}

% ---------------------------------------------------------------------
% R E N D E R
% ---------------------------------------------------------------------

\newif\ifshowproof
\showprooftrue

\NewEnviron{Proof}{%
    \ifshowproof%
        \begin{proof}%
            \BODY
        \end{proof}%
    \fi%
}%

\begin{document}
\begin{center}
    \noindent\textbf{Exercise Sheet 1}
\end{center}
\noindent\textbf{Exercise 2}

Let \(k \in \mathbb{Z}_{>0}\).
\begin{enumerate}
    \item Show that \(k = a^2 + b^2\) for some \(a, b \in \mathbb{Z}\) if and only if for every prime \(p \equiv 3 \mod 4\), the exponent of \(p\) in the prime decomposition of \(k\) (in \(\mathbb{Z}\)) is even.
    \item In this case, describe how to obtain all solutions \((a, b) \in \mathbb{Z}^2\).
\end{enumerate}

\noindent\textbf{Solution}

\noindent\underline{1.}

\noindent Step 1: Let \(z \in \mathbb{Z}_{>0}\) such that \(z \equiv 3 \mod{4}\). Then, \(z = 3 + 4n\) for some \(n \in \mathbb{Z}\). Consider \(\alpha, \beta \in \mathbb{Z}_{>0}\) with \(\alpha \equiv 1 \mod{4}\) and \(\beta \equiv 3 \mod{4}\). For some \(m_\alpha, m_\beta \in \mathbb{N}\), we have

\begin{align}
    z \alpha = (3 + 4n)(1 + 4m_\alpha) = 3 + 4n + 12m_\alpha + 16nm_\alpha &\equiv 3 \mod{4} \\
    z \beta = (3 + 4n)(3 + 4m_\alpha) = 9 + 12n + 12m_\alpha + 16nm_\alpha &\equiv 1 \mod{4} \\
    z 2 = (3 + 4n) 2 = 6 + 8n &\equiv 2 \mod{4} \text{.}
\end{align}

In short, \(z\) must be multiplied with an integer equivalent to \(3\) mod \(4\) if one wants to obtain an integer equivalent to \(1\) mod \(4\).

\noindent Step 2: Similarly as above, let \(z \in \mathbb{Z}_{>0}\) such that \(z \equiv 1 \mod{4}\). Then, \(z = 1 + 4n\) for some \(n \in \mathbb{Z}\). Consider \(\alpha, \beta \in \mathbb{Z}_{>0}\) with \(\alpha \equiv 1 \mod{4}\) and \(\beta \equiv 3 \mod{4}\). For some \(m_\alpha, m_\beta \in \mathbb{N}\), we have

\begin{align}
    z \alpha = (1 + 4n)(1 + 4m_\alpha) = 1 + 4n + 4m_\alpha + 16nm_\alpha &\equiv 1 \mod{4} \\
    z \beta = (1 + 4n)(3 + 4m_\alpha) = 3 + 12n + 4m_\alpha + 16nm_\alpha &\equiv 3 \mod{4} \\
    z 2 = (1 + 4n) 2 = 2 + 8n &\equiv 2 \mod{4} \text{.}
\end{align}

In short, any product of integers equivalent to \(1\) mod \(4\) requires even number of integers equivalent to \(3\) mod \(4\).

\noindent Step 3: Let \(k = a^2 + b^2\). According to Theorem 1.0.1. this is equivalent to \(k = 2\) or \(k \equiv 1 \mod{4}\). If \(k = 2\), then it is clear immediately. So consider the case \(k \neq 2\). From step 1, we know that the prime factorization of \(k\) must contain even number of primes that are equivalent to \(3\) mod \(4\). Therefore, each exponent of such prime must also be even.

\noindent Step 2 shows that 
\end{document}