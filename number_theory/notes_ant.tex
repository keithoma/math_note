\documentclass[a4paper]{book}
\title{Number Theory}
\author{K}


% ---------------------------------------------------------------------
% P A C K A G E S
% ---------------------------------------------------------------------

% typography and formatting
\usepackage[english]{babel}
\usepackage[utf8]{inputenc}
\usepackage{geometry}
\usepackage{exsheets}
\usepackage{environ}

% mathematics
\usepackage{amsthm} % for theorems, and definitions
\usepackage{amssymb}
\usepackage{amsmath}
\usepackage{textcomp}
% \usepackage{MnSymbol} % for \cupdot

% extra
\usepackage{xcolor}
\usepackage{tikz}
\usepackage{multirow}

% ---------------------------------------------------------------------
% S E T T I N G
% ---------------------------------------------------------------------

%maybe delete later, for colorbox
\usepackage{tcolorbox}
\newtcolorbox{defbox}{colback=blue!5!white,colframe=blue!75!black}
\newtcolorbox{thmbox}{colback=orange!5!white,colframe=orange!75!black}
\newtcolorbox{rembox}{colback=purple!5!white,colframe=purple!75!black}

% typography and formatting
\geometry{margin=3cm}

\SetupExSheets{
  counter-format = ch.qu,
  counter-within = chapter,
  question/print = true,
  solution/print = true,
}

% mathematics
\newcounter{global}

\theoremstyle{definition}
\newtheorem{definition}{Definition}[]
\newtheorem{example}{Example}[definition]

\newtheorem{theorem}[definition]{Theorem}
\newtheorem{corollary}{Corollary}
\newtheorem{lemma}[definition]{Lemma}
\newtheorem{proposition}[definition]{Proposition}

\newtheorem*{remark}{Remark}

% extra
\definecolor{mathif}{HTML}{0000A0} % for conditions
\definecolor{maththen}{HTML}{CC5500} % for consequences
\definecolor{mathrem}{HTML}{8b008b} % for notes
\definecolor{mathobj}{HTML}{008800}

\usetikzlibrary{positioning}
\usetikzlibrary{shapes.geometric, arrows}

% ---------------------------------------------------------------------
% C O M M A N D S
% ---------------------------------------------------------------------

\newcommand{\norm}[1]{\left\lVert#1\right\rVert}
\newcommand{\rank}{\text{rank}}
\newcommand{\Vol}{\text{Vol}}

\newcommand{\set}[1]{\left\{\, #1 \,\right\}}
\newcommand{\makeset}[2]{\left\{\, #1 \mid #2 \,\right\}}

\newcommand*\diff{\mathop{}\!\mathrm{d}}
\newcommand*\Diff{\mathop{}\!\mathrm{D}}

\newcommand\restr[2]{{% we make the whole thing an ordinary symbol
  \left.\kern-\nulldelimiterspace % automatically resize the bar with \right
  #1 % the function
  \vphantom{\big|} % pretend it's a little taller at normal size
  \right|_{#2} % this is the delimiter
  }}

% ---------------------------------------------------------------------
% R E N D E R
% ---------------------------------------------------------------------

\newif\ifshowproof
\showprooftrue

\NewEnviron{Proof}{%
    \ifshowproof%
        \begin{proof}%
            \BODY
        \end{proof}%
    \fi%
}%

\begin{document}
\tableofcontents

\part{Linear Algebra}

\part{Field Theory}

\chapter{Algebraic Field Extensions}
\begin{defbox}
    \begin{definition}[Splitting Field]
        A splitting field of a polynomial \(f\) over a field \(K\) is a field extension \(L\) of \(K\) over which \(f\) factors into linear factors that is
        \begin{align*}
            f(X) = c \prod_{i=1}^{\deg f} (X - a_i)
        \end{align*}
        where \(c \in K\) and for each \(1 \leq i \leq \deg f\) we have \(X - a_i \in L[X]\) with \(a_i\) not necessarily distinct and such that the roots \(a_i\) generate \(L\) over \(K\).
    \end{definition}
\end{defbox}
\begin{remark}
    The extension \(L\) is an extension of minimal degree over \(K\) in which \(f\) splits. Such extension always exist and is unique up to isomorphism. The amount of freedom in that isomorphism is known as the Galois group of \(f\) (if we assume it is separable).
\end{remark}

\begin{defbox}
    \begin{definition}[Normal Extension]
        A algebraic extension \(L\) over a field \(K\) is normal if one of the following equivalent conditions are met.
        \begin{enumerate}
            \item I don't quite see this.
            \item \(L\) is a splitting field of a family of polynomials of \(K[X]\).
            \item Every irreducible polynomials of \(K[X]\) that has a root in \(L\) factors into linear factors over \(L\).
        \end{enumerate}
    \end{definition}
\end{defbox}

\chapter{Galois Theory}

\begin{example}
    Let \(\)
\end{example}
\begin{proof}
    The field extension \(\mathbb{Q}(\sqrt{2}, \sqrt{3}) / \mathbb{Q}\) is Galois of degree \(4\). This means that the Galois group is of order \(4\). The minimal polynomial of \(\sqrt{2}\) is \(X^2 - 2\), so the conjugate of \(\sqrt{2}\) is \(-\sqrt{2}\). Similary, the conjugate of \(\sqrt{3}\) is \(-\sqrt{3}\).
\end{proof}

\part{Algebraic Number Theory}

\begin{thmbox}
    \begin{theorem}
        Let \(A\) be an integral domain, and let \(L\) be a field containing \(A\). The elements of \(L\) integral over \(A\) form a ring.
    \end{theorem}
\end{thmbox}
\begin{remark}
    The immediate consequence of this theorem is that the ring of integers is indeed a ring.
\end{remark}
\begin{defbox}
    \begin{definition}
        Symmetric polynomials and elementary symmetric polynomials.
    \end{definition}
\end{defbox}
\begin{thmbox}
    \begin{theorem}
        Let \(A\) be a ring. Every symmetric polynomial \(P(X_1, \ldots, X_r)\) in \(A[X_1, \ldots, X_n]\) can be represented with a linear combination of elementary symmetric polynomials with coeffcients in \(A\).
    \end{theorem}
\end{thmbox}
Proof is constructive and inductive by reducing the polynomial over the lexicographically highest monomial. Not a hard proof, but the indecies are anoying.

The above proof implies:

Let \(f(X) = X^n + a_1 X^{n-1} + \cdots + a_n \in A[X]\), and let \(\alpha_1, \ldots, \alpha_n\) be the roots of \(f(X)\) in some ring containing \(A\), so that \(f(X) = \prod (X - \alpha_i)\) in the larger ring. Then
\begin{align*}
    a_1 = -S_1(\alpha_1, \ldots, \alpha_n), \qquad a_2 = S_2(\alpha_1, \ldots, \alpha_n), \qquad a_n = \pm S_n(\alpha_1, \ldots, \alpha_n) \text{.}
\end{align*}
(I'm not quite sure why this is the case. Maybe use the multi-binomial theorem.)

Thus the elementary symmetric polynomials in the roots of \(f\) lie in \(A\). And so the theorem implies that every symmetric polynomial in the roots of \(f(X)\) lies in \(A\).
\begin{thmbox}
    \begin{proposition}
        Let \(A\) be a integral domain and \(\Omega\) be an algebraically closed field containing \(A\). If \(\alpha_1, \ldots, \alpha_n\) are the roots in \(\Omega\) of a monic polynomial in \(A[X]\), then every polynomial \(g(\alpha_1, \ldots, \alpha_n)\) in \(A[\alpha_1, \ldots, \alpha_n]\) is a root of a monic polynomial in \(A[X]\).
    \end{proposition}
\end{thmbox}
\begin{proof}
    Clearly,
    \begin{align*}
        h(X) := \prod_{\sigma \in \mathrm{Sym}_n} (X -  g(\alpha_{\sigma(1)}, \ldots, \alpha_{\sigma(n  )}))
    \end{align*}
    is a monic polynomial whose coeffcients are symmetric polynomials in the \(\alpha_i\), and therefore lie in \(A\). But \(g(\alpha_1, \ldots, \alpha_n)\) is one of the roots.
\end{proof}
With this we can prove that the above theorem. I don't quite understand few steps ...

\section*{Dedekind's Proof}

\begin{thmbox}
    \begin{proposition}
        Let \(L\) be a field containing \(A\). An element \(\alpha\) of \(L\) is integral over \(A\) if and only if there exists a nonzero finitely generated \(A\)-submodule of \(L\) such that \(\alpha M \subset M\) (in fact, we can take \(M = A[\alpha]\), the \(A\)-subalgebra generated by \(\alpha\)).
    \end{proposition}
\end{thmbox}
\begin{proof}
    \begin{itemize}
        \item Let \(\alpha \in L\) be integral over \(A\). The \(A\)-submodule \(A[\alpha]\) in \(L\) is generated by \(1, \alpha, \ldots, \alpha^{n-1}\), thus finitely generated and clearly nonzero. \(\alpha A[\alpha] \subset A[\alpha]\) also holds.
        \item Let \(M\) be a nonzero, finitely generated \(A\)-submodule in \(L\) such that \(\alpha M \subset M\). Since \(M\) is finitely generated, there is a set of generators \(v_1, \ldots, v_n \in M\). From \(\alpha M \subset M\) we have that
        \begin{align*}
            \alpha v_i = \sum_{j = 1}^n a_{i, j} v_j
        \end{align*}
        for some \(a_{i, j} \in A\). We rewrite this system of equations
        \begin{align*}
            (\alpha - a_{i, i}) v_i \sum_{j = 1, j \neq i}^n a_{i, j} v_j = 0
        \end{align*}
        We have the matrix
        \begin{align*}
            \begin{pmatrix}
                (\alpha - a_{1, 1}) & a_{1, 2} & \cdots & a_{1, n}\\
                a_{2,1} & (\alpha - a_{2, 2}) & \cdots & a_{2, n} \\
                \vdots & & & \vdots \\
                a_{n, 1} & a_{n, 2} & \cdots & (\alpha - a_{n,n})
            \end{pmatrix}
        \end{align*}
        Applying Cramer's Rule we get \(v_i = \frac{\det(C_i)}{\det{C}}\), but \(C_i\) is always \(0\), and at least one \(v_i\) is nonzero, so we have that \(\det(C) = 0\).

        But calculating the determinant of \(C\) gives us

        \begin{align*}
            \alpha^n + c_1 \alpha^{n-1} + \cdots + c_n = 0
        \end{align*}
        as desired.
    \end{itemize}
\end{proof}
Now take \(\alpha\) and \(\beta\) integral over \(A\) and denote \(\alpha M \subset M\) and \(\beta N \subset N\).

\begin{enumerate}
    \item \(MN\) is an \(A\)-submodule of \(L\).
\end{enumerate}

Dedekind's proof is much easier to understand, lol.


\section*{Integral Elements}

\begin{proposition}
    Let \(K\) be the field of fractions of \(A\), and let \(L\) be a field containing \(K\). If \(\alpha \in L\) is algebraic over \(K\), then there exists a nonzero \(d \in A\) such that \(d \alpha \) is integral over \(A\).
\end{proposition}

\begin{corollary}
    Let \(A\) be an integral domain with field of fractions \(K\), and let \(B\) be the integral closure of \(A\) in a field \(L\) containing \(K\). If \(L\) is algebraic over \(K\), then it is the field of fractions \(B\).
\end{corollary}

\begin{example}
    Let \(d\) be a square-free integer. Consider \(A = \mathbb{Z}[\sqrt{d}]\). Show that every element of \(R\) can be written as a product of irreducible elements.
\end{example}
\begin{proof}
    Define \(N: R \longrightarrow \mathbb{N}\) as \(N(a + b \sqrt{d}) = |a^2 - d b^2|\) where \(a, b \in \mathbb{Z}\). Let \(a_1 + b_1 \sqrt{d}\) and \(a_2 + b_2 \sqrt{d}\) be two elements in \(\mathbb{Z}[\sqrt{d}]\) with \(a_1, b_1, a_2, b_2 \in \mathbb{Z}\), then
    \begin{align*}
        N((a_1 + b_1 \sqrt{d}) (a_2 + b_2 \sqrt{d})) &= N((a_1 a_2 + b_1 b_2 d) + (a_1 b_2 + a_2 b_1) \sqrt{d}) \\
        &= |(a_1 a_2 + b_1 b_2 d)^2 - d (a_1 b_2 + a_2 b_1)^2| \\
        &= |a_1^2 a_2^2 + 2 a_1 a_2 b_1 b_2 d + b_1^2 b_2^2 d^2 - a_1^2 b_2^2 d - 2 a_1 a_2 b_1 b_2 d - a_2^2 b_1^2 d | \\
        &= |a_1^2 a_2^2 - a_1^2 b_2^2 d - a_2^2 b_1^2 d + b_1^2 b_2^2 d^2| \\
        \intertext{on the other hand}
        N(a_1 + b_1 \sqrt{d})N(a_2 + b_2 \sqrt{d}) &= |a_1^2 - d b_1^2||a_2^2 - d b_2^2| \\
        &= |a_1^2 a_2^2 - a_1^2 b_2^2 d - a_2^2 b_1^2 d + b_1^2 b^2 d^2 |
    \end{align*}
    so we have \(N((a_1 + b_1 \sqrt{d}) (a_2 + b_2 \sqrt{d})) = N(a_1 + b_1 \sqrt{d})N(a_2 + b_2 \sqrt{d})\). Moreover, let \(u \in \mathbb{Z}[\sqrt{d}]\) be a unit, then there is an element \(v \in \mathbb{Z}[\sqrt{d}]\) such that \(u v = 1\). Applying the function defined above, we get
    \begin{align*}
        1 = N(1) = N(u v) = N(u) N(v)
    \end{align*}
    so \(N(u) = 1\). Now suppose \(N(a + b \sqrt{d}) = 1\) with \(a, b \in \mathbb{Z}\). Consider
    \begin{align*}
        (a + b \sqrt{d})(a - b \sqrt{d}) = a^2 - d b^2 = \pm 1
    \end{align*}
    and therefore \(a + b \sqrt{d}\) is a unit.

    We have shown that \(N\) is a norm map. \(R\) is also an integral domain because if \(x \in R\) is a zero-divisor, then we have \(0 = N(x) = |a^2 - d b^2|\), but this is impossible since \(d\) is square-free. Applying the example before, we get the desired result.
\end{proof}

\begin{example}
    2.1.3. did it before
\end{example}

\begin{example}
    Let \(R\) be a domain in which every element can be written as a product of irreducibles. Show that the following are equivalent.
    \begin{enumerate}
        \item this factorization is unique
        \item if \(\pi\) is irreducible and \(\pi\) divides \(ab\), then \(\pi | a\) or \(\pi | b\)
    \end{enumerate}
\end{example}
\begin{proof}
    Let the factorization be unique, \(\pi \in R\) be irreducible and divide \(ab\). Then \(ab = \pi x\) for some \(x \in R\). On the other hand, \(ab\) has a unique factorization that is the product of the factorization of \(a\) and \(b\) but must contain \(\pi\).

    For the other side let \(p_1^{r_1} \cdot \ldots \cdot p_n^{r_n}\) and \(q_1^{s_1} \cdot \ldots \cdot q_m^{r_m}\) be two factorizations of an element in \(R\). Then \(p_1\) divides \(q_1^{s_1} \cdot \ldots \cdot q_m^{r_m}\) so \(p_1\) divides some \(q_i\). But \(q_i\) is irreducible, so we have \(p_1 = q_i\). Induction yields the desired result.
\end{proof}

\begin{example}
    Show that if \(\pi\) is an irreducible element of a principal ideal domain, then \((\pi)\) is a maximal ideal.
\end{example}
\begin{proof}
    Assume \((\pi)\) is not maximal, then there is an ideal \((a)\) with \(a \neq 1\) such that \((\pi) \subsetneq (a)\). But this implies \(\pi = r a\) for some \(r \in R\) that is not a unit. This is a contradiction.
\end{proof}

\begin{example}
    If \(F\) is a field, prove that \(F[x]\) is Euclidean.
\end{example}
\begin{proof}
    Define \(\phi: F[x] \longrightarrow \mathbb{N}\) as \(\phi(f) = \mathrm{deg}(f)\). Fix two polynomials \(f, g \in F[x]\). If \(\mathrm{deg}(f) >= \mathrm{g}\), then we can do polynomial division to get \(f = g p + r\) where \(\mathrm{deg}(g) > \mathrm{r}\).
\end{proof}

\begin{example}
    Show that \(\mathbb{Z}[i]\) is Euclidean.
\end{example}
\begin{proof}
    Fix two elements \(x, y \in \mathbb{Z}[i]\) and write \(x = a_x + i b_x\) and \(y = a_y + i b_y\). It is
    \begin{align*}
        \frac{x}{y} &= \underbrace{\frac{a_x a_y + b_x b_y}{a_y^2 + b_y^2}}_{=: \alpha} + i \underbrace{\frac{a_y b_x - a_x b_y}{a_y^2 + b_y^2}}_{=: \beta}
    \end{align*}
    Set \(p_x\) to be the closest integer to \(\alpha\) and \(p_y\) to be the closest integer to \(\beta\) and \(p = p_x + i p_y\). Moreover, set \(r = ((\alpha - p_x) + i (\beta - p_y)) y\).

    It is
    \begin{align*}
        r &= y(\alpha + i \beta) - y (p_x + i p_y) \\
        &= y \frac{x}{y} - py \\
        &= x - py
    \end{align*}
    so we got the desired representation.

    Furthermore, we have
    \begin{align*}
        N(r) &= N(y) ((\alpha - p_x)^2 + (\beta - p_y)^2)\\
        &\leq N(y) \frac{1}{2}
    \end{align*}
\end{proof}
\begin{example}
    Prove that if \(p\) is a positive prime, then there is an element \(x \in \mathbb{Z}/p\mathbb{Z}\) such that \(x^2 \equiv -1 \mod{p}\) if and only if either \(p = 2\) or \(p \equiv 1 \mod{4}\).
\end{example}

\begin{proof}
    \begin{enumerate}
        \item Let \(p = 2\), then we can simply choose \(x = 1\). Now let \(p \equiv 1 \mod{4}\). With Wilson's Theorem we have
        \begin{align*}
            -1 \equiv (p - 1)! \equiv 1 \cdot \ldots \frac{p - 1}{2} \cdot \frac{p + 1}{2} \cdot \ldots \cdot p \equiv \left(\left(\frac{p - 1}{2}\right)!\right)^2 \cdot (-1)^{\frac{p - 1}{2}} \equiv \left(\left(\frac{p - 1}{2}\right)!\right)^2
        \end{align*}
        where \(\mod{p}\). So choose the last expression as \(x\) and we are done.
        \item If \(p = 2\), then we are done. Now let \(x^2 \equiv -1 \mod{p}\). If \(p \equiv 3 \mod(4)\), we have
        \begin{align*}
            x^{p - 1} = x^{4n + 2} = x^{4n} x^2 \equiv -1 (x^4)^n \equiv -1 \mod{p}
        \end{align*}
        as \(x^4 \equiv 1 \mod{p}\). But this contradicts Fermat's Little Theorem.
    \end{enumerate}
\end{proof}
\begin{example}
    Find all integer solutions to \(y^2 + 1 = x^3\) with \(x, y \neq 0\).
\end{example}
\begin{proof}
    If \(x\) is even, then \(4 | x^3\), so \(x^3 - 1 \equiv 3 \mod 4\) which cannot be a square since all squares are congruent to either \(0\) or \(1\) \(\mod{4}\). So \(x\) is odd and \(y\) is even. Write \(y^2 + 1 = (y + i)(y - i)\). If a prime divides \((y + i)(y - i)\), then the prime divides also their difference \(2i\). So \(p = 2\) up to units. But then \(p\) divides \(y\) as \(y\) was even, but this is impossible since \(p\) also divides \(y + i\).
\end{proof}

\begin{example}
    What are the primes of \(\mathbb{Z}[i]\)?
\end{example}
\begin{proof}
    We have two types of primes in \(\mathbb{Z}[i]\).
    \begin{enumerate}
        \item \(p\) and \(ip\) where \(p \equiv 3 \mod{4}\).
        \item \(a + i b\) with \(a^2 + b^2 \equiv 1 \mod{4}\) and prime.
    \end{enumerate}
    This is because of the norm function \(N(a + ib) = a^2 + b^2\).
\end{proof}

\begin{example}
    A positive integer \(a\) is the sum of two squares if and only if \(a = b^2 c\) where \(c\) is not divisible by any positive prime \(p \equiv 3 \mod{4}\).
\end{example}
\begin{proof}
    I don't know.
\end{proof}

\begin{example}
    \(\mathbb{Z}[\rho]\) is a ring where
    \begin{align*}
        \rho = \frac{-1 + \sqrt{-3}}{2} \text{.}
    \end{align*}
\end{example}
\begin{proof}
    \begin{enumerate}
        \item \((\mathbb{Z}[\rho], +)\) is an abelian group.
        \begin{enumerate}
            \item If \(a_1 + b_1 \rho\) and \(a_2 + b_2 \rho\) are elements of \(\mathbb{Z}[\rho]\), then \(a_1 + b_1 \rho + a_2 + b_2 \rho = a_1 + a_2 + (b_1 + b_2) \rho\), so the addition is well-defined.
            \item Associativity and commutativity is inhereted from the addition of integers.
            \item The additive identity is \(0\).
            \item If \(a + b \rho\) is in \(\mathbb{Z}[\rho]\), then its inverse is \(-a - b \rho\).
        \end{enumerate}
        \item \((\mathbb{Z}[\rho], \cdot)\) is a monoid.
        \begin{enumerate}
            \item If \(a_1 + b_1 \rho\) and \(a_2 + b_2 \rho\) are two elements of \(\mathbb{Z}[\rho]\), then we have
            \begin{align*}
                (a_1 + b_1 \rho)(a_2 + b_2 \rho) &= a_1 a_2 + b_1 b_2 \rho^2 + (a_1 b_2 + a_2 b_1) \rho \\
                &= a_1 a_2 + b_1 b_2 \overline{\rho} + (a_1 b_2 + a_2 b_1) \rho \\
                &= a_1 a_2 + b_1 b_2 \frac{-1 - \sqrt{3}}{2} + (a_1 b_2 + a_2 b_1) \frac{-1 + \sqrt{3}}{2} \\
                &= a_1 a_2 - \frac{b_1 b_2}{2} - \frac{a_1 b_2 + a_2 b_1}{2} - \frac{b_1 b_2 \sqrt{-3}}{2} + \frac{(a_1 b_2 + a_2 b_1) \sqrt{-3}}{2} \\
                &= a_1 a_2 + \frac{-a_1 b_2 - a_2 b_2 -b_1 b_2}{2} + \frac{(a_1 b_2 + a_2 b_1 - b_1 b_2) \sqrt{-3}}{2}
            \end{align*}
            I made some mistake, but should be right.
            \item The multiplicative identity is \(1\)
        \end{enumerate}
        \item Distributive law is again inherited.
    \end{enumerate}
\end{proof}
\begin{example}
    \begin{enumerate}
        \item Show that \(\mathbb{Z}[\rho]\) is Euclidean.
        \begin{proof}
            Fix two elements \(x_1 + x_2 \rho\) and \(y_1 + y_2 \rho\) of \(\mathbb{Z}[\rho]\). We have
            \begin{align*}
                \frac{x_1 + x_2 \rho}{y_1 + y_2 \rho} &= \frac{x_1 + x_2 \rho}{y_1 + y_2 \rho} \frac{y_1 - y_2 \rho}{y_1 - y_2 \rho} \\
                &= \frac{x_1 y_1 - x_2 y_2 \overline{\rho}  - x_1 y_2 \rho + x_2 y_1 \rho}{y_1^2 + y_2^2 \overline{\rho}}
            \end{align*}
            I think this should work at the end of the day, but I'm too lazy to write it out.
        \end{proof}
        \item Show that the only units in \(\mathbb{Z}[\rho]\) are \(\pm 1\), \(\pm \rho\), and \(\pm \overline{\rho}\).
    \end{enumerate}
\end{example}

\chapter{Cheet Sheet}

\noindent \(K = \mathbb{Q}(\sqrt{d})\) where \(d\) is a square-free integer.
\begin{enumerate}
    \item \(\mathcal{O}_K = \mathbb{Z}[\alpha]\) where
    \begin{align*}
        \alpha := \begin{cases}
            \frac{1 + \sqrt{d}}{2} & d \equiv 1 \mod{4} \\
            \sqrt{d} & d \equiv 2, 3 \mod{4}
        \end{cases}
    \end{align*}
\end{enumerate}

\chapter{Algebraic Numbers and Integers}

\begin{example}
    Show that
    \begin{equation*}
        \alpha := \frac{\sqrt{2}}{3}
    \end{equation*}
    is an algebraic number, but not an algebraic integer.
\end{example}
\begin{proof}
    First of all, \(\alpha\) is the root of
    \begin{equation*}
        X^2 - \frac{2}{9} \in \mathbb{Q}[X] \text{,}
    \end{equation*}
    so it is an algebraic number.

    Now assume \(\alpha\) is an algebraic integer. Then, there is a monic polynomial \(f \in \mathbb{Z}[X]\) such that \(f(\alpha) = 0\). It is
    \begin{align*}
        f(\alpha) = \left(\frac{\sqrt{2}}{3}\right)^n + a_{n-1}\left(\frac{\sqrt{2}}{3}\right)^{n - 1} + \cdots + a_1 \frac{\sqrt{2}}{3} + a_0 &= 0 \\
        (\sqrt{2})^n + 3 a_{n-1} (\sqrt{2})^{n-1} + \cdots + 3^{n-1} a_1 \sqrt{2} + 3^n a_0 &= 0
    \end{align*}
    If \(n\) is odd, then \(\sqrt{2}\) is not an integer, therefore, we can separate the sum into two smaller ones.
    \begin{align*}
        \sum_{k \text{ even}} 3^{n - k} a_k (\sqrt{2})^k = 0
    \end{align*}
    and
    \begin{align*}
        \sum_{k \text{ odd}} 3^{n - k} a_k (\sqrt{2})^k = \sqrt{2} \sum_{k \text{ even}} 3^{n - k} a_k (\sqrt{2})^{\frac{k - 1}{2}} = 0\text{.}
    \end{align*}
    Both sums are divisible by \(3\) as \(3\) divides \(0\) and since all summands except for the very last one contains multiples of \(3\), they are divisible by \(3\), so the last summand must be divisible by \(3\) as well. But this cannot be. Hence \(\alpha\) is not an algebraic integer.
\end{proof}

\begin{example}
    Show that if \(r \in \mathbb{Q}\) is an algebraic integer, then \(r \in \mathbb{Z}\).
\end{example}
\begin{proof}
    Write \(r = \frac{p}{q}\) such that \(q \not| p\) and we have
    \begin{align*}
        p^n + q a_{n-1}p^{n-1} + \cdots + q^n a_0 = 0
    \end{align*}
    \(q\) divides the whole sum, it divides all summands, but it does not divide \(p^n\), therefore \(q = 1\).
\end{proof}

\chapter{3}

\begin{example}
    Let \(K\) be an algebraic number field. If \(\alpha \in K\), then there is a nonzero integer \(m \in \mathbb{Z}\) such that \(m \alpha \in \mathcal{O}_K\).
\end{example}
\begin{proof}
    Since \(\alpha\) is an algebraic number, we have
    \begin{equation*}
        \alpha^n + a_{n-1} \alpha^{n-1} + \cdots + a_1 \alpha + a_0 = 0
    \end{equation*}
    with \(a_0, \ldots, a_{n-1} \in \mathbb{Q}\). So choose \(m \in \mathbb{Z}\) such that \(m \alpha_i\) is an integer for all \(i\). We have
    \begin{align*}
        m^n \alpha^n + m^n a_{n-1} \alpha^{n-1} + \cdots + m^n a_1 \alpha + m^n a_0 = 0 \\
        (m \alpha)^n + m a_{n-1} (m \alpha)^{n-1} + \cdots + m^{n-1} a_1 (m \alpha) + m^n a_0 = 0
    \end{align*}
    so \(m \alpha \in \mathcal{O}_K\).
\end{proof}

\chapter{Integral Bases}

\section{Overview}

\section{Details}

{\color{red}\textbf{Errors}}
\begin{enumerate}
    \item Pretty sure that in the example 9.1. the sign of characteristic polynomial is swapped. We want the definition of characteristic polynomial to be in such a way that it only creates monic polynomials.
    \item Proof of proposition 10 is wrong. The characteristic polynomial is not the same as the minimal polynomial see:
    
    https://feog.github.io/antchap1.pdf

    \item Proof of prop 10 is missing
    \item Proof of Lemma 11 is wrong because of the stuff above.
    \item Second proof of lemma 11 is not complete.
\end{enumerate}

\subsection{Trace, Norm, and Characteristic Polynomial}

\begin{defbox}
    \begin{definition}[Trace, Norm, and Characteristic Polynomial]
        Let \(K\) be an {\color{mathif}algebraic number field} with {\color{mathif}degree} \(n\). Then, \(K\) can be viewed as an {\color{mathif}finite-dimensional vector space} over \(\mathbb{Q}\). If \(\alpha \in K\), we can define a {\color{mathif}linear operator}
        \begin{equation*}
            \Phi_\alpha: K \longrightarrow K, \qquad v \mapsto \alpha v \text{,}
        \end{equation*}
        which may be represented by \(n \times n\) {\color{mathif}matrix} \(A_\Phi = (a_{i, j})_{1 \leq i, j \leq n}\) by requiring
        \begin{equation*}
            \alpha e_i = \sum_{j=1}^n a_{i, j} e_j\text{,} \qquad a_{i, j} \in \mathbb{Q} \text{.}
        \end{equation*}
        We define {\color{maththen}trace} of \(\alpha\) by \(\mathrm{Tr}_K(\alpha) := \mathrm{Tr}(\Phi_\alpha)\), the {\color{maththen}norm} of \(\alpha\) by \(N(\alpha) := \det(\Phi_\alpha)\) and the {\color{maththen}characteristic polynomial} of \(\alpha\) by \(\chi_K(X) := \det(XI - \Phi_\alpha)\). If 
    \end{definition}
\end{defbox}

\begin{example}
    In this example, we will compute the traces, norms and characteristic polynomials of some elements in concrete algebraic number fields.
    \begin{enumerate}
    \item Let \(K = \mathbb{Q}(i)\). If \(\alpha = a + ib\) with \(a, b \in \mathbb{Q}\), then the trace of \(\alpha\) is \(\mathrm{Tr}_K(\alpha) = 2a\), the norm of \(\alpha\) is \(\mathrm{N}_K(\alpha) = a^2 + b^2\), and the characteristic polynomial of \(\alpha\) is \(\chi_\alpha (X) = X^2 - 2a X + a^2 + b^2\).
    \begin{proof}
        A basis of \(K\) is \(\{1, i\}\). Then \(\Phi_\alpha\) is defined by
        \begin{align*}
            1 + 0 \cdot i &\mapsto \alpha = a + ib\\
            0 + 1 \cdot i &\mapsto \alpha i = -b + i a
        \end{align*}
        and we may represent \(\Phi\) by a \(2 \times 2\) matrix
        \begin{align*}
            A_\Phi = \begin{pmatrix}
                a & -b \\ b & a
            \end{pmatrix} \text{.}
        \end{align*}
        Therefore, \(\mathrm{Tr}_K(\alpha) = 2a\), \(\mathrm{N}_K(\alpha) = a^2 + b^2\), and \(\chi_\alpha (X) = X^2 - 2a X + a^2 + b^2\).
    \end{proof}
    \item Let \(K = \mathbb{Q}(\sqrt{2})\). If \(\alpha = a + \sqrt{2}b\) with \(a, b \in \mathbb{Q}\), then the trace of \(\alpha\) is \(\mathrm{Tr}_K(\alpha) = 2a\), the norm of \(\alpha\) is \(\mathrm{N}_K(\alpha) = a^2 - 2b^2\), and the characteristic polynomial of \(\alpha\) is \(\chi_\alpha = X^2 - 2a X + a^2 - 2b^2\).
    \begin{proof}
        A basis of \(K\) is \(\{1, \sqrt{2}\}\). Define \(\Phi_\alpha\) by
        \begin{align*}
            1 + 0 \cdot \sqrt{2} &\mapsto \alpha = a + \sqrt{2} b \\
            0 + 1 \cdot \sqrt{2} &\mapsto \sqrt{2} \alpha = 2b + \sqrt{2}a
        \end{align*}
        then the matrix belonging to \(\Phi_\alpha\) is
        \begin{align*}
            A_\Phi = \begin{pmatrix}
                a & 2b \\
                b & a
            \end{pmatrix} \text{.}
        \end{align*}
        So we have \(\mathrm{Tr}_K(\alpha) = 2a\), \(\mathrm{N}_K(\alpha) = a^2 - 2b^2\), and \(\chi_\alpha = X^2 - 2a X + a^2 - 2b^2\).
    \end{proof}
    \item Let \(K = \mathbb{Q}(\sqrt{5})\). If \(\alpha = a + \sqrt{5}b\), then \(\mathrm{Tr}_K(\alpha) = 2a\) and \(\mathrm{N}_K(\alpha) = a^2 - 5b^2\).
    \begin{proof}
        A basis\footnote{ The given basis is not an integral basis, but an integral basis is not required to find the field trace and field norm of an element.} of \(K\) is \(\{1, \sqrt{5}\}\). As before, the linear operator \(\Phi\) is defined by
        \begin{align*}
            1 + 0 \cdot \omega &\mapsto \alpha = a + \sqrt{5} b \\
            0 + 1 \cdot \omega &\mapsto \omega \alpha = 5b + \sqrt{5}a
        \end{align*}
        and the matrix belonging to \(\Phi\) is given by
        \begin{align*}
            A_\Phi = \begin{pmatrix}
                a & 5b \\
                b & a
            \end{pmatrix}
        \end{align*}
        hence it is \(\mathrm{Tr}_K(\alpha) = 2a\), \(\mathrm{N}_K(\alpha) = a^2 - 5b^2\), and \(\chi_\alpha(X) = X^2 - 2aX + a^2 - 5b^2\).
    \end{proof}
    \item In more general terms, let \(K = \mathbb{Q}(\sqrt{d})\) where \(d\) is a square-free integer. If \(\alpha = a + \sqrt{d}b\), then \(\mathrm{Tr}_K(\alpha) = 2a\) and \(\mathrm{N}_K(\alpha) = a^2 - db^2\).
    \begin{proof}
        A basis of \(K\) is \({1, \sqrt{d}}\). Let \(\Phi_\alpha\) be a linear operator defined by
        \begin{align*}
            1 + 0 \cdot \sqrt{d} &\mapsto \alpha = a + \sqrt{d}b \\
            0 + 1 \cdot \sqrt{d} &\mapsto \sqrt{d}\alpha = db + \sqrt{d}a
        \end{align*}
        which we may represent by a \(2 \times 2\) matrix
        \begin{align*}
            A_{\Phi} = \begin{pmatrix}
                a & db \\
                b & a
            \end{pmatrix} \text{.}
        \end{align*}
        We have \(\mathrm{Tr}_K(\alpha) = 2a\), \(\mathrm{N}_K(\alpha) = a^2 - db^2\), and \(\chi_\alpha(X) = X^2 - 2aX + a^2 - d b^2\) matching the results in our previous examples.
    \end{proof}
    \item Let \(\mathbb{Q}(\sqrt[3]{2})\). If \(\alpha = a + \sqrt[3]{2} b + \sqrt[3]{4} c\), then \(\mathrm{N}_K(\alpha) = a^3 + 2b^3 + 4c^3 - 6abc\).
    \begin{proof}
        A basis of \(K\) is \(\{1, \sqrt[3]{2}, \sqrt[3]{4}\}\). Let \(\Phi_\alpha\) be a linear operator defined by
        \begin{align*}
            1 + 0 \cdot \sqrt[3]{2} + 0 \cdot \sqrt[3]{4} &\mapsto \alpha = a + \sqrt[3]{2} b + \sqrt[3]{4} c \\
            0 + 1 \cdot \sqrt[3]{2} + 0 \cdot \sqrt[3]{4} &\mapsto \sqrt[3]{2} \alpha = 2 c + \sqrt[3]{2}a + \sqrt[3]{4} b \\
            0 + 0 \cdot \sqrt[3]{2} + 1 \cdot \sqrt[3]{4} & \mapsto \sqrt[3]{4} \alpha = 2b + \sqrt[3]{2} (2c) + \sqrt[3]{4} a
        \end{align*}
        which we again represent by
        \begin{align*}
            A_\Phi = 
            \begin{pmatrix}
                a & 2c & 2b \\
                b & a & 2c \\
                c & b & a
            \end{pmatrix} \text{.}
        \end{align*}
        We have \(\mathrm{Tr}_K(\alpha) = 3a\), \(\mathrm{N}_K(\alpha) = a^3 + 2b^3 + 4c^3 - 6abc\) and \(\chi_\alpha(X) = X^3 - 3aX^2 + 3a^2X - 6bcX - a^3 -2b^3 + 6abc - 4c^3\).
    \end{proof}
    \end{enumerate}
\end{example}

\begin{example}
    In this example, we will look at the cases when the field extension is given by a root of a polynomial \footnote{The field extensions are defined by a single root of a polynomial, in other words, how the field extension looks exactly depends on the chosen root, and they are not a splitting field of the polynomial.}.
    \begin{enumerate}
        \item Let \(K = \mathbb{Q}(\theta)\) where \(\theta\) is a root of \(X^3 - X - 1\). If \(\alpha = a + \theta b + \theta^2 c\), then \(\mathrm{Tr}_K(\alpha) = 3a + 2c\) and \(\mathrm{N}_K(\alpha) = a^3 + b^3 + 2 a^2 c - b c^2 + c^3 + a (-b^2 - 3 b c + c^2)\).
        \begin{proof}
            First, we have
            \begin{align*}
                X^3 - X - 1 = 0 \quad &\Rightarrow \quad \theta^3 - \theta - 1 =0 \\
                &\Rightarrow \quad \theta^3 = \theta + 1
            \end{align*}
            so \([K:Q] \leq 3\).
            
            Assume \(X^3 - X - 1\) is reducible, then by Rational Root Theorem, we have that there is a root \(x = pq^{-1}\) with \(p\) is a factor of \(-1\) and \(q\) is a factor of \(1\), but \(\pm 1\) is not a root of the polynomial. Thus, \(X^3 - X - 1\) is irreducible.

            If \(\theta^2\) is rational, then the minimal polynomial of \(\theta\) has a degree of \(2\) and divides \(X^2 - X - 1\) which cannot be. Hence \(\theta^2\) is not rational and \(\{1, \theta, \theta^2\}\) is a basis for \(K\).

            Let \(\alpha = a + \theta b + \theta^2 c\) be an element in \(K\) and define a linear operator \(\Phi_\alpha\) by
            \begin{align*}
                1 + 0 \cdot \theta + 0 \cdot \theta^2 \mapsto \alpha &= a + \theta b + \theta^2 c \\
                0 + 1 \cdot \theta + 0 \cdot \theta^2 \mapsto \theta \alpha &= \theta^3c + \theta a + \theta^2 b \\
                &= (\theta + 1)c + \theta a + \theta^2 b \\
                &= c + \theta (a + c) + \theta^2 b \\
                0 + 0 \cdot \theta + 1 \cdot \theta^2 \mapsto \theta^2\alpha &= \theta^3 b + \theta^4 c + \theta^2 a \\
                &= (\theta + 1)b + (\theta^2 + \theta)c + \theta^2 a\\
                &= b + \theta b + \theta^2 c + \theta c + \theta^2 a \\
                &= b + \theta (b + c) + \theta^2 (a + c) 
            \end{align*}
            which we represent with a \(3 \times 3\) matrix
            \begin{align*}
                A_\Phi = \begin{pmatrix}
                    a & c & b \\
                    b & a + c & b + c \\
                    c & b & a + c
                \end{pmatrix}
            \end{align*}
            so we have \(\mathrm{Tr}_K(\alpha) = 3a + 2c\), \(\mathrm{N}_K(\alpha) = a^3 + b^3 + 2 a^2 c - b c^2 + c^3 + a (-b^2 - 3 b c + c^2)\), and \(\chi_\alpha(X) = \)
        \end{proof}
        \item Let \(K = \mathbb{Q}(\theta)\) where \(\theta\) is a root of \(f(X) = X^4 - X - 1\).
        \begin{proof}
            If \(X^4 - X - 1\) is reducible, then by Rational Root Theorem, there is a root \(pq^{-1}\) with \(p, q \in \mathbb{Z}\) relatively prime such that \(p\) is a factor of \(-1\) and \(q\) is a factor of \(1\). However, \(\pm 1\) is not a root because \(f(\pm 1) = 1 \pm 1 - 1 = \pm 1\). We have that \(f\) is irreducible over the rational numbers.

            Since \(f\) is irreducible, if \(\theta\) is a root of \(f\), then \(f\) is the minimal polynomial of \(\theta\) and we have a basis \(\{1, \theta, \theta^2, \theta^3\}\) for \(K\). Now let \(\alpha = a + \theta b + \theta^2 c + \theta^3 d\) and define a linear operator \(\Phi_\alpha\) by
            \begin{align*}
                1 + 0 \cdot \theta + 0 \cdot \theta^2 + 0 \theta^3 \mapsto \alpha &= a + \theta b + \theta^2 c + \theta^3 d \\
                0 + 1 \cdot \theta + 0 \cdot \theta^2 + 0 \theta^3 \mapsto \theta \alpha &= \theta a + \theta^2 b + \theta^3 c + \theta^4 d \\
                &= \theta a + \theta^2 b + \theta^3 c + (\theta + 1) d \\
                &= d + \theta (a + d) + \theta^2 b + \theta^3 c \\
                0 + 0 \cdot \theta + 1 \cdot \theta^2 + 0 \cdot \theta^3 \mapsto \theta^2 \alpha &= \theta^2 a + \theta^3 b + \theta^4 c + \theta^5 d \\
                &= \theta^2 a + \theta^3 b + (\theta + 1) c + (\theta^2 + \theta) d \\
                &= c + \theta (c + d) + \theta^2 (a + d) + \theta^3 b \\
                0 + 0 \cdot \theta + 0 \cdot \theta^2 + 1 \cdot \theta^3 \mapsto \theta^3 \alpha &= \theta^3 a + \theta^4 b + \theta^5 c + \theta^6 d \\
                &= \theta^3 a + (\theta + 1) b + (\theta^2 + \theta) c + (\theta^3 + \theta^2) d \\
                &= b + \theta (b + c) + \theta^2 (c + d) + \theta^3 (a + d)
            \end{align*}
            which we can represent by a \(4 \times 4\) matrix
            \begin{align*}
                A_\Phi = \begin{pmatrix}
                    a & d & c & b \\
                    b & a + d & c + d & b + c \\
                    c & b & a + d & c + d \\
                    d & c & b & a + d
                \end{pmatrix} \text{.}
            \end{align*}
            So we have \(\mathrm{Tr}_K(\alpha) = 4a + 3d\) and the norm is too unholy to write it out here.
        \end{proof}
    \end{enumerate}
\end{example}

\begin{thmbox}
    \begin{proposition}
        Let \(K\) be an algebraic number field and let \(\alpha \in K\) be an algebraic number. If \(\sigma_1, \ldots, \sigma_n\) are \(n\) distinct embeddings of \(K\) into \(\mathbb{C}\) which fix \(K\), then we have for the trace of \(\alpha\) and the norm of \(\alpha\) that
        \begin{align*}
            \mathrm{Tr}_K(\alpha) = \sum_{i=1}^n \sigma_i(\alpha) \qquad \mathrm{N}_K(\alpha) = \prod_{i=1}^n \sigma_i(\alpha)\text{.}
        \end{align*}
    \end{proposition}
\end{thmbox}
\begin{proof}
    
\end{proof}

\begin{thmbox}
    \begin{proposition}
        If \(\alpha\) is an algebraic number, then its characteristic polynomial is a multiple of its minimal polynomial, i.e. let \(K\) be an algebraic number field and let \(\alpha \in K\) be an algebraic number, then
        \begin{equation*}
            \chi_\alpha(X) = m_\alpha(X) h(X) \qquad h(X) \in \mathbb{Q}[X]
        \end{equation*}
        where \(\chi_\alpha\) is the characteristic polynomial of \(\alpha\) and \(m_\alpha\) is the minimal polynomial of \(\alpha\).
    \end{proposition}
\end{thmbox}
\begin{proof}
    Let \(K\) be an algebraic number field and fix an algebraic number \(\alpha \in K\). Both the characteristic polynomial and the minimal polynomial are monic by definition. If \(m_\alpha\) is a minimal polynomial of \(\alpha\) and has a degree of \(m\), then \([\mathbb{Q}(\alpha) : \mathbb{Q}] = m\), so the characteristic polynomial \(\chi_\alpha\) is also of degree \(m\).

    An intermediate result that we will not prove (because I cannot be bothered by it) is \(\chi_\alpha(\Phi_\alpha) = \Phi_{\chi_\alpha(\alpha)}\). By Cayley-Hamilton theorem we also have \(\chi_\alpha(\Phi_\alpha) = 0\), so \(\Phi_{\chi_\alpha(\alpha)} = 0\). If the linear operator defined by \(v \mapsto \alpha v\) is identical to \(0\), then \(\alpha\) must be \(0\), so \(\chi_\alpha(\alpha) = 0\) which means that \(\alpha\) is a root of its own characteristic polynomial.

    The characteristic polynomial and the minimal polynomial share a root and because of minimality of the minimal polynomial, the minimal polynomial divides the characteristic polynomial, but because they are of the same degree, we have that these polynomials are identical.
\end{proof}

\begin{example}
    Let \(K = \mathbb{Q}(\sqrt{2}, \sqrt{3})\) and \(\alpha = \sqrt{2}+ \sqrt{3}\). The minimal polynomial of \(\alpha\) is
\end{example}
\begin{proof}
    A basis of \(K\) is \(\{1, \sqrt{2}, \sqrt{3}, \sqrt{6}\}\). Define a linear mapping \(\Phi_\alpha\) by
    \begin{align*}
        1 + 0 \cdot \sqrt{2} + 0 \cdot \sqrt{3} + 0 \cdot \sqrt{6} \mapsto (\sqrt{2} + \sqrt{3}) 1 &= 0 + \sqrt{2} + \sqrt{3} + 0 \\
        0 + 1 \cdot \sqrt{2} + 0 \cdot \sqrt{3} + 0 \cdot \sqrt{6} \mapsto (\sqrt{2} + \sqrt{3}) \sqrt{2} &= 2 + 0 + 0 + \sqrt{6} \\
        0 + 0 \cdot \sqrt{2} + 1 \cdot \sqrt{3} + 0 \cdot \sqrt{6} \mapsto (\sqrt{2} + \sqrt{3}) \sqrt{3} &= 3 + 0 + 0 + \sqrt{6} \\
        0 + 0 \cdot \sqrt{2} + 0 \cdot \sqrt{3} + 0 \cdot \sqrt{6} \mapsto (\sqrt{2} + \sqrt{3}) \sqrt{6} &= 0 + 3 \sqrt{2} + 2 \sqrt{3} + 0
    \end{align*}
    which is represented by
    \begin{align*}
        A_\phi = \begin{pmatrix}
            0 & 2 & 3 & 0 \\
            1 & 0 & 0 & 3 \\
            1 & 0 & 0 & 2 \\
            0 & 1 & 1 & 0
        \end{pmatrix} \text{.}
    \end{align*}
    This gives us the minimal polynomial \(m_\alpha(X) = X^4 - 10 X + 1\) of \(\alpha\).
\end{proof}

\begin{thmbox}
    \begin{lemma}
        If \(K\) is an algebraic number field, and \(\alpha \in \mathcal{O}_K\) an element in its ring of integers, then \(\mathrm{Tr}_K(\alpha)\) and \(\mathrm{N}_K(\alpha)\) are in \(\mathbb{Z}\).
    \end{lemma}
\end{thmbox}
\begin{enumerate}
    \item \begin{proof} Let \(\alpha\) be an algebraic integer. Since the characteristic polynomial of \(\alpha\) is a multiple of the minimal polynomial, it has integer coeffcients. Denote the minimal polynomial by \(m_\alpha = \sum_{i=0}^n a_i X^i\) for some \(a_i \in \mathbb{Z}\) for \(1 \leq i \leq n\). The trace is coeffcient of \(X^{n-1}\) in the characteristic polynomial \footnote{I had no idea. Don't want to prove it now, but I'll look into this later.}, therefore it is an integer. Moreover, we have \(N(\alpha) = (-1)^n \chi_\alpha(0) = (-1)^n a_0\) which again is an integer.
    \end{proof}
    \item \begin{proof}
        Let \(K\) be an algebraic number field of degree \(n\) and fix an element \(\alpha \in \mathcal{O}_K\) in its ring of integers. We define a linear operator \(\Phi: K \longrightarrow K\) by \(v \mapsto \alpha v\). If \(e_1, \ldots, e_n\) is a basis of \(K\) viewed as a vector space over \(\mathbb{Q}\), then we may represent \(\Phi\) as a \(n \times n\) matrix by
        \begin{align*}
            \alpha e_i = \sum_{j=1}^n a_{i, j} e_j
        \end{align*}
        for all \(1 \leq i \leq n\) and \(a_{i, j} \in \mathbb{Q}\). Taking the conjugates, we get
        \begin{align*}
            \alpha^{(k)} e_i^{(k)} = \sum_{j=1}^n a_{i, j} e_j^{(k)}
        \end{align*}
        and with Kronecker delta we can write
        \begin{align*}
            \sum_{j=1}^n \delta_{j, k} \alpha^{(j)} e_i^{(j)} = \sum_{j=1}^n a_{i, j} e_j^{(k)} \text{.}
        \end{align*}
        Now set \(\Phi_A := (a_{i, j})\)
    \end{proof}
\end{enumerate}

\begin{example}
    Let \(K = \mathbb{Q}(i)\). Show that \(i \in \mathcal{O}_K\) and verify that \(\mathrm{Tr}_K(i)\) and \(\mathrm{N}_K(i)\) are integers.
\end{example}
\begin{proof}
    \(X^2 + 1 \in \mathbb{Z}[X]\) has the root \(i\), so \(i\) is in \(\mathcal{O}_K\). Since the \(\mathbb{Q}\)-basis of \(\mathbb{Q}(i)\) is \(\set{1, i}\), we have
    \begin{align*}
        \Phi_i(a + ib) = -b + a_i
    \end{align*}
    therefore, the matrix is
    \begin{align*}
        \Phi_i =
        \begin{pmatrix}
            0 & 1 \\ -1 & 0
        \end{pmatrix}
    \end{align*}
    and hence its trace is \(\mathrm{Tr}_K(i) = 0\). Similary, its norm is \(\mathrm{N}_K(i) = 1\).
\end{proof}
\begin{example}
    Determine the algebraic integers of \(\mathbb{Q}(\sqrt{-5})\).
\end{example}

\begin{proof}
    A \(\mathbb{Q}\)-basis for \(\mathbb{Q}(\sqrt{-5})\) is \(\set{1, \sqrt{-5}}\). Let \(\alpha = x + \sqrt{-5}y \in \mathbb{Q}(\sqrt{-5})\). Then
    \begin{align*}
        \Phi_x (a + \sqrt{-5} b) = (x + \sqrt{-5}y)(a + \sqrt{-5}b) = xa -5 yb + (bx + ya) \sqrt{-5} \text{,} 
    \end{align*}
    therefore,
    \begin{align*}
        \Phi_\alpha =
        \begin{pmatrix}
            x & y \\ -5 y & x
        \end{pmatrix}
    \end{align*}
    hence we have \(\mathrm{Tr}_K(\alpha) = 2x\) and \(\mathrm{N}_K = x^2 + 5y^2\).

    If \(x\) is not an integer, then \(2x\) must be, so we must have that \(y^2 \equiv 3 \mod 4\), but this is impossible. Hence \(x, y\) are both integers, therefore, \(\mathcal{O}_K = \mathbb{Z}[\sqrt{-5}]\).
\end{proof}

\begin{thmbox}
    \begin{corollary}
        An element \(u\) in a ring of integers \(\mathcal{O}_K\) is a unit if and only if \(N_K(u) = \pm 1\).
    \end{corollary}
\end{thmbox}
\begin{proof}
    \begin{enumerate}
        \item ``\(\Rightarrow\)'': Let \(u \in \mathcal{O}_K\) be a unit, then there is an element \(v \in \mathcal{O}_K\) such that \(uv = 1\). By the multiplicative property, we have
        \begin{align*}
            1 = N_K(1) = N_K(uv) = N_K(u)N_K(v)
        \end{align*}
        and since \(N_K(u)\) and \(N_K(v)\) are both integers, the only possibilities are \(N_K(u) = \pm 1\).
        \item ``\(\Leftarrow\)'': Conversely, let \(u\) have the norm \(\pm 1\). Denote the characteristic polynomial of \(u\) by \(\chi_u (X) = X^n + a_{n-1} X^{n-1} + \cdots + a_1 X + a_0\). It is
        \begin{align*}
            N(u) = \pm 1 = \det (\Phi_u) = \det(0 I - \Phi_u) = \chi_u (0) = a_0
        \end{align*}
        so we have
        \begin{align*}
            u^n + a_{n-1} u^{n-1} + \cdots + a_1 u \pm 1 = 0 \text{.}
        \end{align*}
        Now let \(v \in K\) be the multiplicative inverse of \(u\), i.e. \(uv = 1\). We want to show that \(v\) is in the ring of integers of \(K\). Multiplying \(v^n\) on both sides of the equation above yields
        \begin{align*}
            & u^n v^n + a_{n-1} u^{n-1} v^n + \cdots + a_1 u v^n \pm v^n = 0 \\
            \iff & 1 + a_{n-1} v + \cdots + a_1 v^{n-1} \pm v^n = 0 \text{.}
        \end{align*}
        Thus \(1 + a_{n-1} X + \cdots + a_1 X^{n-1} \pm X^n\) is a polynomial in \(\mathbb{Z}[X]\) that has \(v\) as a root. This polynomial can be converted to a monic polynomial depending on the sign of the monomial of the highest degree by multiplying \(-1\) if necessary. Therefore, \(u \in \mathcal{O}_K\).
    \end{enumerate}
\end{proof}

\newpage
\subsection{Integral Basis}


\begin{example}
    4.1.5 I'll skip this.
\end{example}

\begin{example}
    Show that there exist \(\omega_1^*, \ldots, \omega_n^* \in K\) such that
    \begin{equation*}
        \mathcal{O}_K \subset \mathbb{Z}\omega_1^* + \cdots + \mathbb{Z} \omega_n^* \text{.}
    \end{equation*}
\end{example}
\begin{proof}
    Let \(\omega_1, \ldots, \omega_n\) be a \(\mathbb{Q}\)-basis for \(K\). For any \(\alpha \in K\), there is a nonzero integer \(m \in \mathbb{Z}\) such that \(m \alpha \in \mathcal{O}_K\).
\end{proof}

I'll skip exercises that require bilinear form for now.

\begin{defbox}
    \begin{definition}
        Let \(K\) be an algebraic number field of degree \(n\) and \(\mathcal{O}_K\) be its ring of integers. We say that \(\omega_1, \ldots, \omega_n\) is an integral basis for \(K\) if \(\omega_i \in \mathcal{O}_K\) for all \(1 \leq i \leq n\) and \(\mathcal{O}_K = \mathbb{Z}\omega_1 + \cdots + \mathbb{Z}\omega_n\).
    \end{definition}
\end{defbox}

\begin{example}
    Show that \(\det{\mathrm{Tr}(\omega_i \omega_j)}\) is independent of the choice of integral basis.
\end{example}

\newpage
\subsection{Discriminant}

\begin{defbox}
    \begin{definition}[Discriminant]
        Let \(K\) be an algebraic number field of degree \(n\) and \(\omega_1, \ldots, \omega_n\) an integral basis. The discriminant of \(K\) is defined as
        \begin{equation*}
            d_K := \det\left(\omega_i^{(j)}\right)^2 \text{.}
        \end{equation*}
    \end{definition}
\end{defbox}

\begin{proof}
    We show that the discriminant is well-defined. In other words, the discriminant is independent of the choice of integral basis.

    Let \(\omega_1, \ldots, \omega_n\) and \(\theta_1, \ldots, \theta_n\) be two integral basis for \(K\).
\end{proof}

\begin{example}
    In this example, we will compute the discriminant of some concrete algebraic number fields.
    \begin{enumerate}
        \item Let \(d\) be a square-free integer and consider the algebraic number field \(K = \mathbb{Q}(\sqrt{d})\). The discriminant of \(K\) is
        \begin{align*}
            \Delta_K = \begin{cases}
                d & \text{if \(d \equiv 1 \mod{4}\)}\\
                4d & \text{if \(d \equiv 2, 3 \mod{4}\).}
            \end{cases}
        \end{align*}
        \begin{proof}
            The ring of integers of \(K\) is \(\mathbb{Z}[\alpha]\) where
            \begin{align*}
                \alpha := \begin{cases}
                    \frac{1 + \sqrt{d}}{2} & d \equiv 1 \mod{4} \\
                    \sqrt{d} & d \equiv 2, 3 \mod{4} \text{.}
                \end{cases}
            \end{align*}
            We will look at each case one by one.
            \begin{enumerate}
                \item If \(\alpha = 2^{-1}(1 + \sqrt{d})\), then a integral basis and its conjugate are
                \begin{align*}
                    \set{1, \frac{1 + \sqrt{d}}{2}} \text{ and } \set{1, \frac{1 - \sqrt{d}}{2}} \text{,}
                \end{align*}
                therefore, the discriminant is
                \begin{align*}
                    \Delta_K = \begin{pmatrix}
                        1 & 1 \\
                        \frac{1 + \sqrt{d}}{2} & \frac{1 - \sqrt{d}}{2}
                    \end{pmatrix}^2
                    = \left(\frac{1 - \sqrt{d}}{2} - \frac{1 + \sqrt{d}}{2}\right)^2
                    = \left(-\frac{2 \sqrt{d}}{2}\right)^2 = d \text{.}
                \end{align*}
                \item On the other hand, if \(\alpha = \sqrt{d}\), then a integral basis and its conjugate are
                \begin{align*}
                    \set{1, \sqrt{d}} \text{ and } \set{1, -\sqrt{d}}
                \end{align*}
                and hence we have
                \begin{align*}
                    \Delta_K = \begin{pmatrix}
                        1 & 1 \\
                        \sqrt{d} & -\sqrt{d}
                    \end{pmatrix}^2
                    = \left(-2\sqrt{d}\right)^2 = 4d \text{.}
                \end{align*}
            \end{enumerate}
            Conclude the stated result above.
        \end{proof}
    \item Let \(K = \mathbb{Q}(\sqrt[3]{2})\), then the discriminant is \(\Delta_K = -108\).
    \begin{proof}
        Denote \(\alpha := \sqrt[3]{2}\). The ring of integers is\footnote{To prove this, there is a lot of work to be done and I will maybe do this someday.} \(\mathcal{O}_K = \mathbb{Z}[\alpha]\), so a integral basis is \(\{1, \alpha, \alpha^2\}\). Moreover, the minimal polynomial of \(\alpha\) is
        \begin{align*}
            m_\alpha (X) = X^3 - 2
        \end{align*}
        which has the roots \(\alpha\), \(\zeta_3 \alpha\) and \(\zeta_3^2 \alpha\) where \(\zeta_3\) is the 3rd root of unity. This gives us the embeddings
        % below, somehow the first character is not rendered
        \begin{alignat*}
            \sigma \sigma_1 \equiv \mathrm{id}_K &\quad \sigma_2(\alpha) = \zeta_3 \alpha &&\quad \sigma_3(\alpha) = \zeta_3^2 \alpha \\
            &\quad \sigma_2(\zeta_3 \alpha) = \zeta_3^2 \alpha &&\quad \sigma_3(\zeta_3 \alpha) = \zeta_3 \alpha\\
            &\quad \sigma_2(\zeta_3^2 \alpha) = \alpha && \quad \sigma_3(\zeta_3^2 \alpha) = \alpha
        \end{alignat*}
        or shorter, \(\sigma_1: \alpha \mapsto \alpha\), \(\sigma_2: \alpha \mapsto \zeta_3 \alpha\) and \(\sigma_3: \alpha \mapsto \zeta_3^2 \alpha\). Applying the embeddings on the integral basis yields
        \begin{alignat*}
            \sigma \sigma_1(1) = 1 & \quad \sigma_1(\alpha) = \alpha && \quad \sigma_1(\alpha^2) = \alpha^2 \\
            \sigma_2(1) = 1 & \quad \sigma_2(\alpha) = \zeta_3 \alpha && \quad \sigma_2(\alpha^2) = \sigma_2(\alpha) \sigma_2(\alpha) = \zeta_3^2 \alpha^2 \\
            \sigma_3(1) = 1 & \quad \sigma_3(\alpha) = \zeta_3^2 \alpha && \quad \sigma_3(\alpha^2) = \sigma_3(\alpha) \sigma_3(\alpha) = \zeta_3 \alpha^2 \text{.}
        \end{alignat*}
        Therefore, the discriminant is
        \begin{align*}
            \Delta_K = \left(\det \begin{pmatrix}
                1 & \alpha & \alpha^2 \\
                1 & \zeta_3 \alpha & \zeta_3^2 \alpha^2 \\
                1 & \zeta_3^2 \alpha & \zeta_3 \alpha^2
            \end{pmatrix}\right)^2
            = \left(-6 i \sqrt{3}\right)^2 = -108 \text{.}
        \end{align*}
    \end{proof}
    \item Let \(K = \mathbb{Q}(\sqrt{2}, \sqrt{3})\).
    \begin{proof}
        An integral basis\footnote{Again, I will show this later, maybe. Apparantly, there is a general formula to find the ring of integers of two squareroots.} is
        \begin{align*}
            \set{1, \sqrt{2}, \sqrt{3}, \underbrace{\frac{\sqrt{2} + \sqrt{6}}{2}}_{=: \alpha}} \text{.}
        \end{align*}
        We need to find \(4\) embeddings of \(K\). \(\sigma_1\) is, as always, the identity on \(K\). \(\sigma_2\) permutates \(\sqrt{2}\) and \(\sigma_3\) permutates \(\sqrt{3}\). Then, \(\sigma_4 = \sigma_2 \circ \sigma_3\). Therefore, we have
        \begin{alignat*}
            \sigma \sigma_1(1) = 1 & \quad \sigma_1(\sqrt{2}) = \sqrt{2} && \quad \sigma_1(\sqrt{3}) = \sqrt{3} &&& \quad \sigma_1 \left(\frac{\sqrt{2} + \sqrt{6}}{2}\right) = \frac{\sqrt{2} + \sqrt{6}}{2} \\
            \sigma_2(1) = 1 & \quad \sigma_2(\sqrt{2}) = -\sqrt{2} && \quad \sigma_2(\sqrt{3}) = \sqrt{3} &&& \quad \sigma_2 \left(\frac{\sqrt{2} + \sqrt{6}}{2}\right) = \frac{-\sqrt{2} - \sqrt{6}}{2} \\
            \sigma_3(1) = 1 & \quad \sigma_3(\sqrt{2}) = \sqrt{2} && \quad \sigma_3(\sqrt{3}) = - \sqrt{3} &&& \quad \sigma_3 \left(\frac{\sqrt{2} + \sqrt{6}}{2}\right) = \frac{\sqrt{2} - \sqrt{6}}{2} \\
            \sigma_4(1) = 1 & \quad \sigma_4(\sqrt{2}) = -\sqrt{2} && \quad \sigma_4(\sqrt{3}) = - \sqrt{3} &&& \quad \sigma_4 \left(\frac{\sqrt{2} + \sqrt{6}}{2}\right) = \frac{-\sqrt{2} + \sqrt{6}}{2}
        \end{alignat*}
        So the discriminant of \(K\) is
        \begin{align*}
            \Delta_K = \left(\det \begin{pmatrix}
                1 & \sqrt{2} & \sqrt{3} & \frac{\sqrt{2} + \sqrt{6}}{2} \\
                1 & -\sqrt{2} & \sqrt{3} & \frac{-\sqrt{2} - \sqrt{6}}{2} \\
                1 & \sqrt{2} & -\sqrt{3} & \frac{\sqrt{2} - \sqrt{6}}{2} \\
                1 & -\sqrt{2} & -\sqrt{3} & \frac{-\sqrt{2} + \sqrt{6}}{2}
            \end{pmatrix}\right)^2
            = 48^2 = 2304
        \end{align*}
    \end{proof}
    \end{enumerate}
\end{example}

\begin{defbox}
    \begin{definition}
        Let \(K\) be an algebraic number field of degree \(n\) and let \(\sigma_1, \ldots, \sigma_n\) be the embeddings of \(K\). For \(a_1, \ldots, a_n \in K\) we can define the discriminant as
        \begin{align*}
            \Delta_K (a_1, \ldots, a_n) = [\det(\sigma_i(a_j))]^2 \text{.}
        \end{align*}
    \end{definition}
\end{defbox}

\begin{thmbox}
    \begin{proposition}
        It is
        \begin{align*}
            \Delta_K(1, a, a^2 \ldots, a^{n-1}) = \prod_{i > j} (\sigma_i(a) - \sigma_j(a)) \text{.}
        \end{align*}
    \end{proposition}
\end{thmbox}
\begin{proof}
    Because \(\sigma_i(a^j) = a^{j + i - 1}\), we have
    \begin{align*}
        \Delta_K(1, a, a^2, \ldots, a^{n-1}) &= \left[\det \begin{pmatrix}
            \sigma_1(1) & \sigma_1(a) & \sigma_1(a^2) & \cdots & \sigma_1(a^{n-1}) \\
            \sigma_2(1) & \sigma_2(a) & \sigma_2(a^2) & \cdots & \sigma_2(a^{n-1}) \\
            \sigma_3(1) & \sigma_3(a) & \sigma_3(a^2) & \cdots & \sigma_3(a^{n-1}) \\
            \vdots & & & & \\
            \sigma_n(1) & \sigma_n(a) & \sigma_n(a^2) & \cdots & \sigma_n(a^{n-1})
        \end{pmatrix}\right]^2 \\
        &= \left[\det \begin{pmatrix}
            1 & a & a^2 & \cdots & a^{n-1} \\
            1 & a^2 & a^3 & \cdots & a^n \\
            1 & a^3 & a^4 & \cdots & a^{n+1} \\
            \vdots & & & & \vdots \\
            1 & a^n & a^{n+1} & \cdots & a^{2n - 1}
        \end{pmatrix}\right]^2 \\
        &= \prod_{i>j} (\sigma_i(a) - \sigma_j(a))
    \end{align*}
    by Vandermonde matrix.
\end{proof}
\begin{rembox}
    \begin{remark}
        We denote \(\Delta_K(1, a, a^2, \ldots, a^{n-1})\) by \(\Delta_K(a)\).
    \end{remark}
\end{rembox}

\chapter{Dedekind Domains}

\begin{defbox}
    \begin{definition}
        An integral domain \(A\) is called Dedekind domain if one of the following equivalent conditions are met.
        \begin{enumerate}
            \item Every nonzero proper ideal of \(A\) factors into product of prime ideals.
            \item \(A\) is
            \begin{enumerate}
                \item integrally closed,
                \item Noetherian,
                \item and every nonzero prime ideal of \(A\) is maximal.
            \end{enumerate}
        \end{enumerate}
    \end{definition}
\end{defbox}

\begin{rembox}
    \begin{remark}
        We have the following chain of inclusions for ring structures.
    \end{remark}
\end{rembox}

\begin{theorem}
    Every ring of integers \(\mathcal{O}_K\) is a Dedekind domain.
\end{theorem}
\begin{proof}
    Let \(\mathcal{O}_K\) be the ring of integers of an algebraic number field \(K\). We show that \(\mathcal{O}_K\) is a Dedekind domain by proving \(\mathcal{O}_K\) is integrally closed, Noetherian, and every nonzero prime ideal of \(A\) is maximal.
\end{proof}

\begin{thmbox}
    \begin{lemma}
        If \(\mathfrak{a} \subset \mathfrak{b}\) are ideals of \(\mathcal{O}_K\), then \(\mathrm{N}(a) > \mathrm{N}(b)\).
    \end{lemma}
\end{thmbox}

\end{document}