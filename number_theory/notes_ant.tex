\documentclass[a4paper]{book}
\title{Number Theory}
\author{K}


% ---------------------------------------------------------------------
% P A C K A G E S
% ---------------------------------------------------------------------

% typography and formatting
\usepackage[english]{babel}
\usepackage[utf8]{inputenc}
\usepackage{geometry}
\usepackage{exsheets}
\usepackage{environ}

% mathematics
\usepackage{amsthm} % for theorems, and definitions
\usepackage{amssymb}
\usepackage{amsmath}
\usepackage{textcomp}
% \usepackage{MnSymbol} % for \cupdot

% extra
\usepackage{xcolor}
\usepackage{tikz}

% ---------------------------------------------------------------------
% S E T T I N G
% ---------------------------------------------------------------------

%maybe delete later, for colorbox
\usepackage{tcolorbox}
\newtcolorbox{defbox}{colback=blue!5!white,colframe=blue!75!black}
\newtcolorbox{thmbox}{colback=orange!5!white,colframe=orange!75!black}
\newtcolorbox{rembox}{colback=purple!5!white,colframe=purple!75!black}

% typography and formatting
\geometry{margin=3cm}

\SetupExSheets{
  counter-format = ch.qu,
  counter-within = chapter,
  question/print = true,
  solution/print = true,
}

% mathematics
\newcounter{global}

\theoremstyle{definition}
\newtheorem{definition}{Definition}[]
\newtheorem{example}{Example}[definition]

\newtheorem{theorem}[definition]{Theorem}
\newtheorem{corollary}{Corollary}
\newtheorem{lemma}[definition]{Lemma}
\newtheorem{proposition}[definition]{Proposition}

\newtheorem*{remark}{Remark}

% extra
\definecolor{mathif}{HTML}{0000A0} % for conditions
\definecolor{maththen}{HTML}{CC5500} % for consequences
\definecolor{mathrem}{HTML}{8b008b} % for notes
\definecolor{mathobj}{HTML}{008800}

\usetikzlibrary{positioning}
\usetikzlibrary{shapes.geometric, arrows}

% ---------------------------------------------------------------------
% C O M M A N D S
% ---------------------------------------------------------------------

\newcommand{\norm}[1]{\left\lVert#1\right\rVert}
\newcommand{\rank}{\text{rank}}
\newcommand{\Vol}{\text{Vol}}

\newcommand{\set}[1]{\left\{\, #1 \,\right\}}
\newcommand{\makeset}[2]{\left\{\, #1 \mid #2 \,\right\}}

\newcommand*\diff{\mathop{}\!\mathrm{d}}
\newcommand*\Diff{\mathop{}\!\mathrm{D}}

\newcommand\restr[2]{{% we make the whole thing an ordinary symbol
  \left.\kern-\nulldelimiterspace % automatically resize the bar with \right
  #1 % the function
  \vphantom{\big|} % pretend it's a little taller at normal size
  \right|_{#2} % this is the delimiter
  }}

% ---------------------------------------------------------------------
% R E N D E R
% ---------------------------------------------------------------------

\newif\ifshowproof
\showprooftrue

\NewEnviron{Proof}{%
    \ifshowproof%
        \begin{proof}%
            \BODY
        \end{proof}%
    \fi%
}%

\begin{document}
\begin{thmbox}
    \begin{theorem}
        Let \(A\) be an integral domain, and let \(L\) be a field containing \(A\). The elements of \(L\) integral over \(A\) form a ring.
    \end{theorem}
\end{thmbox}
\begin{remark}
    The immediate consequence of this theorem is that the ring of integers is indeed a ring.
\end{remark}
\begin{defbox}
    \begin{definition}
        Symmetric polynomials and elementary symmetric polynomials.
    \end{definition}
\end{defbox}
\begin{thmbox}
    \begin{theorem}
        Let \(A\) be a ring. Every symmetric polynomial \(P(X_1, \ldots, X_r)\) in \(A[X_1, \ldots, X_n]\) can be represented with a linear combination of elementary symmetric polynomials with coeffcients in \(A\).
    \end{theorem}
\end{thmbox}
Proof is constructive and inductive by reducing the polynomial over the lexicographically highest monomial. Not a hard proof, but the indecies are anoying.

The above proof implies:

Let \(f(X) = X^n + a_1 X^{n-1} + \cdots + a_n \in A[X]\), and let \(\alpha_1, \ldots, \alpha_n\) be the roots of \(f(X)\) in some ring containing \(A\), so that \(f(X) = \prod (X - \alpha_i)\) in the larger ring. Then
\begin{align*}
    a_1 = -S_1(\alpha_1, \ldots, \alpha_n), \qquad a_2 = S_2(\alpha_1, \ldots, \alpha_n), \qquad a_n = \pm S_n(\alpha_1, \ldots, \alpha_n) \text{.}
\end{align*}
(I'm not quite sure why this is the case. Maybe use the multi-binomial theorem.)

Thus the elementary symmetric polynomials in the roots of \(f\) lie in \(A\). And so the theorem implies that every symmetric polynomial in the roots of \(f(X)\) lies in \(A\).
\begin{thmbox}
    \begin{proposition}
        Let \(A\) be a integral domain and \(\Omega\) be an algebraically closed field containing \(A\). If \(\alpha_1, \ldots, \alpha_n\) are the roots in \(\Omega\) of a monic polynomial in \(A[X]\), then every polynomial \(g(\alpha_1, \ldots, \alpha_n)\) in \(A[\alpha_1, \ldots, \alpha_n]\) is a root of a monic polynomial in \(A[X]\).
    \end{proposition}
\end{thmbox}
\begin{proof}
    Clearly,
    \begin{align*}
        h(X) := \prod_{\sigma \in \mathrm{Sym}_n} (X -  g(\alpha_{\sigma(1)}, \ldots, \alpha_{\sigma(n  )}))
    \end{align*}
    is a monic polynomial whose coeffcients are symmetric polynomials in the \(\alpha_i\), and therefore lie in \(A\). But \(g(\alpha_1, \ldots, \alpha_n)\) is one of the roots.
\end{proof}
With this we can prove that the above theorem. I don't quite understand few steps ...

\section*{Dedekind's Proof}

\begin{thmbox}
    \begin{proposition}
        Let \(L\) be a field containing \(A\). An element \(\alpha\) of \(L\) is integral over \(A\) if and only if there exists a nonzero finitely generated \(A\)-submodule of \(L\) such that \(\alpha M \subset M\) (in fact, we can take \(M = A[\alpha]\), the \(A\)-subalgebra generated by \(\alpha\)).
    \end{proposition}
\end{thmbox}
\begin{proof}
    \begin{itemize}
        \item Let \(\alpha \in L\) be integral over \(A\). The \(A\)-submodule \(A[\alpha]\) in \(L\) is generated by \(1, \alpha, \ldots, \alpha^{n-1}\), thus finitely generated and clearly nonzero. \(\alpha A[\alpha] \subset A[\alpha]\) also holds.
        \item Let \(M\) be a nonzero, finitely generated \(A\)-submodule in \(L\) such that \(\alpha M \subset M\). Since \(M\) is finitely generated, there is a set of generators \(v_1, \ldots, v_n \in M\). From \(\alpha M \subset M\) we have that
        \begin{align*}
            \alpha v_i = \sum_{j = 1}^n a_{i, j} v_j
        \end{align*}
        for some \(a_{i, j} \in A\). We rewrite this system of equations
        \begin{align*}
            (\alpha - a_{i, i}) v_i \sum_{j = 1, j \neq i}^n a_{i, j} v_j = 0
        \end{align*}
        We have the matrix
        \begin{align*}
            \begin{pmatrix}
                (\alpha - a_{1, 1}) & a_{1, 2} & \cdots & a_{1, n}\\
                a_{2,1} & (\alpha - a_{2, 2}) & \cdots & a_{2, n} \\
                \vdots & & & \vdots \\
                a_{n, 1} & a_{n, 2} & \cdots & (\alpha - a_{n,n})
            \end{pmatrix}
        \end{align*}
        Applying Cramer's Rule we get \(v_i = \frac{\det(C_i)}{\det{C}}\), but \(C_i\) is always \(0\), and at least one \(v_i\) is nonzero, so we have that \(\det(C) = 0\).

        But calculating the determinant of \(C\) gives us

        \begin{align*}
            \alpha^n + c_1 \alpha^{n-1} + \cdots + c_n = 0
        \end{align*}
        as desired.
    \end{itemize}
\end{proof}
Now take \(\alpha\) and \(\beta\) integral over \(A\) and denote \(\alpha M \subset M\) and \(\beta N \subset N\).

\begin{enumerate}
    \item \(MN\) is an \(A\)-submodule of \(L\).
\end{enumerate}

Dedekind's proof is much easier to understand, lol.

\section*{Integral Elements}

\begin{proposition}
    Let \(K\) be the field of fractions of \(A\), and let \(L\) be a field containing \(K\). If \(\alpha \in L\) is algebraic over \(K\), then there exists a nonzero \(d \in A\) such that \(d \alpha \) is integral over \(A\).
\end{proposition}

\begin{corollary}
    Let \(A\) be an integral domain with field of fractions \(K\), and let \(B\) be the integral closure of \(A\) in a field \(L\) containing \(K\). If \(L\) is algebraic over \(K\), then it is the field of fractions \(B\).
\end{corollary}

\part{Exercise}

\begin{example}
    Let \(d\) be a square-free integer. Consider \(A = \mathbb{Z}[\sqrt{d}]\). Show that every element of \(R\) can be written as a product of irreducible elements.
\end{example}
\begin{proof}
    Define \(N: R \longrightarrow \mathbb{N}\) as \(N(a + b \sqrt{d}) = |a^2 - d b^2|\) where \(a, b \in \mathbb{Z}\). Let \(a_1 + b_1 \sqrt{d}\) and \(a_2 + b_2 \sqrt{d}\) be two elements in \(\mathbb{Z}[\sqrt{d}]\) with \(a_1, b_1, a_2, b_2 \in \mathbb{Z}\), then
    \begin{align*}
        N((a_1 + b_1 \sqrt{d}) (a_2 + b_2 \sqrt{d})) &= N((a_1 a_2 + b_1 b_2 d) + (a_1 b_2 + a_2 b_1) \sqrt{d}) \\
        &= |(a_1 a_2 + b_1 b_2 d)^2 - d (a_1 b_2 + a_2 b_1)^2| \\
        &= |a_1^2 a_2^2 + 2 a_1 a_2 b_1 b_2 d + b_1^2 b_2^2 d^2 - a_1^2 b_2^2 d - 2 a_1 a_2 b_1 b_2 d - a_2^2 b_1^2 d | \\
        &= |a_1^2 a_2^2 - a_1^2 b_2^2 d - a_2^2 b_1^2 d + b_1^2 b_2^2 d^2| \\
        \intertext{on the other hand}
        N(a_1 + b_1 \sqrt{d})N(a_2 + b_2 \sqrt{d}) &= |a_1^2 - d b_1^2||a_2^2 - d b_2^2| \\
        &= |a_1^2 a_2^2 - a_1^2 b_2^2 d - a_2^2 b_1^2 d + b_1^2 b^2 d^2 |
    \end{align*}
    so we have \(N((a_1 + b_1 \sqrt{d}) (a_2 + b_2 \sqrt{d})) = N(a_1 + b_1 \sqrt{d})N(a_2 + b_2 \sqrt{d})\). Moreover, let \(u \in \mathbb{Z}[\sqrt{d}]\) be a unit, then there is an element \(v \in \mathbb{Z}[\sqrt{d}]\) such that \(u v = 1\). Applying the function defined above, we get
    \begin{align*}
        1 = N(1) = N(u v) = N(u) N(v)
    \end{align*}
    so \(N(u) = 1\). Now suppose \(N(a + b \sqrt{d}) = 1\) with \(a, b \in \mathbb{Z}\). Consider
    \begin{align*}
        (a + b \sqrt{d})(a - b \sqrt{d}) = a^2 - d b^2 = \pm 1
    \end{align*}
    and therefore \(a + b \sqrt{d}\) is a unit.

    We have shown that \(N\) is a norm map. \(R\) is also an integral domain because if \(x \in R\) is a zero-divisor, then we have \(0 = N(x) = |a^2 - d b^2|\), but this is impossible since \(d\) is square-free. Applying the example before, we get the desired result.
\end{proof}
\end{document}