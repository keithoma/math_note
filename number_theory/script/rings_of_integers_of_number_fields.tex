\section{Integral Dependence}
\begin{definition}
    Let \(A \subset B\) be an extension of rings.
    \begin{enumerate}
        \item An element \(x \in B\) is integral over \(A\) if there is a monic polynomial \(f(T) \in A[T]\) such that \(f(x) = 0 \in B\). That is, if for certain \(n \in \mathbb{N}_0\) there are elements \(a_1, \dots, a_n \in A\) such that
        \begin{equation}
            x^n + a_1 x^{n-1} + a_2 x^{n-2} + \dots + a_n = 0.
        \end{equation}
        Such a polynomial \(f(T) = T^n + a_1 T^{n-1} + \dots + a_n\) is called an equation of integral dependence of \(x\) over \(A\).
        \item The integral closure of \(A\) in \(B\) is the subset
        \begin{equation}
            \bar{A} := \{x \in B \mid x \text{ is integral over A}\} \subset B.
        \end{equation}
        \item The extension \(A \subset B\) is called integral if every element of \(B\) is integral over \(A\).
    \end{enumerate}
\end{definition}
\begin{lemma}
    All UFDs are integrally closed.
\end{lemma}
\begin{definition}[\(A\)-Module]
    
\end{definition}
\begin{remark}
    \begin{enumerate}
        \item The submodules of the ring \(A\) itself are precisely the ideals of \(A\).
        \item In an extension of rings \(A \subset B\), \(B\) is an \(A\)-module.
    \end{enumerate}
\end{remark}
\begin{proposition}
    Let \(A \subset B\) be an extension of rings and \(x \in B\). Then the following are equivalent.
    \begin{itemize}
        \item \(x\) is integral over \(A\).
        \item The subring \(A[x] := \{f(x) \mid f(T) \in A[T]\} \subset S\) is a finitely generated \(A\)-module.
        \item There is a finitely generated \(A\)-submodule \(M \subset B\) such that \(1 \in M\) and \(x \cdot M \subset M\).
    \end{itemize}
\end{proposition}

\section{Rings of Integers of Number Fields}
\begin{definition}
    \begin{enumerate}
        \item A {\color{maththen}number field} is a {\color{mathif}finite extension} \(K\) of \(\mathbb{Q}\).
        \item The {\color{maththen}ring of integers} \(\mathcal{O}_K\) of a number field \(K\) is the {\color{mathif}integral closure} of \(\mathbb{Z}\) in \(K\).
    \end{enumerate}
\end{definition}
\begin{example}
    Let \(K = \mathbb{Q}(i)\) be a number field. Then, \(\mathcal{O}_K = \mathbb{Z}[i]\).
\end{example}
\begin{proposition}
    The field \(L\) and the ring \(B\) are related to each other as follows.
    \begin{enumerate}
        \item \(B = \{\alpha \in L \mid \text{the minimal polynomial }\}\)
    \end{enumerate}
\end{proposition}