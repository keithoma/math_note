\documentclass[a4paper]{article}
\title{Number Theory}
\author{K}


% ---------------------------------------------------------------------
% P A C K A G E S
% ---------------------------------------------------------------------

% typography and formatting
\usepackage[english]{babel}
\usepackage[utf8]{inputenc}
\usepackage{geometry}
\usepackage{exsheets}
\usepackage{environ}

% mathematics
\usepackage{amsthm} % for theorems, and definitions
\usepackage{amssymb}
\usepackage{amsmath}
\usepackage{textcomp}
\usepackage{mathtools}
%\usepackage{MnSymbol} % for \cupdot

% extra
\usepackage{xcolor}
\usepackage{tikz}

% ---------------------------------------------------------------------
% S E T T I N G
% ---------------------------------------------------------------------

% typography and formatting
\geometry{margin=3cm}

\SetupExSheets{
  counter-format = ch.qu,
  counter-within = chapter,
  question/print = true,
  solution/print = true,
}

% mathematics

% extra
\definecolor{mathif}{HTML}{0000A0} % for conditions
\definecolor{maththen}{HTML}{CC5500} % for consequences
\definecolor{mathrem}{HTML}{8b008b} % for notes

\usetikzlibrary{positioning}
\usetikzlibrary{shapes.geometric, arrows}

% ---------------------------------------------------------------------
% C O M M A N D S
% ---------------------------------------------------------------------

\newcommand{\norm}[1]{\left\lVert#1\right\rVert}
\newcommand{\rank}{\text{rank}}
\newcommand{\Vol}{\text{Vol}}

\newcommand{\set}[1]{\left\{\, #1 \,\right\}}
\newcommand{\makeset}[2]{\left\{\, #1 \mid #2 \,\right\}}
\newcommand{\bigslant}[2]{{\raisebox{.2em}{$#1$}\left/\raisebox{-.2em}{$#2$}\right.}}

\newcommand*\diff{\mathop{}\!\mathrm{d}}
\newcommand*\Diff{\mathop{}\!\mathrm{D}}

\newcommand\restr[2]{{% we make the whole thing an ordinary symbol
  \left.\kern-\nulldelimiterspace % automatically resize the bar with \right
  #1 % the function
  \vphantom{\big|} % pretend it's a little taller at normal size
  \right|_{#2} % this is the delimiter
  }}

% ---------------------------------------------------------------------
% R E N D E R
% ---------------------------------------------------------------------

\newif\ifshowproof
\showprooftrue

\NewEnviron{Proof}{%
    \ifshowproof%
        \begin{proof}%
            \BODY
        \end{proof}%
    \fi%
}%

\begin{document}
\section*{Exercise 6.1}
Consider \(K := \mathbb{Q}(\sqrt{-10})\).
\begin{enumerate}
    \item Show that \((2) = \mathfrak{p}^2\) for some prime ideal \(\mathfrak{p} \subset \mathcal{O}_K\) and find the generators of \(\mathfrak{p}\) explicitly.
    \begin{proof}
        Because \(-10 \equiv 2 \mod{4}\), the ring of integer of \(K\) is \(\mathcal{O}_K = \mathbb{Z}[\sqrt{-10}]\). The minimal polynomial of \(\sqrt{-10}\) is \(X^2 + 10\) and we have
        \begin{align*}
            X^2 + 10 \equiv X^2 \mod{2} \text{.}
        \end{align*}
        Thus, \((2) = (2, \sqrt{-10})^2\).
    \end{proof}
    \item Prove that the ideal \(\mathfrak{p}\) you just found is prime, but not principal. Deduce the order of \([\mathfrak{p}] \in \mathrm{Cl}(K)\) is \(2\).
    \begin{proof}
        \((2, \sqrt{-10})\) being prime arises from the theorem that gives the method of computation. Now assume \((2, \sqrt{-10})\) is principal, then there is an \(\alpha \in \mathcal{O}_K\) that divides \(2\). Using the multiplicativity of the norm gives \(N(\alpha)\) divides \(N(2) = 4\), so \(N(\alpha) = 2\), but this is impossible. So \(2\) is irreducible in \(\mathcal{O}_K\) and clearly \(\sqrt{-10}\) is not a multiple of \(2\). Hence the generators do not share a divisor and \((2, \sqrt{-10})\) is principal. Moreover, because \((2, \sqrt{-10})^2 = (2)\) is principal (and all principal ideals are equivalent to \((1)\)), the order of \((2, \sqrt{-10})\) is \(2\).
    \end{proof}
        \item Prove that \((3) \subset \mathcal{O}_K\) is prime. Using Minkowski's bound, deduce that \(\mathrm{Cl}(K) \simeq \mathbb{Z}/2\mathbb{Z}\).
        \begin{proof}
            Similarly as in 1., we have
            \begin{align*}
                X^2 + 10 \equiv X^2 + 1 \mod{3}
            \end{align*}
            which is irreducible in \(\mathbb{Z}/3\mathbb{Z}\), so \((3)\) is prime. The Minkowski's bound for \(K\) is
            \begin{align*}
                M_K = \sqrt{|D_K|} \left(\frac{4}{\pi}\right)^{r_2} \frac{n!}{n^n} = \sqrt{40} \frac{4}{\pi} \frac{2}{4} = \frac{4 \sqrt{10}}{\pi} = 4.03 \text{.}
            \end{align*}
            So the ideal class group is generated by the prime ideals with norm not exceeding \(M_K\). For a prime ideal \(\mathfrak{p}\) where \(\mathrm{N}(\mathfrak{p}) < 4\), \(\mathfrak{p}\) divides \((2)\) or \((3)\). While \(3\) is prime, \((2)\) decomposes as \((2) = (2, \sqrt{-10})\). Thus, \(\mathrm{Cl}(K)\) is generated by \([(1)]\) and \([(2, \sqrt{-10})]\) and we have \(\mathrm{Cl}(K) \simeq \mathbb{Z}/3\mathbb{Z}\).
        \end{proof}
\end{enumerate}

\section*{6.2}
Let \(K\) be a number field with ring of integers \(\mathcal{O}_K\). Suppose \(\mathfrak{p} \subset \mathcal{O}_K\) is nonzero prime ideal and \(p \in \mathbb{N}\) a prime number. Denote the numerical norm of some ideal \(\mathfrak{a} \subseteq \mathcal{O}_K\) by \(N_{K / \mathbb{Q}}\). Show that the following are equivalent.
\begin{enumerate}
    \item \(N_{K/\mathbb{Q}} (\mathfrak{p}) \equiv 0 \mod{p}\).
    \item \(\mathfrak{p} \cap \mathbb{Z} = p \mathbb{Z}\).
    \item \(\mathfrak{p}\) appears in the factorization of \((p) \subseteq \mathcal{O}_K\) into prime ideals.
\end{enumerate}
\begin{proof}
    \begin{enumerate}
        \item \textbf{3. \(\Rightarrow\) 1.} Let \((p)\) decomposes into \(\mathfrak{p} \mathfrak{a}\) where \(\mathfrak{p}\) is a prime ideal and \(\mathfrak{a}\) is an integral ideal. If \(\mathfrak{a}^{-1}\) is a fractional ideal with \(\mathfrak{a}\mathfrak{a}^{-1} = (1)\), we have
        \begin{align*}
            \mathrm{N}_{K/\mathbb{Q}}(\mathfrak{p}) = \mathrm{N}_{K/\mathbb{Q}}((p)) \mathrm{N}_{K/\mathbb{Q}}(\mathfrak{a}^{-1}) = |\mathcal{O}_K / (p)|\mathrm{N}_{K/\mathbb{Q}}(\mathfrak{a}^{-1}) \text{.}
        \end{align*}
        Now, \(|\mathcal{O}_K / (p)|\) is divisible by \(p\), we have that \(N_{K/\mathbb{Q}}(\mathfrak{p}) \equiv 0 \mod{p}\).
    \end{enumerate}
\end{proof}
\end{document}