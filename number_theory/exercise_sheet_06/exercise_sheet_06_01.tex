\documentclass[a4paper]{article}
\title{Integration and Integration}
\author{K}


% ---------------------------------------------------------------------
% P A C K A G E S
% ---------------------------------------------------------------------

% typography and formatting
\usepackage[english]{babel}
\usepackage[utf8]{inputenc}
\usepackage{geometry}
\usepackage{exsheets}
\usepackage{environ}

% mathematics
\usepackage{amsthm} % for theorems, and definitions
\usepackage{amssymb}
\usepackage{amsmath}
\usepackage{textcomp}
%\usepackage{MnSymbol} % for \cupdot

% extra
\usepackage{xcolor}
\usepackage{tikz}

% ---------------------------------------------------------------------
% S E T T I N G
% ---------------------------------------------------------------------

% typography and formatting
\geometry{margin=3cm}

\SetupExSheets{
  counter-format = ch.qu,
  counter-within = chapter,
  question/print = true,
  solution/print = true,
}

% mathematics
\theoremstyle{definition}
\newtheorem{definition}{Definition}
\newtheorem{example}{Example}[definition]

\newtheorem{theorem}{Theorem}[definition]
\newtheorem{corollary}{Corollary}
\newtheorem{lemma}{Lemma}[definition]
\newtheorem{proposition}{Proposition}[definition]

\newtheorem*{remark}{Remark}

% extra
\definecolor{mathif}{HTML}{0000A0} % for conditions
\definecolor{maththen}{HTML}{CC5500} % for consequences
\definecolor{mathrem}{HTML}{8b008b} % for notes

\usetikzlibrary{positioning}
\usetikzlibrary{shapes.geometric, arrows}

% ---------------------------------------------------------------------
% C O M M A N D S
% ---------------------------------------------------------------------

\newcommand{\norm}[1]{\left\lVert#1\right\rVert}
\newcommand{\rank}{\text{rank}}
\newcommand{\Vol}{\text{Vol}}
\newcommand*\diff{\mathop{}\!\mathrm{d}}
\newcommand*\Diff{\mathop{}\!\mathrm{D}}

\newcommand\restr[2]{{% we make the whole thing an ordinary symbol
  \left.\kern-\nulldelimiterspace % automatically resize the bar with \right
  #1 % the function
  \vphantom{\big|} % pretend it's a little taller at normal size
  \right|_{#2} % this is the delimiter
  }}

% ---------------------------------------------------------------------
% R E N D E R
% ---------------------------------------------------------------------

\newif\ifshowproof
\showprooftrue

\NewEnviron{Proof}{%
    \ifshowproof%
        \begin{proof}%
            \BODY
        \end{proof}%
    \fi%
}%

\begin{document}
\begin{center}
    \noindent\textbf{Exercise Sheet 2}
\end{center}
\noindent\textbf{Exercise 1}

\noindent\textbf{Solution}

\begin{enumerate}
    \item \(\mathbb{Z}\times\mathbb{Z}\) is not a Dedekind domain as it is not even an integral domain. Take \((1, 0) \in \mathbb{Z}\times\mathbb{Z}\) and \((0, 1) \in \mathbb{Z}\times\mathbb{Z}\) for example. \((1, 0) \cdot (0, 1) = (0, 0)\) even though we chose nonzero elements.
    \item We have \(\mathbb{Z}[X] / (X^2 + 3) \cong \mathbb{Z}[\sqrt{-3}]\); therefore, \(\mathbb{Z}[X] / (X^2 + 3)\) is an integral domain.

    \(\mathbb{Z}[X] / (X^2 + 3)\) is also noetherian as \(\mathbb{Z}\) is noetherian and therefore it's polynomial ring and the quotient ring. 
    % why are they isomorphic? use isomorphism theorem 
    % it is also noetherian
    % is integrally closed in Quot(A)?
    % all above doesn't matter, this is not a Dedekind domain
    %
    % it is not integrally closed for example take (1/2) + (1/2) * sqrt(-3)
    % the third one is a finite field order 11
    % aren't all finite fields with the same order isomorphic to each other?
\end{enumerate}

\end{document}