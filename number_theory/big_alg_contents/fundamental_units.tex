\begin{defbox}
    \begin{definition}
        \begin{enumerate}
            \item An (additive) subgroup \(\Lambda\) of \(\mathbb{R}^m\) is called discrete if any bounded subset of \(\mathbb{R}^m\) contains only finitely many elements of \(\Lambda\).
            \item Let \(\{\gamma_1, \ldots, \gamma_r\}\) be linearly independent set of vectors of \(\mathbb{R}^m\) (so that \(r \leq m\)). The additive subgroup of \(\mathbb{R}^m\) generated by \(\gamma_1, \ldots, \gamma_r\) is called lattice of dimension \(r\), generated by \(\gamma_1, \ldots, \gamma_r\).
        \end{enumerate}
    \end{definition}
\end{defbox}

\begin{thmbox}
    \begin{theorem}
        Any discrete subgroup \(\Lambda\) of \(\mathbb{R}^m\) for \(m \in \mathbb{N}\) is a lattice.
    \end{theorem}
\end{thmbox}

\begin{proof}
    If \(m = 0\), then the discrete subgroup of \(\mathbb{R}^0\) is \(\Lambda = \{0\}\), which is a lattice of dimension \(0\).

    Let \(m = 1\).
\end{proof}