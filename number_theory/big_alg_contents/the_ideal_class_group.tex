\begin{defbox}
    \begin{definition}[Fractional Ideals]
        Let \(R\) be a integral domain with fraction field \(F\). A fractional ideal is a nonzero \(R\)-submodule \(\mathcal{A} \subseteq F\) such that \(d \mathcal{A} \subseteq R\) for some nonzero \(d \in A\).
    \end{definition}
\end{defbox}

\begin{rembox}
    \begin{remark}
        We say integral ideals of \(R\) and simply mean ideals of \(R\) to distinguish them from fractional ideals which are, despite its name and similarities, not true ideals.
    \end{remark}
\end{rembox}

\begin{defbox}
    \begin{definition}[Equivalence of Fractional Ideals]
        Let \(R\) be a integral domain. Two fractional ideals \(\mathcal{A}\) and \(\mathcal{B}\) of \(R\) are said to be equivalent if there exist \(\alpha\) and \(\beta\) in \(R\) such that
        \begin{align*}
            (\alpha) \mathcal{A} = (\beta) \mathcal{B} \text{.}
        \end{align*}
        In this case, we write \(\mathcal{A} \sim \mathcal{B}\) or simply \(\mathcal{A} = \mathcal{B}\).
    \end{definition}
\end{defbox}
\begin{thmbox}
    \begin{proposition}
        The relation defined above \(\mathcal{A} \sim \mathcal{B}\) is indeed a equivalence relation.
    \end{proposition}
\end{thmbox}
\begin{proof}
    Let \(\mathcal{A}\) and \(\mathcal{B}\) be two fractional ideals of an integral domain \(R\). We show that the relation \(\mathcal{A} \sim \mathcal{B}\) as defined above is a equivalence relation.
    \begin{enumerate}
        \item \textbf{Reflexivity.} Trivially, \((\alpha) \mathcal{A} = (\alpha) \mathcal{A}\) for any \(\alpha \in R\), and we have \(\mathcal{A} \sim \mathcal{A}\).
        \item \textbf{Symmetry.} If \(\mathcal{A} \sim \mathcal{B}\), then \((\alpha) \mathcal{A} = (\beta) \mathcal{B}\), and again it is trivial that \((\beta) \mathcal{B} = (\alpha) \mathcal{A}\), hence \(\mathcal{B} \sim \mathcal{A}\).
        \item \textbf{Transitivity.} Let \(\mathcal{A} \sim \mathcal{B}\) and \(\mathcal{B} \sim \mathcal{C}\) hold. There are \(\alpha, \beta, \gamma, \theta \in R\) such that
        \begin{align*}
            (\alpha) \mathcal{A} = (\beta) \mathcal{B} \quad \text{and} \quad (\gamma) \mathcal{B} = (\theta) \mathcal{C} \text{.}
        \end{align*}
        Multiplying both sides of both equalities by \((\gamma)\) and \((\beta)\) respectively yields
        \begin{align*}
            (\gamma) (\alpha) \mathcal{A} = (\gamma) (\beta) \mathcal{B} \quad \text{and} \quad (\beta)(\gamma) \mathcal{B} = (\beta)(\theta) \mathcal{C} \text{.}
        \end{align*}
        Therefore, we have that \((\alpha \gamma) \mathcal{A} = (\beta \theta) \mathcal{C}\) or in other words \(\mathcal{A} \sim \mathcal{C}\).
    \end{enumerate}
\end{proof}

\begin{defbox}
    \begin{definition}
        Let \(K\) be an algebraic number field. The equivalence classes of ideals of \(R\) form a group called the ideal class group of \(K\) or just class group of \(K\), and write it as \(\mathrm{Cl}(K)\).
    \end{definition}
\end{defbox}

\begin{thmbox}
    \begin{theorem}
        Let \(K\) be an algebraic number field.
        \begin{enumerate}
            \item The ideal class group is indeed an abelian group with ideal multiplication as its operation. \([(1)] = [R]\) is the identity element and 
            \item Each ideal class has an integral ideal representant.
            \item The ideal class group is trivial, i.e. \(\mathrm{Cl}(K) = [(1)]\), if and only if all fractional ideals in \(K\) are principal, which is equivalent to \(\mathcal{O}_K\) being a principal ideal domain.
            \item The ideal class group of \(K\) is finite.
        \end{enumerate}
    \end{theorem}
\end{thmbox}

\begin{rembox}
    \begin{remark}
        \begin{enumerate}
            \item For some integral domains not all fractional ideals are invertible, so not all ideal classes are invertible. In other words, the ideal classes need not be a group for arbitary integral domains.
            \item For Dedekind domains fractional ideals are invertible, so the ideal classes form a group, but they need not be finite.
            \item For a Dedekind domain \(R\), the group \(\mathrm{Cl}(R)\) is trivial if and only if \(R\) is a principal domain which is equivalent to \(R\) being a unique factorization domain, so \(\mathrm{Cl}(R)\) is a measure of how far \(R\) is from having unique factorization of elements.
            \item Every abelian group is isomorphic to the ideal class group of some Dedekind domain.
            \item It is believed that every finite abelian group is isomorphic to the ideal class group of some algebraic number field, but this is unsolved.
            \item If \(R\) is Dedekind, \(\mathrm{Cl}(R)\) can be regarded as a quotient Group
            \begin{align*}
                \mathrm{Cl}(R) = \set{\text{fractional \(R\)-ideals}} / \set{\text{principal fractional \(R\)-ideals}}
            \end{align*}
            \item If all fractional ideal of an integral domain is invertible, then it is a Dedekind domain.
        \end{enumerate}
    \end{remark}
\end{rembox}

\begin{defbox}
    \begin{definition}
        The Kronecker Bound (or Hurwitz bound?)
        \begin{align*}
            C = \prod_{\sigma: K \rightarrow \mathbb{C}} \sum_{i=1}^n |\sigma(e_i)|
        \end{align*}
    \end{definition}
\end{defbox}

\begin{thmbox}
    \begin{theorem}
        The ideal classes of \(\mathcal{O}_K\) are
        \begin{enumerate}
            \item represented by ideals in \(\mathcal{O}_K\) with norm at most \(C\)
            \item generated as a group by prime ideals \(\mathfrak{p}\) with norm at most \(C\).
        \end{enumerate}
    \end{theorem}
\end{thmbox}

\begin{thmbox}
    \begin{theorem}
        Let \(K\) be an algebraic number field of degree \(n\), \(\Delta_K\) be the discriminant of \(K / \mathbb{Q}\), and \(2 r_2 = n - r_1\) be the number of complex embeddings where \(r_1\) is the number of real embeddings. Then every class in the ideal class group of \(K\) contains an integral ideal of norm not exceeding Minowski's bound
        \begin{align*}
            M_K = \sqrt{|\Delta_K|} \left( \frac{4}{\pi} \right)^{r_2} \frac{n!}{n^n} \text{.}
        \end{align*}
        Moreover, the ideal class group is generated by the prime ideals with the norm not exceeding this bound.
    \end{theorem}
\end{thmbox}

\begin{example}
    Let \(K = \mathbb{Q}(i)\).
\end{example}