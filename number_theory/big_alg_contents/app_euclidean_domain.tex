\section{Appendix: Euclidean Domain}
\begin{exmbox}
    \begin{example}
        Let \(A = \mathbb{Z}[i]\), then \(A\) is an Euclidean domain.
    \end{example}
\end{exmbox}
\begin{proof}
    First, we define the Euclidean function to be
    \begin{align*}
        \phi: A \setminus \{0\} \longrightarrow \mathbb{N}_0, \quad x = a + i b \mapsto \phi(x) := a^2 + b^2
    \end{align*}
    for some \(a, b \in \mathbb{Z}\). Let \(\alpha = a + ib \in \mathbb{Z}[i]\) and \(\beta = c + id \in \mathbb{Z}[i]\) with \(a, b, c, d \in \mathbb{Z}\). Consider
    \begin{align*}
        \frac{\alpha}{\beta} = \frac{a + ib}{c + id} = \frac{a + ib}{c + id} \cdot \frac{c - id}{c - id} = \underbrace{\frac{ac + bd}{c^2 + d^2}}_{=: \lambda} + i \underbrace{\frac{bc - ad}{c^2 + d^2}}_{=: \mu} \text{.}
    \end{align*}
    Define \(n \in \mathbb{Z}\) to be the integer closest to \(\lambda\) and \(m \in \mathbb{Z}\) to be the integer closest to \(\mu\). Now, set \(q := n + im\) and \(r := x - yq\). Then,
    \begin{align*}
        \phi(r) = \phi(x - yq) = \phi\left( \frac{x}{y} - q \right) \phi(y) = \phi \left(\lambda + i \mu - n + im\right) \phi(y) = \phi \left(\lambda - n + i(\mu -m)\right) \phi(y) \text{.}
    \end{align*}
    Because of the construction, it is
    \begin{align*}
        |\lambda - n| < \frac{1}{2} \quad \text{and} \quad |\mu - m| < \frac{1}{2} \text{,}
    \end{align*}
    hence
    \begin{align*}
        \phi(\lambda - n + i (\mu - m)) < \left(\frac{1}{2}\right)^2 + \left(\frac{1}{2}\right)^2 = \frac{1}{2}
    \end{align*}
    and thus
    \begin{align*}
        \phi(r) = \phi(\lambda - n + i (\mu - m)) \phi(y) < \phi(y)
    \end{align*}
    as desired. For a given \(x, y \in \mathbb{Z}[i]\), we found \(q, r \in \mathbb{Z}[i]\) such that \(x = yq + r\) with either \(r = 0\) or \(\phi(r) < \phi(y)\), therefore \(\mathbb{Z}[i]\) is an Euclidean domain.
\end{proof}

\begin{exmbox}
    \begin{example}
        \(\mathbb{Z}[\sqrt{3}]\) is an Euclidean domain with the Euclidean function
        \begin{align*}
            \phi: \mathbb{Z}[\sqrt{3}] \longrightarrow \mathbb{N}_0, \quad a + b \sqrt{3} \mapsto |a^2 - 3 b^2|
        \end{align*}
        where \(a, b \in \mathbb{Z}\) are integers.
    \end{example}
\end{exmbox}

\begin{exmbox}
    \begin{example}
        \(\mathbb{Z}[\sqrt{-2}]\) is an Euclidean domain.
    \end{example}
\end{exmbox}

\begin{proof}
    Define the Euclidean function to be
    \begin{align*}
        \phi: \mathbb{Z}[\sqrt{-2}] \setminus \{0\} \longrightarrow \mathbb{N}_0, \quad a + ib \mapsto a^2 + 2 b^2
    \end{align*}
    for some \(a, b \in \mathbb{Z}\). Let \(x = a + b \sqrt{-2} \in \mathbb{Z}[\sqrt{-2}]\) and \(y = c + d \sqrt{-2} \in \mathbb{Z}[\sqrt{-2}]\) where \(a, b, c, d \in \mathbb{Z}\) are integers. Consider
    \begin{align*}
        \frac{x}{y} = \frac{a + b \sqrt{-2}}{c + d \sqrt{-2}} = \frac{a + b \sqrt{-2}}{c + d \sqrt{-2}} \cdot \frac{c - d \sqrt{-2}}{c - d \sqrt{-2}} = \underbrace{\frac{ac + 2bd}{c^2 + 2 d^2}}_{=: \lambda} + \underbrace{\left(\frac{bd - ad}{c^2 + 2 d^2} \right)}_{=: \mu} \sqrt{-2} \text{.}
    \end{align*}
    Define \(n \in \mathbb{Z}\) to be the integer closest to \(\lambda\) and \(m \in \mathbb{Z}\) to be the integer closest \(\mu\). Now set \(q := n + m \sqrt{-2}\) and \(r := x - yq\). Then,
    \begin{align*}
        \phi(r) &= \phi(x - yq) \\
        &= \phi \left(\frac{x}{y} - q\right) \phi(y) \\
        &= \phi(\lambda + \mu \sqrt{-2} - n - m \sqrt{-2}) \phi(y) \\
        &= \phi(\lambda - n + (\mu - m) \sqrt{-2}) \phi(y) \text{.}
    \end{align*}
    Because of the construction, it is
    \begin{align*}
        |\lambda - n| < \frac{1}{2} \quad \text{and} \quad |\mu - m| < \frac{1}{2} \text{,}
    \end{align*}
    hence
    \begin{align*}
        \phi(\lambda - n + (\mu - m) \sqrt{-2}) < \left(\frac{1}{2}\right)^2 + 2 \left(\frac{1}{2}\right)^2 = \frac{3}{4}
    \end{align*}
    and thus
    \begin{align*}
        \phi(r) = \phi(\lambda - n + i (\mu - m)) \phi(y) < \phi(y)
    \end{align*}
    as desired. For a given \(x, y \in \mathbb{Z}[i]\), we found \(q, r \in \mathbb{Z}[i]\) such that \(x = yq + r\) with either \(r = 0\) or \(\phi(r) < \phi(y)\), therefore \(\mathbb{Z}[i]\) is an Euclidean domain.
\end{proof}

\begin{exmbox}
    \begin{example}
        Let \(A = \mathbb{Z}[\alpha]\) where
        \begin{align*}
            \alpha = \frac{1 + \sqrt{-7}}{2}
        \end{align*}
        then \(A\) is an Euclidean domain.
    \end{example}
\end{exmbox}
\begin{proof}
\end{proof}
https://math.stackexchange.com/questions/998716/proof-that-mathbbz-left-frac1-sqrt-192-right-is-a-pid
\begin{exmbox}
    \begin{example}
        Let \(K = \mathbb{Q}(\sqrt{-19})\), then the ring of integers of \(K\) consisting of
        \begin{align*}
            \frac{a + b \sqrt{-19}}{2} \text{,}
        \end{align*}
        where \(a, b \in \mathbb{Z}\) are integers both even or both odd, is a principal ideal domain that is not Euclidean.
    \end{example}
\end{exmbox}
