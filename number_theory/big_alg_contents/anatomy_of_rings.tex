\section{Rings and Objects}
\section{Operations}
* polynomial
* quotient
* kartesian product
\section{Properties}
* noether
* artinian
* local
\section{Types}
\subsection{Integral Domains}
\begin{defbox}
    \begin{definition}
        An integral domain \(A\) is a nonzero commutative ring that meets one of the following equivalent conditions.
        \begin{enumerate}
            \item The product of any two nonzero elements of \(A\) is nonzero, i.e. if \(a \in A \setminus \{0\}\) and \(b \in A \setminus \{0\}\), then \(ab \neq 0\).
            \item The only zero divisor in \(A\) is \(0\), i.e. \(\mathrm{ZD}(A) = \{0\}\).
            \item The zero ideal \((0)\) is a prime ideal in \(A\).
        \end{enumerate}
    \end{definition}
\end{defbox}

\begin{thmbox}
    \begin{proposition}
        If \(A\) is a nonzero ring, then \(A \times A\) is not an integral domain.
    \end{proposition}
\end{thmbox}

\begin{thmbox}
    \begin{proposition}
        Let \(A\) be a ring, and \(\mathfrak{p} \subset A\) an ideal. \(\mathfrak{p}\) is prime if and only if \(A / \mathfrak{p}\) is an integral domain.
    \end{proposition}
\end{thmbox}

\begin{exmbox}
    \begin{example}
        The quotient of an integral domain by an arbitary ideal need not be an integral domain.
    \end{example}
\end{exmbox}

\begin{thmbox}
    \begin{proposition}
        If \(A\) is an integral domain, then \(A[X_1, \ldots, X_n]\) for all \(n \in \mathbb{N}\) is an integral domain.
    \end{proposition}
\end{thmbox}

\begin{exmbox}
    \begin{example}
        An integral domain need not be Noetherian and Noetherian rings need not be an integral domain.
    \end{example}
\end{exmbox}

\begin{exmbox}
    \begin{example}
        An integral domain need not be Artinian and Artinian rings need not be an integral domain.
    \end{example}
\end{exmbox}

\begin{thmbox}
    \begin{proposition}
        An integral domain is Artinian if and only if it is a field.
    \end{proposition}
\end{thmbox}

\begin{exmbox}
    \begin{example}
        An integral domain need not be local and a local ring need not be an integral domain.
    \end{example}
\end{exmbox}

\subsection{Integrally Closed Domains}

\subsection{Unique Factorization Domains}

\begin{thmbox}
    \begin{proposition}
        If \(A\) is a unique factorization domain, then \(A[X_1, \ldots, X_n]\) for all \(n \in \mathbb{N}\) is a unique factorization domain.
    \end{proposition}
\end{thmbox}