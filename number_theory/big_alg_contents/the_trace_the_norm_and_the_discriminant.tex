\begin{defbox}
    \begin{definition}
        A prime number \(p \in \mathbb{N}\) is said to be ramified in an algebraic number field \(K\) if the prime ideal factorization
        \begin{align*}
            (p) = p \mathcal{O}_K = \mathfrak{p}_r^{e_1} \cdots \mathfrak{p}_r^{e_r}
        \end{align*}
        has some \(e_i\) greater than \(1\). If every \(e_i\) equals \(1\) for \(1 \leq i \leq r\), we say \(p\) is unramified in \(K\).
    \end{definition}
\end{defbox}

\begin{example}
    In \(\mathbb{Z}[i]\), \(2\) ramifies because \((1 + i)^2 = (2)\), and it is the only prime to do so.
\end{example}

\begin{thmbox}
    \begin{theorem}
        For an algebraic number field \(K\), the primes which ramify are those dividing the integer \(\mathrm{disc}_\mathbb{Z}(\mathcal{O}_K)\). In particular, only finitely many primes ramify.
    \end{theorem}
\end{thmbox}