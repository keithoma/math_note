\documentclass[a4paper]{article}
\title{Number Theory}
\author{K}


% ---------------------------------------------------------------------
% P A C K A G E S
% ---------------------------------------------------------------------

% typography and formatting
\usepackage[english]{babel}
\usepackage[utf8]{inputenc}
\usepackage{geometry}
\usepackage{exsheets}
\usepackage{environ}

% mathematics
\usepackage{amsthm} % for theorems, and definitions
\usepackage{amssymb}
\usepackage{amsmath}
\usepackage{textcomp}
%\usepackage{MnSymbol} % for \cupdot

% extra
\usepackage{xcolor}
\usepackage{tikz}

% ---------------------------------------------------------------------
% S E T T I N G
% ---------------------------------------------------------------------

% typography and formatting
\geometry{margin=3cm}

\SetupExSheets{
  counter-format = ch.qu,
  counter-within = chapter,
  question/print = true,
  solution/print = true,
}

% mathematics

% extra
\definecolor{mathif}{HTML}{0000A0} % for conditions
\definecolor{maththen}{HTML}{CC5500} % for consequences
\definecolor{mathrem}{HTML}{8b008b} % for notes

\usetikzlibrary{positioning}
\usetikzlibrary{shapes.geometric, arrows}

% ---------------------------------------------------------------------
% C O M M A N D S
% ---------------------------------------------------------------------

\newcommand{\norm}[1]{\left\lVert#1\right\rVert}
\newcommand{\rank}{\text{rank}}
\newcommand{\Vol}{\text{Vol}}

\newcommand{\set}[1]{\left\{\, #1 \,\right\}}
\newcommand{\makeset}[2]{\left\{\, #1 \mid #2 \,\right\}}


\newcommand*\diff{\mathop{}\!\mathrm{d}}
\newcommand*\Diff{\mathop{}\!\mathrm{D}}

\newcommand\restr[2]{{% we make the whole thing an ordinary symbol
  \left.\kern-\nulldelimiterspace % automatically resize the bar with \right
  #1 % the function
  \vphantom{\big|} % pretend it's a little taller at normal size
  \right|_{#2} % this is the delimiter
  }}

% ---------------------------------------------------------------------
% R E N D E R
% ---------------------------------------------------------------------

\newif\ifshowproof
\showprooftrue

\NewEnviron{Proof}{%
    \ifshowproof%
        \begin{proof}%
            \BODY
        \end{proof}%
    \fi%
}%

\begin{document}
\section*{Exercise 1.1}
Let \(A \subset B\) be an integral extension of rings and assume that \(B\) is an integral domain. Suppose \(\mathfrak{q} \subset B\) is a prime ideal and let \(\mathfrak{p} := 
\mathfrak{q} \cap A \subset A\).
\begin{enumerate}
  \item Prove that \(A\) is a field if and only if \(B\) is a field.
  \begin{proof}
    Assume \(A\) is a field and let's do this a little different. Let \(\mathfrak{m}\) be a maximal ideal in \(B\) and fix a nonzero element \(b \in \mathfrak{m}\). Because \(b\) is integral over \(A\), we have an expression
    \begin{align*}
      0 = a_0 + a_1 b + a_2 b^2 + \cdots + a_n b^n \iff -a_0 = a_1 b + a_2 b^2 + \cdots a_n b^n \text{.}
    \end{align*}
    On the right side, for each \(1 \leq i \leq n\), we have that \(a_i b^i\) is in \(\mathfrak{m}\), so the whole sum is in \(\mathfrak{m}\). This implies the absurdity that \(-a_0\), an unit, is contained in \(\mathfrak{m}\). So there is no such thing as nonzero \(b\) in \(\mathfrak{m}\) and \(B\) is a field.
    
    For the other direction of the implication, we will do it traditionally. Let \(B\) be a field and fix an \(x \in A\). \(x\) is a unit in \(B\), so there is a \(y \in B\) with \(xy = 1\). Again, for \(y\) we have the expression
    \begin{align*}
      0 = a_0 + a_1 y + a_2 y^2 + \cdots + a_n y^n
    \end{align*}
    and if we multiply \(x^{n-1}\) on both sides, we yield
    \begin{align*}
      & 0 = a_0 x^{n-1} + a_1 x^{n-2} + a_2 x^{n-3} + a_n y \\
      \iff & -a_0 x^{n-1} - a_1 x^{n-2} - a_2 x^{n-3} =  a_n y \\
      \iff & a_n^{-1} (-a_0 x^{n-1} - a_1 x^{n-2} - a_2 x^{n-3}) = y
    \end{align*}
    In other words, \(y\) is in \(A\) or in different words, \(A\) is a field.
  \end{proof}
  \vspace*{250px}
  \item Show that \(\mathfrak{p}\) is a prime ideal of \(A\) and that \(A / \mathfrak{p}\) can be viewed as a subring of \(B / \mathfrak{q}\).
  \begin{proof}
    \begin{enumerate}
      \item We simply have \(A / (\mathfrak{q} \cap A) \cong (A + \mathfrak{q}) / \mathfrak{q}\) and since \(\mathfrak{q}\) is prime in \(A + \mathfrak{q}\), \(A / \mathfrak{p}\) is an integral domain.
      \item That \(A / \mathfrak{p}\) is a subring of \(B / \mathfrak{q}\) follows from above.
    \end{enumerate}
  \end{proof}
  \vspace*{250px}
  \item Show that \(B / \mathfrak{q}\) is integral over \(A / \mathfrak{p}\).
  \begin{proof}
    Should be clear.
  \end{proof}
  \vspace*{250px}
  \item Deduce that \(\mathfrak{q}\) is maximal in \(B\) if and only if \(\mathfrak{p}\) is maximal \(A\).
  \begin{proof}
    This is clear.
  \end{proof}
\end{enumerate}
\section*{Exercise 1.2}
Let \(K\) be a number field with \([K:\mathbb{Q}] = 2\).
\begin{enumerate}
  \item Show that \(K = \mathbb{Q}(\sqrt{d})\) where \(d\) is square-free.
  \begin{proof}
    Since every extension of a field of characteristic \(0\) is separable, \(K\) is separable, and by the primitive element thoerem, we know that \(K\) is simple. Now the algebraic closure of \(\mathbb{Q}\) is \(\mathbb{C}\), there is an element in \(x \in \mathbb{C}\) such that \(K = \mathbb{Q}(x)\). If \(x^2\) is not rational, then \([K : \mathbb{Q}] > 2\). Now assume that \(x^2\) is not square-free, i.e. there is a prime \(p \in \mathbb{N}\) such that \(n \cdot p^2 = x^2\) for some \(n \in \mathbb{Z}\). Then, \(K = \mathbb{Q}(p \sqrt{n}) = \mathbb{Q}(\sqrt{n})\). Moreover, if \(x^2\) is not an integer, another primitive element that is an integer can be found. All in all, there is a square-free integer \(d\) such that \(K = \mathbb{Q}(\sqrt{d})\).
  \end{proof}
  \item In this setting, show that \(\mathcal{O}_K = \mathbb{Z}[\alpha]\) where
  \begin{equation}
    \alpha = \begin{cases}
      \frac{1+\sqrt{d}}{2} &\qquad \text{ if } d \equiv 1 \mod{4}\\
      \sqrt{d} &\qquad \text{ if } d \not\equiv 1 \mod{4}
    \end{cases} \text{.}
  \end{equation}
  \begin{proof}
    Use minimal polynomials
  \end{proof}
  \item No.
\end{enumerate}
\section*{Exercise 1.3}
Consider \(R = \mathbb{Z}[\sqrt{3}]\) with the norm 
\end{document}