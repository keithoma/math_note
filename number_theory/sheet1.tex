\documentclass[a4paper]{article}
\title{Number Theory}
\author{K}


% ---------------------------------------------------------------------
% P A C K A G E S
% ---------------------------------------------------------------------

% typography and formatting
\usepackage[english]{babel}
\usepackage[utf8]{inputenc}
\usepackage{geometry}
\usepackage{exsheets}
\usepackage{environ}

% mathematics
\usepackage{amsthm} % for theorems, and definitions
\usepackage{amssymb}
\usepackage{amsmath}
\usepackage{textcomp}
%\usepackage{MnSymbol} % for \cupdot

% extra
\usepackage{xcolor}
\usepackage{tikz}

% ---------------------------------------------------------------------
% S E T T I N G
% ---------------------------------------------------------------------

% typography and formatting
\geometry{margin=3cm}

\SetupExSheets{
  counter-format = ch.qu,
  counter-within = chapter,
  question/print = true,
  solution/print = true,
}

% mathematics

% extra
\definecolor{mathif}{HTML}{0000A0} % for conditions
\definecolor{maththen}{HTML}{CC5500} % for consequences
\definecolor{mathrem}{HTML}{8b008b} % for notes

\usetikzlibrary{positioning}
\usetikzlibrary{shapes.geometric, arrows}

% ---------------------------------------------------------------------
% C O M M A N D S
% ---------------------------------------------------------------------

\newcommand{\norm}[1]{\left\lVert#1\right\rVert}
\newcommand{\rank}{\text{rank}}
\newcommand{\Vol}{\text{Vol}}

\newcommand{\set}[1]{\left\{\, #1 \,\right\}}
\newcommand{\makeset}[2]{\left\{\, #1 \mid #2 \,\right\}}
\newcommand{\bigslant}[2]{{\raisebox{.2em}{$#1$}\left/\raisebox{-.2em}{$#2$}\right.}}

\newcommand*\diff{\mathop{}\!\mathrm{d}}
\newcommand*\Diff{\mathop{}\!\mathrm{D}}

\newcommand\restr[2]{{% we make the whole thing an ordinary symbol
  \left.\kern-\nulldelimiterspace % automatically resize the bar with \right
  #1 % the function
  \vphantom{\big|} % pretend it's a little taller at normal size
  \right|_{#2} % this is the delimiter
  }}

% ---------------------------------------------------------------------
% R E N D E R
% ---------------------------------------------------------------------

\newif\ifshowproof
\showprooftrue

\NewEnviron{Proof}{%
    \ifshowproof%
        \begin{proof}%
            \BODY
        \end{proof}%
    \fi%
}%

\begin{document}
\section*{Exercise 1.1}
Let \(A \subset B\) be an integral extension of rings and assume that \(B\) is an integral domain. Suppose \(\mathfrak{q} \subset B\) is a prime ideal and let \(\mathfrak{p} := 
\mathfrak{q} \cap A \subset A\).
\begin{enumerate}
  \item Prove that \(A\) is a field if and only if \(B\) is a field.
  \begin{proof}
    Assume \(A\) is a field. Let \(\mathfrak{m}\) be a maximal ideal in \(B\) and fix a nonzero element \(b \in \mathfrak{m}\). Because \(b\) is integral over \(A\), we have an expression with some \(a_0, \ldots, a_n \in A\)
    \begin{align*}
      0 = a_0 + a_1 b + a_2 b^2 + \cdots + a_n b^n \iff -a_0 = a_1 b + a_2 b^2 + \cdots + a_n b^n \text{.}
    \end{align*}
    On the right side, for each \(1 \leq i \leq n\), we have that \(a_i b^i\) is in \(\mathfrak{m}\), so the whole sum \(\sum_{i=1}^n a_i b^i\) is in \(\mathfrak{m}\).
    
    For the other direction of the implication, let \(B\) be a field and fix an \(x \in A\). \(x\) is a unit in \(B\), so there is a \(y \in B\) with \(xy = 1\). Again, for \(y\) we have the expression
    \begin{align*}
      0 = a_0 + a_1 y + a_2 y^2 + \cdots + a_n y^n
    \end{align*}
    and if we multiply \(x^{n-1}\) on both sides, we yield
    \begin{align*}
      & 0 = a_0 x^{n-1} + a_1 x^{n-2} + a_2 x^{n-3} + \cdots + a_n y \\
      \iff & -a_0 x^{n-1} - a_1 x^{n-2} - a_2 x^{n-3} - \cdots - a_{n-1} =  a_n y \\
      \iff & a_n^{-1} (-a_0 x^{n-1} - a_1 x^{n-2} - a_2 x^{n-3} - \cdots - a_{n-1}) = y
    \end{align*}
    In other words, \(y\) is in \(A\) or in different words, \(A\) is a field.
  \end{proof}
  \newpage
  \item Show that \(\mathfrak{p}\) is a prime ideal of \(A\) and that \(A / \mathfrak{p}\) can be viewed as a subring of \(B / \mathfrak{q}\).
  \begin{proof}
    Consider \(A + \mathfrak{q}\). This is a subring of \(B\) and \(\mathfrak{q}\) is also prime in \(A + \mathfrak{q}\). With the second isomorphism theorem we have
    \begin{equation*}
      \bigslant{A}{\mathfrak{p}} = \bigslant{A}{(A \cap \mathfrak{q})} \simeq \bigslant{(A + \mathfrak{q})}{\mathfrak{q}} \text{,}
    \end{equation*}
    and since the last expression is a integral domain, \(A / \mathfrak{p}\) is an integral domain. The last expression also shows that \(A/\mathfrak{p}\) can be viewed as a subring of \(B / \mathfrak{q}\).
  \end{proof}
  %
  %
  %
  %
  %
  \newpage
  \item Show that \(B / \mathfrak{q}\) is integral over \(A / \mathfrak{p}\).
  \begin{proof}
    Fix a \((b + \mathfrak{q}) \in B/\mathfrak{q}\). Because \(B\) is an integral extension, we have an equation for \(b\) with some \(a_0, \ldots, a_n \in A\)
    \begin{equation*}
      b^n + a_{n-1}b^{n-1} + \cdots + a_0 = 0\text{.}
    \end{equation*}
    If \(b \in B\) and \(a \in A\), then \((a + \mathfrak{p})(b + \mathfrak{q})^n = (ab^n + \mathfrak{q})\). Now we have
    \begin{align*}
      &(b + \mathfrak{q})^n + (a_{n-1} + \mathfrak{q}) (b + \mathfrak{q})^{n-1} + \cdots + (a_0 + \mathfrak{q}) \\
      =& (b^n + \mathfrak{q}) + (a_{n-1}b^{n-1} + \mathfrak{q}) + \cdots + (a_0 + \mathfrak{q}) \\
      =& b^n + a_{n-1}b^{n-1} + \cdots + a_0 + \mathfrak{q} \\
      =& 0 + \mathfrak{q} \text{,}
    \end{align*}
    so \(B/\mathfrak{q}\) is integral over \(A / \mathfrak{p}\).
  \end{proof}
  \newpage
  \item Deduce that \(\mathfrak{q}\) is maximal in \(B\) if and only if \(\mathfrak{p}\) is maximal \(A\).
  \begin{proof}
    \(\mathfrak{q}\) is maximal in \(B\) if and only if \(B/\mathfrak{q}\) is a field. We know from 2. that \(A/\mathfrak{p}\) is a subring of \(B/\mathfrak{q}\) and from 3. that \(B/\mathfrak{q}\) is an integral extension of \(A/\mathfrak{p}\). Applying 1. yields that \(A/\mathfrak{p}\) is a field if and only if \(B/\mathfrak{q}\) is a field. Hence \(\mathfrak{p}\) is maximal in \(A\).
  \end{proof}
\end{enumerate}
\newpage
\section*{Exercise 1.2}
Let \(K\) be a number field with \([K:\mathbb{Q}] = 2\).
\begin{enumerate}
  \item Show that \(K = \mathbb{Q}(\sqrt{d})\) where \(d\) is square-free.
  \begin{proof}
    Since every extension of a field of characteristic \(0\) is separable, \(K\) is separable, and by the primitive element thoerem, we know that \(K\) is simple. Now the algebraic closure of \(\mathbb{Q}\) is \(\mathbb{C}\), there is an element in \(x \in \mathbb{C}\) such that \(K = \mathbb{Q}(x)\). If \(x^2\) is not rational, then \([K : \mathbb{Q}] > 2\). Now assume that \(x^2\) is not square-free, i.e. there is a prime \(p \in \mathbb{N}\) such that \(n \cdot p^2 = x^2\) for some \(n \in \mathbb{Z}\). Then, \(K = \mathbb{Q}(p \sqrt{n}) = \mathbb{Q}(\sqrt{n})\). Moreover, if \(x^2\) is not an integer, another primitive element that is an integer can be found. All in all, there is a square-free integer \(d\) such that \(K = \mathbb{Q}(\sqrt{d})\).
  \end{proof}
  \item In this setting, show that \(\mathcal{O}_K = \mathbb{Z}[\alpha]\) where
  \begin{equation}
    \alpha = \begin{cases}
      \frac{1+\sqrt{d}}{2} &\qquad \text{ if } d \equiv 1 \mod{4}\\
      \sqrt{d} &\qquad \text{ if } d \not\equiv 1 \mod{4}
    \end{cases} \text{.}
  \end{equation}
  \begin{proof}
    Use minimal polynomials
  \end{proof}
  \item No.
\end{enumerate}
\newpage
\section*{Exercise 1.3}
Consider \(R = \mathbb{Z}[\sqrt{3}]\) with the norm \(N: R \longrightarrow \mathbb{N}_0\),
\begin{equation*}
  N(a + b\sqrt{3}) = |a^2 - 3b^2| \text{.}
\end{equation*}
Show that \(R\) is euclidian with respect to this norm.
\begin{proof}
  \begin{align*}
    x_a + x_b \sqrt{3} &= q (y_a + y_b \sqrt{3}) + r \\
    &= (q_a + q_b \sqrt{3})(y_a + y_b \sqrt{3}) + r \\
    &= q_a y_a + q_a y_b \sqrt{3} + q_b y_a \sqrt{3} + 3 q_b y_b + r \\
    &= q_a y_a + 3 q_b y_b + (q_a y_b + q_b y_a) \sqrt{3} + r \\
    &= q_a y_a + 3 q_b y_b + r_a + (q_a y_b + q_b y_a + r_b) \sqrt{3}
  \end{align*}
  So
  \begin{align*}
    x_a &= q_a y_a + 3 q_b y_b + r_a \\
    x_b &= q_a y_b + q_b y_a + r_b
  \end{align*}
  Say \(r \neq 0\). We want to show \(|r_a^2 - 3r_b^2| < |y_a^2 - 3y_b^2|\).
  \begin{align*}
    |r^2_a - 3r^2_b| &= |(x_a - q_a y_a - 3 q_b y_b)^2 - 3(x_b - q_a y_b - q_b y_a)^2| \\
    &= |x_a^2 - 2x_a q_a y_a - 6x_a q_b y_b + q_a^2 y_a^2 + 6q_a y_a q_b y_a + 9q_b^2 y_b^2|
  \end{align*}
  It is
  \begin{align*}
    r_a^2 &= (x_a - q_a y_a - 3 q_b y_b)^2 \\
    &= x_a^2 - 2x_a q_a y_a - 6x_a q_b y_b + q_a^2 y_a^2 + 6q_a q_b y_a y_b + 9q_b^2 y_b^2 \\
    %
    %
    %
    -3r_b^2 &= -3(x_b - q_a y_b - q_b y_a)^2 \\
    &= -3(x_b^2 - 2x_b q_a y_b - 2x_b q_b y_a + q_a^2 y_b^2 + 2 q_a q_b y_a q_b + q_b^2 y_a^2) \\
    &= -3x_b^2 + 6x_b q_a y_b + 6 x_b q_b y_a - 3q_a^2 y_b^2 - 6 q_a q_b y_a q_b - 3q_b^2 y_a^2 \\
    %
    %
    %
    r_a^2 - 3r_b^2 &= x_a^2 - 2x_a q_a y_a - 6x_a q_b y_b -3x_b^2 + 6x_b q_a y_b + 6 x_b q_b y_a + q_a^2 y_a^2 + 9q_b^2 y_b^2 - 3q_a^2 y_b^2 - 3q_b^2 y_a^2
  \end{align*}
  It is enough to show
  \begin{align*}
    x_a^2 - 2x_a q_a y_a - 6x_a q_b y_b + q_a^2 y_a^2 + 6q_a q_b y_a y_b + 9q_b^2 y_b^2 < |y^2_a - 3y_b^2|
  \end{align*}
\end{proof}
\end{document}