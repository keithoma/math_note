\documentclass[a4paper]{article}
\title{Number Theory}
\author{K}


% ---------------------------------------------------------------------
% P A C K A G E S
% ---------------------------------------------------------------------

% typography and formatting
\usepackage[english]{babel}
\usepackage[utf8]{inputenc}
\usepackage{geometry}
\usepackage{exsheets}
\usepackage{environ}

% mathematics
\usepackage{amsthm} % for theorems, and definitions
\usepackage{amssymb}
\usepackage{amsmath}
\usepackage{textcomp}
\usepackage{mathtools}
%\usepackage{MnSymbol} % for \cupdot

% extra
\usepackage{xcolor}
\usepackage{tikz}

% ---------------------------------------------------------------------
% S E T T I N G
% ---------------------------------------------------------------------

% typography and formatting
\geometry{margin=3cm}

\SetupExSheets{
  counter-format = ch.qu,
  counter-within = chapter,
  question/print = true,
  solution/print = true,
}

% mathematics

% extra
\definecolor{mathif}{HTML}{0000A0} % for conditions
\definecolor{maththen}{HTML}{CC5500} % for consequences
\definecolor{mathrem}{HTML}{8b008b} % for notes

\usetikzlibrary{positioning}
\usetikzlibrary{shapes.geometric, arrows}

% ---------------------------------------------------------------------
% C O M M A N D S
% ---------------------------------------------------------------------

\newcommand{\norm}[1]{\left\lVert#1\right\rVert}
\newcommand{\rank}{\text{rank}}
\newcommand{\Vol}{\text{Vol}}

\newcommand{\set}[1]{\left\{\, #1 \,\right\}}
\newcommand{\makeset}[2]{\left\{\, #1 \mid #2 \,\right\}}
\newcommand{\bigslant}[2]{{\raisebox{.2em}{$#1$}\left/\raisebox{-.2em}{$#2$}\right.}}

\newcommand*\diff{\mathop{}\!\mathrm{d}}
\newcommand*\Diff{\mathop{}\!\mathrm{D}}

\newcommand\restr[2]{{% we make the whole thing an ordinary symbol
  \left.\kern-\nulldelimiterspace % automatically resize the bar with \right
  #1 % the function
  \vphantom{\big|} % pretend it's a little taller at normal size
  \right|_{#2} % this is the delimiter
  }}

% ---------------------------------------------------------------------
% R E N D E R
% ---------------------------------------------------------------------

\newif\ifshowproof
\showprooftrue

\NewEnviron{Proof}{%
    \ifshowproof%
        \begin{proof}%
            \BODY
        \end{proof}%
    \fi%
}%

\begin{document}
\section*{Exercise 1.1}
Let \(A \subset B\) be an integral extension of rings and assume that \(B\) is an integral domain. Suppose \(\mathfrak{q} \subset B\) is a prime ideal and let \(\mathfrak{p} := 
\mathfrak{q} \cap A \subset A\).
\begin{enumerate}
  \item Prove that \(A\) is a field if and only if \(B\) is a field.
  \begin{proof}
    Assume \(A\) is a field. Let \(\mathfrak{m}\) be a maximal ideal in \(B\) and fix a nonzero element \(b \in \mathfrak{m}\). Because \(b\) is integral over \(A\), we have an expression with some \(a_0, \ldots, a_n \in A\)
    \begin{align*}
      0 = a_0 + a_1 b + a_2 b^2 + \cdots + a_n b^n \iff -a_0 = a_1 b + a_2 b^2 + \cdots + a_n b^n \text{.}
    \end{align*}
    On the right side, for each \(1 \leq i \leq n\), we have that \(a_i b^i\) is in \(\mathfrak{m}\), so the whole sum \(\sum_{i=1}^n a_i b^i\) is in \(\mathfrak{m}\).
    
    For the other direction of the implication, let \(B\) be a field and fix an \(x \in A\). \(x\) is a unit in \(B\), so there is a \(y \in B\) with \(xy = 1\). Again, for \(y\) we have the expression
    \begin{align*}
      0 = a_0 + a_1 y + a_2 y^2 + \cdots + a_n y^n
    \end{align*}
    and if we multiply \(x^{n-1}\) on both sides, we yield
    \begin{align*}
      & 0 = a_0 x^{n-1} + a_1 x^{n-2} + a_2 x^{n-3} + \cdots + a_n y \\
      \iff & -a_0 x^{n-1} - a_1 x^{n-2} - a_2 x^{n-3} - \cdots - a_{n-1} =  a_n y \\
      \iff & a_n^{-1} (-a_0 x^{n-1} - a_1 x^{n-2} - a_2 x^{n-3} - \cdots - a_{n-1}) = y
    \end{align*}
    In other words, \(y\) is in \(A\) or in different words, \(A\) is a field.
  \end{proof}
  \newpage
  \item Show that \(\mathfrak{p}\) is a prime ideal of \(A\) and that \(A / \mathfrak{p}\) can be viewed as a subring of \(B / \mathfrak{q}\).
  \begin{proof}
    Consider \(A + \mathfrak{q}\). This is a subring of \(B\) and \(\mathfrak{q}\) is also prime in \(A + \mathfrak{q}\). With the second isomorphism theorem we have
    \begin{equation*}
      \bigslant{A}{\mathfrak{p}} = \bigslant{A}{(A \cap \mathfrak{q})} \simeq \bigslant{(A + \mathfrak{q})}{\mathfrak{q}} \text{,}
    \end{equation*}
    and since the last expression is a integral domain, \(A / \mathfrak{p}\) is an integral domain. The last expression also shows that \(A/\mathfrak{p}\) can be viewed as a subring of \(B / \mathfrak{q}\).
  \end{proof}
  %
  %
  %
  %
  %
  \newpage
  \item Show that \(B / \mathfrak{q}\) is integral over \(A / \mathfrak{p}\).
  \begin{proof}
    Fix a \((b + \mathfrak{q}) \in B/\mathfrak{q}\). Because \(B\) is an integral extension, we have an equation for \(b\) with some \(a_0, \ldots, a_n \in A\)
    \begin{equation*}
      b^n + a_{n-1}b^{n-1} + \cdots + a_0 = 0\text{.}
    \end{equation*}
    If \(b \in B\) and \(a \in A\), then \((a + \mathfrak{p})(b + \mathfrak{q})^n = (ab^n + \mathfrak{q})\). Now we have
    \begin{align*}
      &(b + \mathfrak{q})^n + (a_{n-1} + \mathfrak{q}) (b + \mathfrak{q})^{n-1} + \cdots + (a_0 + \mathfrak{q}) \\
      =& (b^n + \mathfrak{q}) + (a_{n-1}b^{n-1} + \mathfrak{q}) + \cdots + (a_0 + \mathfrak{q}) \\
      =& b^n + a_{n-1}b^{n-1} + \cdots + a_0 + \mathfrak{q} \\
      =& 0 + \mathfrak{q} \text{,}
    \end{align*}
    so \(B/\mathfrak{q}\) is integral over \(A / \mathfrak{p}\).
  \end{proof}
  \newpage
  \item Deduce that \(\mathfrak{q}\) is maximal in \(B\) if and only if \(\mathfrak{p}\) is maximal \(A\).
  \begin{proof}
    \(\mathfrak{q}\) is maximal in \(B\) if and only if \(B/\mathfrak{q}\) is a field. We know from 2. that \(A/\mathfrak{p}\) is a subring of \(B/\mathfrak{q}\) and from 3. that \(B/\mathfrak{q}\) is an integral extension of \(A/\mathfrak{p}\). Applying 1. yields that \(A/\mathfrak{p}\) is a field if and only if \(B/\mathfrak{q}\) is a field. Hence \(\mathfrak{p}\) is maximal in \(A\).
  \end{proof}
\end{enumerate}
\newpage
\section*{Exercise 1.2}
Let \(K\) be a number field with \([K:\mathbb{Q}] = 2\).
\begin{enumerate}
  \item Show that \(K = \mathbb{Q}(\sqrt{d})\) where \(d\) is square-free.
  \begin{proof}
    Since every extension of a field of characteristic \(0\) is separable, \(K\) is separable, and by the primitive element thoerem, we know that \(K\) is simple. Now the algebraic closure of \(\mathbb{Q}\) is \(\mathbb{C}\), there is an element in \(x \in \mathbb{C}\) such that \(K = \mathbb{Q}(x)\). If \(x^2\) is not rational, then \([K : \mathbb{Q}] > 2\). Now assume that \(x^2\) is not square-free, i.e. there is a prime \(p \in \mathbb{N}\) such that \(n \cdot p^2 = x^2\) for some \(n \in \mathbb{Z}\). Then, \(K = \mathbb{Q}(p \sqrt{n}) = \mathbb{Q}(\sqrt{n})\). Moreover, if \(x^2\) is not an integer, another primitive element that is an integer can be found. All in all, there is a square-free integer \(d\) such that \(K = \mathbb{Q}(\sqrt{d})\).
  \end{proof}
  \item In this setting, show that \(\mathcal{O}_K = \mathbb{Z}[\alpha]\) where
  \begin{equation}
    \alpha = \begin{cases}
      \frac{1+\sqrt{d}}{2} &\qquad \text{ if } d \equiv 1 \mod{4}\\
      \sqrt{d} &\qquad \text{ if } d \not\equiv 1 \mod{4}
    \end{cases} \text{.}
  \end{equation}
  \begin{proof}
    Use minimal polynomials
  \end{proof}
  \item No.
\end{enumerate}
\newpage
\section*{Exercise 1.3}
Consider \(R = \mathbb{Z}[\sqrt{3}]\) with the norm \(N: R \longrightarrow \mathbb{N}_0\),
\begin{equation*}
  N(a + b\sqrt{3}) = |a^2 - 3b^2| \text{.}
\end{equation*}
Show that \(R\) is Euclidean with respect to this norm.
\begin{proof}
  Let \(x, y \in R\) and write \(x = x_a + x_b \sqrt{3}\) and \(y = y_a + y_b \sqrt{3}\). We have
  \begin{align*}
    \frac{x}{y} &= \frac{x_a + x_b \sqrt{3}}{y_a + y_b \sqrt{3}} \\
    &= \frac{x_a + x_b \sqrt{3}}{y_a + y_b \sqrt{3}} \cdot \frac{y_a - y_b \sqrt{3}}{y_a - y_b \sqrt{3}} \\
    &= \frac{x_a y_a - 3x_b y_b + (x_b y_a - x_a y_b) \sqrt{3}}{y^2_a - 3y_b^2} \\
    &= \underbrace{\frac{x_a y_a - 3x_b y_b}{y^2_a - 3y_b^2}}_{=: \alpha} + \underbrace{\frac{x_b y_a - x_a y_b}{y^2_a - 3y_b^2}}_{=: \beta} \sqrt{3} \text{.}
  \end{align*}
  Set \(z_\alpha \in \mathbb{Z}\) to be the closest integer to \(\alpha\) and \(z_\beta \in \mathbb{Z}\) to be the closest integer to \(\beta\).

  To show that \(R\) is Euclidean, we want to find \(p, r \in \mathbb{Z}[\sqrt{3}]\) such that \(x = p y + r\). Set \(\theta := (\alpha - z_\alpha) + (\beta - z_\beta)\sqrt{3}\). We claim that
  \begin{align*}
    p = (z_\alpha + z_\beta \sqrt{3}) \qquad \text{and} \qquad r = y \theta \text{.}
  \end{align*}
  We have
  \begin{align*}
    r &= y \theta \\
    &= y ((\alpha - z_\alpha) + (\beta - z_\beta) \sqrt{3}) \\
    &= y (\alpha - z_\alpha + \beta \sqrt{3} - z_\beta \sqrt{3}) \\
    &= y ((\alpha + \beta \sqrt{3}) - (z_\alpha + z_\beta \sqrt{3})) \\
    &=  y(\alpha + \beta \sqrt{3}) - y(z_\alpha + z_\beta \sqrt{3}) \\
    &= y \frac{x}{y} - py \\
    &= x - py
  \end{align*}
  Addint \(py\) on both ends yields the representation \(x = py + r\) as desired.

  We show \(N(r) < N(y)\). Because \(|\alpha - z_\alpha| < 2\) and \(|\beta - z_\beta| < 2\), we have
  \begin{align*}
    N(r) &= N(y \theta) \\
    &= N(y) N(\theta) \\
    &= N(y) \cdot |(\alpha - z_\alpha)^2 - 3 (\beta - z_\beta)^2| \\
    &\leq N(y) \cdot \max\{(\alpha - z_\alpha)^2, 3 (\beta - z_\beta)^2\} \\
    &\leq N(y) \cdot \frac{3}{4} \\
    &\leq N(y)
  \end{align*}
\end{proof}
\newpage
\section*{Exercise 1.4}
Let \(R = \mathbb{Z}[\sqrt{-5}]\). Show that \(R\) is not a unique factorization domain by taking the following steps.
\begin{enumerate}
  \item Compute the group of units \(R^\times\).
  \begin{proof}
    Define a norm \(N: R \longrightarrow \mathbb{N}_0\) as \(N(a + b\sqrt{5}) = a^2 + 5b^2\) and let \(x + y \sqrt{5} \in \mathbb{Z}[\sqrt{5}]\). If \(x + y \sqrt{5}\) is a unit, then \(N(x + y \sqrt{5}) = 1\), and the only integers that satisfy \(a^2 + 5b^2 = 1\) is \(a = \pm 1\) and \(b = 0\). Therefore, the only units in \(R\) is \(\pm 1\).
  \end{proof}
  \item Find two different factorizations of \(6 \in R\) into irreducible factors.
  \begin{proof}
    Trivially, \(2 \cdot 3 = 6\). Also, it is not hard to find \((1 + \sqrt{-5})(1 - \sqrt{-5}) = 6\).
  \end{proof}
  \item Show that the factors appearing are pairwise non-associated.
  \begin{proof}
    This is clear. Conclude that there are distinct factorizations in \(\mathbb{Z}[\sqrt{-5}]\), hence it is not a unique factorization domain.
  \end{proof}
\end{enumerate}
\newpage
\section*{Exercise 1.4+}
Let \(R = \mathbb{Z}[\sqrt{5}]\). Show that \(R\) is not a unique factorization domain by taking the following steps.
\begin{enumerate}
  \item Compute the group of units \(R^\times\).
  \begin{proof}
    Let \(a + b \sqrt{5} \in \mathbb{Z}[\sqrt{5}]\). We want to find another element \(x + y \sqrt{5} \in \mathbb{Z}[\sqrt{5}]\) such that their product is \(1\). We have
    \begin{align*}
      1 &= (a + b \sqrt{5}) (x + y \sqrt{5}) \\
      &= ax + ay\sqrt{5} + bx \sqrt{5} + 5 by \\
      &= (ax + 5by) + (bx + ay) \sqrt{5}
    \end{align*}
    So we have a system of linear equations
    \begin{align*}
      ax + 5 by &= 1 \\
      bx + ay &= 0 \\
    \end{align*}
    where \(x\) and \(y\) are the variables.

    If \(b = 0\), then
    \begin{align*}
      ax &= 1 \\
      ay &= 0 \text{.}
    \end{align*}
    Because \(a \neq 0\), we have \(y = 0\), and since \(x \in \mathbb{Z}\), the only units in \(\mathbb{Z}[\sqrt{5}]\) with \(b = 0\) is \(\pm 1\).

    If \(b \neq 0\), then multiplying the second equation yields
    \begin{align*}
      \begin{rcases*}
        ax + 5by &= 1 \\
        ax + \frac{a^2}{b}y &= 0
      \end{rcases*}
      \, \Rightarrow \, \frac{b}{5b^2 - a^2} = y
    \end{align*}
    so \((5b - a^2 b^{-1})^{-1} = y\). But \(y\) can 
  \end{proof}
\end{enumerate}
\end{document}