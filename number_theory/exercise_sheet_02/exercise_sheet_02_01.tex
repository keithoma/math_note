\documentclass[a4paper]{article}
\title{Integration and Integration}
\author{K}


% ---------------------------------------------------------------------
% P A C K A G E S
% ---------------------------------------------------------------------

% typography and formatting
\usepackage[english]{babel}
\usepackage[utf8]{inputenc}
\usepackage{geometry}
\usepackage{exsheets}
\usepackage{environ}

% mathematics
\usepackage{amsthm} % for theorems, and definitions
\usepackage{amssymb}
\usepackage{amsmath}
\usepackage{textcomp}
%\usepackage{MnSymbol} % for \cupdot

% extra
\usepackage{xcolor}
\usepackage{tikz}

% ---------------------------------------------------------------------
% S E T T I N G
% ---------------------------------------------------------------------

% typography and formatting
\geometry{margin=3cm}

\SetupExSheets{
  counter-format = ch.qu,
  counter-within = chapter,
  question/print = true,
  solution/print = true,
}

% mathematics
\theoremstyle{definition}
\newtheorem{definition}{Definition}
\newtheorem{example}{Example}[definition]

\newtheorem{theorem}{Theorem}[definition]
\newtheorem{corollary}{Corollary}
\newtheorem{lemma}{Lemma}[definition]
\newtheorem{proposition}{Proposition}[definition]

\newtheorem*{remark}{Remark}

% extra
\definecolor{mathif}{HTML}{0000A0} % for conditions
\definecolor{maththen}{HTML}{CC5500} % for consequences
\definecolor{mathrem}{HTML}{8b008b} % for notes

\usetikzlibrary{positioning}
\usetikzlibrary{shapes.geometric, arrows}

% ---------------------------------------------------------------------
% C O M M A N D S
% ---------------------------------------------------------------------

\newcommand{\norm}[1]{\left\lVert#1\right\rVert}
\newcommand{\rank}{\text{rank}}
\newcommand{\Vol}{\text{Vol}}
\newcommand*\diff{\mathop{}\!\mathrm{d}}
\newcommand*\Diff{\mathop{}\!\mathrm{D}}

\newcommand\restr[2]{{% we make the whole thing an ordinary symbol
  \left.\kern-\nulldelimiterspace % automatically resize the bar with \right
  #1 % the function
  \vphantom{\big|} % pretend it's a little taller at normal size
  \right|_{#2} % this is the delimiter
  }}

% ---------------------------------------------------------------------
% R E N D E R
% ---------------------------------------------------------------------

\newif\ifshowproof
\showprooftrue

\NewEnviron{Proof}{%
    \ifshowproof%
        \begin{proof}%
            \BODY
        \end{proof}%
    \fi%
}%

\begin{document}
\begin{center}
    \noindent\textbf{Exercise Sheet 2}
\end{center}
\noindent\textbf{Exercise 1}

\noindent A polynomial \(f(X) \in \mathbb{Z}[X]\) is primitive if the greatest common divisor of its coefficients is \(1\). Show the following:
\begin{enumerate}
    \item If \(f(X), g(X) \in \mathbb{Z}[X]\) are primitive, then the product \(f(X)g(X)\) is also primitive.
    \item \(f(X) \in \mathbb{Z}[X]\) is irreducible in \(\mathbb{Z}[X]\) if and only if it is primitive and irreducible in \(\mathbb{Q}[X]\).
    \item If a monic \(f(X) \in \mathbb{Z}[X]\) factors as \(f(X) = g(X)h(X)\) with \(g(X), h(X) \in \mathbb{Q}[X]\) monic, then \(g(X), h(X) \in \mathbb{Z}[X]\).
\end{enumerate}
Do the analogous statements hold if we replace \(\mathbb{Z}\) by any UFD A, and \(\mathbb{Q}\) by its field of fractions \(K = \text{Quot}(A)\).

\bigskip

\noindent\textbf{Solution}

\noindent\underline{\textbf{1.}}

\noindent Denote the coefficients of \(f\) and \(g\) with \(a_i\) and \(b_j\) for \(1 \leq i \leq \deg f\) and \(1 \leq j \leq \deg g\) such that

\begin{align}
    f(X) = \sum_{i = 0}^{\deg f} a_i X^i && g(X) = \sum_{j = 0}^{\deg g} b_j X^j
\end{align}

\noindent Assume there is a prime \(p \in \mathbb{Z}\) that divides all coefficients of \(fg\) and let \(a_n\) and \(b_m\) be the first coefficients in \(f\) and \(g\) respectively that are not divisble by \(p\). Such \(a_n\) and \(b_m\) must exist because \(f\) and \(g\) are primitive.

\bigskip

\noindent Consider \(X^{n + m}\) in the polynomial \(fg\). The coefficient for this term is the sum of products of \(a_i\) and \(b_j\) for which \(i + j = n + m\), i.e.

\begin{align}
    a_n b_m + a_{n - 1} b_{m+1} + a_{n + 1} b_{m - 1} + a_{n - 2} b_{m + 2} + \dots \text{.}
\end{align}

\noindent This coefficient is however not divisible by \(p\) as \(p\) divides all but the first term. Hence we have a contradiction.

\noindent \underline{\textbf{2.}}

\noindent Let \(f\) be irreducible in \(\mathbb{Z}[X]\) and assume that is reducible in \(\mathbb{Q}[X]\). From the assumption, we have that \(f(X) = g(X)h(X)\) for some \(g, h \in \mathbb{Q}[X]\) that are not units in \(\mathbb{Q}[X]\).

\bigskip

\noindent For two rational numbers, the greatest common divisor is defined as in the follow manner

\begin{align}
    \gcd \left( \frac{a}{b}, \frac{c}{d} \right) = \frac{\gcd(a \cdot d, b \cdot c)} {b \cdot d} \text{.}
\end{align}

\noindent We also have

\begin{align}
    \left(\frac{b \cdot d}{\gcd(a \cdot d, b \cdot c)}\right) \cdot \left( \frac{a}{b} \right) = \frac{a \cdot d}{\gcd(a \cdot d, b \cdot c)} \in \mathbb{Z}[X] \text{.}
\end{align}

\noindent as \(\gcd(a \cdot d, b \cdot c)\) divides \(a\) or \(d\).

\bigskip

\noindent Define \(d_g\) and \(d_h\) to be the greatest common divisor of the coefficients of \(g\) and \(h\) respectively. Consider \(\tilde{g} := d_g^{-1} g\) and \(\tilde{h} := d_h^{-1} h\). From above, we know that \(\tilde{g}, \tilde{h} \in \mathbb{Z}\), and moreover, these are primitive. Hence their product

\begin{align}
    \tilde{g}\tilde{h} = d_g^{-1} d_h^{-1} g h = d_g^{-1} d_h^{-1} f
\end{align}

\noindent is also primitive.

\bigskip

\noindent But \(f\) is also already primitive and since \(\tilde{g}\tilde{h} \in \mathbb{Z}[X]\) we have that \(d_g^{-1} d_h^{-1} = 1\). In other words, we have a factorization \(f(X) = \tilde{g}(X) \tilde{h}(X)\) which is a contradiction.

\bigskip
\bigskip

\noindent On the other hand, let \(f\) be primitive and irreducible in \(\mathbb{Q}[X]\), but assume it is reducible in \(\mathbb{Z}[X]\).

\bigskip

\noindent If \(f\) is a constant, then it is \(f(X) = \pm 1\) as \(f\) is primitive. This is a contradiction, however, because \(\pm 1\) is a unit in \(\mathbb{Q}[X]\).

\bigskip

\noindent Consider the case where \(\deg \geq 1\). From the assumption, we have a factorization \(f(X) = g(X)h(x)\) with \(g, h \in \mathbb{Z}[X]\) but \(g, h \neq \pm 1\).

\bigskip

\noindent Assume \(g\) is a constant, then \(g\) divides all coefficient of \(f\) in \(\mathbb{Z}\). This cannot be since \(f\) is primitive. Therefore, we have \(\deg g \geq 1\) which means that \(g\) is not a unit in \(\mathbb{Q}[X]\).

\bigskip

\noindent Apply the same argument for \(h\) and we have \(f(X) = g(X)h(X)\) is a non-trivial factorization in \(\mathbb{Q}[X]\). This is a contradiction with the first assumption.

\bigskip

\noindent \underline{\textbf{3.}}

\noindent Let \(f \in \mathbb{Z}[X]\) be monic and \(f(X) = g(X)h(X)\) with \(g\) and \(h\) monic. Assume \(g, h \not\in \mathbb{Z}[X]\). There are some \(n \in \mathbb{Z}\) such that \(nf(X) = \tilde{g}(X) \tilde{h}(X)\) such that \(\tilde{g}, \tilde{h} \in \mathbb{Z}[X]\) (e.g. least common multiple) and let \(n\) be the smallest of such integers.

\bigskip

\noindent Since from the assumption we know that \(n \geq 1\), there is a \(p \in \mathbb{Z}\) that divides \(n\). Now assume \(p\) does not divide all coefficients of \(\tilde{g}\) nor \(\tilde{h}\). Similary to 1., let \(a_n\) and \(b_m\) be the first coefficients that are not divisble by \(p\) and consider the coefficient for \(X^{n+m}\). We again have

\begin{align}
    a_n b_m + a_{n - 1} b_{m+1} + a_{n + 1} b_{m - 1} + a_{n - 2} b_{m + 2} + \dots \text{.}
\end{align}

\noindent which is not divisble by \(p\). Therefore, \(p\) must divide \(\tilde{g}\) or \(\tilde{h}\). We have

\begin{align}
    \frac{n}{p} f(X) = \hat{g} \hat{h} 
\end{align}

\noindent with \(\hat{g}, \hat{h} \in \mathbb{Z}[X]\). This is a contradiction however, as we required \(n\) to be the smallest integer.

\bigskip
\bigskip

\noindent All three proofs can be analogously applied to any UFD \(A\) and \(\text{Quot}(A)\).


\end{document}