\documentclass[a4paper]{article}
\title{Integration and Integration}
\author{K}


% ---------------------------------------------------------------------
% P A C K A G E S
% ---------------------------------------------------------------------

% typography and formatting
\usepackage[english]{babel}
\usepackage[utf8]{inputenc}
\usepackage{geometry}
\usepackage{exsheets}
\usepackage{environ}

% mathematics
\usepackage{amsthm} % for theorems, and definitions
\usepackage{amssymb}
\usepackage{amsmath}
\usepackage{textcomp}
%\usepackage{MnSymbol} % for \cupdot

% extra
\usepackage{xcolor}
\usepackage{tikz}

% ---------------------------------------------------------------------
% S E T T I N G
% ---------------------------------------------------------------------

% typography and formatting
\geometry{margin=3cm}

\SetupExSheets{
  counter-format = ch.qu,
  counter-within = chapter,
  question/print = true,
  solution/print = true,
}

% mathematics
\theoremstyle{definition}
\newtheorem{definition}{Definition}
\newtheorem{example}{Example}[definition]

\newtheorem{theorem}{Theorem}[definition]
\newtheorem{corollary}{Corollary}
\newtheorem{lemma}{Lemma}[definition]
\newtheorem{proposition}{Proposition}[definition]

\newtheorem*{remark}{Remark}

% extra
\definecolor{mathif}{HTML}{0000A0} % for conditions
\definecolor{maththen}{HTML}{CC5500} % for consequences
\definecolor{mathrem}{HTML}{8b008b} % for notes

\usetikzlibrary{positioning}
\usetikzlibrary{shapes.geometric, arrows}

% ---------------------------------------------------------------------
% C O M M A N D S
% ---------------------------------------------------------------------

\newcommand{\norm}[1]{\left\lVert#1\right\rVert}
\newcommand{\rank}{\text{rank}}
\newcommand{\Vol}{\text{Vol}}
\newcommand*\diff{\mathop{}\!\mathrm{d}}
\newcommand*\Diff{\mathop{}\!\mathrm{D}}

\newcommand\restr[2]{{% we make the whole thing an ordinary symbol
  \left.\kern-\nulldelimiterspace % automatically resize the bar with \right
  #1 % the function
  \vphantom{\big|} % pretend it's a little taller at normal size
  \right|_{#2} % this is the delimiter
  }}

% ---------------------------------------------------------------------
% R E N D E R
% ---------------------------------------------------------------------

\newif\ifshowproof
\showprooftrue

\NewEnviron{Proof}{%
    \ifshowproof%
        \begin{proof}%
            \BODY
        \end{proof}%
    \fi%
}%

\begin{document}
\begin{center}
    \noindent\textbf{Exercise Sheet 1}
\end{center}
\noindent\textbf{Exercise 1}

\noindent A polynomial \(f(X) \in \mathbb{Z}[X]\) is primitive if the greatest common divisor of its coefficients is \(1\). Show the following:
\begin{enumerate}
    \item If \(f(X), g(X) \in \mathbb{Z}[X]\) are primitive, then the product \(f(X)g(X)\) is also primitive.
\end{enumerate}

\noindent\textbf{Solution}

\noindent\underline{\textbf{1.}}

\noindent Denote the coefficients of \(f\) and \(g\) with \(a_i\) and \(b_j\) for \(1 \leq i \leq \deg f\) and \(1 \leq j \leq \deg g\) such that

\begin{align}
    f(X) = \sum_{i = 0}^{\deg f} a_i X^i && g(X) = \sum_{j = 0}^{\deg g} b_j X^j
\end{align}

\noindent Assume there is a prime \(x \in \mathbb{Z}\) that divides all coefficients of \(fg\) and let \(a_n\) and \(b_m\) be the first coefficients in \(f\) and \(g\) respectively that are not divisble by \(x\).

\noindent Consider \(X^{n + m}\) in the polynomial \(fg\). The coefficient for this term is the sum of products of \(a_i\) and \(b_j\) for which \(i + j = n + m\), i.e.

\begin{align}
    a_n b_m + a_{n - 1} b_{m+1} + a_{n + 1} b_{m - 1} + a_{n - 2} b_{m + 2} + \dots \text{.}
\end{align}

\noindent This coefficient is however not divisible by \(x\) as \(x\) divides all but the first term. Hence we have a contradiction.

\noindent \underline{\textbf{2.}}

\noindent On the other hand, let \(f\) be primitive and irreducible in \(\mathbb{Q}[X]\), but assume it is reducible in \(\mathbb{Z}[X]\).

If \(f\) is a constant, then it it \(f(X) = \pm 1\) as \(f\) is primitive. This is a contradiction, however, because \(\pm 1\) is a unit in \(\mathbb{Q}[X]\).

Consider the case where \(\deg \geq 1\). From the assumption, we have a factorization \(f(X) = g(X)h(x)\) with \(g, h \in \mathbb{Z}[X]\) but \(g, h \neq \pm 1\).

Assume \(g\) is a constant, then \(g\) divides all coefficient of \(f\) in \(\mathbb{Z}\). This cannot be since \(f\) is primitive. Therefore, we have \(\deg g \geq 1\) which means that \(g\) is not a unit in \(\mathbb{Q}[X]\).

Apply the same argument for \(h\) and we have \(f(X) = g(X)h(X)\) is a non-trivial factorization in \(\mathbb{Q}[X]\). This is a contradiction with the first assumption.
\end{document}